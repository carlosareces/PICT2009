\MySubSection{3. Resumen}

En este proyecto se propone implementar un sistema de di\'alogo con
soporte de interacci\'on en tres dimensiones,  que genere autom\'aticamente
instrucci\'ones en lenguaje natural que ayuden a un usuario a cumplir una tarea determinada. Para este fin, se propone investigar temas fundamentales sobre la
Interacci\'on Hombre-M\'aquina (IHM o HCI de su sigla en Ingl\'es Human-Computer Interaction) en entornos virtuales. Como resultado de este proyecto se espera la implementaci\'on, evaluaci\'on y aplicaci\'on de un sistema prototipo. Las tres
\'areas principales en las que se pueden clasificar los resultados del proyecto son: (1) pragm\'atica de la interacci\'on, (2) representaci\'on
de la informaci\'on e inferencia, (3) evaluaci\'on de sistemas
de di\'alogo. Una vez obtenido
un prototipo, se planea su aplicaci\'on a la tarea espec\'ifica del aprendizaje de idiomas al utilizar el sistema como un ``profesor de idiomas" virtual.

En una primera etapa investigaremos un modelo de interacci\'on unidireccional
(i.e., el flujo de informaci\'on ser\'a desde el sistema hacia el usuario).
En etapas subsiguientes el modelo ser\'a extendido para
permitir intercambio ling\"u\'istico bidireccional, por ejemplo, para que el usuario pida clarificaciones o ayuda al sistema.

Dise\~nar la arquitectura de un sistema de di\'alogo presenta
desaf\'ios tanto te\'oricos como pr\'acticos: en lo te\'orico, se necesitan
heur\'isticas que gobiernen la interacci\'on en cuanto a qu\'e decir,
cu\'ando, y c\'omo teniendo en cuenta el contexto actual, y debe contar con m\'etodos
de inferencia que permitan al sistema adaptarse en funci\'on de la situaci\'on actual para alcanzar un objetivo predefinido.
La complejidad del problema te\'orico se
refleja, en lo pr\'actico, en un sistema de m\'ultiples componentes: un
generador de lenguaje natural, un sistema de planeamiento, un entorno 3D, etc.
Dise\~nar e implementar todos estos componentes requerir\'ia un esfuerzo prohibitivo, por lo que usaremos herramientas ya implementadas y disponibles libremente para el prototipado de sistemas de este tipo, como la plataforma \textit{Generating Instructions in Virtual Environments} (GIVE).

Finalmente, la calidad de cada una de las capacidades del sistema afecta la percepci\'on que el usuario tiene de �ste, y por lo tanto, es imperativo evaluar cada uno de ellos.
%dado que todos los distintos aspectos (o capacidades o funcionalidades) del sistema contribuyen a la calidad del producto en su totalidad. En este sentido,
Se planea utilizar distintas t\'ecnicas de evaluaci\'on originarias de la ingenier\'ia de software (como el testing) o de otras \'areas del procesamiento de lenguaje natural (por ejemplo adaptando y aplicando m\'etricas del \'area de la traducci\'on autom\'atica).
