\MySubSection{3. Resumen}

En este proyecto se propone implementar un sistema de di\'alogo que genere
autom\'aticamente
instrucciones en lenguaje natural para ayudar a un usuario a cumplir una tarea
determinada en un entorno virtual 
en tres dimensiones. Para este fin, se propone investigar temas fundamentales
sobre la
in\-teracci\'on hombre-m\'aquina 
en entornos virtuales. Como resultado de este proyecto, se espera la
implementaci\'on, evaluaci\'on y aplicaci\'on de un sistema prototipo. Las tres
\'areas principales en las que se pueden clasificar los resultados del proyecto
son: (1) pragm\'atica de la interacci\'on, (2) representaci\'on
de la informaci\'on e inferencia, (3) evaluaci\'on de sistemas
de di\'alogo. Una vez obtenido
un prototipo, se planea su aplicaci\'on a la tarea espec\'ifica del 
aprendizaje de idiomas, utilizando el sistema como un ``profesor de idiomas"
virtual.

En una primera etapa investigaremos un modelo de interacci\'on 
ling\"u\'istica unidireccional
(i.e., el flujo de informaci\'on ling\"u\'istica ser\'a desde el sistema hacia
el usuario).
En etapas subsiguientes, el modelo ser\'a extendido para
permitir intercambio ling\"u\'istico bidireccional, por ejemplo, para que el
usuario pida clarificaciones o ayuda al sistema.

Dise\~nar la arquitectura de un sistema de di\'alogo presenta
desaf\'ios tanto te\'oricos como pr\'acticos. En lo te\'orico, se necesitan
heur\'isticas que gobiernen la interacci\'on en cuanto a qu\'e decir,
cu\'ando, y c\'omo (teniendo en cuenta el contexto actual). Adem\'as, se debe
contar con m\'etodos
de inferencia que permitan al sistema adaptarse en funci\'on de la situaci\'on
actual, para alcanzar un objetivo predefinido.
La complejidad del problema te\'orico se
refleja, en lo pr\'actico, en un sistema de m\'ultiples componentes: un
generador de lenguaje natural, un sistema de planning, un entorno de 
interacci\'on en tres dimensiones, etc.
Dise\~nar e implementar todos estos componentes requerir\'ia un esfuerzo
prohibitivo. En este proyecto adaptaremos herramientas ya
implementadas y disponibles libremente para el prototipado de sistemas de este
tipo, como la plataforma \emph{Generating Instructions in Virtual
Environments} (GIVE).

La calidad de cada una de las capacidades del sistema afecta la
percepci\'on que el usuario tiene de \'este. Por lo tanto, es imperativo
evaluar cada una de ellas.
Se planean adaptar y aplicar distintas t\'ecnicas y m\'etricas de evaluaci\'on
del \'area de ingenier\'ia de software y de diversas \'areas del procesamiento
de lenguaje natural (por ejemplo, la traducci\'on autom\'atica).
