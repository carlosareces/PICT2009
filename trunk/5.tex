\MySubSection{5. Composici\'on del Grupo de Investigaci\'on y Areas de Especializaci\'on}

\begin{center}\small
    \begin{minipage}{\linewidth}
        \begin{center}
        \renewcommand{\thefootnote}{\thempfootnote}
        \begin{tabular}{|l|l|}
        \hline\hline
        Nombre &  Especializacion \\
        \hline
        Dr.\ Carlos Areces &
          L\'ogica Computacional\\
        & Representaci\'on de Conocimiento\\
        & Complejidad Algor\'itmica
      \\ \hline
        Dra.\ Paula Estrella &
          Lingu\'istica Computacional \\
        & Evaluacion de Sistemas de PLN
      \\ 
		& Traducci\'on Autom\'atica \\ \hline
        Lic. Luciana Benotti\footnote{L. Benotti completar\'a sus estudios doctorales
          en la Universit\'e Henri Poincare en Enero del 2010.} &
          Pragm\'atica \\
        & Sistemas de Di\'alogo\\
        & L\'ogica Computacional\\
        \hline\hline
        \end{tabular}
        \end{center}
    \end{minipage}
\end{center}


\MyParagraph{Colaboraci\'on Internacional:}
Los investigadores del proyecto colaboran actualmente
con grupos de investigaci\'on internacionales relevantes para
el los objetivos a obtener.  Areces y Benotti fueron miembros
del grupo TALARIS \url{http://talaris.loria.fr}.

Dr.\ Patrick Blackburn \url{http://www.loria.fr/~blackbur} (INRIA Nancy Grand Est, France) has
extensive knowledge on computational and algorithmic aspects of
so-called hybrid languages~\citep{arec:hybr99,blac:hybr98c}.

GLyC \url{http://www.glyc.dc.uba.ar/} XXX

Agust\'in Gravano \url{http://www.glyc.dc.uba.ar/agustin/} y
Jose Casta\~no \url{XXX} XXX

Alexandre Koller \url{http://www.coli.uni-saarland.de/~koller/} XXX

Ron Petrick \url{http://homepages.inf.ed.ac.uk/rpetrick/} y
PKS \url{http://homepages.inf.ed.ac.uk/rpetrick/research/pks/} XXX

Paul Piwek \url{http://mcs.open.ac.uk/pp2464/} XXX

NLG Group at the Open University \url{http://mcs.open.ac.uk/nlg/}

La Dra. Paula Estrella colabora actualmente con las siguientes instituciones especialistas en las \'areas de HCI y  Ling\"u\'istica Computacional:
\begin{itemize}
    \item  Instituto de investigaci\'on Idiap, Martigny, Suiza -  \url{http://www.idiap.ch/}
\item Grupo de Tratamiento Inform\'atico Multiling\"ue, Universidad de Ginebra, Suiza - \url{http://www.issco.unige.ch/}
\item Grupo de inteligencia artificial y pervasiva, Universidad de Friburgo, Suiza - \url{http://diuf.unifr.ch/pai/wiki/doku.php}
\end{itemize}

\MyParagraph{Publicaciones m\'as relevantes del grupo investigador:}
\begin{itemize}
    \item Luciana
\citep{benotti09c}
\citep{benotti09b}
\item Carlos

Referring Expressions
\citep{AKS08}
\citep{AF08}

Logic
\citep{ABM01}
\citep{arec:hybr05b}

%\citep{AFFM08}
Inference
\citep{AG06}
\citep{ANR01}

\item La Dra. Paula Estrella ha trabajado en el desarrollo y mejora de sistemas de TA desde y hacia el idioma Espa\~nol \citep{estr:expe05}, ha realizado su tesis doctoral en la evaluaci\'on contextual de sistemas de TA \citep{estr:impr08, estr:femt09} y tambi\'en ha participado en proyectos que involucran la creaci\'on y evaluaci\'on de corpora multimodales \citep{pope:estr07}.
  
\end{itemize}
