\MySubSection{5. Objetivos Espec\'ificos}


En el presente proyecto nos focalizaremos en el sistema encargado de la
generaci\'on de  instrucciones e interpretaci\'on de las respuestas del usuario.
Dicho sistema deber\'a ser capaz de una interacci\'on natural y debe poder adaptarse a
posibles errores de interpretaci\'on o ejecuci\'on de las instrucciones, dando
instrucciones correctivas cuando sea necesario. El usuario se encontrar\'a
en un universo en el que podr\'a interactuar con objetos y explorar el ambiente
virtual provisto. En este mundo virtual 3D, el usuario intentar\'a seguir las instrucciones
provistas por el sistema realizando acciones f\'isicas.

En una primera etapa, el modelo a utilizar permitir\'a el flujo de informaci\'on
ling\"u\'istica de manera unidireccional, es decir,
desde el sistema hacia el usuario, por lo que
el usuario no podr\'a pedir ningu\'n tipo de ayuda.  La restricci\'on a un
modelo unidireccional tiene como objetivo
simplificar la representaci\'on y manejo del contexto de la interacci\'on.

En etapas m\'as avanzadas del proyecto, el modelo unidireccional ser\'a
extendido para
permitir intercambio ling\"u\'istico bidireccional: por ejemplo, el usuario
podr\'a
pedir clarificaciones o redefinir el objetivo a alcanzar, en ambos casos usando
lenguaje natural.

La generaci\'on de instrucciones para un dominio espec\'ifico requiere
%la adquisici\'on del conocimiento relevante y la compilaci\'on del mismo debe
%incluir cierta
informaci\'on  ling\"u\'istica.
%que permita su posterior procesamiento: en los aspectos
A nivel morfol\'ogico y sint\'actico se requiere una
gr\'amatica del lenguaje seleccionado (Espa\~nol en este caso). A nivel
sem\'antico se necesita un repositorio de informaci\'on l\'exica organizada en
forma ontol\'ogica. A nivel pragm\'atico es necesaria una descripci\'on formal de
las
acciones posibles en el dominio (con sus precondiciones y efectos), e
informaci\'on del estado de cada interacci\'on.

Esta informaci\'on relacionada a un dominio dado es utilizada por distintos
componentes del sistema de di\'alogo que
gestiona la interacci\'on con el usuario. En un sistema t\'ipico, esto
sucede como se detalla a continuaci\'on.  En la direcci\'on sistema $\rightarrow$
usuario, el componente encargado de \textit{planificar} genera una secuencia de
acciones relevantes al momento de la interacci\'on, luego el componente
encargado de  \textit{gestionar el conocimiento} actualiza la informaci\'on
contextual y,  finalmente, el componente encargado de
\textit{generar instrucciones}  transmite las mismas como expresiones
del lenguaje natural.  En la direcci\'on opuesta, es decir usuario $\rightarrow$
sistema,  el componente denominado \textit{discretizador}
transforma el flujo continuo de informaci\'on derivada del comportamiento del
usuario, en acciones relevantes al universo en cuesti\'on, mientras que
los componentes de planeamiento y gesti\'on del conocimiento se ocupan
de mantener el contexto actualizado y de detectar posibles errores.

Los objetivos espec\'ificos de este proyecto se listan a continuaci\'on:

\begin{itemize}
    \item[$\triangleright$] Identificar y delimitar las \'areas de trabajo
pertinentes, seg\'un el estado del arte.\\[-2em]
    \item[$\triangleright$] Dise\~nar pol\'iticas de interacci\'on adecuadas
a la
tarea, implementando y comparando algoritmos de planificaci\'on del contenido y
de gesti\'on de reparaciones.\\[-2em]
    \item[$\triangleright$] Investigar
distintos tipos de l\'ogicas y tareas de inferencia para la representaci\'on
y gesti\'on del conoci\-mien\-to requerido por el sistema.\\[-2em]
    \item[$\triangleright$] Integrar los puntos anteriores en un prototipo de tutor virtual.\\[-2em]
    \item[$\triangleright$] Estudiar la
aplicaci\'on de t\'ecnicas evaluativas de otros dominios del PLN (en particular
de la TA) al dominio de los sistemas de di\'alogos.\\[-2em]
    \item[$\triangleright$] Proponer un modelo de evaluaci\'on basada en el
contexto para la evaluaci\'on de prototipos por usuarios reales,
que permitan identificar las debilidades del sistema.\\[-2em]
    \item[$\triangleright$] Documentar y diseminar la experiencia ganada durante
el proyecto a los fines de contribuir a la comunidad cient\'ifica.
\end{itemize}

En la siguiente secci\'on se detalla el plan de trabajo propuesto para llevar a
cabo los objetivos planteados (la Secci\'on 8
provee un cronograma completo), para luego discutir una posible aplicaci\'on del
sistema a desarrollar.
% y, finalmente, la Subsecci\'on 5.3 provee informaci\'on sobre la plataforma
que se utilizar\'a.

\MySubSubSection{5.1. Tareas a Realizar}

Comenzaremos por el relevamiento y estudio de material
bibliogr\'afico de metodolo\-g\'ias existentes para el desarrollo de sistemas de di\'alogo
 en entornos 3D, enfatizando principalmente el estudio de las pol\'iticas de
interacci\'on y las tareas de inferencia propuestos. Entre la bibliograf\'ia b\'asica a
estudiar podemos nombrar:
Pragm\'atica~\citep{grice75,clark96,kehler04,devault08,ginzburg09},
Planning~\citep{kautz99,hoffmann01,nau04,petrick06} y Evaluaci\'on~\citep{Jones96, EAG96, Dybkjaer07, Petrie09}.


Una vez cumplido el relevamiento inicial, el siguiente objetivo es
dise\~nar e implementar los componentes que
conformar\'ian el sistema encargado de la generaci\'on de instrucciones
y de la interpretaci\'on de respuestas del usuario.
Dise\~nar e implementar todos estos componentes empezando desde cero
requerir\'ia un esfuerzo prohibitivo, por lo que proponemos utilizar recursos
disponibles de forma libre y gratuita. En particular, usaremos la plataforma
desarrollada como parte de la competencia\footnote{En este contexto
``competencia'' (del Ingl\'es \emph{evaluation campaign} o \emph{evaluation challenge})
se refiere a la organizaci\'on de un evento en el que participan distintos
grupos de investigaci\'on, realizando todos el mismo ejercicio con sus sistemas
propios y comparando al final los resultados obtenidos por cada grupo. Este tipo
de eventos es popular en distinas \'areas del PLN (TA, reconocimiento del
habla, recuperaci\'on de informaci\'on, entre otras) y benefician
cada \'area generando y diseminando nuevas
tecnolog\'ias.} \textit{Generating Instructions in Virtual Environments}
(GIVE)\footnote{GIVE se encuentra disponible libremente en
\fnturl{http://www.give-challenge.org}, y posee el respaldo de los grupos de
inter\'es SIGSEM, SIGDIAL, y SIGGEN de la \emph{Association for Computational
Linguistics} (ACL).} que
provee un prototipo b\'asico de este tipo de sistemas.

El objetivo principal de GIVE~\citep{byron09} es actuar como medio de
evaluaci\'on de sistemas de generaci\'on
de lenguaje natural. En esta competencia, los sistemas se eval\'uan de forma
homog\'enea dado que todos los sistemas participantes utilizan los mismos
recursos provistos por los organizadores.
En el escenario propuesto por GIVE, el usuario
humano lleva acabo una tarea de ``b\'usqueda del tesoro" en un entorno
virtual 3D y el trabajo del sistema participante es proveer, en tiempo
real, instrucciones en lenguaje natural que ayuden al usuario
a encontrar el tesoro escondido.  Por lo tanto, GIVE no s\'olo eval\'ua
la generaci\'on de lenguaje natural sino tambi\'en la habilidad del
sistema de participar en una interacci\'on situada en un entorno 3D.

La plataforma GIVE es, entonces, una herramienta propicia para
investigar los distintos aspectos de la interacci\'on situada en un entorno
virtual que se enumeran a continuaci\'on: (1) el problema de realizaci\'on
sint\'actica y morf\'ogica de las
instrucciones (en Ingl\'es \emph{surface realization});
(2) la representaci\'on del conocimiento existente en el contexto de
interacci\'on (contexto del discurso, ontolog\'ias, acciones posibles); (3)
el tipo de inferencia requerido por las tareas involucradas
(planeamiento, construcci\'on y chequeo de modelos); y (4)
las reglas pragm\'aticas de interacci\'on requeridas por la tarea
(administraci\'on de la carga cognitiva, actualizaci\'on y uso de la
informaci\'on contextual. Adicionalmente, GIVE provee
herramientas para registrar todos los detalles de la
interacci\'on, permitiendo de esta forma obtener f\'acilmente un corpus de
interacci\'on en entornos virtuales, anotado autom\'aticamente.

Para la definici\'on correcta de las pol\'iticas de interacci\'on que
utilizaremos en nuestro prototipo, es necesario
contar con un corpus que provea ejemplos de interacci\'on t\'ipica sobre la
tarea~\citep{HCRC-93, byron-06}.  Utilizando herramientas provistas por GIVE,
recopilaremos un corpus de interacci\'on humano-humano (uni y bidireccional)
que contendr\'a la interacci\'on lig\"u\'istica alineada con las acciones
realizadas en el mundo virtual.

Utilizando el corpus recolectado, y bas\'andonos en la plataforma GIVE,
comenzaremos el dise\~no,
implementaci\'on y testing de algoritmos de interacci\'on unidireccionales.
Existen tradicionalmente cuatro tareas que un sistema de generaci\'on
de instrucciones debe implementar: planificaci\'on del contenido, generaci\'on
de expresiones referenciales, gesti\'on del contexto de interacci\'on, e
interpretaci\'on de las respuestas del usuario.

Para planificar el contenido a generar es necesario primero obtener un plan de
acciones que alcance el objetivo desde el estado actual. Dicho plan contrendr\'a
acciones f\'isicas a ejecutar sobre el entorno. El segundo paso es decidir
c\'omo se transmitir\'a al usuario esta secuencia de acciones. Es decir,
se debe decidir cu\'antas acciones comunicar
por instrucci\'on y c\'omo agregarlas coherentemente. El
resultado del proceso de agregaci\'on de acciones es un \'arbol que describe la
estructura de la tarea a diferentes niveles de abstracci\'on. El tercer, y \'ultimo, paso
es decidir c\'omo navegar el \'arbol de acciones para verbalizar las
instruciones (por ejemplo, post o preorden~\citep{foster-etal-ijcai2009}). En este proyecto investigaremos
diferentes pol\'iticas de agregaci\'on (e.g., agregando acciones que manipulan el mismo
objeto) y pol\'iticas no est\'andard de recorrido del \'arbol de acciones (e.g., bajando a un
nivel menor de abstracci\'on en caso de malentendidos).

Para obtener el plan de acciones, utilizaremos la tarea de inferencia de
planning~\citep{nau04}. GIVE provee un sistema de planning muy
limitado.  Existen sistemas de planning m\'as avanzandos (en particular,
que permiten el manejo de informaci\'on incompleta sobre el dominio)
como PKS\footnote{PKS est\'a disponible en \fnturl{http://homepages.inf.ed.ac.uk/rpetrick/research/pks/}}, pero
est\'an todav\'ia en periodo de desarrollo.  En general, el estado del
arte en el \'area de planning no cubre los requerimientos de nuestro
sistema.  Si bien, existen sistemas optimizados que funcionan adecuadamente
en ciertas aplicaciones, ninguno provee servicios como la generaci\'on de
planes alternativos, o la generaci\'on de planes incompletos en caso de
absencia de plan. Por lo que deberemos dise\~nar e implementar estas
extensiones a los algoritmos de planning. Estudiaremos tambi\'en el
comportamiento te\'orico (e.g., complejidad) de estos algoritmos.

Una vez terminada la fase de planificaci\'on de contenido, la siguiente tarea
es la generaci\'on de expresiones referenciales. Esto quiere decir, producir una
frase que describa una entidad referenciable de forma tal que el usuario
la pueda identificar (e.g., ``el jarr\'on que est\'a sobre la mesa'').
Estas expresiones, para ser aceptables, deben ser similares a las que podr\'ia
producir una persona en condiciones normales (por ejemplo, no ser\'ia aceptable,
``el jarr\'on que no est\'a arriba de la silla ni arriba del
sof\'a ni abajo de la mesa'').  En~\citep{AKS08} se propone utilizar la
minimizaci\'on simb\'olica del modelo que representa
el estado del mundo, para as\'i obtener f\'ormulas l\'ogicas que describan
un\'ivocamente a cada objeto. En nuestro proyecto implementaremos este m\'etodo
y lo evaluaremos dentro del sistema de di\'alogo.

Para la gesti\'on del contexto de interacci\'on utilizaremos, en un primer
momento, sistemas existentes de manejo de conocimiento (\emph{knowledge mantainance
systems}) como
FaCT++~\citep{horr:fact99},
RACER~\citep{haar:race99} o Pellet~\citep{siri:pell06}, que soportan tareas de
definici\'on, mantenimiento y consulta de ontolog\'ias.  Estos sistemas
han sido utilizados como motores de inferencia
en numerosas applicaciones en el \'area~\citep{franconi03,koller04}, y en
applicaciones
dise\~nadas por miembros del equipo de
investigaci\'on~\citep{benotti07,benotti09b}.  Una vez observado el
comportamiento de
estos motores de inferencia en la tarea, se analizar\'an sus limitaciones
e investigar\'an extensiones requeridas.

La interpretaci\'on de las respuestas del usuario en el sistema unidireccional
es relativamente simple, y en una primera etapa utilizaremos el m\'odulo
discretizador provisto por GIVE.  Luego de la evaluaci\'on del sistema,
podremos determinar si este m\'odulo satisface o no los requerimientos de
nuestra tarea y cu\'ales son sus limitaciones.  En el sistema bidireccional,
en cambio, este es el m\'odulo que requerir\'a m\'as atenci\'on.

Por empezar, el sistema debe ser extendido con capacidades de procesamiento
del input del usuario en lenguaje natural (parser, construcci\'on sem\'antica,
resoluci\'on de referencias, etc.).  Gracias a que los objetos y las acciones
a los que el usuario se puede referir estan restringidos por el entorno virtual,
podemos restringir el lenguaje
a interpretar y utilizar recursos para interpretaci\'on de lenguaje natural
existentes~\citep{kow06}.  Estudiaremos, en particular, dos
tipos espec\'ificios de contribuciones del usuario: pedidos de aclaraci\'on
de la \'ultima instrucci\'on dada, y redefinici\'on de objetivos.  Elegimos
este tipo de contribuciones dado que representan contribuciones del tipo
`reparaci\'on' a corto y largo plazo, respectivamente. Implementaremos las
reparaciones a corto plazo extendiendo el trabajo de~\cite{purver06}.  Para
las reparaciones a largo plazo utilizaremos los lineamientos
de~\cite{blaylock05a,blaylock05b}.  Obviamente, la integraci\'on de capacidades
ling\"u\'isticas en la direcci\'on usuario $\to$ sistema implica no s\'olo
cambios en el m\'odulo de interpretaci\'on de las respuestas, sino que todos los componentes mencionados
anteriormente son afectados.  Por ejemplo, el m\'odulo de gesti\'on de la
informaci\'on debe ahora tambi\'en representar y mantener actualizada las
contribuciones ling\"u\'isticas del usuario; mientras que el m\'odulo de
planificaci\'on del contenido debe reestructurar el \'arbol de acciones
de la tarea cuando el usuario requiere una reparaci\'on a largo plazo.

Para determinar la calidad del los propotipos obtenidos se propone
estudiar la creaci\'on de un modelo de la calidad seg\'un los est\'andares de
evaluaci\'on de software ISO/IEC 9126 y 14528 \citep{ISO9126-1, ISO14598-1}, los
cuales fueron exitosamente aplicados al dominio de la TA, como lo muestra la
herramienta FEMTI (Framework for the Evaluation of Machine
Translation)\footnote{Disponible libremente en \fnturl{http://www.issco.unige.ch/femti/}}. FEMTI
\citep{Est2005} intenta guiar a los evaluadores hacia la creaci\'on de planes de
evaluaci\'on parametrizables que incluyen diversos aspectos del sistema a evaluar
y ofrece un conjunto de m\'etricas relevantes. El desaf\'io principal de este
estudio es establecer los aspectos del tutor m\'as importantes para un
estudiante de idiomas e identificar el conjunto de m\'etricas relevantes, por
ejemplo bas\'andose en experiencias previas \citep{paradise06, Chu2000,
Litman2002}, realizando encuestas o especificaciones de requerimientos (como en
\citep{Lecoeuche98}) o bien recolectando estos datos a trav\'es de experimentos
llamados ``mago de Oz" (del Ingl\'es \textit{wizard of Oz}) donde el usuario
interact\'ua con un prototipo de sistema (posiblemente incompleto o reducido en
funcionalidades) y un humano (es decir, el ``mago") detr\'as de la interface
responde como si lo hiciera el sistema \citep{Dahlback93, Fabbrizio05}.

Luego de elaborar un modelo de la calidad, se pueden aplicar diversas
metodolog\'ias para evaluar cada uno de estos aspectos: seg\'un corresponda, se
pueden aplicar m\'etricas autom\'aticas, subjetivas (tambi\'en llamadas
``humanas") o basadas en la tarea, tanto para evaluar la contribuci\'on de cada
componente como la calidad del sistema en su totalidad.
Por otro lado, dado que se usar\'a la plataforma GIVE, se planea la
participaci\'on en el evento asociado, lo cual servir\'a como una fuente
adicional de informaci\'on acerca de los aspectos del sistema a mejorar.

Una vez que el sistema de interacci\'on bidireccional fue evaluado y mejorado
con los resultados de esta evaluaci\'on,
se investigar\'a su utilizaci\'on
como tutor virtual de idiomas como describimos en la siguiente secci\'on.


Durante todo el proyecto, nos ocuparemos de la diseminaci\'on de resultados y
lecciones aprendidas.  En particular, trabajaremos en la documentaci\'on del
sistema obtenido, en forma de manuales del usuario y del desarrollador que
incluyan una descripci\'on detallada de los algoritmos implementados con sus
puntos fuertes y d\'ebiles.
Por otro lado, los resultados te\'oricos y aplicados ser\'an presentados a
trav\'es de art\'iculos presentados en conferencias locales o internacionales
as\'i como tambi\'en en revistas cient\'ificas pertinentes a las \'areas de
investigaci\'on afectadas.  Nuestros
planes de diseminaci\'on se definen en m\'as detalle en la Seccion~13.


\MySubSubSection{5.2. Aplicaciones}


El resultado del proyecto ser\'a un sistema capaz de dar instrucciones
en lenguaje natural que deben ser llevadas a cabo por el usuario en un
entorno virtual 3D.  La tecnolog\'ia y los avances te\'oricos del proyecto
pueden utilizarse en distintas aplicaciones (en comercio electr\'onico,
soporte t\'ecnico, control de dispositivos por voz, etc.).  Durante el
\'ultimo a\~no del proyecto investigaremos su uso para la ense\~nanza a
distancia, adaptando el sistema para que funcione como tutor de lenguas
extranjeras~\citep{Eskenazi09,Wik09}.

Dada la arquitectura del sistema, es posible cambiar el lenguaje de
interacci\'on con el sistema (input y output), introduciendo
una gram\'atica y dem\'as recursos sint\'acticos (e.g., informaci\'on
morfol\'ogica) para el lenguaje correcto.  Es decir dados los recursos
sint\'acticos adecuados para, por ejemplo, el ingl\'es que cubran las
estructuras y el vocabulario usados en el sistema, se obtiene un sistema
de di\'alogo que interprete y/o produzca instrucciones en ingl\'es.

Un sistema unidireccional que genere instrucciones en ingl\'es puede
usarse para testear la comprensi\'on del usuario.  La correcta
interpretaci\'on de las instrucciones se puede evaluar a partir de la
correcta ejecuci\'on de las instrucciones dadas.  El sistema
bidireccional permitir\'a al usuario pedir aclaraciones sobre la
\'ultima instrucci\'on (en su lengua natal, en caso de no haber comprendido
la instrucci\'on, o en ingl\'es, si desea practicar su
capacidad de expresarse en el idioma extranjero).  El usuario tambi\'en
podr\'a redefinir el objetivo a alcanzar durante la interacci\'on, y
de esta forma seleccionar el vocabulario y el tipo de estructuras que desea
practicar.

Los mundos virtuales (como Second Life\footnote{Accessible gratuitamente
en \fnturl{http://secondlife.com/}}) est\'an siendo incorporados r\'apidamente a la educaci\'on, tanto inicial como universitaria~\citep{Doswell05,molk:lear09}.  Su principal atractivo
es el de proveer oportunidades que son dif\'iciles o imposibles de
proveer en el mundo actual (por ejemplo, por limitaciones econ\'omicas
o porque la experiencia es peligrosa para el observador).
Por otra parte, el uso de un tutor virtual tiene ciertas ventajas
respecto de un tutor humano. \cite{engwall1020} mencionan las siguientes. (1) \emph{Tiempo de
pr\'actica}: la posibilidad de practicar el nuevo lenguaje es esencial
para el aprendizaje, y un tutor virtual provee oportunidades de pr\'actica
s\'olo limitadas por recursos tecnol\'ogicos. (2) \emph{Prestigio:} un
estudiante puede sentirse avergonzado de cometer errores frente a un tutor
humano, y de esta forma limitar su capacidad de expresi\'on en el lenguaje
extranjero. (3) \emph{Realidad Aumentada}: un tutor virtual puede proveer
material adicional (e.g., ejemplos en contexto, im\'agenes explicativas, etc.) con mayor facilidad y menos esfuerzo que un tutor humano.


Para estimar la eficacia de estos sistemas de tutoring se har\'a una
evaluaci\'on comparativa (i.e., con otros
sistemas de tutoring existentes) o una evaluaci\'on orientada al usuario al ser
utilizado, por ejemplo, por estudiantes de la universidad donde se desarrolla el
proyecto.













