\MySubSection{4. Objetivos Generales}

El objetivo te\'orico principal del proyecto es el estudio de
tres temas fundamentales que posibilitan el desarrollo
de sistemas de di\'alogo en entornos virtuales:
\begin{myitemize}
  \item \emph{Interacci\'on:} La interacci\'on humano-humano esta gobernada
  por numerosas reglas pragm\'aticas que definen nociones b\'asicas como
  las obligaciones conversacionales de los participantes (e.g., quien puede hablar en cada momento), o las implicaciones de un enunciado (e.g., si se
  hizo una pregunta se espera una respuesta).
  Estas reglas aplican tambi\'en a la interacci\'on humano-computadora en
  entornos virtuales, cuando el objetivo del sistema de di\'alogo es
  simular, en la medida de lo posible, el comportamiento humano.

  Modelar formalmente estas reglas pragm\'aticas, para un sistema de di\'alogo
  dado, es uno de los desaf\'ios que condicionan la efectividad del sistema.

  \item \emph{Inferencia:} En t\'erminos generales, podemos entender como
  inferencia toda operaci\'on que transforme informaci\'on impl\'icita en
  expl\'icita.  Bajo esta amplia definici\'on, es posible considerar como
  inferencia tanto inferencia cl\'asica (e.g., consequencia l\'ogica) como
  tareas habituales en inteligencia artificial (e.g., planning), o operaciones estad\'isitcas (e.g., obtener la media de un conjunto de datos).  Un sistema de di\'alogo realiza continuamente operaciones de inferencia, por un lado,
  para interpretar la informaci\'on recibida e incorporarla a su repositorio de informacion; y por otro, para decidir qu\'e parte de la informaci\'on disponible transmitir.

  El problema mismo de decidir qu\'e tarea de inferencia utilizar en una determinada situaci\'on es complejo, y las tareas de inferencia en s\'i son
  computacionalmente costosas.  El desaf\'io en este caso es encontrar el compromiso adecuado entre representaci\'on de la informaci\'on y m\'etodo
  de inferencia a utilizar.

  \item \emph{Evaluaci\'on:} Cuando podemos decir que una interacci\'on dada
  es correcta. Cuando podemos decir que es mejor que otra?  Que p\'arametros podemos utilizar para medir el buen desempe\~no de un sistema de di\'alogo?  Estas son preguntas abiertas que actualmente el \'area de evaluaci\'on sistemas de di\'alogo intenta contestar. \fixme{Paula, algo m\'as ac\'a?}

\end{myitemize}

En el proyecto dise\~naremos un sistema de di\'alogo hombre-computadora con
soporte de interacci\'on 3D.  El sistema generar\'a autom\'aticamente
instrucciones en lenguaje natural para ayudar al usuario a cumplir una tarea determinada.

En nuestro proyecto nos concentraremos en el estudio del m\'odulo generador de instrucciones que deber\'a ser capaz de interactuar naturalmente
con el usuario y adaptarse a posibles errores de interpretaci\'on o
ejecuci\'on de las instrucciones.  En una primera etapa, trabajaremos
con un modelo donde el flujo de informaci\'on ling\"u\'istica ser\'a unidireccional:
desde el sistema hacia el usuario.  El usuario se encontrar\'a en un entorno 3D
en el que puede interactuar con objetos y explorar un mundo virtual. El sistema
le dar\'a instrucciones generadas autom\'aticamente describiendo la tarea que el usuario debe llevar a cabo.  El usuario intentar\'a seguir estas instrucciones
realizando acciones f\'isicas en el mundo virtual.  El sistema ser\'a capaz de detectar errores y proveer correcciones (pero, en esta primera etapa,
el usuario no podr\'a pedir ayuda espec\'ifica).  La restricci\'on a un
modelo de flujo unidireccional de la informaci\'on tiene como objetivo
simplificar la representaci\'on y manejo del contexto de la interacci\'on.
En las \'ultimas etapas del proyecto, el modelo ser\'a extendido para
permitir intercambio ling\"u\'istico bidireccional: por ejemplo, el usuario podr\'a
pedir clarificaciones o redefinir el objetivo a alcanzar usando lenguaje natural.

La generaci\'on de instrucciones requiere de informacion ling\"u\'istica a
distintos niveles. A nivel morfol\'ogico y sint\'actico se requiere una
gr\'amatica de lenguaje natural (e.g., espa\~nol) adaptada a la tarea.
A nivel sem\'antico se necesita un repositorio
de informaci\'on l\'exica organizada en t\'erminos de una ontolog\'ia.
Finalmente, el nivel pragm\'atico se utilizar\'a una representaci\'on de las
acciones posibles con sus precondiciones y efectos en el el mundo virtual,
junto con informaci\'on contextual que describe el estado de la interacci\'on.

Esta informaci\'on ser\'a utilizada por distintos procesos que
gestionar\'an la interacci\'on entre el usuario y el sistema en los
distintos niveles ling\"u\'isticos. En una direcci\'on, un sistema de
planning generar\'a una secuencia de acciones relevantes en ese momento
de la interacci\'on, un sistema de administraci\'on del conocimiento
mantendr\'a la informaci\'on contextual actualizada, y un sistema de
generaci\'on finalmente transmitir\'a las instrucciones como expresiones
en lenguaje natural.  En la otra direcci\'on, un discretizador
transformar\'a el flujo continuo de comportamiento del usuario
en el mundo virtual en acciones relevantes al dominio, mientras que
los sistemas de planning y administraci\'on del conocimiento se ocupan
de mantener el contexto al d\'ia y detectar errores.


El dise\~no de una arquitectura como la que acabamos de describir presenta
desaf\'ios te\'oricos y pr\'acticos.  En lo te\'orico, necesita
heur\'isticas que gu\'ien la interacci\'on con el usuario (i.e., qu\'e decir,
cu\'ando, y c\'omo, teniendo en cuenta el contexto actual) y debe contar con m\'etodos
de inferencia que le permitan no s\'olo describir la situaci\'on actual, sino
tambi\'en indicar como cambiarla para alcanzar un objetivo.
La calidad de cada una de estas capacidades afectar\'a la percepci\'on que
el usuario tendr\'a del sistema, y por lo tanto se necesitan m\'etricas que
permitan evaluar su desempe\~no.  La complejidad del problema te\'orico se
refleja, en lo pr\'actico, en un sistema de m\'ultiples componentes: un
generador de lenguaje natural, un sistema de planeamiento, un entorno 3D, etc.

Dise\~nar e implementar todos estos componentes requerir\'ia un esfuerzo prohibitivo.
En este proyecto utilizaremos la plataforma GIVE (\url{http://www.give-challenge.org}) que
provee un prototipo basico para este tipo de sistemas.
%Concretamente, exploraremos estos tres temas en el contexto del
%Challenge on Generating Instructions in Virtual Environments
%(GIVE, \url{http://www.give-challenge.org}).
El objetivo puntual
de GIVE es actuar como medio de evaluacion de sistemas de generaci\'on
de lenguaje natural. Aunque desde un punto de vista m\'as general, la evaluaci\'on
tambi\'en tiene en cuenta el problema b\'asico de la interaccion en
entornos virtuales.  En el escenario provisto por GIVE, el usuario
humano lleva acabo una tarea de `b\'usqueda del tesoro' en un entorno
virtual 3D.  El trabajo del sistema de di\'alogo es proveer, en tiempo
real, instrucciones en lenguaje natural que ayuden al usuario
a llevar a cabo la tarea.

\fixme{GIVE tiene el apoyo de SIGSEM SIGDIAL SIGGEN grupos de interes de
ACL, la Asociacion de Ling\"u\'istica Computacional.}

Por su dise\~no, GIVE provee un framework interesante para investigar
distintos aspectos de la interacci\'on situada en un entorno virtual:
el problema de realizaci\'on sint\'actica y morfo\'ogica de las
instrucciones (\emph{surface realization}),
la representaci\'on del conocimiento existente en el contexto de
interacci\'on (contexto del discurso, ontolog\'ias, acciones posibles),
el tipo de inferencia requerido por las tareas involucradas
(planning, model checking, construcci\'on de modelos),
las reglas pragm\'aticas de interacci\'on requeridas por la tarea
(administraci\'on de la carga cognitiva, actualizaci\'on y uso de la informaci\'on contextual), etc.

Finalmente, la calidad de la interacci\'on con un
sistema de di\'alogo dado puede medirse en esta tarea en t\'erminos
quantitativos (e.g., tiempo m\'aximo necesitado para completar la tarea)
y qualitativos (e.g., relevancia de la instrucci\'on dada en cada momento).
GIVE provee la infraestructura para loguear todos los detalles de la
interacci\'on y permite, de esta forma, obtener facilmente un corpus
de interacci\'on en entornos virtuales anotado autom\'aticamente con
datos medibles.
\medskip

\noindent
El objetivo concreto del proyecto es obtener un sistema de di\'alogo
situado, para entornos virtuales.  Los par\'ametros del proyecto apuntan
a obtener un balance entre un sistema de uso general, aplicable en
diferentes \'ambitos pero, a la vez, lo suficientemente espec\'ifico,
como para permitir el uso efectivo de las t\'ecnicas actuales de
administraci\'on del conocimiento, planning y procesamiento de lenguaje
natural.

Un sistema de este tipo puede utilizarse en distintas situaciones:
comercio electr\'onico (e.g., avatares de asistencia en la web),
soporte t\'ecnico      (e.g., ayuda al usuario no experto),
control de dispositivos por voz (e.g., sistemas de di\'alogo embebidos),
ense\~nanza a distancia (e.g., tutores de lenguas extranjeras),
etc.  Una vez obtenido
un prototipo, planeamos investigar el uso del sistema como
tutores virtuales para el aprendizaje de idiomas.\fixme{Se puede agregar algo mas aca.}

