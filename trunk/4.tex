\MySubSection{4. Objetivos Generales}

El objetivo \'ultimo de este proyecto es dise\~nar un sistema de di\'alogo hombre-computadora con soporte para interacci\'on en tres dimensiones (3D) que genere autom\'aticamente instrucciones en lenguaje natural (LN) para guiar  al usuario durante la ejecuci\'on de una tarea determinada. Esta propuesta apunta
a obtener un balance entre un sistema de uso general, aplicable en
diferentes \'ambitos, y un sitema lo suficientemente espec\'ifico
como para permitir el uso efectivo de las t\'ecnicas actuales de
administraci\'on del conocimiento, planneamiento y procesamiento de lenguaje natural (PLN).

El sistema resultante podr\'ia utilizarse en distintas situaciones, a modo de ejemplo:
en comercio electr\'onico  al usarse como avatar de asistencia en la web, en soporte t\'ecnico al proveer ayuda al usuario no experto, en
control de dispositivos por voz al usarse en sistemas de di\'alogo embebidos, o
en la ense\~nanza a distancia al constituir un tutor virtual para el aprendizaje de lenguas extranjeras. Esta \'ultima es la tarea seleccionada para testear nuestro prototipo. Para estimar la eficacia y adecuaci\'on del prototipo para tal tarea, se podr\'ia realizar una evaluaci\'on comparativa (compar\'andolo con otros sistemas existentes) o una evaluaci\'on orientada al usuario (al ser utilizado, por ejemplo, por estudiantes de la facultad donde se desarrollar\'a el presente proyecto).


A fin de obtener un sistema como el descripto anteriormente, este proyecto propone estudiar las tres \'areas detalladas a continuaci\'on, las cuales son  fundamentales para el desarrollo de cualquier sistema de di\'alogo:

\paragraph{Pragm\'atica:} La pragm\'atica es un \'area interdisciplinaria a
la que contribuyen teor\'ias ling\"u\'isticas (implicaturas
conversacionales~\citep{Grice75}), sociol\'ogicas (an\'alisis
conversa\-cio\-nal~\citep{schegloff87b}) y filos\'oficas (teor\'ia de los actos
de habla~\citep{austin62}). Su objetivo es estudiar c\'omo el contexto que rodea a una frase expresada en LN
contribuye a determinar su significado, en otras palabras, su sem\'antica. La transmisi\'on de significado depende no s\'olo del
conocimiento ling\"u\'istico, como reglas gramaticales y morfol\'ogicas, sino
tambi\'en extraling\"u\'istico, como la situaci\'on f\'isica donde la comunicaci\'on
est\'a situada, experiencias previas de los hablantes o el objetivo de la
conversaci\'on. Por lo tanto, una misma frase puede cambiar su sem\'antica si se cambia el contexto.  En
pragm\'atica se distingue entre oraci\'on (forma
gramatical que toma el acto del habla) y enunciado (oraci\'on m\'as su contexto). La habilidad de entender un enunciado se conoce como competencia pragm\'atica, la cual explica c\'omo una persona hace
inferencias sobre una oraci\'on y su contexto para interpretar adecuadamente el enunciado que se intenta transmitir.

Para que un sistema de di\'alogo interact\'ue de una forma natural con sus usuarios, debe
demostrar habilidad pragm\'atica, definiendo (1) qu\'e tipo de informaci\'on contextual se debe representar y (2) qu\'e tareas de inferencia sobre la oraci\'on y
su contexto son necesarias para interpretar un enunciado.
%Hacer estas dos tareas de forma correcta tendr\'a un impacto crucial sobre el desempe\~no de un sistema que da instrucciones en tiempo real.
En un sistema como el que proponemos desarrollar
es indispensable que las oraciones hagan expl\'icita la cantidad de informaci\'on
justa: si la informaci\'on a procesar es demasiada, podr\'ia  retrasarse y aburrir al usuario, mientras que
si la informaci\'on es escasa, el usuario podr\'ia no recibir las instrucciones correctas sobre c\'omo llevar a cabo la
tarea y, en consecuencia, cometer errores.


%
% La interacci\'on humano-humano est\'a gobernada
%   por numerosas reglas pragm\'aticas que definen nociones b\'asicas como
%   las obligaciones conversacionales de los participantes (qui\'en puede hablar
% en cada momento), o las implicaciones de un enunciado (por ejemplo, si se
%   hizo una pregunta se espera una respuesta).
%   Estas reglas se aplican tambi\'en a la interacci\'on humano-computadora en
%   entornos virtuales cuando el objetivo del sistema de di\'alogo es
%   simular, en la medida de lo posible, el comportamiento humano.
%
%   Modelar formalmente estas reglas pragm\'aticas para un sistema de di\'alogo
% es uno de los desaf\'ios que condicionan la efectividad del mismo.

  %\item
  \paragraph{Inferencia:} En t\'erminos generales, podemos entender como
  inferencia toda operaci\'on que transforme informaci\'on \textit{impl\'icita} en
  \textit{expl\'icita}.  Bajo esta amplia definici\'on, es posible considerar como
  inferencia tanto la cl\'asica (por caso
  tareas habituales de Inteligencia Artificial (IA) como la planificaci\'on) u operaciones estad\'isticas (por ejemplo para obtener estimadores sobre un conjunto de datos).  Un sistema de di\'alogo realiza continuamente operaciones de inferencia, por un lado   para interpretar la informaci\'on recibida e incorporarla a su repositorio de datos, y por otro, para decidir qu\'e parte de la informaci\'on disponible debe transmitir.

  El problema mismo de decidir qu\'e tipo de representaci\'on l\'ogica y qu\'e tipo de inferencia utilizar (l\'ogica proposicional vs.\ l\'ogica de primer orden, validez vs.\ chequeo de modelos, inferencia l\'ogica vs. inferencia estad\'istica) en una determinada situaci\'on  es complejo, y adem\'as las tareas de inferencia en s\'i son computacionalmente costosas.  El desaf\'io en este caso es encontrar el compromiso adecuado entre representaci\'on de la informaci\'on y el m\'etodo de inferencia a utilizar.

\paragraph{Evaluaci\'on:} La evaluaci\'on de sistemas de generaci\'on de lenguaje natural es una de las m\'as dif\'iciles dentro del \'area del PLN dado que una idea puede expresarse de muchas formas, todas ellas correctas. En general, determinar la calidad de una frase generada no puede hacerse de manera simple y directa, por ejemplo comparando el resultado del sistema con un patr\'on (en Ingl\'es ``gold stadard"). El problema de la falta de patrones es compartido con otra \'area del PLN, la Traducci\'on Autom\'atica (TA), en la cual se han propuesto diversas metodolog\'ias para la evaluaci\'on de sistemas de forma directa (es decir, aplicando alguna m\'etrica al texto generado por un sistema) o indirecta (i.e., evaluando la perfomance del sistema a trav\'es de la utilizaci\'on del texto generado para realizaci\'on de alguna tarea). Sin embargo, en ninguna de estas \'areas existe una metodolog\'ia aceptada como est\'andar y demostrada eficaz de manera general.

En esta propuesta, dado que el objeto a evaluar es un sistema que interact\'ua via la generaci\'on de instrucciones en lenguaje natural, es necesario determinar su performance por medio de evaluaciones cuantitativas (como el tiempo de finalizaci\'on de la tarea), cualitativas (por ejemplo, la calidad de las interacciones) y basadas en el contexto (evaluaciones orientadas al usuario y sus necesidades en una situaci\'on particular). Dada la experiencia de los participantes es de especial inter\'es estudiar la  portabilidad y aplicaci\'on de t\'ecnicas de evaluaci\'on del dominio de la TA y la interacci\'on multimodal humano-computadora al sistema (en su totalidad y por componentes) planteado en este proyecto.
%\end{myitemize}

