\MySubSection{4. Objetivos Generales}

El objetivo \'ultimo de este proyecto es dise\~nar un sistema de di\'alogo hombre-computadora con soporte para interacci\'on en tres dimensiones (3D) que generare autom\'aticamente instrucciones en lenguaje natural para guiar  al usuario durante la ejecuci\'on de una tarea determinada. Esta propuesta apunta
a obtener un balance entre un sistema de uso general, aplicable en
diferentes \'ambitos, y un sitema lo suficientemente espec\'ifico
como para permitir el uso efectivo de las t\'ecnicas actuales de
administraci\'on del conocimiento, planneamiento y procesamiento de lenguaje natural (PLN).

El sistema resultante podr\'ia utilizarse en distintas situaciones, a modo de ejemplo:
en comercio electr\'onico  al usarse como avatar de asistencia en la web, en soporte t\'ecnico al proveer ayuda al usuario no experto, en
control de dispositivos por voz al usarse en sistemas de di\'alogo embebidos,
en la ense\~nanza a distancia al constituir tutores de lenguas extranjeras.  Una vez obtenido
un prototipo, se planea aplicarlo  como
tutor virtual para el aprendizaje de idiomas; para estimar la eficacia del prototipo  en esta tarea, se podr\'ia realizar una evaluaci\'on comparativa (compar\'andolo con otros sistemas existentes) o una evaluaci\'on orientada al usuario al ser utilizado, por ejemplo, estudiantes de FaMAF.


A fin de obtener un sistema como el descripto anteriormente, este proyecto propone estudiar las tres \'areas fundamentales detalladas a continuaci\'on, las cuales posibilitan el desarrollo de este tipo de sistemas:
\begin{myitemize}
  \item \emph{Interacci\'on:} La interacci\'on humano-humano est\'a gobernada
  por numerosas reglas pragm\'aticas que definen nociones b\'asicas como
  las obligaciones conversacionales de los participantes (qui\'en puede hablar en cada momento), o las implicaciones de un enunciado (por ejemplo, si se
  hizo una pregunta se espera una respuesta).
  Estas reglas se aplican tambi\'en a la interacci\'on humano-computadora en
  entornos virtuales cuando el objetivo del sistema de di\'alogo es
  simular, en la medida de lo posible, el comportamiento humano.

  Modelar formalmente estas reglas pragm\'aticas para un sistema de di\'alogo es uno de los desaf\'ios que condicionan la efectividad del mismo.

  \item \emph{Inferencia:} En t\'erminos generales, podemos entender como
  inferencia toda operaci\'on que transforme informaci\'on \textit{impl\'icita} en
  \textit{expl\'icita}.  Bajo esta amplia definici\'on, es posible considerar como
  inferencia tanto la cl\'asica (por caso
  tareas habituales en inteligencia artificial como la planificaci\'on) u operaciones estad\'isticas (por ejemplo para obtener estimadores sobre un conjunto de datos).  Un sistema de di\'alogo realiza continuamente operaciones de inferencia, por un lado   para interpretar la informaci\'on recibida e incorporarla a su repositorio de datos, y por otro, para decidir qu\'e parte de la informaci\'on disponible transmitir.

  El problema mismo de decidir qu\'e tipo de representaci\'on l\'ogica y que tipo de inferencia utilizar en una determinada situaci\'on (e.g., l\'ogica proposicional vs. l\'ogica de primer orden, validez vs.\ chequeo de modelos) es complejo, y las tareas de inferencia en s\'i son computacionalmente costosas.  El desaf\'io en este caso es encontrar el compromiso adecuado entre representaci\'on de la informaci\'on y el m\'etodo de inferencia a utilizar.

\item \emph{Evaluaci\'on:} La evaluaci\'on de sistemas de generaci\'on de lenguaje natural es una de las m\'as dif\'iciles dentro del \'area del Procesamiento de Lenguaje Natural (PLN) dado que una idea puede expresarse de muchas formas, todas ellas correctas, y, en general, determinar la calidad de tales frases no puede hacerse de manera simple y directa, por ejemplo comparando el resultado del sistema con un patr\'on (en Ingl\'es ``gold stadard"). El problema de la falta de patrones es compartido con otra \'area del PLN, la Traducci\'on Autom\'atica (TA), en la cual se han propuesto diversas metodolog\'ias para la evaluaci\'on de sistemas de forma directa (es decir, aplicando alguna m\'etrica al texto generado por un sistema) o indirecta (o sea, evaluando la perfomance del sistema a trav\'es de la utilizaci\'on del texto generado para realizaci\'on de alguna tarea). Sin embargo, en ninguna de estas \'areas existe una metodolog\'ia aceptada como est\'andar y demostrada eficaz de manera general.
En esta propuesta, dado que el objeto a evaluar es un sistema que interact\'ua via la generaci\'on de instrucciones en lenguaje natural, es necesario determinar su performance por medio de evaluaciones cuantitativas (como el tiempo de finalizaci\'on de la tarea), cualitativas (por ejemplo, la calidad de las interacciones) y basadas en el contexto (evaluaciones orientadas al usuario y sus necesidades en una situaci\'on particular). Dada la experiencia de los participantes es de especial inter\'es estudiar la  portabilidad y aplicaci\'on de t\'ecnicas de evaluaci\'on del dominio de la TA y la interacci\'on multimodal humano-computadora al sistema (en su totalidad y por componentes) planteado en este proyecto.
\end{myitemize}

El presente proyecto se focalizar\'a en el  m\'odulo del sistema encargado de generar las instrucciones, el cual deber\'a ser capaz de interactuar naturalmente con el usuario y adaptarse a posibles errores de interpretaci\'on o ejecuci\'on de las instrucciones, dando instrucciones correctivas en consecuencia. Adem\'as, el usuario se encontrar\'a en un universo en el que podr\'a interactuar con objetos y explorar el ambiente virtual provisto. As\'i, el usuario intentar\'a seguir las instrucciones provistas por el sistema,
realizando acciones f\'isicas en un mundo 3D.


En una primera etapa, el modelo a utilizar permitir\'a el flujo de informaci\'on ling\"u\'istica de manera unidireccional, es decir,
desde el sistema hacia el usuario, por lo que
el usuario no podr\'a pedir ningu\'n tipo de ayuda.  La restricci\'on a un
modelo unidireccional tiene como objetivo
simplificar la representaci\'on y manejo del contexto de la interacci\'on.

En etapas m\'as avanzadas del proyecto, el modelo unidireccional ser\'a extendido para
permitir intercambio ling\"u\'istico bidireccional: por ejemplo, el usuario podr\'a
pedir clarificaciones o redefinir el objetivo a alcanzar, en ambos casos usando lenguaje natural.

La generaci\'on de instrucciones para un dominio espec\'ifico requiere la adquisici\'on del conocimiento relevante y la compilaci\'on del mismo debe incluir cierta informaci\'on  ling\"u\'istica que permita su posterior procesamiento: en los aspectos morfol\'ogico y sint\'actico se requiere una gr\'amatica de lenguaje seleccionado (Espa\~nol en este caso), en el aspecto sem\'antico se necesita un repositorio de informaci\'on l\'exica organizada en forma ontol\'ogica y en lo pragm\'atico es necesaria una descripci\'on formal de las
acciones posibles en un universo dado (incluyendo precondiciones, efectos e informaci\'on contextual del estado de cada interacci\'on).

Esta informaci\'on relacionada al conocimiento de un dominio es utilizada por distintos componentes del sistema de di\'alogo que
gestionar\'an la interacci\'on con el usuario. En un sistema t\'ipico, esto sucede como se detalla a continuaci\'on: en la direcci\'on sistema $\rightarrow$ usuario, el componente encargado de \textit{planificar} genera una secuencia de acciones relevantes al momento de la interacci\'on, luego el componente encargado de  \textit{administrar el conocimiento} actualiza la informaci\'on contextual y,  finalmente, el componente encargado de
\textit{generar instrucciones}  transmite las mismas como expresiones
del lenguaje natural.  En la direcci\'on opuesta, es decir usuario $\rightarrow$ sistema,  el componente denominado \textit{discretizador}
transforma el flujo continuo de informaci\'on derivada del comportamiento del usuario, en acciones relevantes al universo en cuesti\'on, mientras que
los componentes de planeamiento y administraci\'on del conocimiento se ocupan
de mantener el contexto actualizado y de detectar posibles errores.


El dise\~no de una tal arquitectura presenta
desaf\'ios tanto te\'oricos como pr\'acticos.  En lo te\'orico, la necesidad de
 guiar la interacci\'on con el usuario (qu\'e decir,
cu\'ando y c\'omo, en un contexto dado) motiva la creaci\'on, adaptaci\'on o mejora de heur\'isticas pertinentes y, por otro lado,  la necesidad de describir la situaci\'on actual as\'i como tambi\'en  de indicar c\'omo alterarla para alcanzar un objetivo pre-establecido, incentivan el estudio de diversos m\'etodos
para la inferencia eficiente y precisa.
La calidad de cada una de estas capacidades afecta la percepci\'on que
el usuario tiene del sistema, y por lo tanto, es imperativo aplicar m\'etricas pertinentes que permitan evaluar cada uno de estos aspectos del sistema.  La complejidad del problema te\'orico se
refleja, en lo pr\'actico, en el desarrollo de un sistema de m\'ultiples componentes: un
generador de lenguaje natural, un componente de planeamiento, un entorno 3D, etc.

Dise\~nar e implementar todos estos componentes requerir\'ia un esfuerzo prohibitivo, por lo que proponemos utilizar recursos disponibles de forma libre y gratuita. En particular, se propone el uso de la plataforma desarrollada com parte de la competencia \textit{Generating Instructions in Virtual Environments} (GIVE\footnote{Disponible libremente en  \fnturl{http://www.give-challenge.org}}) que
provee un prototipo b\'asico de este tipo de sistemas. En este contexto ``competencia" (del Ingl\'es ``evaluation campaign" o ``evaluation challenge") se refiere a la organizaci\'on de un evento en el que participan distintos grupos de investigaci\'on, realizando todos el mismo ejercicio con sus sistemas propios y comparando al final los resultados obtenidos por cada grupo. Este tipo de eventos es muy popular en distinas \'areas del PLN (TA, reconocimiento del habla, recuperaci\'on de informac\'on, entre otras) y benefician ampliamente cada \'area generando y compartiendo con la comunidad de investigadores nuevas tecnolog\'ias.

El objetivo principal de GIVE es actuar como medio de evaluaci\'on de sistemas de generaci\'on
de lenguaje natural, donde \'estos se eval\'uan de forma m\'as precisa y homog\'enea dado que todos los sistemas participantes utilizan los mismos recursos provistos por los organizadores. Adem\'as, desde un punto de vista general, la evaluaci\'on
tambi\'en tiene en cuenta el problema b\'asico de la interacci\'on en
entornos virtuales.  En el escenario propuesto por GIVE, el usuario
humano lleva acabo una tarea de `b\'usqueda del tesoro' en un entorno
virtual 3D y el trabajo del sistema participante es proveer, en tiempo
real, instrucciones en lenguaje natural que ayuden al usuario
a encontrar el tesoro escondido.

\fixme{GIVE tiene el apoyo de SIGSEM SIGDIAL SIGGEN grupos de interes de
ACL, la Asociacion de Ling\"u\'istica Computacional.}

Por su dise\~no, el prototipo GIVE es una herramienta muy propicia para investigar distintos aspectos de la interacci\'on situada en un entorno virtual:
el problema de realizaci\'on sint\'actica y morfo\'ogica de las
instrucciones (en Ingl\'es \emph{surface realization}),
la representaci\'on del conocimiento existente en el contexto de
interacci\'on (contexto del discurso, ontolog\'ias, acciones posibles),
el tipo de inferencia requerido por las tareas involucradas
(planeamiento, construcci\'on y chequeo de modelos),
las reglas pragm\'aticas de interacci\'on requeridas por la tarea
(administraci\'on de la carga cognitiva, actualizaci\'on y uso de la informaci\'on contextual), entre otras. Adicionalmente, GIVE provee herramientas para registrar todos los detalles de la
interacci\'on, permitiendo de esta forma obtener f\'acilmente un corpus de interacci\'on en entornos virtuales anotado autom\'aticamente. \fixme{agregar la importancia de tales corpora}


%Finalmente, la calidad de la interacci\'on con un
%sistema de di\'alogo dado puede medirse en esta tarea en t\'erminos
%cuantitativos (por ejemplo el tiempo m\'aximo para completar la tarea)
%y cualitativos (como la relevancia de una instrucci\'on dada).

\fixme{Terminar con un comentario sobre tutoring!!!.}

