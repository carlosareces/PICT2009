
The PLN\footnote{\url{http://www.cs.famaf.unc.edu.ar/~pln}} research group, 
in which the describe scientific project will be carried out, was funded in 2005.
Te group is developing an important role in human resource training,
delivering courses to undergraduate and postgraduate student at the Universidad
de C\'ordoba and
other universities. It also works in the development of various research
projects and integration with other groups in the region, both within Argentina and
with neighboring countries (Chile, Brazil and Uruguay). 

The current project pools together many of the key areas of expertise of the members
of the group. To begin with, some members of the group specialize in computational
logic, particularly in the theoretical and applied study of languages for
knowledge representation (e.g., modal, hybrid and description logics). They
have also developed  automated theorem provers for these
languages\footnote{\url{http://www.glyc.dc.uba.ar/intohylo/}}. In relation with 
the study of knowledge representation, they have also investigated and developed
algorithms for generating referring expressions~\cite{AKS08}.

The second line of research of the PLN group that is relevant for this project
is context-based evaluation. Members of the group have proposed an evaluation model
for machine translation systems which relates the context of use  to potentially important quality
characteristics~\cite{estr:impr08,estr:femt09}. This model is general enough to
be applied to other systems that produce natural language like the ones
proposed in this paper. 
Thanks to the background on machine translation systems the team has experience evaluating and
comparing natural language output produced in different languages (Spanish and
English in particular), which will be relevant for the development of the
language tutor described in Section~\ref{applications}. 
Finally, the team has experience developing and evaluating   
 multimodal corpora like those described in Section~\ref{description}~\cite{multieval}.

The third line of research that is relevant for this project is pragmatics. In
this area the team has implemented a conversational agent which is able to infer and
negotiate conversational implicatures using inference tasks such as
classical planning and planning under incomplete information~\cite{benotti09b}.
We have also investigated  how to infer conversational implicatures triggered
by comparative utterances~\cite{benotti09a}. Recently we have done corpus-based
work, which shows what kinds of implicatures are inferred and negotiated by
human dialogue participants during a task situated in a 3D
virtual environment~\cite{benotti09c}. 

Other lines of research in the PLN group are not directly related
to the project at this stage, but might become relevant in the
future. They include grammar induction, text mining, statistical syntactic analysis
and ontology population from raw text. 





