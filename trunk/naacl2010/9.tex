% Los temas a investigar en el marco de este proyecto son de relevancia en el
% panorama Argentino actual por, al menos, dos razones de peso.

The topics investigated in the framework of this project are of relevance in the
current Argentine landscape for at least two reasons.

% Por un lado, el proyecto
% integra y desarrolla diferentes aspectos clave del \'area de ling\"u\'istica
% computacional (sintaxis, sem\'antica, pragm\'atica, representaci\'on,
% inferencia, evaluaci\'on). El \'area de ling\"u\'istica computacional y su
% aplicaci\'on al tratamiento autom\'atico del lenguaje natural han tenido un
% gran desarrollo internacional en los \'ultimos a\~nos, con aplicaciones como
% los
% sistemas de b\'usqueda en la web, los sistemas de traducci\'on y resumen
% autom\'aticos, las interfaces de voz, etc. Sin embargo, el \'area es casi
% inexistente actualmente en Argentina. Este proyecto se contar\'a entre 
% las contribuciones que apuntan a revertir esta situaci\'on.
% Por otro lado, el objetivo \'ultimo de este proyecto es investigar el uso de
% la plataforma
% desarrollada en el \'area de educaci\'on a distancia (concretamente, como
% plataforma de aprendizaje de idiomas).

On the one hand, the project integrates and develops various key aspects of the
area of computational linguistics (syntax, semantics, pragmatics,
representation, inference, evaluation). The area of computational linguistics
and its application to automatic processing of natural language have had a major
international development in recent years, systems with applications like Web
search, translation systems and abstract machines, voice interfaces, etc. .
However, the area is almost nonexistent today in Argentina. This project will be
among the contributions that aim to reverse this situation. Moreover, the
ultimate goal of this project is to investigate the use of the platform
developed in the area of distance education (specifically, as a platform for
language learning).

% Claramente, la educaci\'on a distancia es un recurso valioso para superar el 
% problema de la centralizaci\'on de recursos
% educativos en el pa\'is. Sin embargo, desarrollar herramientas adecuadas en
% este \'area espec\'ifica es dif\'icil.

Clearly, distance education is a valuable resource to overcome the problem of
centralization of educational resources in the country. However, developing
appropriate tools in this specific area is difficult.

% Para desarrollar sistemas de ense\~nanza a distancia es esencial modelar el
% avance del aprendizaje del usuario. Esto requiere un sistema capaz de ser
% consciente de la evoluci\'on del usuario, y que tenga en cuenta sus logros y
% sus
% problemas. Este tipo de interacci\'on entre usuario y sistema puede modelarse
% como un di\'alogo, cuyo contexto registra el conocimiento adquirido (del
% usuario
% sobre el material del curso, y del sistema sobre el usuario). La gesti\'on de
% este tipo de di\'alogo es particularmente interesante, ya que el sistema debe
% ser
% capaz de interpretar los requerimientos de, y generar respuestas adecuadas
% para,
% usuarios no expertos cuyo conocimiento evoluciona durante la interacci\'on.
% Adem\'as, el sistema debe ser capaz de representar apropiadamente tanto la
% informaci\'on concerniente al material del curso, como la informaci\'on
% concerniente a la evoluci\'on del usuario. Por ejemplo, el sistema debe ser
% capaz de diagnosticar qu\'e parte del material del curso debe ser revisada a
% partir de las respuestas err\'oneas del usuario.  Por \'ultimo, el sistema
% debe
% poder evaluar la interacci\'on con el usuario, para poder decidir
% que objetivos del aprendizaje fueron alcanzados.
% Los resultados te\'oricos y pr\'acticos obtenidos durante el proyecto,
% contribuyen directamente a la soluci\'on de estos problemas.

To develop distance learning systems is essential to model the user's learning
progress. This requires a system capable of being aware of the evolution of the
user, and taking into account their achievements and their problems. This type
of interaction between user and system can be modeled as a dialogue, which
records the acquired knowledge context (the user about the course material, and
system to user). Managing this kind of dialogue is particularly interesting,
since the system must be able to interpret requirements, and generate
appropriate responses, non-experts whose knowledge evolves during the
interaction. Moreover, the system must be able to properly represent both the
information concerning the course material, such as information concerning the
evolution of the user. For example, the system must be able to diagnose what
part of the course material should be reviewed from the wrong answers of the
user. Finally, the system must be able to evaluate the user interaction in order
to decide which learning objectives were achieved. The theoretical and practical
results obtained during the project directly contribute to solving these
problems.


