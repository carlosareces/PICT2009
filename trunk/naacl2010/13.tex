A continuaci\'on se detallan grupos de investigaci\'on relevantes con los que
los  participantes del presente proyecto colaboran.

El Dr.\ Carlos Areces fue miembro
del grupo TALARIS \url{http://talaris.loria.fr}, parte del
\emph{Laboratoire Lorrain de Recherche en Informatique et ses Applications}
(LORIA - \url{http://www.loria.fr}). El principal tema de investigaci\'on de
TALARIS es la ling\"u\'istica computacional con \'enfasis en sem\'antica e
inferencia. Actualmente, colabora con los siguientes investigadores e
instituciones especialistas en el \'area de inferencia y l\'ogica computacional:

\begin{itemize}
    \item[-]  Patrick Blackburn \url{http://www.loria.fr/~blackbur} (director de
TALARIS) es especialista en el \'area de l\'ogicas modales y aplicaciones
en ling\"u\'istica computacional.
\item[-] Grupo de L\'ogica, Lenguaje y Computabilidad (GLyC)
\url{http://www.glyc.dc.uba.ar/} del Departamento de Computaci\'on de la
Universidad
de Buenos Aires.
\item[-] Ron Petrick \url{http://homepages.inf.ed.ac.uk/rpetrick/} es el
desarrollador de PKS, uno de los pocos sistemas de planning con informaci\'on
incompleta funcionales hoy en d\'ia.
\end{itemize}

La Dra.\ Paula Estrella fue miembro del grupo \emph{Traitement 
Informatique Multilingue} (ISSCO/TIM) de la Facultad de Traducci\'on e
Interpretaci\'on 
de la Universidad de Ginebra y del grupo \emph{Database management and meeting
analysis} 
(IM2.DMA) dentro del Polo de Investigaci\'on \emph{Interactive Multimodal
Information 
Management} (IM2 - \url{http://www.im2.ch/}). Actualmente colabora con las
siguientes 
instituciones especialistas en las \'areas de Interacci\'on Hombre-M\'aquina y 
Ling\"u\'is\-tica Computacional:
\begin{itemize}
\item[-]  Instituto de Investigaci\'on Idiap, Martigny, Suiza - 
\url{http://www.idiap.ch/}
\item[-] Grupo de Tratamiento Inform\'atico Multiling\"ue, Universidad de
Ginebra, Suiza - \\ \url{http://www.issco.unige.ch/}
\item[-]  Grupo de Inteligencia Artificial y Pervasiva, Universidad de Friburgo,
Suiza - \\ \url{http://diuf.unifr.ch/pai/wiki/doku.php}
\end{itemize}

La Lic.\ Luciana Benotti realiza su doctorado en el
grupo TALARIS y colabora actualmente con los siguientes investigadores e
instituciones
especialistas en generaci\'on de lenguaje natural y sistemas de di\'alogo:
\begin{itemize}
    \item[-]  Alexander Koller \url{http://www.coli.uni-saarland.de/~koller/} es
uno de los organizadores 
de la competencia GIVE y es el coordinador de la implementaci\'on de la
plataforma GIVE.
    \item[-] David Traum \url{http://people.ict.usc.edu/~traum/} es, hoy en
d\'ia, uno
de los investigadores mas renombrados del \'area de sistemas de di\'alogo.
Es adem\'as es director del grupo de sistemas de di\'alogo en el Institute
for Creative Techologies \url{http://ict.usc.edu/}.
\item[-] Claire Gardent \url{http://www.loria.fr/~gardent/} (vice-directora de
TALARIS) es especialista en generaci\'on de lenguaje natural, adquisici\'on
lexica y desarrollo de gram\'aticas.
\item[-] Agust\'in Gravano \url{http://www.glyc.dc.uba.ar/agustin/} es
especialista
en el estudio de varia\-ciones pros\'odicas en sistemas de di\'alogo.
\end{itemize}