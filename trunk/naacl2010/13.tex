The members of the PLN in general and the authors of this paper in particular
have several collaborations with national and international research groups in 
computational linguistics and related fields that are relevant for this project. 
 
At the international level, we have ongoing collaboration with the TIM/ISSCO\footnote{\url{http://www.issco.unige.ch/en}} \emph{Multilingual Information Processing Department} at the University of Geneva, with the
Idiap Research Institute\footnote{\url{http://www.idiap.ch}} and  with some
members of the PAI\footnote{\url{http://diuf.unifr.ch/pai/wiki}},
\textit{Pervasive Artificial Intelligence} group of the University of
Fribourg.  These collaborations include the evaluation of NLP systems
and the development of multilingual and multimodal human language
technology systems.

Members of the group have a long standing collaboration  with
the TALARIS\footnote{\url{http://talaris.loria.fr}}
group of the \emph{Laboratoire Lorrain de Recherche en Informatique et ses
Applications (LORIA)}. The main research topic at TALARIS is computational
linguistics
with strong emphasis on semantics and inference. In the framework of this
collaboration we are participating in the 2010 edition of the GIVE
Challenge. In the process of designing the systems that will participate in
the challenge we jointly investigated the use of different referring strategies
in situated instruction
giving~\cite{amoia10}. 

We have also collaborated with the Virtual Humans group of the Institute for
Creative Technologies\footnote{\url{http://ict.usc.edu/projects/virtual_humans}}
from the University of Southern California. In particular we computationally
modeled the inference of conversational implicatures triggered by comparative
utterances~\cite{benotti09a}. The Institute for Creative Technologies offers
Internship programs every year that we plan to use in order to strengthen our
collaboration.

All these collaborations are directly related to the main theme of the project 
described in this article.  The PLN group has also research collaborations with 
other international research teams in the framework of other scientific programs. 
For example, the PLN group has being part of a recently finished international project 
MICROBIO\footnote{\url{http://www.microbioamsud.net}} on ontology population from raw text.
The project was funded by the Stic-Amsud\footnote{\url{http://www.sticamsud.org}} program, 
a scientific-technological cooperation program integrated by France, Argentine, Brazil, 
Chile, Paraguay, Peru and Uruguay. The expertise obtained during this project might
be useful in the future when trying to extend our GIVE ontologies to new domains. 
Similarly, the team maintain scientific relations with the University of Texas at Austin
(mainly with Dr.\ J. Moore in projects related to the development of the ACL2\footnote{\url{http://www.cs.utexas.edu/users/moore/acl2}} prover); and with the 
Research team Symbiose\footnote{\url{http://www.irisa.fr/symbiose}} of the Institut de
Recherche en Informatique et Syst\'emes Al\'eatoires (working on the use of linguistic
techniques for the modelisation of genomic sequences).

At the national level, the group has intensively collaborated with GLyC\footnote{\url{http://www.glyc.dc.uba.ar}}, \emph{Grupo de L\'ogica,
Lenguaje y Computabilidad} on knowledge representation and inference 
(see, e.g.~\cite{AG06,AFFM08}). GLyC is part
of the Computer Science Department of the Universidad de Buenos Aires. 
During 2010, teams PLN and GLyC will join forces and collaborate in
the organization of ELiC\footnote{\url{http://www.glyc.dc.uba.ar/elic2010}}, 
the \emph{First School in Computational Linguistics} in Argentina,  which will
take place in July at the Universidad de Buenos Aires.  ELiC 2010 will be co-located 
with the ECI\footnote{\url{http://www.dc.uba.ar/events/eci/2009/eci2009}}, Escuela de Ciencias Inform\'aticas which has a
long standing reputation as a high-quality winter school in Computer Science in
Argentina, and is being organized yearly since 1987.
With ELiC we aim at creating, for the first time, a space to introduce the field of 
computational linguistics to graduate students in Argentina.  Thanks to the 
support of the North American Chapter of the Association for
Computational Linguistics (NAACL) and of the Universidad de Buenos Aires, 
ELiC is offering student travel grants and fee waivers to encourage participation.

The PLN group is also contacting other groups working in computational linguistics
in Argentina like the research group in Artificial Intelligence from the 
Universidad Nacional del Comahue\footnote{\url{http://www.uncoma.edu.ar/}}. Taking 
advantage of previous co-participation in different project we plan to organize
exchange programs in the framework of a research network. 

Finally, the PLN group is planning to organize a workshop on Computational
Linguistics as a satellite event of IBERAMIA 2010\footnote{\url{http://cs.uns.edu.ar/iberamia2010}}, 
the Ibero-American Conference on Artificial Intelligence, that will be organized by the Universidad
del Sur, in the city of Bah\'ia Blanca, Argentina. 
