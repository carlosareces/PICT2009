
% Se puede clasificar el impacto cient\'ifico
% esperado del proyecto en tres grandes areas, como se detalla a continuaci\'on:

One can classify the expected scientific impact of the project in three main
areas, as detailed below:

% \paragraph{Interacci\'on:}
% Una de las contribuciones m\'as
% importantes del proyecto ser\'a un laboratorio virtual para
% teor\'ias pragm\'aticas que consistir\'a de un
% entorno controlado para el estudio de la interacci\'on situada en un mundo
% donde
% se entremezclan acciones f\'isicas y ling\"u\'isticas.

\paragraph{Interaction:}
One of the major contributions of the project will be a virtual laboratory for
pragmatic theories that consist of a controlled environment for studying the
interaction set in a world where physical actions and language intermingle.

% El prototipo unidireccional permitir\'a
% investigar el impacto de distintas
% pol\'iticas para dar instrucciones (post o
% preorden en el \'arbol de acciones de la tarea)
% sobre la realizaci\'on exitosa de la tarea.
% Estudios de este tipo se han realizado anteriormente
% (e.g.,~\cite{foster-etal-ijcai2009}) pero
% asumiendo una tarea prefijada.
% Dado que nuestro prototipo permitir\'a la especificaci\'on
% del mundo virtual, las posibles acciones y el objetivo
% a alcanzar, podremos determinar cu\'ando el impacto resultante
% del uso de una determinada pol\'itica
% es dependiente de la tarea.

The prototype will allow unidirectional investigate the impact of different
policies to give instructions (post or pre-order on the tree stock of the task)
on the successful completion of the task. Such studies have been performed
previously (e.g.,~\cite{foster-etal-ijcai2009}) but assuming a predetermined
task. Since our prototype allows the specification of the virtual world,
possible actions and the goal to achieve, we can determine when the impact
resulting from the use of a particular policy is dependent on the task.

% El prototipo bidireccional nos permitir\'a investigar el
% fen\'omeno de reparaciones a corto y largo plazo,
% dise\~nando y evaluando un sistema de predicci\'on de reparaciones
% contextualizadas.  Estas reparaciones est\'an usualmente
% causadas por implicaturas conversacionales.  Modelar
% estas implicaturas en un sistema de di\'alogo gen\'erico
% es dif\'icil.  Sin embargo, dado que el presente prototipo provee una
% interacci\'on situada
% y restringida al mundo virtual, ser\'a posible testear la relaci\'on entre
% las implicaturas, el tipo de reparaciones a las que
% dan origen y las tareas de inferencia necesarias para
% predecirlas.

The prototype will allow us to investigate the two-way phenomenon repair the
short and long term, designing and evaluating a prediction system repairs
contextualized. These repairs are usually caused by conversational implicatures.
Modeling these implicatures in a generic dialogue system is difficult. However,
since the present prototype provides a situated interaction and restricted to
the virtual world will be possible to test the relationship between
implicatures, the type of repairs that give rise and inference tasks necessary
to predict them. 

% \paragraph{Inferencia:} La principal contribuci\'on del
% proyecto en el \'area de l\'ogica e inferencia es en el
% dise\~no, desarrollo y estudio de algoritmos de planning.
% Un sistema de planning t\'ipico toma tres inputs -- un
% estado inicial, una especificaci\'on de posibles acciones y
% un objetivo esperado -- y retorna una secuencia de acciones (un plan)
% que al ser aplicadas secuencialmente al estado inicial, termina
% en un estado que satisface el objetivo pedido.  Distintos
% m\'etodos para obtener un plan han sido estudiados (forward chaining, backward
% chaining, codificaci\'on en t\'erminos de satisfiabilidad proposicional,
% etc.); y existen actualmente sistemas
% implementados que pueden resolver esta tarea eficientemente.

\paragraph{Inference:} 
The main contribution of the project in the area of logic and inference is in
the design, development and planning study of algorithms. A typical planning
system takes three inputs - an initial state, a specification of possible
actions and an expected objective - and returns a sequence of actions (a plan)
that when applied sequentially to the initial state, ends in a state that
satisfies the goal order. Different methods to obtain a scheme have been studied
(forward chaining, backward chaining, coding in terms of propositional
satisfiable, etc..) And are currently deployed systems that can solve this task
efficiently.

% Sin embargo, la mayor\'ia de estos sistemas asumen condiciones
% que simplifican el problema (tiempo at\'omico y
% determin\'istico, informaci\'on completa, ausencia de una teor\'ia
% background, etc.) y retornan un \'unico plan.  En el transcurso
% del proyecto se investigar\'an algoritmos que eliminan algunas
% de las simplificaciones mencionadas (en particular, investigaremos
% el caso de planning con informaci\'on incompleta y en base a una
% teor\'ia background) y que ofrecen adem\'as servicios de planning
% extendidos (retorno de planes alternativos, planes m\'inimos, planes
% condicionales, planes incompletos, acciones posibles en un estado dado, etc.)

However, most of these systems assume conditions that simplify the problem
(deterministic atomic time, complete, absence of a background theory, etc.). And
return a single plan. In the course of the project will investigate algorithms
that eliminate some of the simplifications (in particular, investigate the case
of planning with incomplete information and theory based on a background) and
also provide extended services planning (return on alternative plans, plans
minimum conditional plans, incomplete plans, possible actions in a given state,
etc.).

% \paragraph{Evaluaci\'on:}
% En este \'area se espera que una de las contribuciones principales sea la
% integraci\'on de t\'ecnicas de evaluaci\'on de distintas \'areas en una
% metodolog\'ia que permita evaluar sistemas de di\'alogo para entornos
% virtuales
% de manera que se estime su usabilidad y eficacia. Esta metodolog\'ia
% podr\'ia usarse tanto para determinar si un sistema es adecuado para un tipo
% de
% tarea y de usuario, como para comparar la performance de distintos sistemas
% del
% mismo tipo.

\paragraph{Evaluation:}
In this area it is expected that one of the main contributions is the
integration of assessment techniques from different areas on a methodology for
evaluating dialog systems for virtual environments so as to estimate its
usability and effectiveness. This methodology could be used both to determine
whether a system is suitable for a task type and user, and to compare the
performance of different systems of the same type.

% Otra contribuci\'on ser\'a el estudio y aplicaci\'on de est\'andares de
% evaluaci\'on de software a los sistemas desarrollados, generando un modelo de
% la
% calidad estandarizado y proponiendo un conjunto de m\'etricas apropiadas para
% evaluar cada uno de los aspectos contenidos en el modelo. Este trabajo
% incluir\'a tambi\'en un estudio profundo de varias m\'etricas a fines de
% incluir
% en el modelo de calidad algunos consejos sobre las m\'etricas elegidas; este
% estudio se considera como una meta-evaluaci\'on del modelo propuesto.

Another contribution is the study and application of appraisal standards
developed software systems, creating a standardized quality model and proposing
a set of appropriate metrics to assess each of the aspects of the model. This
work will also include a thorough study of several metrics in late in the model
include some tips on quality metrics chosen, this study is considered a
meta-evaluation of the proposed model.

% Finalmente,
% el corpus anotado de interacci\'on humano-humano, m\'as
% los corpus de interacci\'on humano-m\'aquina recopilados
% durante el proyecto se har\'an p\'ublicos.  Este tipo de
% corpora servir\'a, por ejemplo, para dise\~nar plataformas
% m\'as generales de evaluaci\'on de sistemas de di\'alogo,
% que van m\'as all\'a de los aspectos evaluados actualmente
% por plataformas existentes como GIVE.

Finally, the annotated corpus of human-human interaction, over the corpus of
human-machine interaction collected during the project be made public. Such
corpora will serve, for example, to design more general platform for evaluating
dialog systems, which go beyond the aspects evaluated by currently existing
platforms as GIVE.


