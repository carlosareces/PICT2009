
% 
% En este proyecto prestaremos principal atenci\'on al sistema encargado de la
% generaci\'on de  instrucciones e interpretaci\'on de las respuestas del
% usuario.
% Dicho sistema deber\'a ser capaz de una interacci\'on natural y debe poder
% adaptarse a
% posibles errores de interpretaci\'on o ejecuci\'on de las instrucciones, dando
% instrucciones correctivas cuando sea necesario. El usuario se encontrar\'a
% en un universo en el que podr\'a interactuar con objetos y explorar el
% ambiente
% virtual provisto. En este mundo virtual 3D, el usuario intentar\'a seguir las
% instrucciones
% provistas por el sistema realizando acciones f\'isicas.

In this project, we focused primarily on the system responsible for the
generation of instructions and interpretation of user responses. The system must
be capable of
a natural interaction and be able to adapt to possible errors of interpretation
or execution of the instructions, giving corrective instructions when necessary.
You will find yourself in a universe where you can interact with objects and
explore the virtual environment provided. In this 3D virtual world, the user
will try to follow instructions provided by the system by physical actions.

% En una primera etapa, el modelo a utilizar permitir\'a el flujo de
% informaci\'on
% ling\"u\'istica de manera unidireccional, desde el sistema hacia el usuario,
% por lo que
% el usuario no podr\'a pedir ning\'un tipo de ayuda.  La restricci\'on a un
% modelo unidireccional tiene como objetivo
% simplificar la representaci\'on y el manejo del contexto de la interacci\'on.

Initially, the model to use will flow one-way linguistic information from the
system to the user so the user can not ask for any help. The restriction to a
unidirectional model is to simplify the representation and management of the
context of the interaction.

% En etapas m\'as avanzadas del proyecto, el modelo unidireccional ser\'a
% extendido para
% permitir intercambio ling\"u\'istico bidireccional: por ejemplo, el usuario
% podr\'a
% pedir clarificaciones o redefinir el objetivo a alcanzar, en ambos casos
% usando
% lenguaje natural.

At later stages of the project, the unidirectional model will be extended to
allow bidirectional language exchange: for example, the user may request
clarification or redefine the objective to be achieved in both cases using
natural language.

% La generaci\'on de instrucciones para un dominio espec\'ifico requiere
% informaci\'on  ling\"u\'istica a distintos niveles.
% A nivel morfol\'ogico y sint\'actico se requiere una
% gram\'atica del lenguaje seleccionado (Espa\~nol en este caso). A nivel
% sem\'antico se necesita un repositorio de informaci\'on l\'exica organizada en
% forma ontol\'ogica. A nivel pragm\'atico es necesaria una descripci\'on formal
% de
% las
% acciones posibles en el dominio (con sus precondiciones y efectos), e
% informaci\'on del estado de la interacci\'on.

The generation of instructions for a specific domain requires different levels
of linguistic information. A morphological and syntactic level requires a
selected language grammar (Spanish in this case). A semantic level requires a
repository of lexical information organized as ontological. A pragmatic level
requires a formal description of possible actions in the domain (with their
preconditions and effects), and information of prior interaction.


% Esta informaci\'on relacionada a un dominio dado es utilizada por distintos
% componentes del sistema de di\'alogo que
% gestiona la interacci\'on con el usuario. En un sistema t\'ipico, esto
% sucede como se detalla a continuaci\'on.  En la direcci\'on sistema
% $\rightarrow$
% usuario, el componente encargado de \emph{planificar el contenido} genera una
% secuencia de
% instrucciones relevantes al momento de la interacci\'on, luego el componente
% encargado de  \emph{gestionar el conocimiento} actualiza la informaci\'on
% contextual y,  finalmente, el componente encargado de
% \emph{generar instrucciones}  transmite las mismas como expresiones
% de lenguaje natural.  En la direcci\'on opuesta, es decir usuario
% $\rightarrow$
% sistema,  el componente denominado \emph{discretizador}
% transforma el flujo continuo de informaci\'on derivada del comportamiento del
% usuario, en acciones relevantes a la tarea en cuesti\'on, mientras que
% los componentes de planificaci\'on del contenido y gesti\'on del conocimiento
% se ocupan
% de mantener el contexto actualizado y de detectar posibles errores.

This information relating to a given domain is used by various components of the
dialogue system that manages the user interaction. In a typical system, this is
as detailed below. The address system $\rightarrow$ user, the component
responsible for planning the content generates a sequence of instructions
relevant to the moment of interaction, then the component will manage the
knowledge updated contextual information and, finally, the component will
generate transmits the same instructions as natural language expressions. In the
opposite direction, ie user $\rightarrow$ system, the component called
discretized transforms the continuous flow of information derived from user
behavior in actions relevant to the task at hand, while the components of
content planning and management knowledge concerned with maintaining the current
context and possible errors.

% En la siguiente secci\'on se detalla el plan de trabajo propuesto para llevar
% a
% cabo los objetivos planteados (la Secci\'on 8
% provee un cronograma completo), para luego discutir una posible aplicaci\'on
% del
% sistema a desarrollar.


In the next sections, we detail the proposed work plan to carry out the stated
objectives and then we discuss a possible application of the system to be
developed. 

\subsection{Tasks}
% 
% Dise\~nar e implementar todos estos componentes empezando desde cero
% requerir\'ia un esfuerzo prohibitivo, por lo que proponemos utilizar recursos
% disponibles de forma libre y gratuita. En particular, usaremos la plataforma
% desarrollada como parte de la competencia \emph{Generating Instructions in
% Virtual Environments}
% (GIVE)\footnote{GIVE se encuentra disponible libremente en
% \url{http://www.give-challenge.org}, y posee el respaldo de los grupos de
% inter\'es SIGSEM, SIGDIAL, y SIGGEN de la \emph{Association for Computational
% Linguistics} (ACL).} que
% provee una plataforma b\'asica para este tipo de sistemas.

As we said, designing and implementing all the necessary components from scratch
would require a prohibitive effort, therefore we will use resources
available for free. In particular, we will use the platform developed as part of
the competencia1 Generating Instructions in Virtual Environments
(GIVE)\footnote{GIVE is freely available at \url{http://www.give-challenge.org}
and has the support of interest groups SIGSEM, SIGDIAL, and SIGGEN of the
Association for Computational Linguistics (ACL).} which provides a basic
platform for this type of system.

% El objetivo principal de GIVE~\cite{byron09} es actuar como medio de
% evaluaci\'on de sistemas de generaci\'on
% de lenguaje natural. En esta competencia, los sistemas se eval\'uan de forma
% homog\'enea dado que todos los sistemas participantes utilizan los mismos
% recursos provistos por los organizadores.
% En el escenario propuesto por GIVE, el usuario
% humano lleva acabo una tarea de ``b\'usqueda del tesoro" en un entorno
% virtual 3D y el trabajo del sistema participante es proveer, en tiempo
% real, instrucciones en lenguaje natural que ayuden al usuario
% a encontrar el tesoro escondido.  Por lo tanto, GIVE no s\'olo eval\'ua
% la generaci\'on de lenguaje natural sino tambi\'en la habilidad del
% sistema de participar en una interacci\'on situada en un entorno 3D.

The main aim of GIVE~\cite{byron09}  is to act as a means of evaluation of
systems for natural language generation. In this competition, systems are
evaluated in a homogeneous since all participating systems use the same
resources provided by the organizers. In the scenario proposed by GIVE, the
human user carries out a task of `` treasure hunts in a 3D virtual environment
and work of the participant system is to provide real-time, natural language
instructions that help users find the hidden treasure. So GIVE not only assesses
the generation of natural language but also the system's ability to participate
in an interaction set in a 3D environment.
% 
% La plataforma GIVE es, entonces, una herramienta propicia para
% investigar los distintos aspectos de la interacci\'on situada en un entorno
% virtual que se enumeran a continuaci\'on: el problema de realizaci\'on
% sint\'actica y morf\'ogica de las
% instrucciones (en Ingl\'es, \emph{surface realization});
% la representaci\'on del conocimiento existente en el contexto de
% interacci\'on (contexto del discurso, ontolog\'ias, acciones posibles);
% el tipo de inferencia requerido por las tareas involucradas
% (planning, construcci\'on y chequeo de modelos); y
% las reglas pragm\'aticas de interacci\'on requeridas por la tarea
% (administraci\'on de la carga cognitiva, actualizaci\'on y uso de la
% informaci\'on contextual). GIVE provee, adem\'as, 
% herramientas para registrar todos los detalles de la
% interacci\'on, permitiendo de esta forma obtener f\'acilmente un corpus de
% interacci\'on en entornos virtuales, anotado autom\'aticamente.

GIVE platform is thus a favorable tool for investigating various aspects of
interaction situated in a virtual environment as listed below: the problem of
morphogen syntactic realization of the instructions (in English, surface
realization), the representation of existing knowledge in the context of
interaction (discourse context, ontologies, possible actions), the type of
inference required by the tasks involved (planning, construction and checking of
designs), and pragmatic rules of interaction required by the task (management
cognitive load, update and use of contextual information). GIVE provides further
tools to record all details of the interaction, thus allowing to easily obtain a
corpus of interaction in virtual environments annotated automatically.

% Para la definici\'on correcta de las pol\'iticas de interacci\'on que
% utilizaremos en nuestro prototipo es necesario
% contar con un corpus que provea ejemplos de interacci\'on t\'ipica sobre la
% tarea~\cite{HCRC-93,byron-06}.  Usando herramientas provistas por GIVE,
% recopilaremos un corpus de interacci\'on humano-humano (uni y bidireccional)
% que contendr\'a la interacci\'on lig\"u\'istica alineada con las acciones
% realizadas en el mundo virtual.

For the correct definition of interaction policies we use in our prototype is a
need for a body that provides examples of typical interaction on the
task~\cite{HCRC-93,byron-06}. Using tools provided by GIVE, we collect a corpus
of human-human interaction (uni and bidirectional) containing the interaction
linguistics has aligned with actions in the virtual world.

% A partir del corpus recolectado comenzaremos el dise\~no,
% implementaci\'on y testing de algoritmos de interacci\'on unidireccionales,
% los 
% cuales ser\'an integrados a la plataforma GIVE.
% Existen tradicionalmente cuatro tareas que un sistema de generaci\'on
% de instrucciones debe implementar: (1) planificaci\'on del contenido, (2)
% generaci\'on
% de expresiones referenciales, (3) gesti\'on del contexto de interacci\'on, y
% (4) interpretaci\'on de las respuestas del usuario.

From the corpus collected begin the design, implementation and testing of
unidirectional interaction algorithms, which will be integrated into the
platform GIVE. There are traditionally four tasks that a generation system must
implement instructions: (1) planning the content, (2) generation of referring
expressions, (3) managing the interaction context, and (4) interpretation of
user responses. 
% 
% \emph{(1) Planificaci\'on del contenido:} Para planificar el contenido a
% generar
% es necesario primero obtener un plan de
% acciones que alcance el objetivo desde el estado actual. Dicho plan
% contrendr\'a
% acciones f\'isicas a ejecutar sobre el entorno. El segundo paso es decidir
% c\'omo se transmitir\'a al usuario esta secuencia de acciones. Es decir,
% se debe decidir cu\'antas acciones comunicar
% por instrucci\'on y c\'omo agregarlas coherentemente. El
% resultado del proceso de agregaci\'on de acciones (en Ingl\'es, \emph{action
% agreggation}) 
% es un \'arbol que describe la
% estructura de la tarea a diferentes niveles de abstracci\'on. El tercer y
% \'ultimo paso
% es decidir c\'omo navegar el \'arbol de acciones para verbalizar las
% instruciones (por ejemplo, en post o preorden~\cite{foster-etal-ijcai2009}).
% En
% este proyecto investigaremos
% diferentes pol\'iticas de agregaci\'on (e.g., agregando acciones que manipulan
% el mismo
% objeto) y pol\'iticas no est\'andard de recorrido del \'arbol de acciones
% (e.g.,
% bajando a un
% nivel menor de abstracci\'on en caso de malentendidos).

\emph{(1) Content planning:} To plan the content to be generated is first
necessary to obtain a plan of action to reach the target from the current state.
The plan to run contrendrá physical actions on the environment. The second step
is to decide how to transmit this sequence of user actions. That is, they must
decide how many shares communicate instruction and how to add coherently. The
result of the aggregation process shares (in English, action agreggation) is a
tree describing the task structure at different levels of abstraction. The third
and final step is to decide how to navigate the tree of actions to verbalize the
instructions (for example, post or preorder~\cite{foster-etal-ijcai2009}). This
project will investigate different aggregation policies (eg, adding actions that
manipulate the same object) and non-standard travel policy of the tree stock
(eg, dropping to a lower level of abstraction in case of misunderstanding).

% Para obtener el plan de acciones, utilizaremos la tarea de inferencia de
% planning~\cite{nau04}. GIVE provee un sistema de planning muy
% limitado.  Existen sistemas de planning m\'as avanzandos (en particular,
% que permiten el manejo de informaci\'on incompleta sobre el dominio)
% como PKS\footnote{PKS est\'a disponible en
% \url{http://homepages.inf.ed.ac.uk/rpetrick/research/pks/}}, pero
% est\'an todav\'ia en desarrollo.  En general, el estado del
% arte en el \'area de planning no cubre los requerimientos de nuestro
% sistema.  Si bien existen sistemas optimizados que funcionan adecuadamente
% en ciertas aplicaciones, ninguno provee servicios como la generaci\'on de
% planes alternativos, o la generaci\'on de planes incompletos en caso de
% ausencia de plan. Por lo que deberemos dise\~nar e implementar estas
% extensiones a los algoritmos de planning. Estudiaremos tambi\'en el
% comportamiento te\'orico (e.g., complejidad) de estos algoritmos.

For the action plan, we use the inference task planning~\cite{nau04}. GIVE
provides a very limited system planning. There are more avanzandos
planning systems (in particular, to enable the handling of incomplete
information on the domain) as
PKS\footnote{PKS is
freely available in \url{http://homepages.inf.ed.ac.uk/rpetrick/research/pks/}},
but they are still developing. In general,
the state of the art in the planning area does not cover the requirements of our
system. While there are systems that work well optimized for certain
applications, none provides services such as the generation of alternative
plans, or the generation of incomplete plans in case of absence of plan. As we
design and implement these extensions to the algorithms of planning. Also study
the theoretical behavior (e.g., complexity) of these algorithms. 
% 
% \emph{(2) Generaci\'on de Expresiones Referenciales:} Una vez terminada la
% fase
% de planificaci\'on de contenido, la siguiente tarea
% es la generaci\'on de expresiones referenciales. Esta tarea implica producir
% una
% frase que describa una entidad referenciable de forma tal que el usuario
% la pueda identificar (e.g., ``el jarr\'on que est\'a sobre la mesa'').
% Estas expresiones, para ser aceptables, deben ser similares a las que podr\'ia
% producir una persona en condiciones normales (por ejemplo, no ser\'ia
% aceptable,
% ``el jarr\'on que no est\'a arriba de la silla ni arriba del
% sof\'a ni abajo de la mesa'').  En~\cite{AKS08} se propone utilizar la
% minimizaci\'on simb\'olica del modelo que representa
% el estado del mundo, para as\'i obtener f\'ormulas l\'ogicas que describan
% un\'ivocamente a cada objeto. En nuestro proyecto implementaremos este
% m\'etodo
% y lo evaluaremos dentro del sistema de di\'alogo.

\emph{(2) Generating referring expressions:} Once the planning phase content,
the next task is the generation of referring expressions. This task involves
producing a sentence that describes a referenceable entity so that the user can
identify (eg, `` the vase on the table''). These expressions, to be acceptable,
must be similar to those that could produce a person in normal conditions (for
example, would not be acceptable, `` the vase who is not above the top of the
chair or sofa or under the table''). In~\cite{AKS08} proposes to use the
symbolic minimization model that represents the state of the world, in order to
obtain logical formulas that describe each object uniquely. In our project we
will implement this method and evaluate within the dialogue system.
% 
% \emph{(3) Gesti\'on del Contexto de Interacci\'on:} Para la gesti\'on del
% contexto de interacci\'on utilizaremos, en un primer
% momento, sistemas existentes de manejo de conocimiento (\emph{knowledge
% mantainance
% systems}) como
% FaCT++~\cite{horr:fact99},
% RACER~\cite{haar:race99} o Pellet~\cite{siri:pell06}, que soportan tareas de
% definici\'on, mantenimiento y consulta de ontolog\'ias.  Estos sistemas
% han sido utilizados como motores de inferencia
% en numerosas aplicaciones en el \'area~\cite{franconi03,koller04} y,
% en particular,  en aplicaciones dise\~nadas por miembros del equipo de
% investigaci\'on~\cite{benotti09b}.  Una vez observado el
% comportamiento de
% estos motores de inferencia en la tarea, se analizar\'an sus limitaciones
% e investigar\'an extensiones requeridas.

\emph{(3) Management of the Interaction Context:} To manage the use of
interaction context in the first instance, existing systems of knowledge
management (knowledge mantainance systems) as RACER~\cite{haar:race99} or
Pellet~\cite{siri:pell06}, which support tasks of definition, maintenance and
querying of ontologies. These systems have been used as inference engines in
numerous applications in the area~\cite{franconi03,koller04} and, in particular
in applications designed by members of the research team~\cite{benotti09b}. Once
observed the behavior of these inference engines on the task, analyzing the
limitations and investigate extensions required.
% 
% \emph{(4) Interpretaci\'on de las Respuestas del Usuario:} La
% interpretaci\'on 
% de las respuestas del usuario en el sistema unidireccional
% es relativamente simple, y en una primera etapa utilizaremos el m\'odulo
% discretizador provisto por GIVE.  Luego de la evaluaci\'on del sistema,
% podremos determinar si este m\'odulo satisface o no los requerimientos de
% nuestra tarea y cu\'ales son sus limitaciones.  En el sistema bidireccional,
% en cambio, \'este es el m\'odulo que requerir\'a m\'as atenci\'on.

\emph{(4) Interpretation of User Responses:} The interpretation of user
responses in the unidirectional system is relatively simple, and in a first
stage we use the discretized form provided by GIVE. After evaluating the system,
this module can determine whether or not it meets the requirements of our task
and what are its limitations. In the two-way system, however, this is the module
that will require more attention.
% 
% Por empezar, el sistema bidireccional debe ser extendido con capacidades de
% procesamiento
% de enunciados provenientes del usuario (an\'alisis sint\'actico,
% construcci\'on
% sem\'antica,
% resoluci\'on de referencias, etc.).  Gracias a que, en este sistema,  los
% objetos y las acciones
% a los que el usuario se puede referir son los del entorno virtual,
% el lenguaje a interpretar est\'a naturalmente restringido y es posible 
% utilizar recursos para interpretaci\'on de lenguaje natural
% existentes~\cite{kow06}.  Estudiaremos, en particular, dos
% tipos espec\'ificios de contribuciones del usuario: pedidos de aclaraci\'on
% de la \'ultima instrucci\'on dada, y redefinici\'on de objetivos.  Elegimos
% este tipo de contribuciones dado que representan contribuciones del tipo
% `reparaci\'on' a corto y largo plazo, respectivamente. Implementaremos las
% reparaciones a corto plazo extendiendo el trabajo de~\cite{purver06}.  Para
% las reparaciones a largo plazo utilizaremos los lineamientos
% de~\cite{blaylock05a,blaylock05b}.  Obviamente, la integraci\'on de
% capacidades
% ling\"u\'isticas en la direcci\'on usuario $\to$ sistema implica no s\'olo
% cambios en el m\'odulo de interpretaci\'on de las respuestas, sino que todos
% los
% componentes mencionados
% anteriormente son afectados.  Por ejemplo, el m\'odulo de gesti\'on de la
% informaci\'on debe ahora tambi\'en representar y mantener actualizada las
% contribuciones ling\"u\'isticas del usuario; mientras que el m\'odulo de
% planificaci\'on del contenido debe reestructurar el \'arbol de acciones
% de la tarea cuando el usuario requiere una reparaci\'on a largo plazo.


To start with, the two-way system should be expanded processing capabilities of
statements coming from user input (parsing, semantic construction, resolution of
references, etc.).. With that in this system, the objects and actions that the
user can refer are the virtual environment, to interpret the language is
naturally limited and may utilize resources for existing natural language
interpretation~\cite{kow06}. Study, in particular, two types of user
contributions specificity:
requests for clarification of the instruction given, and redefinition of goals.
We chose such contributions represent contributions because the type `fix it
'short and long term respectively. We will implement short-term repairs to
extend the work of~\cite{purver06}. For long-term repairs will use the
guidelines of~\cite{blaylock05a,blaylock05b}. Obviously, the integration
of
language skills in the management user $\to$ system involves not only changes
in the form of interpretation of the answers, but all the above components are
affected. For example, the management module of information must now also
represent and update the user's linguistic contributions, while the content
planning module must restructure the tree stock of the task when the user
requires a long-term repair. 
\medskip

% \noindent
% Para determinar la calidad de los propotipos obtenidos se propone
% estudiar la creaci\'on de un modelo de la calidad seg\'un los est\'andares de
% evaluaci\'on de software ISO/IEC 9126 y 14528 \cite{ISO9126-1,ISO14598-1}, los
% cuales fueron exitosamente aplicados al dominio de la TA, como lo muestra la
% herramienta FEMTI (Framework for the Evaluation of Machine
% Translation)\footnote{Disponible libremente en
% \url{http://www.issco.unige.ch/femti/}}. FEMTI
% \cite{Est2005} intenta guiar a los evaluadores hacia la creaci\'on de planes
% de
% evaluaci\'on parametrizables que incluyen diversos aspectos del sistema a
% evaluar
% y ofrece un conjunto de m\'etricas relevantes. La identificaci\'on de
% m\'etricas
% relevantes se puede realizar usando distintos m\'etodos, por
% ejemplo bas\'andose en experiencias previas
% \cite{paradise06,Chu2000,Litman2002}, realizando encuestas o especificaciones
% de
% requerimientos (como en
% \cite{Lecoeuche98}) o bien recolectando estos datos a trav\'es de experimentos
% llamados ``mago de Oz" (del Ingl\'es \emph{wizard of Oz}) donde el usuario
% interact\'ua con un prototipo de sistema (posiblemente incompleto o reducido
% en
% funcionalidades) y un humano (el ``mago") detr\'as de la interface
% responde como si lo hiciera el sistema \cite{Dahlback93,Fabbrizio05}.

\noindent
To determine the quality of the obtained propotipos is to study the creation of
a model of quality assessment by the standards of software ISO/IEC 9126 and
14528~\cite{ISO9126-1,ISO14598-1}, which were successfully applied to the
TA domain, as shown in the tool FEMTI (Framework for the Evaluation of Machine
Translation)\footnote{Freely available in
\url{http://www.issco.unige.ch/femti/}}. FEMTI~\cite{Est2005}
try to guide the evaluators towards creating parameterized evaluation
plans that include various aspects of the system to evaluate and offer a
relevant set of metrics. The identification of relevant metrics can be performed
using various methods, eg based on previous
experience~\cite{paradise06,Chu2000,Litman2002}, conducting
surveys or specification requirements (as in \cite{Lecoeuche98}) or
collecting such data through experiments called ``Wizard of Oz"(the
English wizard of Oz) where the user interacts with a prototype system (possibly
incomplete or reduced functionality) and a human (the ``Wizard") behind the
interface responds as if I did the system~\cite{Dahlback93,Fabbrizio05}.

% Luego de elaborar un modelo de la calidad, se pueden aplicar diversas
% metodolog\'ias para evaluar distintos aspectos del sistema. Seg\'un
% corresponda,
% se
% pueden aplicar m\'etricas autom\'aticas, subjetivas (tambi\'en llamadas
% ``humanas") o basadas en la tarea, tanto para evaluar la contribuci\'on de
% cada
% componente como la calidad del sistema en su totalidad.
% Por otro lado, dado que se usar\'a la plataforma GIVE, se planea la
% participaci\'on en el evento asociado, lo cual servir\'a como una fuente
% adicional de informaci\'on acerca de los aspectos del sistema a mejorar.

After developing a quality model can be applied several methodologies to assess
various aspects of the system. As appropriate, you can apply automatic metrics,
subjective (also called `` human ") or based on the task, both to evaluate the
contribution of each component as the quality of the whole system. On the other
hand, since it will be used GIVE platform is planned participation in the
associated event, which will serve as an additional source of information about
aspects of the system to improve.

% Una vez que el sistema de interacci\'on bidireccional fue evaluado y mejorado
% con los resultados de esta evaluaci\'on,
% se investigar\'a su utilizaci\'on
% como tutor virtual de idiomas como describimos en la siguiente secci\'on.

Once the two-way interaction system was evaluated and improved over the results
of this evaluation, we will investigate its use as a virtual language tutor as
described in the next section.

% Durante todo el proyecto, nos ocuparemos de la diseminaci\'on de resultados y
% lecciones aprendidas.  En particular, trabajaremos en la documentaci\'on del
% sistema obtenido, en forma de manuales del usuario y del desarrollador que
% incluyan una descripci\'on detallada de los algoritmos implementados con sus
% puntos fuertes y d\'ebiles.
% Los resultados te\'oricos y aplicados ser\'an presentados en conferencias
% locales e internacionales,
% y en revistas cient\'ificas pertinentes a las \'areas de
% investigaci\'on afectadas.  Nuestros
% planes de diseminaci\'on se definen en m\'as detalle en la Seccion~10.

Throughout the project, we will ensure the dissemination of results and lessons
learned. In particular, we will work in your system documentation obtained in
the form of user manuals and developer to include a detailed description of the
algorithms implemented with their strengths and weaknesses. The theoretical and
applied results will be presented at local and international conferences and in
journals relevant to the research areas concerned. Our plans to spread more
fully defined in Section 10. 

\fixme{quede aca}

\subsection{Aplications}

El resultado del proyecto ser\'a un sistema capaz de dar instrucciones
en lenguaje natural que deben ser llevadas a cabo por el usuario en un
entorno virtual 3D.  La tecnolog\'ia y los avances te\'oricos del proyecto
pueden utilizarse en distintas aplicaciones (en comercio electr\'onico,
soporte t\'ecnico, control de dispositivos por voz, etc.).  Durante el
\'ultimo a\~no del proyecto investigaremos su uso para la ense\~nanza a
distancia, adaptando el sistema para que funcione como tutor de lenguas
extranjeras~\cite{Eskenazi09,Wik09}.

Dada la arquitectura del sistema, es posible cambiar el lenguaje de
interacci\'on con el sistema (input y output), introduciendo
una gram\'atica y dem\'as recursos sint\'acticos (e.g., informaci\'on
morfol\'ogica) para el lenguaje correcto.  Es decir, dados los recursos
sint\'acticos adecuados para, por ejemplo, el ingl\'es que cubran las
estructuras y el vocabulario usados en el sistema, se obtiene un sistema
de di\'alogo que interprete y/o produzca instrucciones en ingl\'es.

Un sistema unidireccional que genere instrucciones en ingl\'es puede
usarse para testear la comprensi\'on del usuario.  La correcta
interpretaci\'on de las instrucciones se puede evaluar a partir de la
correcta ejecuci\'on de las instrucciones dadas.  El sistema
bidireccional permitir\'a al usuario pedir aclaraciones sobre la
\'ultima instrucci\'on (en su lengua natal, en caso de no haber comprendido
la instrucci\'on, o en ingl\'es, si desea practicar su
capacidad de expresarse en el idioma extranjero).  El usuario tambi\'en
podr\'a redefinir el objetivo a alcanzar durante la interacci\'on, y
de esta forma seleccionar el vocabulario y el tipo de estructuras que desea
practicar.

Los mundos virtuales (como Second Life\footnote{Accessible gratuitamente
en \url{http://secondlife.com/}}) est\'an siendo incorporados r\'apidamente a
la educaci\'on, tanto inicial como universitaria~\cite{Doswell05,molk:lear09}. 
Su principal atractivo
es el de proveer oportunidades que son dif\'iciles o imposibles de
proveer en el mundo actual (por ejemplo, por limitaciones econ\'omicas
o porque la experiencia es peligrosa para el observador).
Por otra parte, el uso de un tutor virtual tiene ciertas ventajas
respecto de un tutor humano. \cite{engwall1020} mencionan las siguientes. (1) \emph{Tiempo de
pr\'actica}: la posibilidad de practicar el nuevo lenguaje es esencial
para el aprendizaje, y un tutor virtual provee oportunidades de pr\'actica
s\'olo limitadas por recursos tecnol\'ogicos. (2) \emph{Prestigio:} un
estudiante puede sentirse avergonzado de cometer errores frente a un tutor
humano, y de esta forma limitar su capacidad de expresi\'on en el lenguaje
extranjero. (3) \emph{Realidad Aumentada}: un tutor virtual puede proveer
material adicional (e.g., ejemplos en contexto, im\'agenes explicativas, etc.) con mayor facilidad y menos esfuerzo que un tutor humano.

Para estimar la eficacia de estos sistemas de tutoring se har\'a una
evaluaci\'on comparativa (i.e., con otros
sistemas de tutoring existentes) o una evaluaci\'on orientada al usuario al ser
utilizado, por ejemplo, por estudiantes de la universidad donde se desarrolla el
proyecto.
El desaf\'io principal de este
estudio es establecer los aspectos del tutor m\'as importantes para un
estudiante de idiomas e identificar el conjunto de m\'etricas relevantes.












