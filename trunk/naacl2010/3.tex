
The goal of this project is to implement a dialogue system which automatically
generates instructions in order to help a user to fulfill a
given task in a 3D virtual environment. In this context, we will investigate
fundamental issues about human-computer interaction. The expected results of the
project can be classified in three areas: pragmatics
of interaction;  information representation and inference; and evaluation of
dialogue systems. Once a working prototype is finished, we will adapt it to
the specific task of language learning, using the system as a virtual language
teacher. Our prototype will teach English to native Spanish speakers. Hence, it
will need to understand and produce both languages.

Initially, we will investigate a model of unidirectional linguistic
interaction (i.e., linguistic information flows only from the system to the
user). In subsequent stages, the model will be
extended to allow bidirectional language exchange. For example, the user may
ask clarifications to the system or redefine the goal of the interaction.

The architecture of the envisioned dialogue system presents both theoretical and
practical challenges. On the theoretical side, heuristics are needed in order to
govern
decisions such as what to say, when, and how (given the current
context). In addition, the system should implement inference methods in order
to figure out how to modify the current situation and reach the task goal.
The complexity of
the theoretical issues is reflected, in practice, in a system of
multiple components: a natural language generator, a planner,
a 3D interactive environment, to mention a few. Designing
and implementing all these components from scratch would 
require a prohibitive effort. Instead we will adapt tools already implemented 
and freely available for prototyping this kind of 
systems, such as the platform
\emph{GIVE\footnote{\url{http://www.give-challenge.org}}, Generating
Instructions in Virtual
Environments}~\cite{byron09}. 

The quality of each of the components of the system affects the
perception users have of it. It is imperative to carry out 
extensive evaluation. 
We plan to adapt and apply different evaluation techniques and
metrics from the area of Machine Translation to assess the  
performance of the system.
