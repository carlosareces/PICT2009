

% El objetivo \'ultimo de este proyecto es dise\~nar un sistema de di\'alogo
% hombre-computadora con soporte para interacci\'on en tres dimensiones (3D) que
% genere autom\'aticamente instrucciones en lenguaje natural para guiar  al
% usuario durante la ejecuci\'on de una tarea determinada. Este proyecto apunta
% a obtener un balance entre un sistema de uso general (aplicable en
% diferentes \'ambitos) y un sistema lo suficientemente espec\'ifico
% como para permitir el uso efectivo de las t\'ecnicas actuales de
% gesti\'on del conocimiento, planning y procesamiento de lenguaje
% natural (PLN).

% The ultimate goal of this project is to design a dialogue system with
% human-computer interaction support for three dimensional (3D) to automatically
% generate natural language instructions to guide the user during the execution
% of a task. 

This project aims to achieve a balance between a system which is generic in the
sense that it is applicable in different areas, and a specific enough in order
to allow for the efficient use of existing techniques for knowledge management,
planning and natural language processing (NLP).

% El sistema resultante podr\'ia utilizarse en distintas situaciones, a modo de
% ejemplo:
% en comercio electr\'onico  al usarse como avatar de asistencia en la web, en
% soporte t\'ecnico al proveer ayuda al usuario no experto, en
% control de dispositivos por voz al usarse en sistemas de di\'alogo embebidos,
% o
% en la ense\~nanza a distancia al constituir un tutor virtual para el
% aprendizaje
% de lenguas extranjeras. Esta \'ultima es la tarea seleccionada para testear
% nuestro prototipo. La eficacia y adecuaci\'on del prototipo para
% tal tarea puede ser estimada de diferentes formas. Por un lado, se puede
% realizar 
% una evaluaci\'on comparativa con
% otros sistemas existentes. Por otra parte, estudiantes de la facultad donde se
% desarrollar\'a
% el presente proyecto pueden participar en una evaluaci\'on orientada al
% usuario.

The resulting system could be used in different applications in which the
system needs to give instructions. For example in electronic commerce providing 
assistance on the web for buying a product, in technical support helping a
novice user to fix a problem, and in language learning by providing a
virtual tutor for learning foreign languages. The latter is the task selected to
test our prototype. The effectiveness and suitability of a prototype for such a
task can be estimated in different ways. On the one hand, we will compare it
with other existing and similar systems. On the other hand, we will
carry out a user evaluation. 

% A fin de obtener un sistema como el descripto anteriormente, este proyecto
% propone estudiar las tres \'areas detalladas a continuaci\'on, las cuales son 
% fundamentales para el desarrollo de cualquier sistema de di\'alogo:

Implementing a dialogue system is a multidisciplinary effort. During the
development of our dialogue system we will particularly concentrate our research
efforts in the following areas:

% \paragraph{Pragm\'atica:} La pragm\'atica es un \'area interdisciplinaria a
% la que contribuyen teor\'ias ling\"u\'isticas (e.g., implicaturas
% conversacionales~\cite{grice75}), sociol\'ogicas (e.g., an\'alisis
% conversa\-cio\-nal~\cite{schegloff87b}) y filos\'oficas (e.g., teor\'ia de los
% actos
% de habla~\cite{austin62}). Su objetivo es estudiar c\'omo el contexto (en el
% que la conversaci\'on esta situada) contribuye al significado (de cada cosa
% que
% se diga durante esa conversaci\'on). La transmisi\'on de significado depende
% no
% s\'olo de la informaci\'on ling\"u\'istica (entidades en foco, reglas
% gramaticales y morfol\'ogicas, etc.), sino tambien extraling\"u\'istica
% (situaci\'on f\'isica donde la comunicaci\'on est\'a situada, experiencias
% previas de los hablantes, objetivo de la conversaci\'on, etc.). Por lo tanto,
% una misma oraci\'on puede significar cosas diferentes en distintos contextos;
% el
% \'area de pragm\'atica estudia el proceso por el cual una oraci\'on es
% desambig\"uada usando su contexto. En pragm\'atica se distingue entre
% oraci\'on
% (forma gramatical que toma el acto ling\"u\'istico) y enunciado (oraci\'on
% m\'as su
% contexto). La habilidad de entender una oraci\'on usando su contexto, es
% decir,
% la habilidad de entender un enunciado, se conoce como competencia
% pragm\'atica.
% Explicar la competencia pragm\'atica implica explicar c\'omo una persona hace
% inferencias sobre una oraci\'on y su contexto para interpretar adecuadamente
% el
% enunciado que el emisor intenta transmitir. 

\paragraph{Pragmatics:} Pragmatics is an interdisciplinary field which
integrates insights from linguistics (e.g., 
conversational implicatures~\cite{grice75}),
sociology (e.g., conversational analysis~\cite{schegloff87b}) and
philosophy (e.g., theory of speech acts~\cite{austin62}). It aims to explore how
the context (in which a conversation is situated) contributes to the meaning (of
everything that is said during that conversation). The meaning conveyed during
a conversation depends not only of linguistic information (entities in focus,
grammatical and morphological rules, etc.) but also from extralinguistic
information (physical situation where the conversation is located, previous
experiences of speakers, etc.). As a result, the same sentence may mean
different things in different contexts. The area of pragmatics studies the
process by which a sentence is disambiguated using its context. In pragmatics,
there is a distinction between sentence (grammatical form that
the linguistic act takes) and utterance (sentence plus its context). The ability
to understand a sentence using its context, i.e. the ability to understand an
utterance, is called pragmatic competence. Explaining the pragmatic competence
involves explaining how a person makes inferences about a sentence and its
context to properly interpret the meaning that the speaker intends to convey.
% 
% Para que un sistema de di\'alogo interact\'ue de una forma natural con sus
% usuarios, debe demostrar habilidad pragm\'atica. Por lo tanto, dicho sistema
% debe definir (1) qu\'e tipo de informacion contextual se debe representar y
% (2)
% qu\'e tareas de inferencia sobre la oraci\'on y el contexto son necesarias
% para
% interpretar un enunciado. Hacer estas dos tareas de forma correcta tendr\'a un
% impacto crucial sobre el desempe\~no de un sistema como el que proponemos
% desarrollar. En dicho sistema es indispensable que las oraciones hagan
% expl\'icita la cantidad de informaci\'on justa: si la informaci\'on es
% demasiada, se retrasar\'a y aburrir\'a al usuario, si la informaci\'on es muy
% poca, el usuario no sabr\'a c\'omo llevar a cabo la tarea y cometer\'a
% errores. 

A dialogue system needs to have pragmatic capabilities in order to interact in a
natural way with its users. Therefore, the system must define (1) what kind of
contextual information should be represented and (2) what inference tasks on a
sentence and context are necessary in order to interpret an utterance. Doing
these tasks correctly will have a crucial impact on the performance of a system
like the one we propose to develop. In such a system it is indispensable that
the sentences makes explicit the right ammount of information: if
information is too much, the user will be delayed and get bored, if the
information is too little, the user will not know how to perform the task and
will make mistakes.

% \paragraph{Inferencia:} Podemos entender como
% inferencia toda operaci\'on que transforme informaci\'on \emph{impl\'icita} en
% \emph{expl\'icita}.  Esta definici\'on es lo suficientemente general 
% como para cubrir tareas que van desde la inferencia l\'ogica (i.e.,
% deducci\'on 
% en un lenguaje formal), hasta tareas de inferencia habituales en inteligencia
% artificial (e.g., planning e inferencia no mon\'otona), y operaciones
% estad\'isticas (por
% ejemplo obtener estimadores sobre un conjunto de datos).  Un sistema de
% di\'alogo realiza continuamente operaciones de inferencia. Por un lado, 
% se necesita inferencia para
% interpretar la informaci\'on recibida e incorporarla al repositorio de datos,
% y por otro, para decidir qu\'e parte de la informaci\'on disponible se debe
% transmitir.

\paragraph{Inference:} Inference can be understood as any operation that
transforms implicit information in explicit information. This definition is
general enough to cover tasks ranging from logical inference (i.e., deduction in
a formal language) to inference tasks common in AI (e.g., planning and
non-monotonic inference), as well as statistical operations (e.g. obtaining
estimators on a data set). A dialogue system has to
continually perform inference operations. On the one hand, inference is needed
to interpret the information received and incorporate it to the data repository,
and, on the other hand, it is needed in order to decide how much of the
available information should be conveyed.

%   El problema mismo de decidir qu\'e tipo de representaci\'on l\'ogica y qu\'e
% tipo de inferencia utilizar  en una determinada situaci\'on  es complejo
% (l\'ogica proposicional vs.\ l\'ogica de primer
% orden, validez vs.\ chequeo de modelos, inferencia l\'ogica vs.\ inferencia
% estad\'istica). Adem\'as las
% tareas de inferencia en s\'i son computacionalmente costosas.  El desaf\'io en
% este caso es encontrar el compromiso adecuado entre la representaci\'on de la
% informaci\'on y el m\'etodo de inferencia a utilizar.

The very problem of deciding what kind of logical representation and what type
of inference to use in a given situation is complex (propositional logic vs.
First-order logic, validity vs. model checking, logical inference vs.
statistical inference). Furthermore inference tasks themselves are
computationally expensive. The challenge here is to find the appropriate
balance between the expressivity of the representation formalism and the
cost of the required inference methods.


% \paragraph{Evaluaci\'on:} La evaluaci\'on de sistemas de generaci\'on de
% lenguaje natural es una de las m\'as dif\'iciles dentro del \'area del PLN,
% dado
% que una idea puede expresarse de muchas formas, todas ellas correctas. En
% general, determinar la calidad de una frase generada no puede hacerse de
% manera
% simple y directa, por ejemplo, comparando el resultado del sistema con un
% patr\'on (en Ingl\'es \emph{gold standard}). El problema de la falta de
% patrones es
% compartido con otra \'area del PLN, la Traducci\'on Autom\'atica (TA), en la
% que se han propuesto diversas metodolog\'ias para la evaluaci\'on de sistemas 
% que pueden clasificarse en directas e indirectas. 
% Una metodolog\'ia directa aplica alguna m\'etrica al texto generado por un
% sistema. Una metodolog\'ia indirecta, en cambio, eval\'ua la perfomance del 
% sistema a trav\'es de la utilizaci\'on del texto generado para realizar alguna
% tarea. Sin
% embargo, en ninguna de estas \'areas existe una metodolog\'ia aceptada como
% est\'andar y demostrada eficaz de manera general.

The evaluation of natural language generation systems is one of the
most difficult ones in the area of NLP; an idea can be expressed in many
forms, all of them correct. In general, determining the quality of a
generated sentence can not be done simply and directly, for example, comparing
the result with a gold standard. The problem of lack of gold standards is shared
with another area of the NLP, namely the Machine Translation (MT), for which
various evaluation methodologies, which can be classified into direct and
indirect, have been proposed. A direct method applies a metric to the text
generated by the system. An indirect method evaluates the perfomance of the
system through the use of the generated text to perform some task. However, 
none of these methods there is a standard and accepted methodology which has
been generally proven to be effective.

% Dado que el objeto a evaluar en este proyecto es un sistema que interact\'ua
% via la generaci\'on de instrucciones en lenguaje natural, podemos
% determinar su performance por medio de evaluaciones cuantitativas (como el
% tiempo de finalizaci\'on de la tarea), cualitativas (e.g., la calidad de
% las interacciones) y basadas en el contexto (evaluaciones orientadas al
% usuario
% y sus necesidades en una situaci\'on particular). Dada la experiencia previa
% de los
% integrantes del proyecto, es de especial inter\'es estudiar la  portabilidad y
% aplicaci\'on
% de t\'ecnicas de evaluaci\'on del dominio de la TA y la interacci\'on
% multimodal
% humano-computadora, al sistema  planteado en
% este proyecto (en su totalidad y por componentes).

Since what is being evaluated in this project is a system that interacts via the
generation of natural language instructions, we can determine its performance
through quantitative metrics (such as average task completion time), 
qualitative metrics (e.g., general user satisfaction) and metrics based on the
context (how well the system addressed the user needs in particular situations).
Given our previous experience, it is particularly
interesting for us to study the portability of evaluation
techniques from the domain of MT
and multimodal human-computer interaction to the evaluation of the system
proposed in this project.
