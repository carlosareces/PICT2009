\MySubSection{12. Insersi\'on en el Programa de Investigaci\'on.}

El Procesamiento de Lenguaje Natural (PLN), y en particular el
desarrollo de Sistemas de Di\'alogo, es un \'area en crecimiento
en los pa\'ises desarrollados, por su gran aplicabilidad en el
actual contexto de la Sociedad de la Informaci\'on. Efectivamente,
dado el gran aumento de la informaci\'on textual en formato
digital, el procesamiento automatizado de los textos se ha
convertido en una capacidad estrat\'egica para empresas,
instituciones y comunidades ling\"u\'isticas. Por esta raz\'on
el \'area de PLN se encuentra en crecimiento en las principales
universidades y empresas tecnol\'ogicas del mundo, con un
presupuesto que crece a\~no a a\~no. Sin embargo, en la Argentina
este \'area se encuentra muy poco desarrollada. Esto se puede
atribuir a diferentes factores:
\begin{myitemize}
\item la relativa juventud del \'area de PLN, lo que implica una relativa escasez de profesionales bien formados en todo el mundo,
\item el desarrollo insuficiente de todo el \'area de Ciencias de la Computaci\'on, por razones hist\'oricas y demanda de la industria,
\item el poco desarrollo del \'area de Inteligencia Artificial y
Ling\"u\'istica Formal en la Argentina, tambi\'en por razones hist\'oricas
y acad\'emicas,
\item la escasa interacci\'on entre los pocos investigadores en PLN que se encuentran en la regi\'on.
\end{myitemize}

Parece claro que el PLN es un \'area de investigaci\'on estrat\'egica
para la Argentina, en la que se puede alcanzar la excelencia acad\'emica
e industrial a nivel internacional. Creemos que hay que apoyar el
desarrollo de este \'area favoreciendo los siguientes aspectos:
\begin{myitemize}
\item formaci\'on de recursos humanos, con una fuerte componente de
formaci\'on en el centros de excelencia extranjeros, ya sea a trav\'es
de programas de doctorado o de cursos dictados por profesionales de
reconocido prestigio en la Argentina,
\item incorporaci\'on de recursos humanos formados, para contribuir
al aumento y diversificaci\'on de la masa cr\'itica en el \'area,
\item mejora de la interacci\'on entre los diversos grupos o investigadores aislados en PLN, a
trav\'es de la organizaci\'on de workshops, cursos de profesores visitantes, co-tutor\'ias,
programas de especializaci\'on coordinados, etc.
\end{myitemize}

En FaMAF existe un grupo de PLN desde 2005 (\url{http://www.cs.famaf.unc.edu.ar/~pln}). Este
grupo est\'a desarrollando una importante labor de formaci\'on de recursos humanos, con el dictado de cursos de grado y de postgrado en la FaMAF y otras Universidades del pa\'is.
Tambi\'en trabaja en el desarrollo de diversos proyectos de investigaci\'on y en la integraci\'on
con otros grupos de la regi\'on, tanto en Argentina como en Chile, Brasil y Uruguay. Este proyecto de investigaci\'on se integra al programa del grupo PLN de Procesamiento de Lenguaje Natural del
FaMAF.


