\documentclass[11pt]{article}
\usepackage{times,url,latexsym,amssymb,tabularx,a4wide,color}
\usepackage{colortbl}
\usepackage[draft,silent]{fixme}

%Por algun motivo babel spanish interfiere con natbib.
%Notar que los ~ aparecen ahora en el texto (en vez
%de actuar como non-breaking space
%\usepackage[spanish]{babel}
%\hyphenation{es-pe-ra-dos, na-tu-ral-men-te}
\usepackage{natbib}

%% Define a new 'leo' style for the package that will use a smaller font.
\makeatletter
\def\url@leostyle{%
  \@ifundefined{selectfont}{\def\UrlFont{\sf}}{\def\UrlFont{\small\sffamily}}}

\def\url@tinyleostyle{%
  \@ifundefined{selectfont}{\def\UrlFont{\sf}}{\def\UrlFont{\scriptsize\sffamily}}}
\makeatother

%% Now actually use the newly defined style.
\urlstyle{leo}

\newcommand{\fnturl}[1]{\urlstyle{tinyleo}\url{#1}\urlstyle{leo}}

\renewcommand{\refname}{\normalfont\large\sffamily\textbf{15. Bibliograf\'ia}}

\addtolength{\topmargin}{-1cm}
% \oddsidemargin 0.1 in
% \evensidemargin 0.15 in
% \marginparwidth 1 in
% \oddsidemargin 0.125 in
% \evensidemargin 0.125 in
% \marginparwidth 0.75 in
% \textwidth 6.125 in
%
\addtolength{\textheight}{1.5cm}
% \addtolength{\textheight}{2cm}
% \addtolength{\voffset}{-1cm}
% % \addtolength{\textwidth}{2cm}
% \addtolength{\textwidth}{.4cm}
% % \addtolength{\hoffset}{-1cm}
% \addtolength{\hoffset}{-.4cm}

\makeatother

\pagestyle{plain}

\newcommand{\MySubSection}[1]{\vspace*{-.1\baselineskip}%
\subsection*{\sffamily\textbf
#1}\vspace*{-.2\baselineskip}}

\newcommand{\MySubSubSection}[1]{\vspace*{-.1\baselineskip}%
\subsubsection*{\sffamily\textbf
#1}\vspace*{-.2\baselineskip}}

\newcommand{\MyParagraph}[1]{\vspace*{-.35\baselineskip}\paragraph*{{\sffamily\textbf
#1}}}
\newcommand{\MySubParagraph}[1]{\vspace*{-.35\baselineskip}\subparagraph*{{\sffamily\textit{#1}}}}

\newenvironment{mylist}{%
  \renewcommand{\labelitemi}{\mbox{\tiny
$\blacksquare$}}%
  \renewcommand{\labelitemii}{$\bullet$}%
  \addtolength{\topsep}{-3\parskip}%
  \begin{itemize}\setlength{\itemsep}{-1.8pt}}%
  {\end{itemize}}

\newenvironment{myitemize}{%
  \renewcommand{\labelitemi}{\mbox{\tiny
$\blacksquare$}}%
  \renewcommand{\labelitemii}{$\bullet$}%
  \addtolength{\topsep}{-3\parskip}%
  \begin{itemize}\setlength{\itemsep}{-1.8pt}}%
  {\end{itemize}}

\newcommand{\nextchunk}{\vspace*{.65\baselineskip}\noindent}

\begin{document}
%\bibliographystyle{alpha}
\thispagestyle{plain}

\section*{\sffamily\textbf{Convocatoria PICT-2009}}
\mbox{}

\vspace*{-.5\baselineskip}
\MySubSection{1.1. T\'itulo del Proyecto:
{\rm \emph{\Large Sistemas de Di\'alogo para Entornos Virtuales.}}}

\MySubSection{1.2. Grupo Responsable}

\hspace*{.5cm} Dr.\ Carlos Areces (Investigador Responsable)

Dra.\ Paula Estrella (Investigadora Integrante)

Lic.\ Luciana Benotti (Investigadora Integrante)\footnote{La Lic.\ Benotti
obtendr\'a su t\'itulo de Dra.\ en Ciencias de la Computaci\'on en Enero de 2010.}

\MySubSection{1.3. Grupo Colaborador}

\hspace*{.5cm}
Dra.\ Laura Alonso Alemani

Dr.\ Gabriel Infante L\'opez

Lic.\ Franco Luque

\MySubSection{2.  Palabras Clave}

Sistemas de Di\'alogo -- Representaci\'on del Conocimiento -- Planning
-- Evaluaci\'on

\MySubSection{3. Resumen}

En este proyecto se propone implementar un sistema de di\'alogo que genere
autom\'aticamente
instrucciones en lenguaje natural para ayudar a un usuario a cumplir una tarea
determinada en un entorno virtual 
en tres dimensiones. Para este fin, se propone investigar temas fundamentales
sobre la
in\-teracci\'on hombre-m\'aquina 
en entornos virtuales. Como resultado de este proyecto, se espera la
implementaci\'on, evaluaci\'on y aplicaci\'on de un sistema prototipo. Las tres
\'areas principales en las que se pueden clasificar los resultados del proyecto
son: (1) pragm\'atica de la interacci\'on, (2) representaci\'on
de la informaci\'on e inferencia, (3) evaluaci\'on de sistemas
de di\'alogo. Una vez obtenido
un prototipo, se planea su aplicaci\'on a la tarea espec\'ifica del 
aprendizaje de idiomas, utilizando el sistema como un ``profesor de idiomas"
virtual.

En una primera etapa investigaremos un modelo de interacci\'on 
ling\"u\'istica unidireccional
(i.e., el flujo de informaci\'on ling\"u\'istica ser\'a desde el sistema hacia
el usuario).
En etapas subsiguientes, el modelo ser\'a extendido para
permitir intercambio ling\"u\'istico bidireccional, por ejemplo, para que el
usuario pida clarificaciones o ayuda al sistema.

Dise\~nar la arquitectura de un sistema de di\'alogo presenta
desaf\'ios tanto te\'oricos como pr\'acticos. En lo te\'orico, se necesitan
heur\'isticas que gobiernen la interacci\'on en cuanto a qu\'e decir,
cu\'ando, y c\'omo (teniendo en cuenta el contexto actual). Adem\'as, se debe
contar con m\'etodos
de inferencia que permitan al sistema adaptarse en funci\'on de la situaci\'on
actual, para alcanzar un objetivo predefinido.
La complejidad del problema te\'orico se
refleja, en lo pr\'actico, en un sistema de m\'ultiples componentes: un
generador de lenguaje natural, un sistema de planning, un entorno de 
interacci\'on en tres dimensiones, etc.
Dise\~nar e implementar todos estos componentes requerir\'ia un esfuerzo
prohibitivo. En este proyecto adaptaremos herramientas ya
implementadas y disponibles libremente para el prototipado de sistemas de este
tipo, como la plataforma \emph{Generating Instructions in Virtual
Environments} (GIVE).

La calidad de cada una de las capacidades del sistema afecta la
percepci\'on que el usuario tiene de \'este. Por lo tanto, es imperativo
evaluar cada una de ellas.
Se planean adaptar y aplicar distintas t\'ecnicas y m\'etricas de evaluaci\'on
del \'area de ingenier\'ia de software y de diversas \'areas del procesamiento
de lenguaje natural (por ejemplo, la traducci\'on autom\'atica).
  % Resumen

% El objetivo \'ultimo de este proyecto es dise\~nar un sistema de di\'alogo
% hombre-computadora con soporte para interacci\'on en tres dimensiones (3D) que
% genere autom\'aticamente instrucciones en lenguaje natural para guiar  al
% usuario durante la ejecuci\'on de una tarea determinada. Este proyecto apunta
% a obtener un balance entre un sistema de uso general (aplicable en
% diferentes \'ambitos) y un sistema lo suficientemente espec\'ifico
% como para permitir el uso efectivo de las t\'ecnicas actuales de
% gesti\'on del conocimiento, planning y procesamiento de lenguaje
% natural (PLN).

% The ultimate goal of this project is to design a dialogue system with
% human-computer interaction support for three dimensional (3D) to automatically
% generate natural language instructions to guide the user during the execution
% of a task. 

This project aims to achieve a balance between a system which is generic in the
sense that it is applicable in different areas, and a specific enough in order
to allow for the efficient use of existing techniques for knowledge management,
planning and natural language processing (NLP).

% El sistema resultante podr\'ia utilizarse en distintas situaciones, a modo de
% ejemplo:
% en comercio electr\'onico  al usarse como avatar de asistencia en la web, en
% soporte t\'ecnico al proveer ayuda al usuario no experto, en
% control de dispositivos por voz al usarse en sistemas de di\'alogo embebidos,
% o
% en la ense\~nanza a distancia al constituir un tutor virtual para el
% aprendizaje
% de lenguas extranjeras. Esta \'ultima es la tarea seleccionada para testear
% nuestro prototipo. La eficacia y adecuaci\'on del prototipo para
% tal tarea puede ser estimada de diferentes formas. Por un lado, se puede
% realizar 
% una evaluaci\'on comparativa con
% otros sistemas existentes. Por otra parte, estudiantes de la facultad donde se
% desarrollar\'a
% el presente proyecto pueden participar en una evaluaci\'on orientada al
% usuario.

The resulting system could be used in different applications in which the
system needs to give instructions. For example in electronic commerce providing 
assistance on the web for buying a product, in technical support helping a
novice user to fix a problem, and in language learning by providing a
virtual tutor for learning foreign languages. The latter is the task selected to
test our prototype. The effectiveness and suitability of a prototype for such a
task can be estimated in different ways. On the one hand, we will compare it
with other existing and similar systems. On the other hand, we will
carry out a user evaluation. 

% A fin de obtener un sistema como el descripto anteriormente, este proyecto
% propone estudiar las tres \'areas detalladas a continuaci\'on, las cuales son 
% fundamentales para el desarrollo de cualquier sistema de di\'alogo:

Implementing a dialogue system is a multidisciplinary effort. During the
development of our dialogue system we will particularly concentrate our research
efforts in the following areas:

% \paragraph{Pragm\'atica:} La pragm\'atica es un \'area interdisciplinaria a
% la que contribuyen teor\'ias ling\"u\'isticas (e.g., implicaturas
% conversacionales~\cite{grice75}), sociol\'ogicas (e.g., an\'alisis
% conversa\-cio\-nal~\cite{schegloff87b}) y filos\'oficas (e.g., teor\'ia de los
% actos
% de habla~\cite{austin62}). Su objetivo es estudiar c\'omo el contexto (en el
% que la conversaci\'on esta situada) contribuye al significado (de cada cosa
% que
% se diga durante esa conversaci\'on). La transmisi\'on de significado depende
% no
% s\'olo de la informaci\'on ling\"u\'istica (entidades en foco, reglas
% gramaticales y morfol\'ogicas, etc.), sino tambien extraling\"u\'istica
% (situaci\'on f\'isica donde la comunicaci\'on est\'a situada, experiencias
% previas de los hablantes, objetivo de la conversaci\'on, etc.). Por lo tanto,
% una misma oraci\'on puede significar cosas diferentes en distintos contextos;
% el
% \'area de pragm\'atica estudia el proceso por el cual una oraci\'on es
% desambig\"uada usando su contexto. En pragm\'atica se distingue entre
% oraci\'on
% (forma gramatical que toma el acto ling\"u\'istico) y enunciado (oraci\'on
% m\'as su
% contexto). La habilidad de entender una oraci\'on usando su contexto, es
% decir,
% la habilidad de entender un enunciado, se conoce como competencia
% pragm\'atica.
% Explicar la competencia pragm\'atica implica explicar c\'omo una persona hace
% inferencias sobre una oraci\'on y su contexto para interpretar adecuadamente
% el
% enunciado que el emisor intenta transmitir. 

\paragraph{Pragmatics:} Pragmatics is an interdisciplinary field which
integrates insights from linguistics (e.g., 
conversational implicatures~\cite{grice75}),
sociology (e.g., conversational analysis~\cite{schegloff87b}) and
philosophy (e.g., theory of speech acts~\cite{austin62}). It aims to explore how
the context (in which a conversation is situated) contributes to the meaning (of
everything that is said during that conversation). The meaning conveyed during
a conversation depends not only of linguistic information (entities in focus,
grammatical and morphological rules, etc.) but also from extralinguistic
information (physical situation where the conversation is located, previous
experiences of speakers, etc.). As a result, the same sentence may mean
different things in different contexts. The area of pragmatics studies the
process by which a sentence is disambiguated using its context. In pragmatics,
there is a distinction between sentence (grammatical form that
the linguistic act takes) and utterance (sentence plus its context). The ability
to understand a sentence using its context, i.e. the ability to understand an
utterance, is called pragmatic competence. Explaining the pragmatic competence
involves explaining how a person makes inferences about a sentence and its
context to properly interpret the meaning that the speaker intends to convey.
% 
% Para que un sistema de di\'alogo interact\'ue de una forma natural con sus
% usuarios, debe demostrar habilidad pragm\'atica. Por lo tanto, dicho sistema
% debe definir (1) qu\'e tipo de informacion contextual se debe representar y
% (2)
% qu\'e tareas de inferencia sobre la oraci\'on y el contexto son necesarias
% para
% interpretar un enunciado. Hacer estas dos tareas de forma correcta tendr\'a un
% impacto crucial sobre el desempe\~no de un sistema como el que proponemos
% desarrollar. En dicho sistema es indispensable que las oraciones hagan
% expl\'icita la cantidad de informaci\'on justa: si la informaci\'on es
% demasiada, se retrasar\'a y aburrir\'a al usuario, si la informaci\'on es muy
% poca, el usuario no sabr\'a c\'omo llevar a cabo la tarea y cometer\'a
% errores. 

A dialogue system needs to have pragmatic capabilities in order to interact in a
natural way with its users. Therefore, the system must define (1) what kind of
contextual information should be represented and (2) what inference tasks on a
sentence and context are necessary in order to interpret an utterance. Doing
these tasks correctly will have a crucial impact on the performance of a system
like the one we propose to develop. In such a system it is indispensable that
the sentences makes explicit the right ammount of information: if
information is too much, the user will be delayed and get bored, if the
information is too little, the user will not know how to perform the task and
will make mistakes.

% \paragraph{Inferencia:} Podemos entender como
% inferencia toda operaci\'on que transforme informaci\'on \emph{impl\'icita} en
% \emph{expl\'icita}.  Esta definici\'on es lo suficientemente general 
% como para cubrir tareas que van desde la inferencia l\'ogica (i.e.,
% deducci\'on 
% en un lenguaje formal), hasta tareas de inferencia habituales en inteligencia
% artificial (e.g., planning e inferencia no mon\'otona), y operaciones
% estad\'isticas (por
% ejemplo obtener estimadores sobre un conjunto de datos).  Un sistema de
% di\'alogo realiza continuamente operaciones de inferencia. Por un lado, 
% se necesita inferencia para
% interpretar la informaci\'on recibida e incorporarla al repositorio de datos,
% y por otro, para decidir qu\'e parte de la informaci\'on disponible se debe
% transmitir.

\paragraph{Inference:} Inference can be understood as any operation that
transforms implicit information in explicit information. This definition is
general enough to cover tasks ranging from logical inference (i.e., deduction in
a formal language) to inference tasks common in AI (e.g., planning and
non-monotonic inference), as well as statistical operations (e.g. obtaining
estimators on a data set). A dialogue system has to
continually perform inference operations. On the one hand, inference is needed
to interpret the information received and incorporate it to the data repository,
and, on the other hand, it is needed in order to decide how much of the
available information should be conveyed.

%   El problema mismo de decidir qu\'e tipo de representaci\'on l\'ogica y qu\'e
% tipo de inferencia utilizar  en una determinada situaci\'on  es complejo
% (l\'ogica proposicional vs.\ l\'ogica de primer
% orden, validez vs.\ chequeo de modelos, inferencia l\'ogica vs.\ inferencia
% estad\'istica). Adem\'as las
% tareas de inferencia en s\'i son computacionalmente costosas.  El desaf\'io en
% este caso es encontrar el compromiso adecuado entre la representaci\'on de la
% informaci\'on y el m\'etodo de inferencia a utilizar.

The very problem of deciding what kind of logical representation and what type
of inference to use in a given situation is complex (propositional logic vs.
First-order logic, validity vs. model checking, logical inference vs.
statistical inference). Furthermore inference tasks themselves are
computationally expensive. The challenge here is to find the appropriate
balance between the expressivity of the representation formalism and the
cost of the required inference methods.


% \paragraph{Evaluaci\'on:} La evaluaci\'on de sistemas de generaci\'on de
% lenguaje natural es una de las m\'as dif\'iciles dentro del \'area del PLN,
% dado
% que una idea puede expresarse de muchas formas, todas ellas correctas. En
% general, determinar la calidad de una frase generada no puede hacerse de
% manera
% simple y directa, por ejemplo, comparando el resultado del sistema con un
% patr\'on (en Ingl\'es \emph{gold standard}). El problema de la falta de
% patrones es
% compartido con otra \'area del PLN, la Traducci\'on Autom\'atica (TA), en la
% que se han propuesto diversas metodolog\'ias para la evaluaci\'on de sistemas 
% que pueden clasificarse en directas e indirectas. 
% Una metodolog\'ia directa aplica alguna m\'etrica al texto generado por un
% sistema. Una metodolog\'ia indirecta, en cambio, eval\'ua la perfomance del 
% sistema a trav\'es de la utilizaci\'on del texto generado para realizar alguna
% tarea. Sin
% embargo, en ninguna de estas \'areas existe una metodolog\'ia aceptada como
% est\'andar y demostrada eficaz de manera general.

The evaluation of natural language generation systems is one of the
most difficult ones in the area of NLP; an idea can be expressed in many
forms, all of them correct. In general, determining the quality of a
generated sentence can not be done simply and directly, for example, comparing
the result with a gold standard. The problem of lack of gold standards is shared
with another area of the NLP, namely the Machine Translation (MT), for which
various evaluation methodologies, which can be classified into direct and
indirect, have been proposed. A direct method applies a metric to the text
generated by the system. An indirect method evaluates the perfomance of the
system through the use of the generated text to perform some task. However, 
none of these methods there is a standard and accepted methodology which has
been generally proven to be effective.

% Dado que el objeto a evaluar en este proyecto es un sistema que interact\'ua
% via la generaci\'on de instrucciones en lenguaje natural, podemos
% determinar su performance por medio de evaluaciones cuantitativas (como el
% tiempo de finalizaci\'on de la tarea), cualitativas (e.g., la calidad de
% las interacciones) y basadas en el contexto (evaluaciones orientadas al
% usuario
% y sus necesidades en una situaci\'on particular). Dada la experiencia previa
% de los
% integrantes del proyecto, es de especial inter\'es estudiar la  portabilidad y
% aplicaci\'on
% de t\'ecnicas de evaluaci\'on del dominio de la TA y la interacci\'on
% multimodal
% humano-computadora, al sistema  planteado en
% este proyecto (en su totalidad y por componentes).

Since what is being evaluated in this project is a system that interacts via the
generation of natural language instructions, we can determine its performance
through quantitative metrics (such as average task completion time), 
qualitative metrics (e.g., general user satisfaction) and metrics based on the
context (how well the system addressed the user needs in particular situations).
Given our previous experience, it is particularly
interesting for us to study the portability of evaluation
techniques from the domain of MT
and multimodal human-computer interaction to the evaluation of the system
proposed in this project.
  % Objetivos Generales
% 
% En este proyecto prestaremos principal atenci\'on al sistema encargado de la
% generaci\'on de  instrucciones e interpretaci\'on de las respuestas del
% usuario.
% Dicho sistema deber\'a ser capaz de una interacci\'on natural y debe poder
% adaptarse a
% posibles errores de interpretaci\'on o ejecuci\'on de las instrucciones, dando
% instrucciones correctivas cuando sea necesario. El usuario se encontrar\'a
% en un universo en el que podr\'a interactuar con objetos y explorar el
% ambiente
% virtual provisto. En este mundo virtual 3D, el usuario intentar\'a seguir las
% instrucciones
% provistas por el sistema realizando acciones f\'isicas.

In this project, we focused primarily on the system responsible for the
generation of instructions and interpretation of user responses. The system must
be capable of
a natural interaction and be able to adapt to possible errors of interpretation
or execution of the instructions, giving corrective instructions when necessary.
You will find yourself in a universe where you can interact with objects and
explore the virtual environment provided. In this 3D virtual world, the user
will try to follow instructions provided by the system by physical actions.

% En una primera etapa, el modelo a utilizar permitir\'a el flujo de
% informaci\'on
% ling\"u\'istica de manera unidireccional, desde el sistema hacia el usuario,
% por lo que
% el usuario no podr\'a pedir ning\'un tipo de ayuda.  La restricci\'on a un
% modelo unidireccional tiene como objetivo
% simplificar la representaci\'on y el manejo del contexto de la interacci\'on.

Initially, the model to use will flow one-way linguistic information from the
system to the user so the user can not ask for any help. The restriction to a
unidirectional model is to simplify the representation and management of the
context of the interaction.

% En etapas m\'as avanzadas del proyecto, el modelo unidireccional ser\'a
% extendido para
% permitir intercambio ling\"u\'istico bidireccional: por ejemplo, el usuario
% podr\'a
% pedir clarificaciones o redefinir el objetivo a alcanzar, en ambos casos
% usando
% lenguaje natural.

At later stages of the project, the unidirectional model will be extended to
allow bidirectional language exchange: for example, the user may request
clarification or redefine the objective to be achieved in both cases using
natural language.

% La generaci\'on de instrucciones para un dominio espec\'ifico requiere
% informaci\'on  ling\"u\'istica a distintos niveles.
% A nivel morfol\'ogico y sint\'actico se requiere una
% gram\'atica del lenguaje seleccionado (Espa\~nol en este caso). A nivel
% sem\'antico se necesita un repositorio de informaci\'on l\'exica organizada en
% forma ontol\'ogica. A nivel pragm\'atico es necesaria una descripci\'on formal
% de
% las
% acciones posibles en el dominio (con sus precondiciones y efectos), e
% informaci\'on del estado de la interacci\'on.

The generation of instructions for a specific domain requires different levels
of linguistic information. A morphological and syntactic level requires a
selected language grammar (Spanish in this case). A semantic level requires a
repository of lexical information organized as ontological. A pragmatic level
requires a formal description of possible actions in the domain (with their
preconditions and effects), and information of prior interaction.


% Esta informaci\'on relacionada a un dominio dado es utilizada por distintos
% componentes del sistema de di\'alogo que
% gestiona la interacci\'on con el usuario. En un sistema t\'ipico, esto
% sucede como se detalla a continuaci\'on.  En la direcci\'on sistema
% $\rightarrow$
% usuario, el componente encargado de \emph{planificar el contenido} genera una
% secuencia de
% instrucciones relevantes al momento de la interacci\'on, luego el componente
% encargado de  \emph{gestionar el conocimiento} actualiza la informaci\'on
% contextual y,  finalmente, el componente encargado de
% \emph{generar instrucciones}  transmite las mismas como expresiones
% de lenguaje natural.  En la direcci\'on opuesta, es decir usuario
% $\rightarrow$
% sistema,  el componente denominado \emph{discretizador}
% transforma el flujo continuo de informaci\'on derivada del comportamiento del
% usuario, en acciones relevantes a la tarea en cuesti\'on, mientras que
% los componentes de planificaci\'on del contenido y gesti\'on del conocimiento
% se ocupan
% de mantener el contexto actualizado y de detectar posibles errores.

This information relating to a given domain is used by various components of the
dialogue system that manages the user interaction. In a typical system, this is
as detailed below. The address system $\rightarrow$ user, the component
responsible for planning the content generates a sequence of instructions
relevant to the moment of interaction, then the component will manage the
knowledge updated contextual information and, finally, the component will
generate transmits the same instructions as natural language expressions. In the
opposite direction, ie user $\rightarrow$ system, the component called
discretized transforms the continuous flow of information derived from user
behavior in actions relevant to the task at hand, while the components of
content planning and management knowledge concerned with maintaining the current
context and possible errors.

% En la siguiente secci\'on se detalla el plan de trabajo propuesto para llevar
% a
% cabo los objetivos planteados (la Secci\'on 8
% provee un cronograma completo), para luego discutir una posible aplicaci\'on
% del
% sistema a desarrollar.


In the next sections, we detail the proposed work plan to carry out the stated
objectives and then we discuss a possible application of the system to be
developed. 

\subsection{Tasks}
% 
% Dise\~nar e implementar todos estos componentes empezando desde cero
% requerir\'ia un esfuerzo prohibitivo, por lo que proponemos utilizar recursos
% disponibles de forma libre y gratuita. En particular, usaremos la plataforma
% desarrollada como parte de la competencia \emph{Generating Instructions in
% Virtual Environments}
% (GIVE)\footnote{GIVE se encuentra disponible libremente en
% \url{http://www.give-challenge.org}, y posee el respaldo de los grupos de
% inter\'es SIGSEM, SIGDIAL, y SIGGEN de la \emph{Association for Computational
% Linguistics} (ACL).} que
% provee una plataforma b\'asica para este tipo de sistemas.

As we said, designing and implementing all the necessary components from scratch
would require a prohibitive effort, therefore we will use resources
available for free. In particular, we will use the platform developed as part of
the competencia1 Generating Instructions in Virtual Environments
(GIVE)\footnote{GIVE is freely available at \url{http://www.give-challenge.org}
and has the support of interest groups SIGSEM, SIGDIAL, and SIGGEN of the
Association for Computational Linguistics (ACL).} which provides a basic
platform for this type of system.

% El objetivo principal de GIVE~\cite{byron09} es actuar como medio de
% evaluaci\'on de sistemas de generaci\'on
% de lenguaje natural. En esta competencia, los sistemas se eval\'uan de forma
% homog\'enea dado que todos los sistemas participantes utilizan los mismos
% recursos provistos por los organizadores.
% En el escenario propuesto por GIVE, el usuario
% humano lleva acabo una tarea de ``b\'usqueda del tesoro" en un entorno
% virtual 3D y el trabajo del sistema participante es proveer, en tiempo
% real, instrucciones en lenguaje natural que ayuden al usuario
% a encontrar el tesoro escondido.  Por lo tanto, GIVE no s\'olo eval\'ua
% la generaci\'on de lenguaje natural sino tambi\'en la habilidad del
% sistema de participar en una interacci\'on situada en un entorno 3D.

The main aim of GIVE~\cite{byron09}  is to act as a means of evaluation of
systems for natural language generation. In this competition, systems are
evaluated in a homogeneous since all participating systems use the same
resources provided by the organizers. In the scenario proposed by GIVE, the
human user carries out a task of `` treasure hunts in a 3D virtual environment
and work of the participant system is to provide real-time, natural language
instructions that help users find the hidden treasure. So GIVE not only assesses
the generation of natural language but also the system's ability to participate
in an interaction set in a 3D environment.
% 
% La plataforma GIVE es, entonces, una herramienta propicia para
% investigar los distintos aspectos de la interacci\'on situada en un entorno
% virtual que se enumeran a continuaci\'on: el problema de realizaci\'on
% sint\'actica y morf\'ogica de las
% instrucciones (en Ingl\'es, \emph{surface realization});
% la representaci\'on del conocimiento existente en el contexto de
% interacci\'on (contexto del discurso, ontolog\'ias, acciones posibles);
% el tipo de inferencia requerido por las tareas involucradas
% (planning, construcci\'on y chequeo de modelos); y
% las reglas pragm\'aticas de interacci\'on requeridas por la tarea
% (administraci\'on de la carga cognitiva, actualizaci\'on y uso de la
% informaci\'on contextual). GIVE provee, adem\'as, 
% herramientas para registrar todos los detalles de la
% interacci\'on, permitiendo de esta forma obtener f\'acilmente un corpus de
% interacci\'on en entornos virtuales, anotado autom\'aticamente.

GIVE platform is thus a favorable tool for investigating various aspects of
interaction situated in a virtual environment as listed below: the problem of
morphogen syntactic realization of the instructions (in English, surface
realization), the representation of existing knowledge in the context of
interaction (discourse context, ontologies, possible actions), the type of
inference required by the tasks involved (planning, construction and checking of
designs), and pragmatic rules of interaction required by the task (management
cognitive load, update and use of contextual information). GIVE provides further
tools to record all details of the interaction, thus allowing to easily obtain a
corpus of interaction in virtual environments annotated automatically.

% Para la definici\'on correcta de las pol\'iticas de interacci\'on que
% utilizaremos en nuestro prototipo es necesario
% contar con un corpus que provea ejemplos de interacci\'on t\'ipica sobre la
% tarea~\cite{HCRC-93,byron-06}.  Usando herramientas provistas por GIVE,
% recopilaremos un corpus de interacci\'on humano-humano (uni y bidireccional)
% que contendr\'a la interacci\'on lig\"u\'istica alineada con las acciones
% realizadas en el mundo virtual.

For the correct definition of interaction policies we use in our prototype is a
need for a body that provides examples of typical interaction on the
task~\cite{HCRC-93,byron-06}. Using tools provided by GIVE, we collect a corpus
of human-human interaction (uni and bidirectional) containing the interaction
linguistics has aligned with actions in the virtual world.

% A partir del corpus recolectado comenzaremos el dise\~no,
% implementaci\'on y testing de algoritmos de interacci\'on unidireccionales,
% los 
% cuales ser\'an integrados a la plataforma GIVE.
% Existen tradicionalmente cuatro tareas que un sistema de generaci\'on
% de instrucciones debe implementar: (1) planificaci\'on del contenido, (2)
% generaci\'on
% de expresiones referenciales, (3) gesti\'on del contexto de interacci\'on, y
% (4) interpretaci\'on de las respuestas del usuario.

From the corpus collected begin the design, implementation and testing of
unidirectional interaction algorithms, which will be integrated into the
platform GIVE. There are traditionally four tasks that a generation system must
implement instructions: (1) planning the content, (2) generation of referring
expressions, (3) managing the interaction context, and (4) interpretation of
user responses. 
% 
% \emph{(1) Planificaci\'on del contenido:} Para planificar el contenido a
% generar
% es necesario primero obtener un plan de
% acciones que alcance el objetivo desde el estado actual. Dicho plan
% contrendr\'a
% acciones f\'isicas a ejecutar sobre el entorno. El segundo paso es decidir
% c\'omo se transmitir\'a al usuario esta secuencia de acciones. Es decir,
% se debe decidir cu\'antas acciones comunicar
% por instrucci\'on y c\'omo agregarlas coherentemente. El
% resultado del proceso de agregaci\'on de acciones (en Ingl\'es, \emph{action
% agreggation}) 
% es un \'arbol que describe la
% estructura de la tarea a diferentes niveles de abstracci\'on. El tercer y
% \'ultimo paso
% es decidir c\'omo navegar el \'arbol de acciones para verbalizar las
% instruciones (por ejemplo, en post o preorden~\cite{foster-etal-ijcai2009}).
% En
% este proyecto investigaremos
% diferentes pol\'iticas de agregaci\'on (e.g., agregando acciones que manipulan
% el mismo
% objeto) y pol\'iticas no est\'andard de recorrido del \'arbol de acciones
% (e.g.,
% bajando a un
% nivel menor de abstracci\'on en caso de malentendidos).

\emph{(1) Content planning:} To plan the content to be generated is first
necessary to obtain a plan of action to reach the target from the current state.
The plan to run contrendrá physical actions on the environment. The second step
is to decide how to transmit this sequence of user actions. That is, they must
decide how many shares communicate instruction and how to add coherently. The
result of the aggregation process shares (in English, action agreggation) is a
tree describing the task structure at different levels of abstraction. The third
and final step is to decide how to navigate the tree of actions to verbalize the
instructions (for example, post or preorder~\cite{foster-etal-ijcai2009}). This
project will investigate different aggregation policies (eg, adding actions that
manipulate the same object) and non-standard travel policy of the tree stock
(eg, dropping to a lower level of abstraction in case of misunderstanding).

% Para obtener el plan de acciones, utilizaremos la tarea de inferencia de
% planning~\cite{nau04}. GIVE provee un sistema de planning muy
% limitado.  Existen sistemas de planning m\'as avanzandos (en particular,
% que permiten el manejo de informaci\'on incompleta sobre el dominio)
% como PKS\footnote{PKS est\'a disponible en
% \url{http://homepages.inf.ed.ac.uk/rpetrick/research/pks/}}, pero
% est\'an todav\'ia en desarrollo.  En general, el estado del
% arte en el \'area de planning no cubre los requerimientos de nuestro
% sistema.  Si bien existen sistemas optimizados que funcionan adecuadamente
% en ciertas aplicaciones, ninguno provee servicios como la generaci\'on de
% planes alternativos, o la generaci\'on de planes incompletos en caso de
% ausencia de plan. Por lo que deberemos dise\~nar e implementar estas
% extensiones a los algoritmos de planning. Estudiaremos tambi\'en el
% comportamiento te\'orico (e.g., complejidad) de estos algoritmos.

For the action plan, we use the inference task planning~\cite{nau04}. GIVE
provides a very limited system planning. There are more avanzandos
planning systems (in particular, to enable the handling of incomplete
information on the domain) as
PKS\footnote{PKS is
freely available in \url{http://homepages.inf.ed.ac.uk/rpetrick/research/pks/}},
but they are still developing. In general,
the state of the art in the planning area does not cover the requirements of our
system. While there are systems that work well optimized for certain
applications, none provides services such as the generation of alternative
plans, or the generation of incomplete plans in case of absence of plan. As we
design and implement these extensions to the algorithms of planning. Also study
the theoretical behavior (e.g., complexity) of these algorithms. 
% 
% \emph{(2) Generaci\'on de Expresiones Referenciales:} Una vez terminada la
% fase
% de planificaci\'on de contenido, la siguiente tarea
% es la generaci\'on de expresiones referenciales. Esta tarea implica producir
% una
% frase que describa una entidad referenciable de forma tal que el usuario
% la pueda identificar (e.g., ``el jarr\'on que est\'a sobre la mesa'').
% Estas expresiones, para ser aceptables, deben ser similares a las que podr\'ia
% producir una persona en condiciones normales (por ejemplo, no ser\'ia
% aceptable,
% ``el jarr\'on que no est\'a arriba de la silla ni arriba del
% sof\'a ni abajo de la mesa'').  En~\cite{AKS08} se propone utilizar la
% minimizaci\'on simb\'olica del modelo que representa
% el estado del mundo, para as\'i obtener f\'ormulas l\'ogicas que describan
% un\'ivocamente a cada objeto. En nuestro proyecto implementaremos este
% m\'etodo
% y lo evaluaremos dentro del sistema de di\'alogo.

\emph{(2) Generating referring expressions:} Once the planning phase content,
the next task is the generation of referring expressions. This task involves
producing a sentence that describes a referenceable entity so that the user can
identify (eg, `` the vase on the table''). These expressions, to be acceptable,
must be similar to those that could produce a person in normal conditions (for
example, would not be acceptable, `` the vase who is not above the top of the
chair or sofa or under the table''). In~\cite{AKS08} proposes to use the
symbolic minimization model that represents the state of the world, in order to
obtain logical formulas that describe each object uniquely. In our project we
will implement this method and evaluate within the dialogue system.
% 
% \emph{(3) Gesti\'on del Contexto de Interacci\'on:} Para la gesti\'on del
% contexto de interacci\'on utilizaremos, en un primer
% momento, sistemas existentes de manejo de conocimiento (\emph{knowledge
% mantainance
% systems}) como
% FaCT++~\cite{horr:fact99},
% RACER~\cite{haar:race99} o Pellet~\cite{siri:pell06}, que soportan tareas de
% definici\'on, mantenimiento y consulta de ontolog\'ias.  Estos sistemas
% han sido utilizados como motores de inferencia
% en numerosas aplicaciones en el \'area~\cite{franconi03,koller04} y,
% en particular,  en aplicaciones dise\~nadas por miembros del equipo de
% investigaci\'on~\cite{benotti09b}.  Una vez observado el
% comportamiento de
% estos motores de inferencia en la tarea, se analizar\'an sus limitaciones
% e investigar\'an extensiones requeridas.

\emph{(3) Management of the Interaction Context:} To manage the use of
interaction context in the first instance, existing systems of knowledge
management (knowledge mantainance systems) as RACER~\cite{haar:race99} or
Pellet~\cite{siri:pell06}, which support tasks of definition, maintenance and
querying of ontologies. These systems have been used as inference engines in
numerous applications in the area~\cite{franconi03,koller04} and, in particular
in applications designed by members of the research team~\cite{benotti09b}. Once
observed the behavior of these inference engines on the task, analyzing the
limitations and investigate extensions required.
% 
% \emph{(4) Interpretaci\'on de las Respuestas del Usuario:} La
% interpretaci\'on 
% de las respuestas del usuario en el sistema unidireccional
% es relativamente simple, y en una primera etapa utilizaremos el m\'odulo
% discretizador provisto por GIVE.  Luego de la evaluaci\'on del sistema,
% podremos determinar si este m\'odulo satisface o no los requerimientos de
% nuestra tarea y cu\'ales son sus limitaciones.  En el sistema bidireccional,
% en cambio, \'este es el m\'odulo que requerir\'a m\'as atenci\'on.

\emph{(4) Interpretation of User Responses:} The interpretation of user
responses in the unidirectional system is relatively simple, and in a first
stage we use the discretized form provided by GIVE. After evaluating the system,
this module can determine whether or not it meets the requirements of our task
and what are its limitations. In the two-way system, however, this is the module
that will require more attention.
% 
% Por empezar, el sistema bidireccional debe ser extendido con capacidades de
% procesamiento
% de enunciados provenientes del usuario (an\'alisis sint\'actico,
% construcci\'on
% sem\'antica,
% resoluci\'on de referencias, etc.).  Gracias a que, en este sistema,  los
% objetos y las acciones
% a los que el usuario se puede referir son los del entorno virtual,
% el lenguaje a interpretar est\'a naturalmente restringido y es posible 
% utilizar recursos para interpretaci\'on de lenguaje natural
% existentes~\cite{kow06}.  Estudiaremos, en particular, dos
% tipos espec\'ificios de contribuciones del usuario: pedidos de aclaraci\'on
% de la \'ultima instrucci\'on dada, y redefinici\'on de objetivos.  Elegimos
% este tipo de contribuciones dado que representan contribuciones del tipo
% `reparaci\'on' a corto y largo plazo, respectivamente. Implementaremos las
% reparaciones a corto plazo extendiendo el trabajo de~\cite{purver06}.  Para
% las reparaciones a largo plazo utilizaremos los lineamientos
% de~\cite{blaylock05a,blaylock05b}.  Obviamente, la integraci\'on de
% capacidades
% ling\"u\'isticas en la direcci\'on usuario $\to$ sistema implica no s\'olo
% cambios en el m\'odulo de interpretaci\'on de las respuestas, sino que todos
% los
% componentes mencionados
% anteriormente son afectados.  Por ejemplo, el m\'odulo de gesti\'on de la
% informaci\'on debe ahora tambi\'en representar y mantener actualizada las
% contribuciones ling\"u\'isticas del usuario; mientras que el m\'odulo de
% planificaci\'on del contenido debe reestructurar el \'arbol de acciones
% de la tarea cuando el usuario requiere una reparaci\'on a largo plazo.


To start with, the two-way system should be expanded processing capabilities of
statements coming from user input (parsing, semantic construction, resolution of
references, etc.).. With that in this system, the objects and actions that the
user can refer are the virtual environment, to interpret the language is
naturally limited and may utilize resources for existing natural language
interpretation~\cite{kow06}. Study, in particular, two types of user
contributions specificity:
requests for clarification of the instruction given, and redefinition of goals.
We chose such contributions represent contributions because the type `fix it
'short and long term respectively. We will implement short-term repairs to
extend the work of~\cite{purver06}. For long-term repairs will use the
guidelines of~\cite{blaylock05a,blaylock05b}. Obviously, the integration
of
language skills in the management user $\to$ system involves not only changes
in the form of interpretation of the answers, but all the above components are
affected. For example, the management module of information must now also
represent and update the user's linguistic contributions, while the content
planning module must restructure the tree stock of the task when the user
requires a long-term repair. 
\medskip

% \noindent
% Para determinar la calidad de los propotipos obtenidos se propone
% estudiar la creaci\'on de un modelo de la calidad seg\'un los est\'andares de
% evaluaci\'on de software ISO/IEC 9126 y 14528 \cite{ISO9126-1,ISO14598-1}, los
% cuales fueron exitosamente aplicados al dominio de la TA, como lo muestra la
% herramienta FEMTI (Framework for the Evaluation of Machine
% Translation)\footnote{Disponible libremente en
% \url{http://www.issco.unige.ch/femti/}}. FEMTI
% \cite{Est2005} intenta guiar a los evaluadores hacia la creaci\'on de planes
% de
% evaluaci\'on parametrizables que incluyen diversos aspectos del sistema a
% evaluar
% y ofrece un conjunto de m\'etricas relevantes. La identificaci\'on de
% m\'etricas
% relevantes se puede realizar usando distintos m\'etodos, por
% ejemplo bas\'andose en experiencias previas
% \cite{paradise06,Chu2000,Litman2002}, realizando encuestas o especificaciones
% de
% requerimientos (como en
% \cite{Lecoeuche98}) o bien recolectando estos datos a trav\'es de experimentos
% llamados ``mago de Oz" (del Ingl\'es \emph{wizard of Oz}) donde el usuario
% interact\'ua con un prototipo de sistema (posiblemente incompleto o reducido
% en
% funcionalidades) y un humano (el ``mago") detr\'as de la interface
% responde como si lo hiciera el sistema \cite{Dahlback93,Fabbrizio05}.

\noindent
To determine the quality of the obtained propotipos is to study the creation of
a model of quality assessment by the standards of software ISO/IEC 9126 and
14528~\cite{ISO9126-1,ISO14598-1}, which were successfully applied to the
TA domain, as shown in the tool FEMTI (Framework for the Evaluation of Machine
Translation)\footnote{Freely available in
\url{http://www.issco.unige.ch/femti/}}. FEMTI~\cite{Est2005}
try to guide the evaluators towards creating parameterized evaluation
plans that include various aspects of the system to evaluate and offer a
relevant set of metrics. The identification of relevant metrics can be performed
using various methods, eg based on previous
experience~\cite{paradise06,Chu2000,Litman2002}, conducting
surveys or specification requirements (as in \cite{Lecoeuche98}) or
collecting such data through experiments called ``Wizard of Oz"(the
English wizard of Oz) where the user interacts with a prototype system (possibly
incomplete or reduced functionality) and a human (the ``Wizard") behind the
interface responds as if I did the system~\cite{Dahlback93,Fabbrizio05}.

% Luego de elaborar un modelo de la calidad, se pueden aplicar diversas
% metodolog\'ias para evaluar distintos aspectos del sistema. Seg\'un
% corresponda,
% se
% pueden aplicar m\'etricas autom\'aticas, subjetivas (tambi\'en llamadas
% ``humanas") o basadas en la tarea, tanto para evaluar la contribuci\'on de
% cada
% componente como la calidad del sistema en su totalidad.
% Por otro lado, dado que se usar\'a la plataforma GIVE, se planea la
% participaci\'on en el evento asociado, lo cual servir\'a como una fuente
% adicional de informaci\'on acerca de los aspectos del sistema a mejorar.

After developing a quality model can be applied several methodologies to assess
various aspects of the system. As appropriate, you can apply automatic metrics,
subjective (also called `` human ") or based on the task, both to evaluate the
contribution of each component as the quality of the whole system. On the other
hand, since it will be used GIVE platform is planned participation in the
associated event, which will serve as an additional source of information about
aspects of the system to improve.

% Una vez que el sistema de interacci\'on bidireccional fue evaluado y mejorado
% con los resultados de esta evaluaci\'on,
% se investigar\'a su utilizaci\'on
% como tutor virtual de idiomas como describimos en la siguiente secci\'on.

Once the two-way interaction system was evaluated and improved over the results
of this evaluation, we will investigate its use as a virtual language tutor as
described in the next section.

% Durante todo el proyecto, nos ocuparemos de la diseminaci\'on de resultados y
% lecciones aprendidas.  En particular, trabajaremos en la documentaci\'on del
% sistema obtenido, en forma de manuales del usuario y del desarrollador que
% incluyan una descripci\'on detallada de los algoritmos implementados con sus
% puntos fuertes y d\'ebiles.
% Los resultados te\'oricos y aplicados ser\'an presentados en conferencias
% locales e internacionales,
% y en revistas cient\'ificas pertinentes a las \'areas de
% investigaci\'on afectadas.  Nuestros
% planes de diseminaci\'on se definen en m\'as detalle en la Seccion~10.

Throughout the project, we will ensure the dissemination of results and lessons
learned. In particular, we will work in your system documentation obtained in
the form of user manuals and developer to include a detailed description of the
algorithms implemented with their strengths and weaknesses. The theoretical and
applied results will be presented at local and international conferences and in
journals relevant to the research areas concerned. Our plans to spread more
fully defined in Section 10. 

\fixme{quede aca}

\subsection{Aplications}

El resultado del proyecto ser\'a un sistema capaz de dar instrucciones
en lenguaje natural que deben ser llevadas a cabo por el usuario en un
entorno virtual 3D.  La tecnolog\'ia y los avances te\'oricos del proyecto
pueden utilizarse en distintas aplicaciones (en comercio electr\'onico,
soporte t\'ecnico, control de dispositivos por voz, etc.).  Durante el
\'ultimo a\~no del proyecto investigaremos su uso para la ense\~nanza a
distancia, adaptando el sistema para que funcione como tutor de lenguas
extranjeras~\cite{Eskenazi09,Wik09}.

Dada la arquitectura del sistema, es posible cambiar el lenguaje de
interacci\'on con el sistema (input y output), introduciendo
una gram\'atica y dem\'as recursos sint\'acticos (e.g., informaci\'on
morfol\'ogica) para el lenguaje correcto.  Es decir, dados los recursos
sint\'acticos adecuados para, por ejemplo, el ingl\'es que cubran las
estructuras y el vocabulario usados en el sistema, se obtiene un sistema
de di\'alogo que interprete y/o produzca instrucciones en ingl\'es.

Un sistema unidireccional que genere instrucciones en ingl\'es puede
usarse para testear la comprensi\'on del usuario.  La correcta
interpretaci\'on de las instrucciones se puede evaluar a partir de la
correcta ejecuci\'on de las instrucciones dadas.  El sistema
bidireccional permitir\'a al usuario pedir aclaraciones sobre la
\'ultima instrucci\'on (en su lengua natal, en caso de no haber comprendido
la instrucci\'on, o en ingl\'es, si desea practicar su
capacidad de expresarse en el idioma extranjero).  El usuario tambi\'en
podr\'a redefinir el objetivo a alcanzar durante la interacci\'on, y
de esta forma seleccionar el vocabulario y el tipo de estructuras que desea
practicar.

Los mundos virtuales (como Second Life\footnote{Accessible gratuitamente
en \url{http://secondlife.com/}}) est\'an siendo incorporados r\'apidamente a
la educaci\'on, tanto inicial como universitaria~\cite{Doswell05,molk:lear09}. 
Su principal atractivo
es el de proveer oportunidades que son dif\'iciles o imposibles de
proveer en el mundo actual (por ejemplo, por limitaciones econ\'omicas
o porque la experiencia es peligrosa para el observador).
Por otra parte, el uso de un tutor virtual tiene ciertas ventajas
respecto de un tutor humano. \cite{engwall1020} mencionan las siguientes. (1) \emph{Tiempo de
pr\'actica}: la posibilidad de practicar el nuevo lenguaje es esencial
para el aprendizaje, y un tutor virtual provee oportunidades de pr\'actica
s\'olo limitadas por recursos tecnol\'ogicos. (2) \emph{Prestigio:} un
estudiante puede sentirse avergonzado de cometer errores frente a un tutor
humano, y de esta forma limitar su capacidad de expresi\'on en el lenguaje
extranjero. (3) \emph{Realidad Aumentada}: un tutor virtual puede proveer
material adicional (e.g., ejemplos en contexto, im\'agenes explicativas, etc.) con mayor facilidad y menos esfuerzo que un tutor humano.

Para estimar la eficacia de estos sistemas de tutoring se har\'a una
evaluaci\'on comparativa (i.e., con otros
sistemas de tutoring existentes) o una evaluaci\'on orientada al usuario al ser
utilizado, por ejemplo, por estudiantes de la universidad donde se desarrolla el
proyecto.
El desaf\'io principal de este
estudio es establecer los aspectos del tutor m\'as importantes para un
estudiante de idiomas e identificar el conjunto de m\'etricas relevantes.












  % Objetivos Especificos
\MySubSection{7. Trabajo Previo y Trabajos Relacionados.}

  % Metodo Cientifico
% Se puede clasificar el impacto cient\'ifico
% esperado del proyecto en tres grandes areas, como se detalla a continuaci\'on:

One can classify the expected scientific impact of the project in three main
areas, as detailed below:

% \paragraph{Interacci\'on:}
% Una de las contribuciones m\'as
% importantes del proyecto ser\'a un laboratorio virtual para
% teor\'ias pragm\'aticas que consistir\'a de un
% entorno controlado para el estudio de la interacci\'on situada en un mundo
% donde
% se entremezclan acciones f\'isicas y ling\"u\'isticas.

\paragraph{Interaction:}
One of the major contributions of the project will be a virtual laboratory for
pragmatic theories that consist of a controlled environment for studying the
interaction set in a world where physical actions and language intermingle.

% El prototipo unidireccional permitir\'a
% investigar el impacto de distintas
% pol\'iticas para dar instrucciones (post o
% preorden en el \'arbol de acciones de la tarea)
% sobre la realizaci\'on exitosa de la tarea.
% Estudios de este tipo se han realizado anteriormente
% (e.g.,~\cite{foster-etal-ijcai2009}) pero
% asumiendo una tarea prefijada.
% Dado que nuestro prototipo permitir\'a la especificaci\'on
% del mundo virtual, las posibles acciones y el objetivo
% a alcanzar, podremos determinar cu\'ando el impacto resultante
% del uso de una determinada pol\'itica
% es dependiente de la tarea.

The prototype will allow unidirectional investigate the impact of different
policies to give instructions (post or pre-order on the tree stock of the task)
on the successful completion of the task. Such studies have been performed
previously (e.g.,~\cite{foster-etal-ijcai2009}) but assuming a predetermined
task. Since our prototype allows the specification of the virtual world,
possible actions and the goal to achieve, we can determine when the impact
resulting from the use of a particular policy is dependent on the task.

% El prototipo bidireccional nos permitir\'a investigar el
% fen\'omeno de reparaciones a corto y largo plazo,
% dise\~nando y evaluando un sistema de predicci\'on de reparaciones
% contextualizadas.  Estas reparaciones est\'an usualmente
% causadas por implicaturas conversacionales.  Modelar
% estas implicaturas en un sistema de di\'alogo gen\'erico
% es dif\'icil.  Sin embargo, dado que el presente prototipo provee una
% interacci\'on situada
% y restringida al mundo virtual, ser\'a posible testear la relaci\'on entre
% las implicaturas, el tipo de reparaciones a las que
% dan origen y las tareas de inferencia necesarias para
% predecirlas.

The prototype will allow us to investigate the two-way phenomenon repair the
short and long term, designing and evaluating a prediction system repairs
contextualized. These repairs are usually caused by conversational implicatures.
Modeling these implicatures in a generic dialogue system is difficult. However,
since the present prototype provides a situated interaction and restricted to
the virtual world will be possible to test the relationship between
implicatures, the type of repairs that give rise and inference tasks necessary
to predict them. 

% \paragraph{Inferencia:} La principal contribuci\'on del
% proyecto en el \'area de l\'ogica e inferencia es en el
% dise\~no, desarrollo y estudio de algoritmos de planning.
% Un sistema de planning t\'ipico toma tres inputs -- un
% estado inicial, una especificaci\'on de posibles acciones y
% un objetivo esperado -- y retorna una secuencia de acciones (un plan)
% que al ser aplicadas secuencialmente al estado inicial, termina
% en un estado que satisface el objetivo pedido.  Distintos
% m\'etodos para obtener un plan han sido estudiados (forward chaining, backward
% chaining, codificaci\'on en t\'erminos de satisfiabilidad proposicional,
% etc.); y existen actualmente sistemas
% implementados que pueden resolver esta tarea eficientemente.

\paragraph{Inference:} 
The main contribution of the project in the area of logic and inference is in
the design, development and planning study of algorithms. A typical planning
system takes three inputs - an initial state, a specification of possible
actions and an expected objective - and returns a sequence of actions (a plan)
that when applied sequentially to the initial state, ends in a state that
satisfies the goal order. Different methods to obtain a scheme have been studied
(forward chaining, backward chaining, coding in terms of propositional
satisfiable, etc..) And are currently deployed systems that can solve this task
efficiently.

% Sin embargo, la mayor\'ia de estos sistemas asumen condiciones
% que simplifican el problema (tiempo at\'omico y
% determin\'istico, informaci\'on completa, ausencia de una teor\'ia
% background, etc.) y retornan un \'unico plan.  En el transcurso
% del proyecto se investigar\'an algoritmos que eliminan algunas
% de las simplificaciones mencionadas (en particular, investigaremos
% el caso de planning con informaci\'on incompleta y en base a una
% teor\'ia background) y que ofrecen adem\'as servicios de planning
% extendidos (retorno de planes alternativos, planes m\'inimos, planes
% condicionales, planes incompletos, acciones posibles en un estado dado, etc.)

However, most of these systems assume conditions that simplify the problem
(deterministic atomic time, complete, absence of a background theory, etc.). And
return a single plan. In the course of the project will investigate algorithms
that eliminate some of the simplifications (in particular, investigate the case
of planning with incomplete information and theory based on a background) and
also provide extended services planning (return on alternative plans, plans
minimum conditional plans, incomplete plans, possible actions in a given state,
etc.).

% \paragraph{Evaluaci\'on:}
% En este \'area se espera que una de las contribuciones principales sea la
% integraci\'on de t\'ecnicas de evaluaci\'on de distintas \'areas en una
% metodolog\'ia que permita evaluar sistemas de di\'alogo para entornos
% virtuales
% de manera que se estime su usabilidad y eficacia. Esta metodolog\'ia
% podr\'ia usarse tanto para determinar si un sistema es adecuado para un tipo
% de
% tarea y de usuario, como para comparar la performance de distintos sistemas
% del
% mismo tipo.

\paragraph{Evaluation:}
In this area it is expected that one of the main contributions is the
integration of assessment techniques from different areas on a methodology for
evaluating dialog systems for virtual environments so as to estimate its
usability and effectiveness. This methodology could be used both to determine
whether a system is suitable for a task type and user, and to compare the
performance of different systems of the same type.

% Otra contribuci\'on ser\'a el estudio y aplicaci\'on de est\'andares de
% evaluaci\'on de software a los sistemas desarrollados, generando un modelo de
% la
% calidad estandarizado y proponiendo un conjunto de m\'etricas apropiadas para
% evaluar cada uno de los aspectos contenidos en el modelo. Este trabajo
% incluir\'a tambi\'en un estudio profundo de varias m\'etricas a fines de
% incluir
% en el modelo de calidad algunos consejos sobre las m\'etricas elegidas; este
% estudio se considera como una meta-evaluaci\'on del modelo propuesto.

Another contribution is the study and application of appraisal standards
developed software systems, creating a standardized quality model and proposing
a set of appropriate metrics to assess each of the aspects of the model. This
work will also include a thorough study of several metrics in late in the model
include some tips on quality metrics chosen, this study is considered a
meta-evaluation of the proposed model.

% Finalmente,
% el corpus anotado de interacci\'on humano-humano, m\'as
% los corpus de interacci\'on humano-m\'aquina recopilados
% durante el proyecto se har\'an p\'ublicos.  Este tipo de
% corpora servir\'a, por ejemplo, para dise\~nar plataformas
% m\'as generales de evaluaci\'on de sistemas de di\'alogo,
% que van m\'as all\'a de los aspectos evaluados actualmente
% por plataformas existentes como GIVE.

Finally, the annotated corpus of human-human interaction, over the corpus of
human-machine interaction collected during the project be made public. Such
corpora will serve, for example, to design more general platform for evaluating
dialog systems, which go beyond the aspects evaluated by currently existing
platforms as GIVE.


  % Cronograma de trabajo
Los temas a investigar en el marco de este proyecto son de relevancia en el
panorama Argentino actual por, al menos, dos razones de peso.

Por un lado, el proyecto
integra y desarrolla diferentes aspectos clave del \'area de ling\"u\'istica
computacional (sintaxis, sem\'antica, pragm\'atica, representaci\'on,
inferencia, evaluaci\'on). El \'area de ling\"u\'istica computacional y su
aplicaci\'on al tratamiento autom\'atico del lenguaje natural han tenido un
gran desarrollo internacional en los \'ultimos a\~nos, con aplicaciones como los
sistemas de b\'usqueda en la web, los sistemas de traducci\'on y resumen
autom\'aticos, las interfaces de voz, etc. Sin embargo, el \'area es casi
inexistente actualmente en Argentina. Este proyecto se contar\'a entre 
las contribuciones que apuntan a revertir esta situaci\'on.
Por otro lado, el objetivo \'ultimo de este proyecto es investigar el uso de la plataforma
desarrollada en el \'area de educaci\'on a distancia (concretamente, como
plataforma de aprendizaje de idiomas).

Claramente, la educaci\'on a distancia es un recurso valioso para superar el 
problema de la centralizaci\'on de recursos
educativos en el pa\'is. Sin embargo, desarrollar herramientas adecuadas en
este \'area espec\'ifica es dif\'icil.

Para desarrollar sistemas de ense\~nanza a distancia es esencial modelar el
avance del aprendizaje del usuario. Esto requiere un sistema capaz de ser
consciente de la evoluci\'on del usuario, y que tenga en cuenta sus logros y sus
problemas. Este tipo de interacci\'on entre usuario y sistema puede modelarse
como un di\'alogo, cuyo contexto registra el conocimiento adquirido (del usuario
sobre el material del curso, y del sistema sobre el usuario). La gesti\'on de
este tipo de di\'alogo es particularmente interesante, ya que el sistema debe ser
capaz de interpretar los requerimientos de, y generar respuestas adecuadas para,
usuarios no expertos cuyo conocimiento evoluciona durante la interacci\'on.
Adem\'as, el sistema debe ser capaz de representar apropiadamente tanto la
informaci\'on concerniente al material del curso, como la informaci\'on
concerniente a la evoluci\'on del usuario. Por ejemplo, el sistema debe ser
capaz de diagnosticar qu\'e parte del material del curso debe ser revisada a
partir de las respuestas err\'oneas del usuario.  Por \'ultimo, el sistema debe
poder evaluar la interacci\'on con el usuario, para poder decidir
que objetivos del aprendizaje fueron alcanzados.
Los resultados te\'oricos y pr\'acticos obtenidos durante el proyecto,
contribuyen directamente a la soluci\'on de estos problemas.

  % Trabajo Previo y Trabajos Relacionados
\MySubSection{10. Impacto Cient\'ifico}

Una vez m\'as podemos clasificar el impacto cient\'ifico
esperado del proyecto en tres grandes areas.

\paragraph{Interacci\'on:}
Una de las contribuciones m\'as
importantes del proyecto ser\'a un `laboratorio virtual' para
teor\'ias pragm\'aticas. El proyecto proveer\'a de un
entorno controlado, que permite
el estudio de interacci\'on situada en un mundo donde
se entremezclan acciones f\'isicas y acciones ling\"u\'isticas.

El prototipo unidireccional nos permitir\'a
investigar el impacto que distintas
pol\'iticas para dar instrucciones (e.g., post o
preorden en el \'arbol de acciones de la tarea)
tienen sobre la realizaci\'on exitosa de la tarea.
Evaluaciones de este tipo se han realizado anteriormente
(e.g.,~\citep{foster-etal-ijcai2009}) pero siempre
asumiendo una tarea prefijada.
Dado que nuestro prototipo permite la especificaci\'on
del mundo virtual, las posibles acciones y el objetivo
a alcanzar, podremos evaluar cu\'ando estas pol\'iticas
son dependientes o no de la tarea.
\fixme{Requiere haber definido post y preorden antes.}

El prototipo bidireccional nos permitir\'a investigar el
fen\'omeno de reparaciones a corto y largo plazo,
evaluando un sistema de predicci\'on de reparaciones
contextualizadas.  Estas reparaciones est\'an usualmente
causadas por implicaturas conversacionales.  Modelar
estas implicaturas en un sistema de di\'alogo gen\'erico
es dif\'icil.  Dado que nuestro prototipo provee una interacci\'on situada
y restringida al mundo virtual, podremos testear la relaci\'on entre
estas implicaturas, el tipo de reparaciones a las que
dan origen, y las tareas de inferencia necesarias para
predecirlas.

Finalmente,
el corpus anotado de interacci\'on humano-humano, m\'as
los corpus de interacci\'on humano-m\'aquina recopilados
durante el proyecto se har\'an p\'ublicos.  Este tipo de
corpora servir\'a, por ejemplo, para dise\~nar plataformas
m\'as generales de evaluaci\'on de sistemas de di\'alogo,
que van m\'as all\'a de los aspectos evaluados actualmente
por plataformas existentes como GIVE.

\paragraph{Inferencia:} La principal contribuci\'on del
proyecto en el \'area de l\'ogica e inferencia es en el
dise\~no, desarrollo y estudio de algoritmos de planning.
Un sistema de planning t\'ipico toma tres inputs -- un
estado inicial, una especificaci\'on de posibles acciones y
un objetivo esperado -- y retorna una secuencia de acciones (un plan)
que al ser aplicadas secuencialmente al estado inicial, termina
en un estado que satisface el objetivo pedido.  Distintos
m\'etodos para obtener un plan han sido estudiados (e.g.,
forward chaining, backward chaining, codification as propositional
satisfiability, etc.); y existen actualmente sistemas
implementados que pueden resolver esta tarea eficientemente.

Sin embargo, la mayor\'ia de estos sistemas asumen condiciones
que simplifican el problema (e.g., tiempo at\'omico y
determin\'istico, informaci\'on completa, ausencia de una teor\'ia
background, etc.) y retornan un \'unico plan.  En el transcurso
del proyecto se investigar\'an algoritmos que eliminan algunas
de las simplificaciones mencionadas (en particular, investigaremos
el caso de planning con informaci\'on incompleta y en base a una
teor\'ia background) y que ofrecen adem\'as servicios de planning
extendidos (retorno de planes alternativos, planes m\'inimos, planes
condicionados, planes incompletos, acciones posibles en un estado dado, etc.)

\paragraph{Evaluaci\'on:}
\fixme{escribir dos parrafos como los anteriores}
 % Impacto Cientifico
%\MySubSection{11. Insersi\'on en el Programa de Investigaci\'on}

% El procesamiento de lenguaje natural, y en particular el
% desarrollo de sistemas de di\'alogo, es un \'area en crecimiento
% en los pa\'ises desarrollados, por su gran aplicabilidad en el
% actual contexto de la sociedad de la informaci\'on. Efectivamente,
% dado el gran aumento de la informaci\'on textual en formato
% digital, el procesamiento automatizado de los textos se ha
% convertido en una capacidad estrat\'egica para empresas,
% instituciones y la comunidad en general. Por esta raz\'on
% el \'area de PLN se encuentra en crecimiento en las principales
% universidades y empresas tecnol\'ogicas del mundo, con un
% presupuesto que crece a\~no a a\~no. Sin embargo, 
% este \'area se encuentra muy poco desarrollada en la Argentina. Esto se puede
% atribuir a diferentes factores:

% \begin{itemize}
% \item la relativa juventud del \'area de PLN, lo que implica una relativa
% escasez de profesionales bien formados en todo el mundo,
% \item el desarrollo insuficiente de todo el \'area de investigaci\'on en
% Ciencias de la Computaci\'on, por razones hist\'oricas y demanda de la
% industria,
% \item el poco desarrollo del \'area de Inteligencia Artificial y
% Ling\"u\'istica Formal en la Argentina, tambi\'en por razones hist\'oricas
% y acad\'emicas,
% \item la escasa interacci\'on entre los pocos investigadores en PLN que se
% encuentran en la regi\'on.
% \end{itemize}

Natural language processing, and in particular the development of dialogue
systems is a growth area in developed countries, for their great applicability
in the current context of information society. Indeed, given the large increase
of textual information in digital form, the automated processing of texts has
become a strategic capability for companies, institutions and the wider
community. For this reason the area is growing PLN at major universities and
technology companies in the world with a budget that grows every year. However,
this area is very underdeveloped in Argentina. This can be attributed to several
factors. (a) The relative youth of the area of PLN, which implies a relative
dearth of trained professionals throughout the world. (b) The underdevelopment
of the whole area of research in computer science, for historical reasons and
industry demand. (c) The undeveloped area of Artificial Intelligence and
Formal Linguistics in Argentina, also for historical and academic. (d) Poor
interaction between the few researchers in NLP that are in the region. 

% Parece claro que el PLN es un \'area de investigaci\'on estrat\'egica
% para la Argentina, en la que se puede alcanzar la excelencia acad\'emica
% e industrial a nivel internacional. Creemos que hay que apoyar el
% desarrollo de este \'area favoreciendo los siguientes aspectos:
% 
% \begin{itemize}
% \item formaci\'on de recursos humanos a trav\'es
% de programas de doctorado y de cursos dictados en la Argentina por
% profesionales de
% reconocido prestigio internacional,
% 
% \item incorporaci\'on de recursos humanos formados, para contribuir
% al aumento y diversificaci\'on de la masa cr\'itica en el \'area,
% 
% \item mejora de la interacci\'on entre los diversos grupos o investigadores
% aislados en PLN, a
% trav\'es de la organizaci\'on de workshops, cursos de profesores visitantes,
% co-tutor\'ias,
% programas de especializaci\'on coordinados, etc.
% \end{itemize}

It seems clear that PLN is a strategic research area for Argentina, which can
achieve academic excellence and industry worldwide. We believe in supporting the
development of this area by promoting the following. (a) Training of human
resources through doctoral programs and courses taught in Argentina by
internationally renowned professionals. (b) Incorporation of trained human
resources to contribute to the growth and diversification of the critical mass
in the area. (c) Improving the interaction between various groups or
individual researchers in NLP, through the organization of workshops, courses,
visiting professors, co-tutoring, coordinated specialty programs, etc.

 % Impacto Socio-Economico
% En la FaMAF existe un grupo de PLN desde 2005
% (\url{http://www.cs.famaf.unc.edu.ar/~pln}). Este
% grupo est\'a desarrollando una importante labor de formaci\'on de recursos
% humanos, con el dictado 
% de cursos de grado y de postgrado en la FaMAF y en otras universidades del
% pa\'is.
% Tambi\'en trabaja en el desarrollo de diversos proyectos de investigaci\'on y
% en la integraci\'on
% con otros grupos de la regi\'on, tanto en Argentina como en Chile, Brasil y
% Uruguay. Este proyecto 
% de investigaci\'on se integra al programa del grupo de procesamiento de
% lenguaje % natural de la
% FaMAF.

The PLN\footnote{\url{http://www.cs.famaf.unc.edu.ar/~pln}} research group, 
in which the describe scientific project will be carried out, was funded in 2005.
Te group is developing an important role in human resource training,
delivering courses to undergraduate and postgraduate studies at the FaMAF and
other universities. It also works in the development of various research
projects and integration with other groups in the region, both in Argentina and
Chile, Brazil and Uruguay. 

% El Dr.\ Carlos Areces se especializa en l\'ogica
% computacional,
% en particular en el estudio te\'orico y aplicado de lenguajes para la
% representaci\'on del conocimiento (e.g., l\'ogicas modales, h\'ibridas y
% para la descripci\'on).  Las publicaciones
% m\'as importantes sobre aspectos te\'oricos en estos lenguajes
% son~\cite{ABM01,arec:hybr05b}.  Ha desarrollado tambi\'en demostradores
% autom\'aticos para estos lenguajes~\cite{ANR01,arec:hylo02a,AG06,Hoffmann2007}
% accessibles en \url{http://www.glyc.dc.uba.ar/intohylo/}.
% Ha trabajado tambi\'en en el dise\~no de algoritmos para la generaci\'on de
% expresiones referenciales~\cite{AKS08}, y en el estudio de lenguajes l\'ogicos
% adecuados para la representaci\'on sem\'antica~\cite{AF08}.
 

To begin with, some members of the group specialize in computational
logic, particularly in the theoretical and applied study of languages for
knowledge representation (e.g., modal, hybrid and description logics). They
have also developed  automated theorem provers for these
languages\footnote{\url{http://www.glyc.dc.uba.ar/intohylo/}}. In relation with 
the study of knowledge representation, they have also investigated and developed
algorithms for generating referring expressions~\cite{AKS08}.

% La Dra.\ Paula Estrella ha realizado su tesis doctoral en la 
% evaluaci\'on contextual de sistemas de TA~\cite{estr:impr08,estr:femt09},
% proponiendo 
% un modelo generalizable y aplicable a otras \'areas del PLN, como se muestra
% en 
% \cite{Miller2008}. Tambi\'en ha trabajado en el desarrollo y mejora de
% sistemas de TA 
% estad\'istica desde y hacia varios idiomas, por ejemplo el
% Espa\~nol~\cite{estr:expe05} 
% y entre los idiomas Franc\'es, Ingl\'es, Japon\'es y Arabe
% \cite{rayner-EtAl:2009:GEAF}. 
% Adem\'as particip\'o en proyectos que involucran la creaci\'on y evaluaci\'on
% de % corpora 
% correspondientes a anotaciones multimodales~\cite{pope:estr07} y trabaj\'o en
% un % prototipo 
% para la evaluaci\'on autom\'atica de anotaciones de di\'alogos multimodales
% \cite{multieval}, 
% investigando tambi\'en la aplicaci\'on de m\'etricas provenientes de otras
% \'areas; el 
% prototipo est\'a disponible p\'ublicamente en
% \url{http://www.issco.unige.ch:8080/multieval/}.

The second line of research of the PLN group that is relevant for this project
is context-based evaluation. In particular, we have proposed a evaluation model
for machine translation (MT) systems which relates the context of use of the
system to potentially important quality
characteristics~\cite{estr:impr08,estr:femt09}. This model is general enough to
be applied to the other systems that produce natural language such as the system
proposed in this paper. 
% Given our background with MT systems we have experience in evaluating and
% comparing natural language output produced in different languages (Spanish and
% English in particular), which will be relevant for the development of the
% language tutor described in Section~\ref{applications}. 
Finally, we have
experience developing and evaluating   
 multimodal corpora as is the corpora described in Section~\ref{description}~\cite{multieval}.


% La Lic.\ Luciana Benotti ha trabajado en el desarrollo de un sistema de
% di\'alogo completo situado en una aventura de
% texto~\cite{benotti09b}. La implementaci\'on abarca desde tareas de an\'alisis
% sint\'actico hasta tareas pragm\'aticas. Su investigaci\'on se ha enfocado en
% tareas pragmaticas que requieren de tareas de inferencia practicas. En
% particular ha utilizado la tarea de inferencia de planning clasico y de
% planning con informacion incompleta. Luego extendi\'o el sistema de di\'alogo
% desarrollado por el grupo de David Traum~\cite{TraumSigdialBook08} para
% analizar pragmaticamente oraciones comparativas~\cite{benotti09a}. Finalmente,
% trabaj\'o en el an\'alisis emp\'irico de corpora de di\'alogo humano-humano
% utilizando
% diferentes t\'ecnicas de planning para precedir sub-dialogos de
% reparaci\'on~\cite{benotti09c}.

The third line of research that is relevant for this project is pragmatics. In
this area we have implemented a conversational agent which is able to infer and
negotiate conversational implicatures using inference tasks such as
classical planning and planning under incomplete information~\cite{benotti09b}.
We have also investigated the inference of conversational implicatures triggered
by comparative utterances~\cite{benotti09a}. Recently we have done corpus-based
work, which shows what kinds of implicatures are inferred and negotiated by
human dialogue participants during a task situated in a 3D
virtual environment~\cite{benotti09c}. 

There are other lines of research in the PLN group that are not directly related
to this project at this stage but might become more relevant in the
future: grammar induction, text mining, statistical syntactic analysis
and ontology population from raw text. 





 % Insersion en el Programa de Investigacion
The members of the PLN in general and the authors of this paper in particular
have several collaborations with national and international research groups in 
computational linguistics and related fields that are relevant for this project. 
 
At the international level, we have ongoing collaboration with the TIM/ISSCO\footnote{\url{http://www.issco.unige.ch/en}} \emph{Multilingual Information Processing Department} at the University of Geneva, with the
Idiap Research Institute\footnote{\url{http://www.idiap.ch}} and  with some
members of the PAI\footnote{\url{http://diuf.unifr.ch/pai/wiki}},
\textit{Pervasive Artificial Intelligence} group of the University of
Fribourg.  These collaborations include the evaluation of NLP systems
and the development of multilingual and multimodal human language
technology systems.

Members of the group have a long standing collaboration  with
the TALARIS\footnote{\url{http://talaris.loria.fr}}
group of the \emph{Laboratoire Lorrain de Recherche en Informatique et ses
Applications (LORIA)}. The main research topic at TALARIS is computational
linguistics
with strong emphasis on semantics and inference. In the framework of this
collaboration we are participating in the 2010 edition of the GIVE
Challenge. In the process of designing the systems that will participate in
the challenge we jointly investigated the use of different referring strategies
in situated instruction
giving~\cite{amoia10}. 

We have also collaborated with the Virtual Humans group of the Institute for
Creative Technologies\footnote{\url{http://ict.usc.edu/projects/virtual_humans}}
from the University of Southern California. In particular we computationally
modeled the inference of conversational implicatures triggered by comparative
utterances~\cite{benotti09a}. The Institute for Creative Technologies offers
Internship programs every year that we plan to use in order to strengthen our
collaboration.

All these collaborations are directly related to the main theme of the project 
described in this article.  The PLN group has also research collaborations with 
other international research teams in the framework of other scientific programs. 
For example, the PLN group has being part of a recently finished international project 
MICROBIO\footnote{\url{http://www.microbioamsud.net}} on ontology population from raw text.
The project was funded by the Stic-Amsud\footnote{\url{http://www.sticamsud.org}} program, 
a scientific-technological cooperation program integrated by France, Argentine, Brazil, 
Chile, Paraguay, Peru and Uruguay. The expertise obtained during this project might
be useful in the future when trying to extend our GIVE ontologies to new domains. 
Similarly, the team maintain scientific relations with the University of Texas at Austin
(mainly with Dr.\ J. Moore in projects related to the development of the ACL2\footnote{\url{http://www.cs.utexas.edu/users/moore/acl2}} prover); and with the 
Research team Symbiose\footnote{\url{http://www.irisa.fr/symbiose}} of the Institut de
Recherche en Informatique et Syst\'emes Al\'eatoires (working on the use of linguistic
techniques for the modelisation of genomic sequences).

At the national level, the group has intensively collaborated with GLyC\footnote{\url{http://www.glyc.dc.uba.ar}}, \emph{Grupo de L\'ogica,
Lenguaje y Computabilidad} on knowledge representation and inference 
(see, e.g.~\cite{AG06,AFFM08}). GLyC is part
of the Computer Science Department of the Universidad de Buenos Aires. 
During 2010, teams PLN and GLyC will join forces and collaborate in
the organization of ELiC\footnote{\url{http://www.glyc.dc.uba.ar/elic2010}}, 
the \emph{First School in Computational Linguistics} in Argentina,  which will
take place in July at the Universidad de Buenos Aires.  ELiC 2010 will be co-located 
with the ECI\footnote{\url{http://www.dc.uba.ar/events/eci/2009/eci2009}}, Escuela de Ciencias Inform\'aticas which has a
long standing reputation as a high-quality winter school in Computer Science in
Argentina, and is being organized yearly since 1987.
With ELiC we aim at creating, for the first time, a space to introduce the field of 
computational linguistics to graduate students in Argentina.  Thanks to the 
support of the North American Chapter of the Association for
Computational Linguistics (NAACL) and of the Universidad de Buenos Aires, 
ELiC is offering student travel grants and fee waivers to encourage participation.

The PLN group is also contacting other groups working in computational linguistics
in Argentina like the research group in Artificial Intelligence from the 
Universidad Nacional del Comahue\footnote{\url{http://www.uncoma.edu.ar/}}. Taking 
advantage of previous co-participation in different project we plan to organize
exchange programs in the framework of a research network. 

Finally, the PLN group is planning to organize a workshop on Computational
Linguistics as a satellite event of IBERAMIA 2010\footnote{\url{http://cs.uns.edu.ar/iberamia2010}}, 
the Ibero-American Conference on Artificial Intelligence, that will be organized by the Universidad
del Sur, in the city of Bah\'ia Blanca, Argentina. 
 % Diseminacion
\MySubSection{6. Descripci\'on Detallada del Plan de Investigaci\'on Propuesto}

\vspace*{.2cm}%

\MySubSubSection{6.1. Definici\'on del Problema.}


\MySubSubSection{6.2. Resultados Espec\'ificos Esperados.}
Los resultados esperados pueden clasificarse usando los
tres temas principales que articulan el proyecto.


Al comienzo del proyecto recopilaremos
un corpus de interacci\'on humano-humano (uni y bidireccional).  El
corpus contendr\'a la interacci\'on lig\"u\'istica alineada con
las acciones realizadas en el mundo virtual.  Este corpus
ser\'a utilizado para guiar el dise\~no de las pol\'iticas
de interacci\'on a implementar.


\fixme{cosas comentadas en el source aca abajo que pueden usarse}
% XXX
%
% Si se conceptualiza
% la estructura de la tarea como un \'arbol en el cual
% la ra\'iz
%
% (e.g., s\'olo pr\'oxima
% instrucci\'on a realizar, objetivo general a alcanzar m\'as
% proxima instrucci\'on, etc.).
%
%
%
% La teor\'ia
% de relevancia ~\citep{} distingue t
%
% (premisas implicadas, conclusiones implicadas
% y explicaturas)
%
%
% Dicho sistema de predicci\'on
% requerir\'a el uso de distintas tareas de inferencia.
% El contenido de reparaciones coherentes
% dentro de una interacci\'on tiene restricciones
% pragm\'aticas. Podemos clasificar las reparaciones
% coherentes en premisas implicadas, conclusiones implicadas,
% y explicaturas. Para predecir premisas implicadas se requiere
% planning con informaci\'on incompleta y planes alternativos,
% para predecir las
% conclusiones implicadas se requieren acciones posibles en
% un estado dado,
%
%
% Mapping entre tareas pragm\'aticas/cognitivas y servicios
% de inferencia / representaci\'on de conocimientos del sistema.



\paragraph{Interacci\'on:}

\emph{Generaci\'on de expresiones referenciales}. Un problema importante en
s\'intesis de lenguaje natural es el de la generaci\'on de expresiones
referenciales. Esto quiere decir, producir una frase nominal que describa
un\'ivocamente a un objeto (e.g. ``el jarr\'on que est\'a sobre la mesa'').
Estas expresiones, para ser aceptables, deben ser similares a las que podr\'ia
producir una persona en condiciones normales (no ser\'ia aceptable, en lugar del
ejemplo anterior, ``el jarr\'on que no est\'a arriba de la silla ni arriba del
sof\'a ni abajo de la mesa'').  En~\cite{AKS08} se propone utilizar la
minimizaci\'on simb\'olica del modelo que representa
el estado del mundo para as\'i obtener f\'ormulas l\'ogicas que describan
un\'ivocamente a cada objeto. En particular se observa que con un
lenguaje l\'ogico sin negaci\'on, no se pueden obtener resultados poco
naturales como el anteriormente expuesto. Esta \'area de aplicaci\'on se
investigar\'a en conjunto con el grupo TALARIS (INRIA,
Nancy, Francia).

\paragraph{Inferencia:}
\citep{arec:logi00}.


\paragraph{Evaluaci\'on:}


\MySubSubSection{6.3. M\'etodo Scient\'ifico a Utilizar.}
\fixme{Mencionar Corpus}

Describimos a continuaci\'on el m\'etodo cient\'ifico a utilizar
respecto a los aspectos te\'oricos, pr\'acticos y emp\'iricos
que conciernen al proyecto.

\paragraph{Aspectos Te\'oricos.}
La metodolog\'a de trabajo a utilizar para investigar los aspectos
te\'oricos del proyecto es el estandard en investigaci\'on de base.
Comenzaremos con el estudio de la bibliograf\'ia existente, para luego
familiarizarnos con las t\'ecnicas de demostraci\'on espec\'ificas de esta
\'area y poder con ellas demostrar los resultados deseados. El grupo
responsable esta formado por expertos en las tres \'areas principales
que el proyecto planea investigar: Benotti es experta en el \'area de
pragm\'atica de la interacci\'on y sistemas de di\'alogos; Areces es
experto en m\'etodos de inferencia; y Estrella es experta en t\'ecnicas
de evaluci\'on de sistemas de procesamiento natural.
Los tres han desarrollado extensa investigaci\'on previa en sus respectivas
\'areas
de experiencia.

\paragraph{Aspectos Pr\'acticos.}
Una vez se haya avanzado con las tareas de investigaci\'on te\'orica,
estas dar\'an origen al dise\~no de t\'ecnicas y algoritmos que se
implementar\'an en el framework de GIVE. Para ello se utilizar\'a la
metodolog\'ia
usual en ingenier\'ia del software (desarrollo por
m\'odulos con interfaces claras y bien documentadas,
que garanticen alta cohesi\'on y bajo acoplamiento).
Al finalizar la implementaci\'on
se espera tambi\'en tener un documento al estilo de manual de
desarrollador, en donde queden expl\'icitas las principales interfaces
de los m\'odulos desarrollados y la forma de interacci\'on con la
herramienta.


\paragraph{An\'alisis Emp\'irico.}
Una vez
implementados los algoritmos, se proceder\'a a realizar tests
completos de unidad y de integraci\'on.

\fixme{mas sobre evaluacion y testing aca.}

\MySubSubSection{6.4. Cronograma de trabajo}

El cronograma de trabajo se estructura de la siguiente manera:

\paragraph{A\~no 1.} El objetivo de los primeros meses es obtener un
prototimo del sistema unidireccional.  Comenzaremos con un relevamiento
de la bibliograf\'ia.  Dado que el grupo de investigaci\'on est\'a
constitu\'io de expertos en las distintas \'areas pertinentes al
proyecto, nos concentraremos durante los primeros meses en obtener un
entendimiento com\'un de los distintos aspectos del problema, donde
cada experto contribuir\'a bibliograf\'ia adecuada de su \'area para los
dem\'as miembros del grupo.  A continuaci\'on comenzaremos a estudiar
la plataforma GIVE, y a adaptar sus componentes para nuestro objetivo
espec\'ifico.  Como parte del proyecto, planeamos participar del GIVE
challenge, lo que nos impondr\'a deadlines concretos para obtener un
primer prototipo.  La siguiente tarea es la definici\'on  y el dise\~no,
por un lado, de los par\'ametros de interacci\'on que queremos incorporar
en el sistema, y por otro, los esquemas de representaci\'on de la
informaci\'on que el sistema necesita, y las tareas de inferencia a
utilizar.

\fixme{Agregar algo concreto aca sobre `parametros de interacci\'on'}

Respecto de la representaci\'on del conocimiento, en un primer momento
utilizaremos sistemas de inferencia \emph{of-the-shelf} como
FaCT++~\citep{horr:fact99},
RACER~\citep{haar:race99} o Pellet~\citep{XXX}, que soportan tareas de
definici\'on, mantenimiento y consulta de ontolog\'ias.  Estos sistemas
han sido utilizados como motores de inferencia
en numerosas applicaciones en el \'area, y en applicaciones
dise\~nadas por miembros del equipo de
investigaci\'on~(e.g., \citep{FROLOG}).  Una vez observado el comportamiento de
estos motores de inferencia en la tarea, se analizar\'an sus limitaciones
e investigar\'an extensiones en el segundo a\~no del proyecto.  La tarea
de inferencia que ser\'a investigada en m\'as detalle durante el primer
a\~no ser\'a la tarea de planning.  GIVE provee un sistema de plannig muy
limitado.  Existen, sistemas de planning m\'as avanzandos (en particular,
que permiten el manejo de infirmaci\'on incompleta sobre el dominio)
como PKS \url{http://homepages.inf.ed.ac.uk/rpetrick/research/pks/}, pero
est\'an todav\'ia en periodo de desarrollo.  En general, el estado del
arte en el \'area de planning no cubre los requerimientos de nuestro
sistema.  Si bien, existen sistemas optimizados que funcionan adecuadamente
en ciertas aplicaciones, ninguno provee servicios como la generaci\'on de
planes alternativos, o la generaci\'on de planes incompletos en caso de
absencia de plan.

Una vez que obtengamos una version del sistema con su
correspondiente testeo y documentaci\'on, podemos comenzar con
su evaluaci\'on.  Debe notarse de todas formas que es preciso tener en cuenta
el tipo de evaluaci\'on a realizar (que por lo tanto debe estar definido
adecuadamente) durante el periodo de desarrollo, para asegurar que el sistema
es capaz de prover la informaci\'on necesaria requerida durante la evaluaci\'on
(e.g., logueo de eventos, etc.).

Al final del a\~no comenzar\'a la preparaci\'on de articulos y reportes para
la presentaci\'on de los resultados obtenidos hasta el momento.

\paragraph{A\~no 2.} El objetivo principal del segundo a\~no del proyecto es,
por un lado, extender y completar el prototipo de sistema con informaci\'on
unidireccional en base a la evaluaci\'on realizada al fin del a\~no anterior,
y utilizando el feedback obtenido de la
participaci\'on en el GIVE challenge.  Por otro lado, se comenzar\'a a agregar
capacidades de interacci\'on ling\"u\'istica bidireccional.  Un sistema
bidireccional es mucho m\'as complejo que un sistema unidireccional.  Por
empezar, deberemos integrar al sistema un m\'odulo de interpretaci\'on de
lenguaje natural (e.g., parser, construcci\'on sem\'antica, etc.).  Nos
concentraremos en dos tipos espec\'ificos de instrucciones que el usuario
podr\'a dar al sistema: a) pedidos de aclaraci\'on de la \'ultima
instrucci\'on dada, y b) redefinici\'on del goal de la interacci\'on.  Estos
dos tipos de instrucci\'on son representativos de `reparaciones' a corto y
largo plazo.  Cada uno de ellos requerir\'a la redefinici\'on adecuada de
las reglas pragm\'aticas que gobiernan la interacci\'on (e.g., XXX), \fixme{resolver
tareas pragmaticas bidireccionales} y de servicios
de infer\'encia a medida (e.g., interpretaci\'on de expresiones referenciales,
redefinici\'on del plan en ejecuci\'on, etc.).  Una vez m\'as, a medida que
estas nuevas capacidades sean agregadas al sistema, deberemos llevar a
cabo testeo y documentaci\'on, para finalmente dar paso a la evaluaci\'on.



\paragraph{A\~no 3.} El objetivo del \'ultimo a\~no es transformar el prototipo
de sistema de di\'alogo bidireccional obtenido durante el a\~no anterior, en un
tutor virtual para el aprendizaje de idiomas.  Adem\'as del beneficio claro de
obtener un sistema para una aplicaci\'on concreta que pueda ser utilizado
m\'as alla del \'ambito acad\'emico, XXX \fixme{terminar.}

\medskip

\noindent
A continuaci\'on se sumarizan las tareas a desarrollar en los tres a\~nos de
trabajo (organizadas trimestralmente):

{\footnotesize
\begin{center}
\begin{tabular}{|p{7cm}||p{2mm}|p{2mm}|p{2mm}|p{2mm}||p{2mm}|p{2mm}|p{2mm}|p{2mm
}||p{2mm}|p{2mm}|p{2mm}|p{2mm}||}
\hline
 \rowcolor[rgb]{0.8,0.8,0.8}\hspace{3.5cm}Tarea & 1 & 2 & 3 & 4 & 1 & 2 & 3 & 4
& 1 & 2 & 3 & 4\\
\hline 1. Relevamiento bibliogr\'afico
& $\times$ & $\times$ &&&&&&&$\times$&&&\\
\hline 2. Estudio del material bibliogr\'afico
& $\times$ & $\times$ & $\times$ &  &&&&&$\times$&&&\\
\hline 3. Estudio de la plataforma GIVE
& & $\times$ & &&&&&&&&&\\
\hline 4. Dise\~no e impl.\ de alg.\ de interacci\'on (unidireccional)
& & & $\times$ & $\times$&&&$\times$&$\times$&&&&\\
\hline 5. Dise\~no e impl.\ de alg.\ de interacci\'on (bidireccional)
& & &  & &&&$\times$&$\times$&&&$\times$&$\times$\\
\hline 6. Dise\~no e impl.\ de alg.\ de inferencia
& & & $\times$ & $\times$&&&$\times$&$\times$&&&$\times$&$\times$\\
\hline 7. Testing
&&&&$\times$&&&&$\times$&&&&$\times$\\
\hline 8. Documentaci\'on
&&&&$\times$&&&&$\times$&&&&$\times$\\
\hline 9. Evaluaci\'on
&&&$\times$&$\times$&$\times$&&$\times$&$\times$&$\times$&&$\times$&$\times$\\
\hline 10. Desarrollo de tutor virtual
&&&&&&&&&&$\times$&$\times$&$\times$\\
\hline 11. Elaborac.\ y presentaci\'on de resultados te\'oricos
&&&&$\times$&$\times$&$\times$&&$\times$&$\times$&$\times$&&$\times$\\
\hline 12. Elaborac.\ y presentaci\'on de resultados aplicados
&&&&$\times$&$\times$&$\times$&&$\times$&$\times$&$\times$&&$\times$\\\hline
\end{tabular}\end{center}
}

%Se comienza con el relevamiento bibliogr\'afico (1) y estudio del
%material (2). A continuaci\'on se desarrollan las investigaciones
%te\'oricas en c\'alculo de complejidades y desarrollo de algoritmos para
% extensiones de la l\'ogica modal b\'asica, en particular la l\'ogica
% h\'ibrida (3 y 4) y para l\'ogicas modales sub-booleanas (5 y 6). Hacia los
% finales del primer a\~no (y una vez obtenidos avances te\'oricos sustanciales)
% se empieza paralelamente con la implementaci\'on de los algoritmos (7). Una vez
% que esta tarea termina, se procede con la etapa del testing (8). Se dedicar\'an
% tres meses a la documentaci\'on del software (9). El final del proyecto est\'a
% reservado para la elaboraci\'on y presentaci\'on final de los resultados
% te\'oricos (10) y aplicados (11).

  % Composicion del Grupo de Investigacion y Areas de Especializacion



% % % % % % % % % % % % % % % % % % % % % % % % % % % % % % % % % % % % % %
% \MySubSection{8a.  Material Requerido} For day-to-day
% work standard machine equipment will be used.  Distributed
% computational experiments will be carried out on computing facilities
% available at SARA. Implementations will mostly be done in Standard ML
% of New Jersey and Allegro Common Lisp (for which licenses are
% available), while  dedicated Perl scripts will be used for
% data manipulation and preprocessing.


% % % % % % % % % % % % % % % % % % % % % % % % % % % % % % % % % % % % % %
%% header automatically generated from bbl file

% \citep{byron09}
% \citep{blaylock05}
% \citep{lochbaum98}
% \citep{litman90}
% \citep{purver06}
% \citep{fikes72}
% \citep{gerevini05}
% \citep{kautz99}
% \citep{hoffmann01}
% \citep{hsu06}
% \citep{allwood95}
% \citep{skantze07}
% \citep{stoia07}
% \citep{gabsdil03}
% \citep{stoia08}
% \citep{rieser05}
% \citep{schegloff87b}
% \citep{Grice75}
% \citep{rodriguez04}
%
% \citep{clark96}
% \citep{ginzburg09}
% \citep{levinson87}
%
% \citep{BBW06}
% \citep{KR99}
% \citep{KR97}
% \citep{PT87}
% \citep{S08}
%
% \citep{citeulike:386317}

%\paragraph{BLA BLA}

%La Universidad tiene un importante rol que desempe\~nar en este nuevo espacio digital, multicultural y pluriling\"u\'istico llamado "Cibercultura", promoviendo una verdadera interacci\'on entre:

%   	El gobierno   	Las empresas   Las Universidades y Centros de Investigaci\'on y Desarrollo
%De esta manera se contribuye a la formaci\'on de capital intelectual capaz de comprender los procesos de la nueva sociedad, desde un enfoque genuinamente interdisciplinario.
%Las nuevas tecnolog\'ias ofrecen al presente y futuro de la Argentina, un sinf\'in de posibilidades, por el hecho de atravesar todos los sectores, pol\'iticos, econ\'omicos y sociales, siendo su ``desarrollo estrat\'egico'' un pilar fundamental para crecimiento equitativo y sustentable del pa\'is.

%\url{http://www.uai.edu.ar/ciiti/2009/bsas/congreso.html}

\bibliographystyle{plainnat}
\bibliography{pict09}


\end{document}
