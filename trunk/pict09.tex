\documentclass[11pt]{article}

\usepackage{times,url,latexsym,amssymb,tabularx,a4wide,color}
\usepackage{colortbl}
\usepackage[draft,silent]{fixme}

%Por algun motivo babel spanish interfiere con natbib.
%Notar que los ~ aparecen ahora en el texto (en vez
%de actuar como non-breaking space
%\usepackage[spanish]{babel}

\hyphenation{es-pe-ra-dos}
\usepackage{natbib}

%% Define a new 'leo' style for the package that will use a smaller font.
\makeatletter
\def\url@leostyle{%
  \@ifundefined{selectfont}{\def\UrlFont{\sf}}{\def\UrlFont{\small\sffamily}}}

\def\url@tinyleostyle{%
  \@ifundefined{selectfont}{\def\UrlFont{\sf}}{\def\UrlFont{\scriptsize\sffamily}}}
\makeatother

%% Now actually use the newly defined style.
\urlstyle{leo}

\newcommand{\fnturl}[1]{\urlstyle{tinyleo}\url{#1}\urlstyle{leo}}

\renewcommand{\refname}{\normalfont\large\sffamily\textbf{12. Bibliograf\'ia}}

\addtolength{\topmargin}{-1cm}
% \oddsidemargin 0.1 in
% \evensidemargin 0.15 in
% \marginparwidth 1 in
% \oddsidemargin 0.125 in
% \evensidemargin 0.125 in
% \marginparwidth 0.75 in
% \textwidth 6.125 in
%
\addtolength{\textheight}{1.5cm}
% \addtolength{\textheight}{2cm}
% \addtolength{\voffset}{-1cm}
% % \addtolength{\textwidth}{2cm}
% \addtolength{\textwidth}{.4cm}
% % \addtolength{\hoffset}{-1cm}
% \addtolength{\hoffset}{-.4cm}

\makeatother

\pagestyle{plain}

\newcommand{\MySubSection}[1]{\vspace*{-.1\baselineskip}%
\subsection*{\sffamily\textbf
#1}\vspace*{-.2\baselineskip}}

\newcommand{\MySubSubSection}[1]{\vspace*{-.1\baselineskip}%
\subsubsection*{\sffamily\textbf
#1}\vspace*{-.2\baselineskip}}

\newcommand{\MyParagraph}[1]{\vspace*{-.35\baselineskip}\paragraph*{{\sffamily\textbf
#1}}}
\newcommand{\MySubParagraph}[1]{\vspace*{-.35\baselineskip}\subparagraph*{{\sffamily\textit{#1}}}}

\newenvironment{mylist}{%
  \renewcommand{\labelitemi}{\mbox{\tiny
$\blacksquare$}}%
  \renewcommand{\labelitemii}{$\bullet$}%
  \addtolength{\topsep}{-3\parskip}%
  \begin{itemize}\setlength{\itemsep}{-1.8pt}}%
  {\end{itemize}}

\newenvironment{myitemize}{%
  \renewcommand{\labelitemi}{\mbox{\tiny
$\blacksquare$}}%
  \renewcommand{\labelitemii}{$\bullet$}%
  \addtolength{\topsep}{-3\parskip}%
  \begin{itemize}\setlength{\itemsep}{-1.8pt}}%
  {\end{itemize}}

\newcommand{\nextchunk}{\vspace*{.65\baselineskip}\noindent}

\begin{document}
%\bibliographystyle{alpha}
\thispagestyle{plain}

\section*{\sffamily\textbf{Convocatioria PICT-2009}}
\mbox{}

\vspace*{-.5\baselineskip}
\MySubSection{1a. T\'itulo del Proyecto:
{\rm \emph{\Large Sistemas de Di\'alogo para Entornos Virtuales.}}}

\MySubSection{1b. Grupo Responsable}

\hspace*{.5cm} Dr.\ Carlos Areces (Investigador Responsable)

Dra.\ Paula Estrella (Investigadora Integrante)

Lic.\ Luciana Benotti (Investigadora Integrante)\footnote{La Lic.\ Benotti
obtendr\'a su t\'itulo de Dra.\ en Ciencias de la Computacion en Enero/2010.}

\MySubSection{1c. Grupo Colaborador}

\hspace*{.5cm}
Dra.\ Laura Alonso Alemani

Dr.\ Gabriel Infante L\'opez

Lic.\ Franco Luque

\MySubSection{2.  Palabras Clave}

Sistemas de Di\'alogo -- Representaci\'on del Conocimiento -- Planning
-- Evaluaci\'on


% En este proyecto se propone implementar un sistema de di\'alogo que genere
% autom\'aticamente
% instrucciones en lenguaje natural para ayudar a un usuario a cumplir una tarea
% determinada en un entorno virtual 
% en tres dimensiones. Para este fin, se propone investigar temas fundamentales
% sobre la
% in\-teracci\'on hombre-m\'aquina 
% en entornos virtuales. Como resultado de este proyecto, se espera la
% implementaci\'on, evaluaci\'on y aplicaci\'on de un sistema prototipo. Las
% tres
% \'areas principales en las que se pueden clasificar los resultados del
% proyecto
% son: (1) pragm\'atica de la interacci\'on, (2) representaci\'on
% de la informaci\'on e inferencia, (3) evaluaci\'on de sistemas
% de di\'alogo. Una vez obtenido
% un prototipo, se planea su aplicaci\'on a la tarea espec\'ifica del 
% aprendizaje de idiomas, utilizando el sistema como un ``profesor de idiomas"
% virtual.

This project aims to implement a dialogue system to automatically generate
natural language instructions to help a user to fulfill a particular task in a
3D virtual environment. For this purpose, we investigate fundamental issues
about human-computer interaction in virtual environments. As a result of this
project a prototype system will be implemented and evaluated. We can
classify the results of the project in three areas: (1) pragmatics of
interaction, (2) information representation and inference, (3) evaluation of
dialogue systems. Once the working prototype is finished it will be applied to
the specific task of language learning, using the system as a virtual language
teacher. 

% En una primera etapa investigaremos un modelo de interacci\'on 
% ling\"u\'istica unidireccional
% (i.e., el flujo de informaci\'on ling\"u\'istica ser\'a desde el sistema hacia
% el usuario).
% En etapas subsiguientes, el modelo ser\'a extendido para
% permitir intercambio ling\"u\'istico bidireccional, por ejemplo, para que el
% usuario pida clarificaciones o ayuda al sistema.

In a first stage we will investigate a model of unidirectional linguistic
interaction (i.e., linguistic information flows only from the system to the
user). In subsequent stages, the model will be extended to allow bidirectional
language exchange, for example, the user may ask for clarification or help to
the system.

% Dise\~nar la arquitectura de un sistema de di\'alogo presenta
% desaf\'ios tanto te\'oricos como pr\'acticos. En lo te\'orico, se necesitan
% heur\'isticas que gobiernen la interacci\'on en cuanto a qu\'e decir,
% cu\'ando, y c\'omo (teniendo en cuenta el contexto actual). Adem\'as, se debe
% contar con m\'etodos
% de inferencia que permitan al sistema adaptarse en funci\'on de la situaci\'on
% actual, para alcanzar un objetivo predefinido.
% La complejidad del problema te\'orico se
% refleja, en lo pr\'actico, en un sistema de m\'ultiples componentes: un
% generador de lenguaje natural, un sistema de planning, un entorno de 
% interacci\'on en tres dimensiones, etc.
% Dise\~nar e implementar todos estos componentes requerir\'ia un esfuerzo
% prohibitivo. En este proyecto adaptaremos herramientas ya
% implementadas y disponibles libremente para el prototipado de sistemas de este
% tipo, como la plataforma \emph{Generating Instructions in Virtual
% Environments} (GIVE).

Designing the architecture of a dialogue system presents both theoretical and
practical challenges. On the theoretical side, heuristics are needed to govern
the interaction in terms of what to say, when, and how (given the current
context). In addition, there must be inference methods that allow the system to
adapt to the current situation, to reach a predefined goal. The complexity of
the theoretical issues is reflected, in practice, in a system of
multiple components: a natural language generator, a planner,
a 3D interactive environment, to mention a few. Designing
and implementing all these components require a prohibitive effort. This project
will adapt tools already implemented and freely available for prototyping such
systems, such as the platform \emph{Generating Instructions in Virtual
Environments (GIVE)}\cite{byron09}. 

% La calidad de cada una de las capacidades del sistema afecta la
% percepci\'on que el usuario tiene de \'este. Por lo tanto, es imperativo
% evaluar cada una de ellas.
% Se planean adaptar y aplicar distintas t\'ecnicas y m\'etricas de evaluaci\'on
% del \'area de ingenier\'ia de software y de diversas \'areas del procesamiento
% de lenguaje natural (por ejemplo, la traducci\'on autom\'atica).

The quality of each of the capabilities of the system affects the perception
that users have of it. Therefore, it is imperative to evaluate each of such
capabilities. We plan to adapt and apply different evaluation techniques and
metrics from the area of software engineering and various areas of
natural language processing (e.g. machine translation).
  % Resumen
\MySubSection{4. Objetivos Generales}

El objetivo \'ultimo de este proyecto es dise\~nar un sistema de di\'alogo hombre-computadora con soporte para interacci\'on en tres dimensiones (3D) que generare autom\'aticamente instrucciones en lenguaje natural para guiar  al usuario durante la ejecuci\'on de una tarea determinada. Esta propuesta apunta
a obtener un balance entre un sistema de uso general, aplicable en
diferentes \'ambitos, y un sitema lo suficientemente espec\'ifico
como para permitir el uso efectivo de las t\'ecnicas actuales de
administraci\'on del conocimiento, planneamiento y procesamiento de lenguaje natural (PLN).

El sistema resultante podr\'ia utilizarse en distintas situaciones, a modo de ejemplo:
en comercio electr\'onico  al usarse como avatar de asistencia en la web, en soporte t\'ecnico al proveer ayuda al usuario no experto, en
control de dispositivos por voz al usarse en sistemas de di\'alogo embebidos,
en la ense\~nanza a distancia al constituir tutores de lenguas extranjeras.  Una vez obtenido
un prototipo, se planea aplicarlo  como
tutor virtual para el aprendizaje de idiomas; para estimar la eficacia del prototipo  en esta tarea, se podr\'ia realizar una evaluaci\'on comparativa (compar\'andolo con otros sistemas existentes) o una evaluaci\'on orientada al usuario al ser utilizado, por ejemplo, estudiantes de FaMAF.


A fin de obtener un sistema como el descripto anteriormente, este proyecto propone estudiar las tres \'areas fundamentales detalladas a continuaci\'on, las cuales posibilitan el desarrollo de este tipo de sistemas:
\begin{myitemize}
  \item \emph{Interacci\'on:} La interacci\'on humano-humano est\'a gobernada
  por numerosas reglas pragm\'aticas que definen nociones b\'asicas como
  las obligaciones conversacionales de los participantes (qui\'en puede hablar en cada momento), o las implicaciones de un enunciado (por ejemplo, si se
  hizo una pregunta se espera una respuesta).
  Estas reglas se aplican tambi\'en a la interacci\'on humano-computadora en
  entornos virtuales cuando el objetivo del sistema de di\'alogo es
  simular, en la medida de lo posible, el comportamiento humano.

  Modelar formalmente estas reglas pragm\'aticas para un sistema de di\'alogo es uno de los desaf\'ios que condicionan la efectividad del mismo.

  \item \emph{Inferencia:} En t\'erminos generales, podemos entender como
  inferencia toda operaci\'on que transforme informaci\'on \textit{impl\'icita} en
  \textit{expl\'icita}.  Bajo esta amplia definici\'on, es posible considerar como
  inferencia tanto la cl\'asica (por caso
  tareas habituales en inteligencia artificial como la planificaci\'on) u operaciones estad\'isticas (por ejemplo para obtener estimadores sobre un conjunto de datos).  Un sistema de di\'alogo realiza continuamente operaciones de inferencia, por un lado   para interpretar la informaci\'on recibida e incorporarla a su repositorio de datos, y por otro, para decidir qu\'e parte de la informaci\'on disponible transmitir.

  El problema mismo de decidir qu\'e tipo de representaci\'on l\'ogica y que tipo de inferencia utilizar en una determinada situaci\'on (e.g., l\'ogica proposicional vs.\ l\'ogica de primer orden, validez vs.\ chequeo de modelos) es complejo, y las tareas de inferencia en s\'i son computacionalmente costosas.  El desaf\'io en este caso es encontrar el compromiso adecuado entre representaci\'on de la informaci\'on y el m\'etodo de inferencia a utilizar.

\item \emph{Evaluaci\'on:} La evaluaci\'on de sistemas de generaci\'on de lenguaje natural es una de las m\'as dif\'iciles dentro del \'area del Procesamiento de Lenguaje Natural (PLN) dado que una idea puede expresarse de muchas formas, todas ellas correctas, y, en general, determinar la calidad de tales frases no puede hacerse de manera simple y directa, por ejemplo comparando el resultado del sistema con un patr\'on (en Ingl\'es ``gold stadard"). El problema de la falta de patrones es compartido con otra \'area del PLN, la Traducci\'on Autom\'atica (TA), en la cual se han propuesto diversas metodolog\'ias para la evaluaci\'on de sistemas de forma directa (es decir, aplicando alguna m\'etrica al texto generado por un sistema) o indirecta (o sea, evaluando la perfomance del sistema a trav\'es de la utilizaci\'on del texto generado para realizaci\'on de alguna tarea). Sin embargo, en ninguna de estas \'areas existe una metodolog\'ia aceptada como est\'andar y demostrada eficaz de manera general.
En esta propuesta, dado que el objeto a evaluar es un sistema que interact\'ua via la generaci\'on de instrucciones en lenguaje natural, es necesario determinar su performance por medio de evaluaciones cuantitativas (como el tiempo de finalizaci\'on de la tarea), cualitativas (por ejemplo, la calidad de las interacciones) y basadas en el contexto (evaluaciones orientadas al usuario y sus necesidades en una situaci\'on particular). Dada la experiencia de los participantes es de especial inter\'es estudiar la  portabilidad y aplicaci\'on de t\'ecnicas de evaluaci\'on del dominio de la TA y la interacci\'on multimodal humano-computadora al sistema (en su totalidad y por componentes) planteado en este proyecto.
\end{myitemize}

El presente proyecto se focalizar\'a en el  m\'odulo del sistema encargado de generar las instrucciones, el cual deber\'a ser capaz de interactuar naturalmente con el usuario y adaptarse a posibles errores de interpretaci\'on o ejecuci\'on de las instrucciones, dando instrucciones correctivas en consecuencia. Adem\'as, el usuario se encontrar\'a en un universo en el que podr\'a interactuar con objetos y explorar el ambiente virtual provisto. As\'i, el usuario intentar\'a seguir las instrucciones provistas por el sistema,
realizando acciones f\'isicas en un mundo 3D.


En una primera etapa, el modelo a utilizar permitir\'a el flujo de informaci\'on ling\"u\'istica de manera unidireccional, es decir,
desde el sistema hacia el usuario, por lo que
el usuario no podr\'a pedir ningu\'n tipo de ayuda.  La restricci\'on a un
modelo unidireccional tiene como objetivo
simplificar la representaci\'on y manejo del contexto de la interacci\'on.

En etapas m\'as avanzadas del proyecto, el modelo unidireccional ser\'a extendido para
permitir intercambio ling\"u\'istico bidireccional: por ejemplo, el usuario podr\'a
pedir clarificaciones o redefinir el objetivo a alcanzar, en ambos casos usando lenguaje natural.

La generaci\'on de instrucciones para un dominio espec\'ifico requiere la adquisici\'on del conocimiento relevante y la compilaci\'on del mismo debe incluir cierta informaci\'on  ling\"u\'istica que permita su posterior procesamiento: en los aspectos morfol\'ogico y sint\'actico se requiere una gr\'amatica de lenguaje seleccionado (Espa\~nol en este caso), en el aspecto sem\'antico se necesita un repositorio de informaci\'on l\'exica organizada en forma ontol\'ogica y en lo pragm\'atico es necesaria una descripci\'on formal de las
acciones posibles en un universo dado (incluyendo precondiciones, efectos e informaci\'on contextual del estado de cada interacci\'on).

Esta informaci\'on relacionada al conocimiento de un dominio es utilizada por distintos componentes del sistema de di\'alogo que
gestionar\'an la interacci\'on con el usuario. En un sistema t\'ipico, esto sucede como se detalla a continuaci\'on: en la direcci\'on sistema $\rightarrow$ usuario, el componente encargado de \textit{planificar} genera una secuencia de acciones relevantes al momento de la interacci\'on, luego el componente encargado de  \textit{administrar el conocimiento} actualiza la informaci\'on contextual y,  finalmente, el componente encargado de
\textit{generar instrucciones}  transmite las mismas como expresiones
del lenguaje natural.  En la direcci\'on opuesta, es decir usuario $\rightarrow$ sistema,  el componente denominado \textit{discretizador}
transforma el flujo continuo de informaci\'on derivada del comportamiento del usuario, en acciones relevantes al universo en cuesti\'on, mientras que
los componentes de planeamiento y administraci\'on del conocimiento se ocupan
de mantener el contexto actualizado y de detectar posibles errores.


El dise\~no de una tal arquitectura presenta
desaf\'ios tanto te\'oricos como pr\'acticos.  En lo te\'orico, la necesidad de
 guiar la interacci\'on con el usuario (qu\'e decir,
cu\'ando y c\'omo, en un contexto dado) motiva la creaci\'on, adaptaci\'on o mejora de heur\'isticas pertinentes y, por otro lado,  la necesidad de describir la situaci\'on actual as\'i como tambi\'en  de indicar c\'omo alterarla para alcanzar un objetivo pre-establecido, incentivan el estudio de diversos m\'etodos
para la inferencia eficiente y precisa.
La calidad de cada una de estas capacidades afecta la percepci\'on que
el usuario tiene del sistema, y por lo tanto, es imperativo aplicar m\'etricas pertinentes que permitan evaluar cada uno de estos aspectos del sistema.  La complejidad del problema te\'orico se
refleja, en lo pr\'actico, en el desarrollo de un sistema de m\'ultiples componentes: un
generador de lenguaje natural, un componente de planeamiento, un entorno 3D, etc.

Dise\~nar e implementar todos estos componentes requerir\'ia un esfuerzo prohibitivo, por lo que proponemos utilizar recursos disponibles de forma libre y gratuita. En particular, se propone el uso de la plataforma desarrollada com parte de la competencia \textit{Generating Instructions in Virtual Environments} (GIVE\footnote{Disponible libremente en  \fnturl{http://www.give-challenge.org}}) que
provee un prototipo b\'asico de este tipo de sistemas. En este contexto ``competencia" (del Ingl\'es ``evaluation campaign" o ``evaluation challenge") se refiere a la organizaci\'on de un evento en el que participan distintos grupos de investigaci\'on, realizando todos el mismo ejercicio con sus sistemas propios y comparando al final los resultados obtenidos por cada grupo. Este tipo de eventos es muy popular en distinas \'areas del PLN (TA, reconocimiento del habla, recuperaci\'on de informac\'on, entre otras) y benefician ampliamente cada \'area generando y compartiendo con la comunidad de investigadores nuevas tecnolog\'ias.

El objetivo principal de GIVE es actuar como medio de evaluaci\'on de sistemas de generaci\'on
de lenguaje natural, donde \'estos se eval\'uan de forma m\'as precisa y homog\'enea dado que todos los sistemas participantes utilizan los mismos recursos provistos por los organizadores. Adem\'as, desde un punto de vista general, la evaluaci\'on
tambi\'en tiene en cuenta el problema b\'asico de la interacci\'on en
entornos virtuales.  En el escenario propuesto por GIVE, el usuario
humano lleva acabo una tarea de `b\'usqueda del tesoro' en un entorno
virtual 3D y el trabajo del sistema participante es proveer, en tiempo
real, instrucciones en lenguaje natural que ayuden al usuario
a encontrar el tesoro escondido.

\fixme{GIVE tiene el apoyo de SIGSEM SIGDIAL SIGGEN grupos de interes de
ACL, la Asociacion de Ling\"u\'istica Computacional.}

Por su dise\~no, el prototipo GIVE es una herramienta muy propicia para investigar distintos aspectos de la interacci\'on situada en un entorno virtual:
el problema de realizaci\'on sint\'actica y morfo\'ogica de las
instrucciones (en Ingl\'es \emph{surface realization}),
la representaci\'on del conocimiento existente en el contexto de
interacci\'on (contexto del discurso, ontolog\'ias, acciones posibles),
el tipo de inferencia requerido por las tareas involucradas
(planeamiento, construcci\'on y chequeo de modelos),
las reglas pragm\'aticas de interacci\'on requeridas por la tarea
(administraci\'on de la carga cognitiva, actualizaci\'on y uso de la informaci\'on contextual), entre otras. Adicionalmente, GIVE provee herramientas para registrar todos los detalles de la
interacci\'on, permitiendo de esta forma obtener f\'acilmente un corpus de interacci\'on en entornos virtuales anotado autom\'aticamente. \fixme{agregar la importancia de tales corpora}


%Finalmente, la calidad de la interacci\'on con un
%sistema de di\'alogo dado puede medirse en esta tarea en t\'erminos
%cuantitativos (por ejemplo el tiempo m\'aximo para completar la tarea)
%y cualitativos (como la relevancia de una instrucci\'on dada).

\fixme{Terminar con un comentario sobre tutoring!!!.}

  % Objetivos Generales
As we mentioned above, we will use the GIVE platform as the basic architecture 
for our dialogue system.  In the scenario proposed by GIVE, a
human user carries out a ``treasure hunt'' in a 3D virtual environment
and the task of the generation system is to provide real-time, natural
language instructions that help the user find the hidden treasure.

For the correct definition of the interaction policies of our prototype we need
a corpus that provides examples of typical interactions in the domain. GIVE
provides tools for collecting such corpus in the form of a Wizard of Oz
platform that record all details of the interaction, thus
allowing to easily obtain a corpus of interaction in virtual environments
annotated automatically.

From the collected corpus we will begin the design, implementation and testing
of our dialogue system.  The main components that we will have to design and 
implement can be organized using the traditional four tasks that a dialogue 
system should address: (1) content planning, (2)
generation of referring expressions, (3) management of the interaction context, and
(4) interpretation of user responses. 

\emph{(1) Content Planning:} Given the envisioned setup we described in Section~\ref{intro},
the first task of the system is to obtain a plan to reach the desired goal, from the current state.
The plan will contain physical actions to be performed in the virtual environment. The second
step is to decide how to transmit this sequence of actions to the user. E.g, to decide
how many actions to communicate per instruction, and how to aggregate them
coherently. The result of the action aggregation process can be represented as a
tree describing the task structure at different levels of abstraction. The third
and final step is to decide how to navigate the tree of actions to verbalize the
instructions (for example, post or preorder as
explored in~\cite{foster-etal-ijcai2009}). We will investigate
different aggregation policies (e.g., aggregating actions that
manipulate similar objects) and innovative ways in which to navigate the task tree
(e.g., moving to a lower level of abstraction in case of misunderstandings).
Plan computation can be solved using classical planners~\cite{nau04}.
However, while there are planners that work well when optimized for certain
applications, none provides services such as the generation of alternative
plans, or the generation of incomplete plans in case of the absence of plan.
One of the goals of the project is to design and implement these extensions to 
classical planning algorithms. We will also study the theoretical behavior (e.g., complexity) of
these new algorithms. 

\emph{(2) Generation of Referring Expressions:} Once content planning is
complete, the next step is to generation adequate referring expressions. 
This task involves producing a phrase that describes a referable entity so that the user can
identify it (e.g., ``the vase on the table''). To be
acceptable, these expressions should be ``natural:'' they should be at the same time
sufficiently but not overly constrained, and they should not impose on the user a heavier 
cognitive load than necessary. For example, producing the expression 
``the vase that is not above the chair or sofa or under the
table'' would probably not be acceptable. Areces et al.~\shortcite{AKS08} propose to
use symbolic minimization of the model that represents the state of the world, in
order to obtain logical formulas that describe each object uniquely. In our
project we will implement this method and evaluate it within the dialogue system.

\emph{(3) Management of the Interaction Context:} To manage the use of
the interaction context we will use existing knowledge maintenance systems such as
RACER\footnote{\url{http://www.racer-systems.com}} or Pellet\footnote{\url{http://clarkparsia.com/pellet}}, which support inference tasks such as
definition, maintenance and querying of ontologies. These systems have been used
as inference engines in numerous applications in
the area and, in particular, in dialogue systems for text adventures~\cite{benotti09b}. Once we
have studied the behavior of these inference engines on the task, we will
analyze its limitations and investigate the required extensions.

\emph{(4) Interpretation of User Responses:} The interpretation of user
responses in the unidirectional system is relatively simple: it amounts to
discretizing the continuous flow of user behavior in the 3D world into actions
meaningful for the domain task. In a first
stage, we will use the discretizer provided by GIVE. After evaluating it we can
determine whether or not this module meets the requirements of
our task and what are its limitations. In the bidirectional system, however,
the interpretation of user responses is the task that will require more
attention.
To start with, the bidirectional system should be expanded with capabilities
for  processing statements coming from the user (namely, parsing, semantic
construction, resolution of references, etc.). We will study, in particular, two
types of user contributions: requests for clarification of the instruction
given (what we call `short-term repairs'), and for redefinition of goals (what we
call `long-term repairs'). We will implement short-term repairs
using the approach described in~\cite{purver06}. For long-term repairs we will use the
guidelines of~\cite{blaylock05a}. 

\subsection{Evaluation}
To determine the quality of the obtained prototypes we propose to create a
quality model following the ISO/IEC
9126 and 14528 standards for the evaluation of software
products~\cite{ISO9126-1,ISO14598-1}. These standards were successfully applied
to
the Machine Translation domain, resulting in the
\emph{FEMTI\footnote{\url{http://www.issco.unige.ch/femti/}}, Framework for the
Evaluation of
Machine Translation}~\cite{Est2005}. FEMTI
guides evaluators towards creating parameterized evaluation
plans that include various aspects of the to-be-evaluated system and offer a
relevant set of metrics. The identification of relevant metrics can be performed
using various methods, e.g., based on previous
experience~\cite{paradise06,Litman2002}, conducting
surveys or requirement specifications~\cite{Lecoeuche98}, or
collecting such data through Wizard of Oz
experiments~\cite{Dahlback93}.
After developing a quality model, several methodologies to assess
various aspects of the system can be applied: automatic metrics,
subjective metrics or metrics based on the task (to
evaluate both the contribution of each component and the quality of the whole
system). 

The GIVE platform is used every year as a unified framework for evaluating
generation systems. Systems have to generate natural language
instructions and be able to participate in a real-time interaction situated in a
3D environment. The GIVE Challenge is one of the shared tasks endorsed by
ACL's special interests groups in generation, dialogue and semantics. We plan
to participate in the challenge, which will serve as an additional
source of information about aspects of the system that need
improvement.
Once the prototype is evaluated and improved using the results
of this evaluation, we will investigate its use as a virtual language tutor as
described in the next section.

\subsection{An Application: A Virtual Tutor}\label{applications}

The project outcome will be a system capable of giving natural language
instructions situated in a virtual 3D environment. The technology and
theoretical advances of the project could be used in various applications, but
one of the most interesting characteristics we plan to investigate is that, 
a priori, by just changing the linguistic resources, the language of interaction
with the system (input and output) can be changed as desired. After obtaining a
first prototype of an instruction giving dialogue system, 
 we will investigate its use for distance learning,
adapting the system to operate as a foreign language tutor~\cite{Wik09}.

A one-way system that generates instructions in English can be used to test the
user understanding of a foreign language. The correct interpretation of the
instructions can be evaluated from the proper execution of the instructions. The
two-way system will allow the user to formulate clarifications (either in their
native language or in the foreign language). The user may also redefine the
objective to be achieved during the interaction, and thus select the type of
vocabulary he wants to practice.

Virtual worlds (like Second Life) are being rapidly incorporated into
education, both initial and superior~\cite{Doswell05,molk:lear09}. The use of a
virtual tutor has certain advantages over a human tutor.
Engwall~\shortcite{engwall1020} mentioned the following. (1) Amount of
practice: the chance to practice the new language is essential for learning, and
a virtual tutor provides opportunities only limited by the
technological resources. (2) Prestige: a student
may feel embarrassed about making mistakes with a human tutor, and this
might limit his willingness to speak in the foreign language. (3) Augmented
Reality: a virtual
tutor can provide additional material (e.g., examples in context, explanatory
images, etc.) with less effort than a human tutor.


Such a virtual tutor can be used in distance learning. To develop distance learning systems, it is essential to model the user's learning
progress. This requires a system aware of the evolution of the
user, and that takes into account their achievements and their problems. This type
of interaction between user and system can be modeled as a dialogue, which
records the acquired knowledge context (of the user about the course material, and
of the system about to user). The system must be able to interpret requirements, and generate
appropriate responses, for non-experts uses whose knowledge evolves during the
interaction. Moreover, the system must be able to properly represent both the
information concerning the course material, and information about the
evolution of the user. For example, the system must be able to diagnose what
part of the course material should be reviewed from the wrong answers of the
user. Finally, the system must be able to evaluate the user interaction in order
to decide which learning objectives have been achieved. The theoretical and practical
results of the project contribute to solving these difficult
problems.








  % Composicion del Grupo de Investigacion y Areas de Especializacion
\MySubSection{6. Descripci\'on Detallada del Plan de Investigaci\'on Propuesto}

\vspace*{.2cm}%

\MySubSubSection{6.1. Definici\'on del Problema.}


\MySubSubSection{6.2. Resultados Espec\'ificos Esperados.}
Los resultados esperados pueden clasificarse usando los
tres temas principales que articulan el proyecto.


Al comienzo del proyecto recopilaremos
un corpus de interacci\'on humano-humano (uni y bidireccional).  El
corpus contendr\'a la interacci\'on lig\"u\'istica alineada con
las acciones realizadas en el mundo virtual.  Este corpus
ser\'a utilizado para guiar el dise\~no de las pol\'iticas
de interacci\'on a implementar.


\fixme{cosas comentadas en el source aca abajo que pueden usarse}
% XXX
%
% Si se conceptualiza
% la estructura de la tarea como un \'arbol en el cual
% la ra\'iz
%
% (e.g., s\'olo pr\'oxima
% instrucci\'on a realizar, objetivo general a alcanzar m\'as
% proxima instrucci\'on, etc.).
%
%
%
% La teor\'ia
% de relevancia ~\citep{} distingue t
%
% (premisas implicadas, conclusiones implicadas
% y explicaturas)
%
%
% Dicho sistema de predicci\'on
% requerir\'a el uso de distintas tareas de inferencia.
% El contenido de reparaciones coherentes
% dentro de una interacci\'on tiene restricciones
% pragm\'aticas. Podemos clasificar las reparaciones
% coherentes en premisas implicadas, conclusiones implicadas,
% y explicaturas. Para predecir premisas implicadas se requiere
% planning con informaci\'on incompleta y planes alternativos,
% para predecir las
% conclusiones implicadas se requieren acciones posibles en
% un estado dado,
%
%
% Mapping entre tareas pragm\'aticas/cognitivas y servicios
% de inferencia / representaci\'on de conocimientos del sistema.



\paragraph{Interacci\'on:}

\emph{Generaci\'on de expresiones referenciales}. Un problema importante en
s\'intesis de lenguaje natural es el de la generaci\'on de expresiones
referenciales. Esto quiere decir, producir una frase nominal que describa
un\'ivocamente a un objeto (e.g. ``el jarr\'on que est\'a sobre la mesa'').
Estas expresiones, para ser aceptables, deben ser similares a las que podr\'ia
producir una persona en condiciones normales (no ser\'ia aceptable, en lugar del
ejemplo anterior, ``el jarr\'on que no est\'a arriba de la silla ni arriba del
sof\'a ni abajo de la mesa'').  En~\cite{AKS08} se propone utilizar la
minimizaci\'on simb\'olica del modelo que representa
el estado del mundo para as\'i obtener f\'ormulas l\'ogicas que describan
un\'ivocamente a cada objeto. En particular se observa que con un
lenguaje l\'ogico sin negaci\'on, no se pueden obtener resultados poco
naturales como el anteriormente expuesto. Esta \'area de aplicaci\'on se
investigar\'a en conjunto con el grupo TALARIS (INRIA,
Nancy, Francia).

\paragraph{Inferencia:}
\citep{arec:logi00}.


\paragraph{Evaluaci\'on:}


\MySubSubSection{6.3. M\'etodo Scient\'ifico a Utilizar.}
\fixme{Mencionar Corpus}

Describimos a continuaci\'on el m\'etodo cient\'ifico a utilizar
respecto a los aspectos te\'oricos, pr\'acticos y emp\'iricos
que conciernen al proyecto.

\paragraph{Aspectos Te\'oricos.}
La metodolog\'a de trabajo a utilizar para investigar los aspectos
te\'oricos del proyecto es el estandard en investigaci\'on de base.
Comenzaremos con el estudio de la bibliograf\'ia existente, para luego
familiarizarnos con las t\'ecnicas de demostraci\'on espec\'ificas de esta
\'area y poder con ellas demostrar los resultados deseados. El grupo
responsable esta formado por expertos en las tres \'areas principales
que el proyecto planea investigar: Benotti es experta en el \'area de
pragm\'atica de la interacci\'on y sistemas de di\'alogos; Areces es
experto en m\'etodos de inferencia; y Estrella es experta en t\'ecnicas
de evaluci\'on de sistemas de procesamiento natural.
Los tres han desarrollado extensa investigaci\'on previa en sus respectivas
\'areas
de experiencia.

\paragraph{Aspectos Pr\'acticos.}
Una vez se haya avanzado con las tareas de investigaci\'on te\'orica,
estas dar\'an origen al dise\~no de t\'ecnicas y algoritmos que se
implementar\'an en el framework de GIVE. Para ello se utilizar\'a la
metodolog\'ia
usual en ingenier\'ia del software (desarrollo por
m\'odulos con interfaces claras y bien documentadas,
que garanticen alta cohesi\'on y bajo acoplamiento).
Al finalizar la implementaci\'on
se espera tambi\'en tener un documento al estilo de manual de
desarrollador, en donde queden expl\'icitas las principales interfaces
de los m\'odulos desarrollados y la forma de interacci\'on con la
herramienta.


\paragraph{An\'alisis Emp\'irico.}
Una vez
implementados los algoritmos, se proceder\'a a realizar tests
completos de unidad y de integraci\'on.

\fixme{mas sobre evaluacion y testing aca.}

\MySubSubSection{6.4. Cronograma de trabajo}

El cronograma de trabajo se estructura de la siguiente manera:

\paragraph{A\~no 1.} El objetivo de los primeros meses es obtener un
prototimo del sistema unidireccional.  Comenzaremos con un relevamiento
de la bibliograf\'ia.  Dado que el grupo de investigaci\'on est\'a
constitu\'io de expertos en las distintas \'areas pertinentes al
proyecto, nos concentraremos durante los primeros meses en obtener un
entendimiento com\'un de los distintos aspectos del problema, donde
cada experto contribuir\'a bibliograf\'ia adecuada de su \'area para los
dem\'as miembros del grupo.  A continuaci\'on comenzaremos a estudiar
la plataforma GIVE, y a adaptar sus componentes para nuestro objetivo
espec\'ifico.  Como parte del proyecto, planeamos participar del GIVE
challenge, lo que nos impondr\'a deadlines concretos para obtener un
primer prototipo.  La siguiente tarea es la definici\'on  y el dise\~no,
por un lado, de los par\'ametros de interacci\'on que queremos incorporar
en el sistema, y por otro, los esquemas de representaci\'on de la
informaci\'on que el sistema necesita, y las tareas de inferencia a
utilizar.

\fixme{Agregar algo concreto aca sobre `parametros de interacci\'on'}

Respecto de la representaci\'on del conocimiento, en un primer momento
utilizaremos sistemas de inferencia \emph{of-the-shelf} como
FaCT++~\citep{horr:fact99},
RACER~\citep{haar:race99} o Pellet~\citep{XXX}, que soportan tareas de
definici\'on, mantenimiento y consulta de ontolog\'ias.  Estos sistemas
han sido utilizados como motores de inferencia
en numerosas applicaciones en el \'area, y en applicaciones
dise\~nadas por miembros del equipo de
investigaci\'on~(e.g., \citep{FROLOG}).  Una vez observado el comportamiento de
estos motores de inferencia en la tarea, se analizar\'an sus limitaciones
e investigar\'an extensiones en el segundo a\~no del proyecto.  La tarea
de inferencia que ser\'a investigada en m\'as detalle durante el primer
a\~no ser\'a la tarea de planning.  GIVE provee un sistema de plannig muy
limitado.  Existen, sistemas de planning m\'as avanzandos (en particular,
que permiten el manejo de infirmaci\'on incompleta sobre el dominio)
como PKS \url{http://homepages.inf.ed.ac.uk/rpetrick/research/pks/}, pero
est\'an todav\'ia en periodo de desarrollo.  En general, el estado del
arte en el \'area de planning no cubre los requerimientos de nuestro
sistema.  Si bien, existen sistemas optimizados que funcionan adecuadamente
en ciertas aplicaciones, ninguno provee servicios como la generaci\'on de
planes alternativos, o la generaci\'on de planes incompletos en caso de
absencia de plan.

Una vez que obtengamos una version del sistema con su
correspondiente testeo y documentaci\'on, podemos comenzar con
su evaluaci\'on.  Debe notarse de todas formas que es preciso tener en cuenta
el tipo de evaluaci\'on a realizar (que por lo tanto debe estar definido
adecuadamente) durante el periodo de desarrollo, para asegurar que el sistema
es capaz de prover la informaci\'on necesaria requerida durante la evaluaci\'on
(e.g., logueo de eventos, etc.).

Al final del a\~no comenzar\'a la preparaci\'on de articulos y reportes para
la presentaci\'on de los resultados obtenidos hasta el momento.

\paragraph{A\~no 2.} El objetivo principal del segundo a\~no del proyecto es,
por un lado, extender y completar el prototipo de sistema con informaci\'on
unidireccional en base a la evaluaci\'on realizada al fin del a\~no anterior,
y utilizando el feedback obtenido de la
participaci\'on en el GIVE challenge.  Por otro lado, se comenzar\'a a agregar
capacidades de interacci\'on ling\"u\'istica bidireccional.  Un sistema
bidireccional es mucho m\'as complejo que un sistema unidireccional.  Por
empezar, deberemos integrar al sistema un m\'odulo de interpretaci\'on de
lenguaje natural (e.g., parser, construcci\'on sem\'antica, etc.).  Nos
concentraremos en dos tipos espec\'ificos de instrucciones que el usuario
podr\'a dar al sistema: a) pedidos de aclaraci\'on de la \'ultima
instrucci\'on dada, y b) redefinici\'on del goal de la interacci\'on.  Estos
dos tipos de instrucci\'on son representativos de `reparaciones' a corto y
largo plazo.  Cada uno de ellos requerir\'a la redefinici\'on adecuada de
las reglas pragm\'aticas que gobiernan la interacci\'on (e.g., XXX), \fixme{resolver
tareas pragmaticas bidireccionales} y de servicios
de infer\'encia a medida (e.g., interpretaci\'on de expresiones referenciales,
redefinici\'on del plan en ejecuci\'on, etc.).  Una vez m\'as, a medida que
estas nuevas capacidades sean agregadas al sistema, deberemos llevar a
cabo testeo y documentaci\'on, para finalmente dar paso a la evaluaci\'on.



\paragraph{A\~no 3.} El objetivo del \'ultimo a\~no es transformar el prototipo
de sistema de di\'alogo bidireccional obtenido durante el a\~no anterior, en un
tutor virtual para el aprendizaje de idiomas.  Adem\'as del beneficio claro de
obtener un sistema para una aplicaci\'on concreta que pueda ser utilizado
m\'as alla del \'ambito acad\'emico, XXX \fixme{terminar.}

\medskip

\noindent
A continuaci\'on se sumarizan las tareas a desarrollar en los tres a\~nos de
trabajo (organizadas trimestralmente):

{\footnotesize
\begin{center}
\begin{tabular}{|p{7cm}||p{2mm}|p{2mm}|p{2mm}|p{2mm}||p{2mm}|p{2mm}|p{2mm}|p{2mm
}||p{2mm}|p{2mm}|p{2mm}|p{2mm}||}
\hline
 \rowcolor[rgb]{0.8,0.8,0.8}\hspace{3.5cm}Tarea & 1 & 2 & 3 & 4 & 1 & 2 & 3 & 4
& 1 & 2 & 3 & 4\\
\hline 1. Relevamiento bibliogr\'afico
& $\times$ & $\times$ &&&&&&&$\times$&&&\\
\hline 2. Estudio del material bibliogr\'afico
& $\times$ & $\times$ & $\times$ &  &&&&&$\times$&&&\\
\hline 3. Estudio de la plataforma GIVE
& & $\times$ & &&&&&&&&&\\
\hline 4. Dise\~no e impl.\ de alg.\ de interacci\'on (unidireccional)
& & & $\times$ & $\times$&&&$\times$&$\times$&&&&\\
\hline 5. Dise\~no e impl.\ de alg.\ de interacci\'on (bidireccional)
& & &  & &&&$\times$&$\times$&&&$\times$&$\times$\\
\hline 6. Dise\~no e impl.\ de alg.\ de inferencia
& & & $\times$ & $\times$&&&$\times$&$\times$&&&$\times$&$\times$\\
\hline 7. Testing
&&&&$\times$&&&&$\times$&&&&$\times$\\
\hline 8. Documentaci\'on
&&&&$\times$&&&&$\times$&&&&$\times$\\
\hline 9. Evaluaci\'on
&&&$\times$&$\times$&$\times$&&$\times$&$\times$&$\times$&&$\times$&$\times$\\
\hline 10. Desarrollo de tutor virtual
&&&&&&&&&&$\times$&$\times$&$\times$\\
\hline 11. Elaborac.\ y presentaci\'on de resultados te\'oricos
&&&&$\times$&$\times$&$\times$&&$\times$&$\times$&$\times$&&$\times$\\
\hline 12. Elaborac.\ y presentaci\'on de resultados aplicados
&&&&$\times$&$\times$&$\times$&&$\times$&$\times$&$\times$&&$\times$\\\hline
\end{tabular}\end{center}
}

%Se comienza con el relevamiento bibliogr\'afico (1) y estudio del
%material (2). A continuaci\'on se desarrollan las investigaciones
%te\'oricas en c\'alculo de complejidades y desarrollo de algoritmos para
% extensiones de la l\'ogica modal b\'asica, en particular la l\'ogica
% h\'ibrida (3 y 4) y para l\'ogicas modales sub-booleanas (5 y 6). Hacia los
% finales del primer a\~no (y una vez obtenidos avances te\'oricos sustanciales)
% se empieza paralelamente con la implementaci\'on de los algoritmos (7). Una vez
% que esta tarea termina, se procede con la etapa del testing (8). Se dedicar\'an
% tres meses a la documentaci\'on del software (9). El final del proyecto est\'a
% reservado para la elaboraci\'on y presentaci\'on final de los resultados
% te\'oricos (10) y aplicados (11).

  % Descripcion Detallada del Plan de Investigacion Propuesto
\MySubSection{7. Trabajo Previo y Trabajos Relacionados.}



\fixme{Que podemos poner como `Trabajo Relacionado'?}  % Trabajo Previo y Trabajos Relacionados
\MySubSection{8. Cronograma de trabajo}

El cronograma de trabajo se estructura de la siguiente manera:

\paragraph{A\~no 1.} El objetivo de los primeros meses es obtener un
prototimo del sistema unidireccional.  Comenzaremos con un relevamiento
de la bibliograf\'ia.  Dado que el grupo de investigaci\'on est\'a
constitu\'io de expertos en las distintas \'areas pertinentes al
proyecto, nos concentraremos durante los primeros meses en obtener un
entendimiento com\'un de los distintos aspectos del problema, donde
cada experto contribuir\'a bibliograf\'ia adecuada de su \'area para los
dem\'as miembros del grupo.  A continuaci\'on comenzaremos a estudiar
la plataforma GIVE, y a adaptar sus componentes para nuestro objetivo
espec\'ifico.  Como parte del proyecto, planeamos participar del GIVE
challenge, lo que nos impondr\'a deadlines concretos para obtener un
primer prototipo.  La siguiente tarea es la definici\'on  y el dise\~no,
por un lado, de los par\'ametros de interacci\'on que queremos incorporar
en el sistema, y por otro, los esquemas de representaci\'on de la
informaci\'on que el sistema necesita, y las tareas de inferencia a
utilizar.

\fixme{Agregar algo concreto aca sobre `parametros de interacci\'on'}

Respecto de la representaci\'on del conocimiento, en un primer momento
utilizaremos sistemas de inferencia \emph{of-the-shelf} como
FaCT++~\citep{horr:fact99},
RACER~\citep{haar:race99} o Pellet~\citep{XXX}, que soportan tareas de
definici\'on, mantenimiento y consulta de ontolog\'ias.  Estos sistemas
han sido utilizados como motores de inferencia
en numerosas applicaciones en el \'area, y en applicaciones
dise\~nadas por miembros del equipo de
investigaci\'on~(e.g., \citep{FROLOG}).  Una vez observado el comportamiento de
estos motores de inferencia en la tarea, se analizar\'an sus limitaciones
e investigar\'an extensiones en el segundo a\~no del proyecto.  La tarea
de inferencia que ser\'a investigada en m\'as detalle durante el primer
a\~no ser\'a la tarea de planning.  GIVE provee un sistema de plannig muy
limitado.  Existen, sistemas de planning m\'as avanzandos (en particular,
que permiten el manejo de infirmaci\'on incompleta sobre el dominio)
como PKS \url{http://homepages.inf.ed.ac.uk/rpetrick/research/pks/}, pero
est\'an todav\'ia en periodo de desarrollo.  En general, el estado del
arte en el \'area de planning no cubre los requerimientos de nuestro
sistema.  Si bien, existen sistemas optimizados que funcionan adecuadamente
en ciertas aplicaciones, ninguno provee servicios como la generaci\'on de
planes alternativos, o la generaci\'on de planes incompletos en caso de
absencia de plan.

Una vez que obtengamos una version del sistema con su
correspondiente testeo y documentaci\'on, podemos comenzar con
su evaluaci\'on.  Debe notarse de todas formas que es preciso tener en cuenta
el tipo de evaluaci\'on a realizar (que por lo tanto debe estar definido
adecuadamente) durante el periodo de desarrollo, para asegurar que el sistema
es capaz de prover la informaci\'on necesaria requerida durante la evaluaci\'on
(e.g., logueo de eventos, etc.).

Al final del a\~no comenzar\'a la preparaci\'on de articulos y reportes para
la presentaci\'on de los resultados obtenidos hasta el momento.

\paragraph{A\~no 2.} El objetivo principal del segundo a\~no del proyecto es,
por un lado, extender y completar el prototipo de sistema con informaci\'on
unidireccional en base a la evaluaci\'on realizada al fin del a\~no anterior,
y utilizando el feedback obtenido de la
participaci\'on en el GIVE challenge.  Por otro lado, se comenzar\'a a agregar
capacidades de interacci\'on ling\"u\'istica bidireccional.  Un sistema
bidireccional es mucho m\'as complejo que un sistema unidireccional.  Por
empezar, deberemos integrar al sistema un m\'odulo de interpretaci\'on de
lenguaje natural (e.g., parser, construcci\'on sem\'antica, etc.).  Nos
concentraremos en dos tipos espec\'ificos de instrucciones que el usuario
podr\'a dar al sistema: a) pedidos de aclaraci\'on de la \'ultima
instrucci\'on dada, y b) redefinici\'on del goal de la interacci\'on.  Estos
dos tipos de instrucci\'on son representativos de `reparaciones' a corto y
largo plazo.  Cada uno de ellos requerir\'a la redefinici\'on adecuada de
las reglas pragm\'aticas que gobiernan la interacci\'on (e.g., XXX), \fixme{resolver
tareas pragmaticas bidireccionales} y de servicios
de infer\'encia a medida (e.g., interpretaci\'on de expresiones referenciales,
redefinici\'on del plan en ejecuci\'on, etc.).  Una vez m\'as, a medida que
estas nuevas capacidades sean agregadas al sistema, deberemos llevar a
cabo testeo y documentaci\'on, para finalmente dar paso a la evaluaci\'on.



\paragraph{A\~no 3.} El objetivo del \'ultimo a\~no es transformar el prototipo
de sistema de di\'alogo bidireccional obtenido durante el a\~no anterior, en un
tutor virtual para el aprendizaje de idiomas.  Adem\'as del beneficio claro de
obtener un sistema para una aplicaci\'on concreta que pueda ser utilizado
m\'as alla del \'ambito acad\'emico, XXX \fixme{terminar.}

\medskip

\noindent
A continuaci\'on se sumarizan las tareas a desarrollar en los tres a\~nos de
trabajo (organizadas trimestralmente):

{\footnotesize
\begin{center}
\begin{tabular}{|p{7cm}||p{2mm}|p{2mm}|p{2mm}|p{2mm}||p{2mm}|p{2mm}|p{2mm}|p{2mm
}||p{2mm}|p{2mm}|p{2mm}|p{2mm}||}
\hline
 \rowcolor[rgb]{0.8,0.8,0.8}\hspace{3.5cm}Tarea & 1 & 2 & 3 & 4 & 1 & 2 & 3 & 4
& 1 & 2 & 3 & 4\\
\hline 1. Relevamiento bibliogr\'afico
& $\times$ & $\times$ &&&&&&&$\times$&&&\\
\hline 2. Estudio del material bibliogr\'afico
& $\times$ & $\times$ & $\times$ &  &&&&&$\times$&&&\\
\hline 3. Estudio de la plataforma GIVE
& & $\times$ & &&&&&&&&&\\
\hline 4. Dise\~no e impl.\ de alg.\ de interacci\'on (unidireccional)
& & & $\times$ & $\times$&&&$\times$&$\times$&&&&\\
\hline 5. Dise\~no e impl.\ de alg.\ de interacci\'on (bidireccional)
& & &  & &&&$\times$&$\times$&&&$\times$&$\times$\\
\hline 6. Dise\~no e impl.\ de alg.\ de inferencia
& & & $\times$ & $\times$&&&$\times$&$\times$&&&$\times$&$\times$\\
\hline 7. Testing
&&&&$\times$&&&&$\times$&&&&$\times$\\
\hline 8. Documentaci\'on
&&&&$\times$&&&&$\times$&&&&$\times$\\
\hline 9. Evaluaci\'on
&&&$\times$&$\times$&$\times$&&$\times$&$\times$&$\times$&&$\times$&$\times$\\
\hline 10. Desarrollo de tutor virtual
&&&&&&&&&&$\times$&$\times$&$\times$\\
\hline 11. Elaborac.\ y presentaci\'on de resultados te\'oricos
&&&&$\times$&$\times$&$\times$&&$\times$&$\times$&$\times$&&$\times$\\
\hline 12. Elaborac.\ y presentaci\'on de resultados aplicados
&&&&$\times$&$\times$&$\times$&&$\times$&$\times$&$\times$&&$\times$\\\hline
\end{tabular}\end{center}
}

%Se comienza con el relevamiento bibliogr\'afico (1) y estudio del
%material (2). A continuaci\'on se desarrollan las investigaciones
%te\'oricas en c\'alculo de complejidades y desarrollo de algoritmos para
% extensiones de la l\'ogica modal b\'asica, en particular la l\'ogica
% h\'ibrida (3 y 4) y para l\'ogicas modales sub-booleanas (5 y 6). Hacia los
% finales del primer a\~no (y una vez obtenidos avances te\'oricos sustanciales)
% se empieza paralelamente con la implementaci\'on de los algoritmos (7). Una vez
% que esta tarea termina, se procede con la etapa del testing (8). Se dedicar\'an
% tres meses a la documentaci\'on del software (9). El final del proyecto est\'a
% reservado para la elaboraci\'on y presentaci\'on final de los resultados
% te\'oricos (10) y aplicados (11).

  % Impacto Scientifico
\MySubSection{9. Impacto Socio-Econ\'omico.}

Los temas a investigar en el marco de este proyecto son de relevancia en el panorama Argentino actual por, al menos, dos razones de peso.

Por un lado, el proyecto
integra y desarrolla diferentes aspectos clave del \'area de ling\"u\'istica computacional (sintaxis, sem\'antica, pragm\'atica, representaci\'on,
inferencia, evaluaci\'on). El \'area de ling\"u\'istica computacional y su
aplicaci\'on al tratamiento autom\'atico del lenguaje natural han tenido un
gran desarrollo internacional en los \'ultimos a\~nos, con aplicaciones como los sistemas de b\'usqueda en la web, los sistemas de traducci\'on y resumen autom\'aticos, las interfaces de voz, etc. Sin embargo, el \'area es casi inexistente actualmente en Argentina.


Por otro lado, durante las \'ultimas etapas
del proyecto se propone investigar el uso de la plataforma
desarrollada en el \'area de educaci\'on a distancia (concretamente, como
plataforma de aprendizaje de idiomas).
Claramente, la educaci\'on a distancia es una herramienta
que contribuye a superar el problema de la centralizaci\'on de recursos
educativos en el pa\'is, pero desarrollar herramientas adecuadas en
este \'area espec\'ifica es dif\'icil.

Para desarrollar sistemas de ense\~nanza a distancia es esencial modelar el avance del aprendizaje del usuario. Esto requiere un sistema capaz de ser consciente de la evoluci\'on del usuario, y que tenga en cuenta sus logros y sus problemas. Este tipo de interacci\'on entre usuario y sistema puede modelarse como un dialogo, cuyo contexto registra el conocimiento adquirido (del usuario sobre el material del curso, y del sistema sobre el usuario). La gesti\'on de este tipo de dialogo es particularmente interesante, ya que el sistema debe ser capaz de interpretar los requerimientos de, y generar respuestas adecuadas para, usuarios no expertos cuyo conocimiento evoluciona durante la interacci\'on. Adem\'as, el sistema debe ser capaz de representar apropiadamente tanto la informaci\'on concerniente al material del curso, como la informaci\'on concerniente a la evoluci\'on del usuario. Por ejemplo, el sistema debe ser capaz de diagnosticar que parte del material del curso debe ser revisada a partir de las respuestas err\'oneas del usuario.  Por \'ultimo, el sistema debe poder evaluar la interacci\'on con el usuario, para poder decidir
que objetivos del aprendizaje fueron alcanzados.

Los resultados te\'oricos y pr\'acticos obtenidos durante el proyecto,
contribuyen directamente a la soluci\'on de estos problemas.

  % Impacto Socio-Economico
\MySubSection{10. Impacto Cient\'ifico}

Se puede clasificar el impacto cient\'ifico
esperado del proyecto en tres grandes areas, como se detalla a continuaci\'on:

\paragraph{Interacci\'on:}
Una de las contribuciones m\'as
importantes del proyecto ser\'a un laboratorio virtual para
teor\'ias pragm\'aticas que consistir\'a de un
entorno controlado para el estudio de la interacci\'on situada en un mundo donde
se entremezclan acciones f\'isicas y ling\"u\'isticas. 

El prototipo unidireccional permitir\'a
investigar el impacto de distintas
pol\'iticas para dar instrucciones (post o
preorden en el \'arbol de acciones de la tarea)
sobre la realizaci\'on exitosa de la tarea.
Estudios de este tipo se han realizado anteriormente
(~\citep{foster-etal-ijcai2009}) pero 
asumiendo una tarea prefijada.
Dado que nuestro prototipo permitir\'a la especificaci\'on
del mundo virtual, las posibles acciones y el objetivo
a alcanzar, podremos determinar cu\'ando estas pol\'iticas
son dependientes de la tarea.

El prototipo bidireccional nos permitir\'a investigar el
fen\'omeno de reparaciones a corto y largo plazo,
evaluando un sistema de predicci\'on de reparaciones
contextualizadas.  Estas reparaciones est\'an usualmente
causadas por implicaturas conversacionales.  Modelar
estas implicaturas en un sistema de di\'alogo gen\'erico
es dif\'icil.  Sin embargo, dado que el presente prototipo provee una
interacci\'on situada
y restringida al mundo virtual, ser'a posible testear la relaci\'on entre
las implicaturas, el tipo de reparaciones a las que
dan origen y las tareas de inferencia necesarias para
predecirlas.


\fixme{estan bien ``acciones linguisticas''? Cambi\'e de lugar ultimo parrafo q
hablaba de evaluacion}

\paragraph{Inferencia:} La principal contribuci\'on del
proyecto en el \'area de l\'ogica e inferencia es en el
dise\~no, desarrollo y estudio de algoritmos de planning.
Un sistema de planning t\'ipico toma tres inputs -- un
estado inicial, una especificaci\'on de posibles acciones y
un objetivo esperado -- y retorna una secuencia de acciones (un plan)
que al ser aplicadas secuencialmente al estado inicial, termina
en un estado que satisface el objetivo pedido.  Distintos
m\'etodos para obtener un plan han sido estudiados (forward chaining, backward
chaining, codification as propositional
satisfiability, etc.); y existen actualmente sistemas
implementados que pueden resolver esta tarea eficientemente.

Sin embargo, la mayor\'ia de estos sistemas asumen condiciones
que simplifican el problema (tiempo at\'omico y
determin\'istico, informaci\'on completa, ausencia de una teor\'ia
background, etc.) y retornan un \'unico plan.  En el transcurso
del proyecto se investigar\'an algoritmos que eliminan algunas
de las simplificaciones mencionadas (en particular, investigaremos
el caso de planning con informaci\'on incompleta y en base a una
teor\'ia background) y que ofrecen adem\'as servicios de planning
extendidos (retorno de planes alternativos, planes m\'inimos, planes
condicionados, planes incompletos, acciones posibles en un estado dado, etc.)

\paragraph{Evaluaci\'on:}
En este \'area se espera que una de las contribuciones principales sea la
intergaci\'on de t\'ecnicas de evaluaci\'on de distintas \'areas en una
metodolog\'ia que permita evaluar sistemas de di\'alogo para entornos virtuales
de manera que se estime la usabilidad y eficacia del mismo. Esta metodolog\'ia
podr\'ia usarse tanto para determinar si un sistema es adecuado para un tipo de
tarea y de usuario, como para comparar la performance de distintos sistemas del
mismo tipo.

Otra contribuci\'on ser\'a el estudio y aplicaci\'on de est\'andares de
evaluaci\'on de software a los sistemas desarrollados, generando un modelo de la
calidad estandarizado y proponiendo un conjunto de m\'etricas apropiadas para
evaluar cada uno de los aspectos contenidos en el modelo. Este trabajo
incluir\'a tambi\'en un estudio profundo de varias m\'etricas a fines de incluir
en el modelo de calidad algunos consejos sobre las m\'etricas elegidas; este
estudio se considera como una meta-evaluaci\'on del modelo propuesto.

Finalmente,
el corpus anotado de interacci\'on humano-humano, m\'as
los corpus de interacci\'on humano-m\'aquina recopilados
durante el proyecto se har\'an p\'ublicos.  Este tipo de
corpora servir\'a, por ejemplo, para dise\~nar plataformas
m\'as generales de evaluaci\'on de sistemas de di\'alogo,
que van m\'as all\'a de los aspectos evaluados actualmente
por plataformas existentes como GIVE.


 % Insersion en el Programa de Investigacion
%\MySubSection{11. Insersi\'on en el Programa de Investigaci\'on}

% El procesamiento de lenguaje natural, y en particular el
% desarrollo de sistemas de di\'alogo, es un \'area en crecimiento
% en los pa\'ises desarrollados, por su gran aplicabilidad en el
% actual contexto de la sociedad de la informaci\'on. Efectivamente,
% dado el gran aumento de la informaci\'on textual en formato
% digital, el procesamiento automatizado de los textos se ha
% convertido en una capacidad estrat\'egica para empresas,
% instituciones y la comunidad en general. Por esta raz\'on
% el \'area de PLN se encuentra en crecimiento en las principales
% universidades y empresas tecnol\'ogicas del mundo, con un
% presupuesto que crece a\~no a a\~no. Sin embargo, 
% este \'area se encuentra muy poco desarrollada en la Argentina. Esto se puede
% atribuir a diferentes factores:

% \begin{itemize}
% \item la relativa juventud del \'area de PLN, lo que implica una relativa
% escasez de profesionales bien formados en todo el mundo,
% \item el desarrollo insuficiente de todo el \'area de investigaci\'on en
% Ciencias de la Computaci\'on, por razones hist\'oricas y demanda de la
% industria,
% \item el poco desarrollo del \'area de Inteligencia Artificial y
% Ling\"u\'istica Formal en la Argentina, tambi\'en por razones hist\'oricas
% y acad\'emicas,
% \item la escasa interacci\'on entre los pocos investigadores en PLN que se
% encuentran en la regi\'on.
% \end{itemize}

Natural language processing, and in particular the development of dialogue
systems is a growth area in developed countries, for their great applicability
in the current context of information society. Indeed, given the large increase
of textual information in digital form, the automated processing of texts has
become a strategic capability for companies, institutions and the wider
community. For this reason the area is growing PLN at major universities and
technology companies in the world with a budget that grows every year. However,
this area is very underdeveloped in Argentina. This can be attributed to several
factors. (a) The relative youth of the area of PLN, which implies a relative
dearth of trained professionals throughout the world. (b) The underdevelopment
of the whole area of research in computer science, for historical reasons and
industry demand. (c) The undeveloped area of Artificial Intelligence and
Formal Linguistics in Argentina, also for historical and academic. (d) Poor
interaction between the few researchers in NLP that are in the region. 

% Parece claro que el PLN es un \'area de investigaci\'on estrat\'egica
% para la Argentina, en la que se puede alcanzar la excelencia acad\'emica
% e industrial a nivel internacional. Creemos que hay que apoyar el
% desarrollo de este \'area favoreciendo los siguientes aspectos:
% 
% \begin{itemize}
% \item formaci\'on de recursos humanos a trav\'es
% de programas de doctorado y de cursos dictados en la Argentina por
% profesionales de
% reconocido prestigio internacional,
% 
% \item incorporaci\'on de recursos humanos formados, para contribuir
% al aumento y diversificaci\'on de la masa cr\'itica en el \'area,
% 
% \item mejora de la interacci\'on entre los diversos grupos o investigadores
% aislados en PLN, a
% trav\'es de la organizaci\'on de workshops, cursos de profesores visitantes,
% co-tutor\'ias,
% programas de especializaci\'on coordinados, etc.
% \end{itemize}

It seems clear that PLN is a strategic research area for Argentina, which can
achieve academic excellence and industry worldwide. We believe in supporting the
development of this area by promoting the following. (a) Training of human
resources through doctoral programs and courses taught in Argentina by
internationally renowned professionals. (b) Incorporation of trained human
resources to contribute to the growth and diversification of the critical mass
in the area. (c) Improving the interaction between various groups or
individual researchers in NLP, through the organization of workshops, courses,
visiting professors, co-tutoring, coordinated specialty programs, etc.

% En la FaMAF existe un grupo de PLN desde 2005
% (\url{http://www.cs.famaf.unc.edu.ar/~pln}). Este
% grupo est\'a desarrollando una importante labor de formaci\'on de recursos
% humanos, con el dictado 
% de cursos de grado y de postgrado en la FaMAF y en otras universidades del
% pa\'is.
% Tambi\'en trabaja en el desarrollo de diversos proyectos de investigaci\'on y
% en la integraci\'on
% con otros grupos de la regi\'on, tanto en Argentina como en Chile, Brasil y
% Uruguay. Este proyecto 
% de investigaci\'on se integra al programa del grupo de procesamiento de
% lenguaje % natural de la
% FaMAF.


In FaMAF there is an NLP group since 2005 (\url{http://www.cs.famaf.unc.edu.ar/
~ pln}). This group is developing an important role in human resource training,
delivering courses to undergraduate and postgraduate studies at the FaMAF and
other universities. It also works in the development of various research
projects and integration with other groups in the region, both in Argentina and
Chile, Brazil and Uruguay. This research program is integrated into the group of
natural language processing of FaMAF. 


 % Diseminacion




% % % % % % % % % % % % % % % % % % % % % % % % % % % % % % % % % % % % % %
% \MySubSection{8a.  Material Requerido} For day-to-day
% work standard machine equipment will be used.  Distributed
% computational experiments will be carried out on computing facilities
% available at SARA. Implementations will mostly be done in Standard ML
% of New Jersey and Allegro Common Lisp (for which licenses are
% available), while  dedicated Perl scripts will be used for
% data manipulation and preprocessing.


% % % % % % % % % % % % % % % % % % % % % % % % % % % % % % % % % % % % % %
%% header automatically generated from bbl file

\citep{byron09}
\citep{blaylock05}
\citep{lochbaum98}
\citep{litman90}
\citep{purver06}
\citep{fikes72}
\citep{gerevini05}
\citep{kautz99}
\citep{hoffmann01}
\citep{hsu06}
\citep{allwood95}
\citep{skantze07}
\citep{stoia07}
\citep{gabsdil03}
\citep{stoia08}
\citep{rieser05}
\citep{schegloff87b}
\citep{Grice75}
\citep{rodriguez04}

\citep{clark96}
\citep{ginzburg09}
\citep{levinson87}

\citep{BBW06}
\citep{KR99}
\citep{KR97}
\citep{PT87}
\citep{S08}

\citep{citeulike:386317}

%\paragraph{BLA BLA}

%La Universidad tiene un importante rol que desempe\~nar en este nuevo espacio digital, multicultural y pluriling\"u\'istico llamado "Cibercultura", promoviendo una verdadera interacci\'on entre:

%   	El gobierno   	Las empresas   Las Universidades y Centros de Investigaci\'on y Desarrollo
%De esta manera se contribuye a la formaci\'on de capital intelectual capaz de comprender los procesos de la nueva sociedad, desde un enfoque genuinamente interdisciplinario.
%Las nuevas tecnolog\'ias ofrecen al presente y futuro de la Argentina, un sinf\'in de posibilidades, por el hecho de atravesar todos los sectores, pol\'iticos, econ\'omicos y sociales, siendo su ``desarrollo estrat\'egico'' un pilar fundamental para crecimiento equitativo y sustentable del pa\'is.

%\url{http://www.uai.edu.ar/ciiti/2009/bsas/congreso.html}

\bibliographystyle{plainnat}
\bibliography{pict09}


\end{document}
