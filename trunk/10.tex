\MySubSection{10. Impacto Cient\'ifico}

Se puede clasificar el impacto cient\'ifico
esperado del proyecto en tres grandes areas, como se detalla a continuaci\'on:

\paragraph{Interacci\'on:}
Una de las contribuciones m\'as
importantes del proyecto ser\'a un laboratorio virtual para
teor\'ias pragm\'aticas que consistir\'a de un
entorno controlado para el estudio de la interacci\'on situada en un mundo donde
se entremezclan acciones f\'isicas y ling\"u\'isticas. 

El prototipo unidireccional permitir\'a
investigar el impacto de distintas
pol\'iticas para dar instrucciones (post o
preorden en el \'arbol de acciones de la tarea)
sobre la realizaci\'on exitosa de la tarea.
Estudios de este tipo se han realizado anteriormente
(~\citep{foster-etal-ijcai2009}) pero 
asumiendo una tarea prefijada.
Dado que nuestro prototipo permitir\'a la especificaci\'on
del mundo virtual, las posibles acciones y el objetivo
a alcanzar, podremos determinar cu\'ando estas pol\'iticas
son dependientes de la tarea.

El prototipo bidireccional nos permitir\'a investigar el
fen\'omeno de reparaciones a corto y largo plazo,
evaluando un sistema de predicci\'on de reparaciones
contextualizadas.  Estas reparaciones est\'an usualmente
causadas por implicaturas conversacionales.  Modelar
estas implicaturas en un sistema de di\'alogo gen\'erico
es dif\'icil.  Sin embargo, dado que el presente prototipo provee una
interacci\'on situada
y restringida al mundo virtual, ser'a posible testear la relaci\'on entre
las implicaturas, el tipo de reparaciones a las que
dan origen y las tareas de inferencia necesarias para
predecirlas.


\fixme{estan bien ``acciones linguisticas''? Cambi\'e de lugar ultimo parrafo q
hablaba de evaluacion}

\paragraph{Inferencia:} La principal contribuci\'on del
proyecto en el \'area de l\'ogica e inferencia es en el
dise\~no, desarrollo y estudio de algoritmos de planning.
Un sistema de planning t\'ipico toma tres inputs -- un
estado inicial, una especificaci\'on de posibles acciones y
un objetivo esperado -- y retorna una secuencia de acciones (un plan)
que al ser aplicadas secuencialmente al estado inicial, termina
en un estado que satisface el objetivo pedido.  Distintos
m\'etodos para obtener un plan han sido estudiados (forward chaining, backward
chaining, codification as propositional
satisfiability, etc.); y existen actualmente sistemas
implementados que pueden resolver esta tarea eficientemente.

Sin embargo, la mayor\'ia de estos sistemas asumen condiciones
que simplifican el problema (tiempo at\'omico y
determin\'istico, informaci\'on completa, ausencia de una teor\'ia
background, etc.) y retornan un \'unico plan.  En el transcurso
del proyecto se investigar\'an algoritmos que eliminan algunas
de las simplificaciones mencionadas (en particular, investigaremos
el caso de planning con informaci\'on incompleta y en base a una
teor\'ia background) y que ofrecen adem\'as servicios de planning
extendidos (retorno de planes alternativos, planes m\'inimos, planes
condicionados, planes incompletos, acciones posibles en un estado dado, etc.)

\paragraph{Evaluaci\'on:}
En este \'area se espera que una de las contribuciones principales sea la
intergaci\'on de t\'ecnicas de evaluaci\'on de distintas \'areas en una
metodolog\'ia que permita evaluar sistemas de di\'alogo para entornos virtuales
de manera que se estime la usabilidad y eficacia del mismo. Esta metodolog\'ia
podr\'ia usarse tanto para determinar si un sistema es adecuado para un tipo de
tarea y de usuario, como para comparar la performance de distintos sistemas del
mismo tipo.

Otra contribuci\'on ser\'a el estudio y aplicaci\'on de est\'andares de
evaluaci\'on de software a los sistemas desarrollados, generando un modelo de la
calidad estandarizado y proponiendo un conjunto de m\'etricas apropiadas para
evaluar cada uno de los aspectos contenidos en el modelo. Este trabajo
incluir\'a tambi\'en un estudio profundo de varias m\'etricas a fines de incluir
en el modelo de calidad algunos consejos sobre las m\'etricas elegidas; este
estudio se considera como una meta-evaluaci\'on del modelo propuesto.

Finalmente,
el corpus anotado de interacci\'on humano-humano, m\'as
los corpus de interacci\'on humano-m\'aquina recopilados
durante el proyecto se har\'an p\'ublicos.  Este tipo de
corpora servir\'a, por ejemplo, para dise\~nar plataformas
m\'as generales de evaluaci\'on de sistemas de di\'alogo,
que van m\'as all\'a de los aspectos evaluados actualmente
por plataformas existentes como GIVE.


