\MySubSection{10. Impacto Cient\'ifico.}

Una vez m\'as podemos clasificar el impacto cient\'ifico
esperado del proyecto en tres grandes areas.

\paragraph{Interacci\'on:}
Una de las contribuciones m\'as
importantes del proyecto ser\'a un `laboratorio virtual' para
teor\'ias pragm\'aticas. El proyecto proveer\'a de un
entorno controlado, que permite
el estudio de interacci\'on situada en un mundo donde
se entremezclan acciones f\'isicas y acciones ling\"u\'isticas.

El prototipo unidireccional nos permitir\'a
investigar el impacto que distintas
pol\'iticas para dar instrucciones (e.g., post o
preorden en el \'arbol de acciones de la tarea)
tienen sobre la realizaci\'on exitosa de la tarea.
Evaluaciones de este tipo se han realizado anteriormente
(e.g.,~\citep{foster-etal-ijcai2009}) pero siempre
asumiendo una tarea prefijada.
Dado que nuestro prototipo permite la especificaci\'on
del mundo virtual, las posibles acciones y el objetivo
a alcanzar, podremos evaluar cu\'ando estas pol\'iticas
son dependientes o no de la tarea.
\fixme{Requiere haber definido post y preorden antes.}

El prototipo bidireccional nos permitir\'a investigar el
fen\'omeno de reparaciones a corto y largo plazo,
evaluando un sistema de predicci\'on de reparaciones
contextualizadas.  Estas reparaciones est\'an usualmente
causadas por implicaturas conversacionales.  Modelar
estas implicaturas en un sistema de di\'alogo gen\'erico
es dif\'icil.  Dado que nuestro prototipo provee una interacci\'on situada
y restringida al mundo virtual, podremos testear la relaci\'on entre
estas implicaturas, el tipo de reparaciones a las que
dan origen, y las tareas de inferencia necesarias para
predecirlas.

Finalmente,
el corpus anotado de interacci\'on humano-humano, m\'as
los corpus de interacci\'on humano-m\'aquina recopilados
durante el proyecto se har\'an p\'ublicos.  Este tipo de
corpora servir\'a, por ejemplo, para dise\~nar plataformas
m\'as generales de evaluaci\'on de sistemas de di\'alogo,
que van m\'as all\'a de los aspectos evaluados actualmente
por plataformas existentes como GIVE.

\paragraph{Inferencia:} La principal contribuci\'on del
proyecto en el \'area de l\'ogica e inferencia es en el
dise\~no, desarrollo y estudio de algoritmos de planning.
Un sistema de planning t\'ipico toma tres inputs -- un
estado inicial, una especificaci\'on de posibles acciones y
un objetivo esperado -- y retorna una secuencia de acciones (un plan)
que al ser aplicadas secuencialmente al estado inicial, termina
en un estado que satisface el objetivo pedido.  Distintos
m\'etodos para obtener un plan han sido estudiados (e.g.,
forward chaining, backward chaining, codification as propositional
satisfiability, etc.); y existen actualmente sistemas
implementados que pueden resolver esta tarea eficientemente.

Sin embargo, la mayor\'ia de estos sistemas asumen condiciones
que simplifican el problema (e.g., tiempo at\'omico y
determin\'istico, informaci\'on completa, ausencia de una teor\'ia
background, etc.) y retornan un \'unico plan.  En el transcurso
del proyecto se investigar\'an algoritmos que eliminan algunas
de las simplificaciones mencionadas (en particular, investigaremos
el caso de planning con informaci\'on incompleta y en base a una
teor\'ia background) y que ofrecen adem\'as servicios de planning
extendidos (retorno de planes alternativos, planes m\'inimos, planes
condicionados, planes incompletos, acciones posibles en un estado dado, etc.)

\paragraph{Evaluaci\'on:}
\fixme{escribir dos parrafos como los anteriores}
