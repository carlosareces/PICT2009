\MySubSection{6. Descripci\'on Detallada del Plan de Investigaci\'on Propuesto}

\vspace*{.2cm}%

\MySubSubSection{6.1. Definici\'on del Problema.}


\MySubSubSection{6.2. Resultados Espec\'ificos Esperados.}
Los resultados esperados pueden clasificarse usando los
tres temas principales que articulan el proyecto.


\paragraph{Interacci\'on:}

\emph{Generaci\'on de expresiones referenciales}. Un problema importante en
s\'intesis de lenguaje natural es el de la generaci\'on de expresiones referenciales. Esto quiere decir, producir una frase nominal que describa un\'ivocamente a un objeto (e.g. ``el jarr\'on que est\'a sobre la mesa''). Estas expresiones, para ser aceptables, deben ser similares a las que podr\'ia producir una persona en condiciones normales (no ser\'ia aceptable, en lugar del ejemplo anterior, ``el jarr\'on que no est\'a arriba de la silla ni arriba del sof\'a ni abajo de la mesa'').  En~\cite{AKS08} se propone utilizar la minimizaci\'on simb\'olica del modelo que representa
el estado del mundo para as\'i obtener f\'ormulas l\'ogicas que describan
un\'ivocamente a cada objeto. En particular se observa que con un
lenguaje l\'ogico sin negaci\'on, no se pueden obtener resultados poco
naturales como el anteriormente expuesto. Esta \'area de aplicaci\'on se
investigar\'a en conjunto con el grupo TALARIS (INRIA,
Nancy, Francia).

\paragraph{Inferencia:}
\citep{arec:logi00}.


\paragraph{Evaluaci\'on:}


\MySubSubSection{6.3. M\'etodo Scient\'ifico a Utilizar.}
\fixme{Mencionar Corpus}

Describimos a continuaci\'on el m\'etodo cient\'ifico a utilizar
respecto a los aspectos te\'oricos, pr\'acticos y emp\'iricos
que conciernen al proyecto.

\paragraph{Aspectos Te\'oricos.}
La metodolog\'a de trabajo a utilizar para investigar los aspectos
te\'oricos del proyecto es el estandard en investigaci\'on de base.
Comenzaremos con el estudio de la bibliograf\'ia existente, para luego
familiarizarnos con las t\'ecnicas de demostraci\'on espec\'ificas de esta
\'area y poder con ellas demostrar los resultados deseados. El grupo
responsable esta formado por expertos en las tres \'areas principales
que el proyecto planea investigar: Benotti es experta en el \'area de
pragm\'atica de la interacci\'on y sistemas de di\'alogos; Areces es
experto en m\'etodos de inferencia; y Estrella es experta en t\'ecnicas
de evaluci\'on de sistemas de procesamiento natural.
Los tres han desarrollado extensa investigaci\'on previa en sus respectivas \'areas
de experiencia.

\paragraph{Aspectos Pr\'acticos.}
Una vez se haya avanzado con las tareas de investigaci\'on te\'orica,
estas dar\'an origen al dise\~no de t\'ecnicas y algoritmos que se
implementar\'an en el framework de GIVE. Para ello se utilizar\'a la metodolog\'ia
usual en ingenier\'ia del software (desarrollo por
m\'odulos con interfaces claras y bien documentadas,
que garanticen alta cohesi\'on y bajo acoplamiento).
Al finalizar la implementaci\'on
se espera tambi\'en tener un documento al estilo de manual de
desarrollador, en donde queden expl\'icitas las principales interfaces
de los m\'odulos desarrollados y la forma de interacci\'on con la
herramienta.


\paragraph{An\'alisis Emp\'irico.}
Una vez
implementados los algoritmos, se proceder\'a a realizar tests
completos de unidad y de integraci\'on.

\fixme{mas sobre evaluacion y testing aca.}

\MySubSubSection{6.4. Cronograma de trabajo}

El cronograma de trabajo se estructura de la siguiente manera:

\paragraph{A\~no 1.} El objetivo de los primeros meses es obtener un
prototimo del sistema unidireccional.  Comenzaremos con un relevamiento
de la bibliograf\'ia.  Dado que el grupo de investigaci\'on est\'a
constitu\'io de expertos en las distintas \'areas pertinentes al
proyecto, nos concentraremos durante los primeros meses en obtener un
entendimiento com\'un de los distintos aspectos del problema, donde
cada experto contribuir\'a bibliograf\'ia adecuada de su \'area para los
dem\'as miembros del grupo.  A continuaci\'on comenzaremos a estudiar
la plataforma GIVE, y a adaptar sus componentes para nuestro objetivo
espec\'ifico.  Como parte del proyecto, planeamos participar del GIVE
challenge, lo que nos impondr\'a deadlines concretos para obtener un
primer prototipo.  La siguiente tarea es la definici\'on  y el dise\~no,
por un lado, de los par\'ametros de interacci\'on que queremos incorporar
en el sistema, y por otro, los esquemas de representaci\'on de la
informaci\'on que el sistema necesita, y las tareas de inferencia a
utilizar.

\fixme{Agregar algo concreto aca sobre `parametros de interacci\'on'}

Respecto de la representaci\'on del conocimiento, en un primer momento
utilizaremos sistemas de inferencia \emph{of-the-shelf} como
FaCT++~\citep{horr:fact99},
RACER~\citep{haar:race99} o Pellet~\citep{XXX}, que soportan tareas de
definici\'on, mantenimiento y consulta de ontolog\'ias.  Estos sistemas
han sido utilizados como motores de inferencia
en numerosas applicaciones en el \'area, y en applicaciones
dise\~nadas por miembros del equipo de
investigaci\'on~(e.g., \citep{FROLOG}).  Una vez observado el comportamiento de
estos motores de inferencia en la tarea, se analizar\'an sus limitaciones
e investigar\'an extensiones en el segundo a\~no del proyecto.  La tarea
de inferencia que ser\'a investigada en m\'as detalle durante el primer
a\~no ser\'a la tarea de planning.  GIVE provee un sistema de plannig muy
limitado.  Existen, sistemas de planning m\'as avanzandos (en particular,
que permiten el manejo de infirmaci\'on incompleta sobre el dominio)
como PKS \url{http://homepages.inf.ed.ac.uk/rpetrick/research/pks/}, pero
est\'an todav\'ia en periodo de desarrollo.  En general, el estado del
arte en el \'area de planning no cubre los requerimientos de nuestro
sistema.  Si bien, existen sistemas optimizados que funcionan adecuadamente
en ciertas aplicaciones, ninguno provee servicios como la generaci\'on de
planes alternativos, o la generaci\'on de planes incompletos en caso de
absencia de plan.

Una vez que obtengamos una sistema version del sistema con su
correspondiente testeo y documentaci\'on, podemos comenzar con
su evaluaci\'on.  Debe notarse de todas formas que es preciso tener en cuenta
el tipo de evaluaci\'on a realizar (que por lo tanto debe estar definido
adecuadamente) durante el periodo de desarrollo, para asegurar que el sistema
es capaz de prover la informaci\'on necesaria requerida durante la evaluaci\'on
(e.g., logueo de eventos, etc.).

Al final del a\~no comenzar\'a la preparaci\'on de articulos y reportes para
la presentaci\'on de los resultados obtenidos hasta el momento.

\paragraph{A\~no 2.} Los meses 13--24

\paragraph{A\~no 3.} El \'ultimo a\~no

\medskip

\noindent
A continuaci\'on se sumarizan las tareas a desarrollar en los tres a\~nos de trabajo (organizadas trimestralmente):

{\footnotesize
\begin{center}
\begin{tabular}{|p{7cm}||p{2mm}|p{2mm}|p{2mm}|p{2mm}||p{2mm}|p{2mm}|p{2mm}|p{2mm}||p{2mm}|p{2mm}|p{2mm}|p{2mm}||}
\hline
 \rowcolor[rgb]{0.8,0.8,0.8}\hspace{3.5cm}Tarea & 1 & 2 & 3 & 4 & 1 & 2 & 3 & 4 & 1 & 2 & 3 & 4\\
\hline 1. Relevamiento bibliogr\'afico
& $\times$ & $\times$ &&&&&&&$\times$&&&\\
\hline 2. Estudio del material bibliogr\'afico
& $\times$ & $\times$ & $\times$ &  &&&&&$\times$&&&\\
\hline 3. Estudio de la plataforma GIVE
& & $\times$ & &&&&&&&&&\\
\hline 4. Dise\~no e impl.\ de alg.\ de interacci\'on (unidireccional)
& & & $\times$ & $\times$&&&$\times$&$\times$&&&&\\
\hline 5. Dise\~no e impl.\ de alg.\ de interacci\'on (bidireccional)
& & &  & &&&$\times$&$\times$&&&$\times$&$\times$\\
\hline 6. Dise\~no e impl.\ de alg.\ de inferencia
& & & $\times$ & $\times$&&&$\times$&$\times$&&&$\times$&$\times$\\
\hline 7. Testing
&&&&$\times$&&&&$\times$&&&&$\times$\\
\hline 8. Documentaci\'on
&&&&$\times$&&&&$\times$&&&&$\times$\\
\hline 9. Evaluaci\'on
&&&$\times$&$\times$&$\times$&&$\times$&$\times$&$\times$&&$\times$&$\times$\\
\hline 10. Desarrollo de tutor virtual
&&&&&&&&&&$\times$&$\times$&$\times$\\
\hline 11. Elaborac.\ y presentaci\'on de resultados te\'oricos
&&&&$\times$&$\times$&$\times$&&$\times$&$\times$&$\times$&&$\times$\\
\hline 12. Elaborac.\ y presentaci\'on de resultados aplicados
&&&&$\times$&$\times$&$\times$&&$\times$&$\times$&$\times$&&$\times$\\\hline
\end{tabular}\end{center}
}

%Se comienza con el relevamiento bibliogr\'afico (1) y estudio del
%material (2). A continuaci\'on se desarrollan las investigaciones
%te\'oricas en c\'alculo de complejidades y desarrollo de algoritmos para
% extensiones de la l\'ogica modal b\'asica, en particular la l\'ogica
% h\'ibrida (3 y 4) y para l\'ogicas modales sub-booleanas (5 y 6). Hacia los finales del primer a\~no (y una vez obtenidos avances te\'oricos sustanciales) se empieza paralelamente con la implementaci\'on de los algoritmos (7). Una vez que esta tarea termina, se procede con la etapa del testing (8). Se dedicar\'an tres meses a la documentaci\'on del software (9). El final del proyecto est\'a reservado para la elaboraci\'on y presentaci\'on final de los resultados te\'oricos (10) y aplicados (11).

