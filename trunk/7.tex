\MySubSubSection{7. Metodolog\'ia de trabajo}
\fixme{Mencionar Corpus}

Describimos a continuaci\'on el m\'etodo cient\'ifico a utilizar
respecto a los aspectos te\'oricos, pr\'acticos y emp\'iricos
que conciernen al proyecto.

\begin{myitemize}
\item \textbf{Aspectos Te\'oricos.}
La metodolog\'a de trabajo a utilizar para investigar los aspectos
te\'oricos del proyecto es el estandard en investigaci\'on de base.
Comenzaremos con el estudio de la bibliograf\'ia existente, para luego
familiarizarnos con las t\'ecnicas de demostraci\'on espec\'ificas de esta
\'area y poder con ellas demostrar los resultados deseados. El grupo
responsable esta formado por expertos en las tres \'areas principales
que el proyecto planea investigar: Benotti es experta en el \'area de
pragm\'atica de la interacci\'on y sistemas de di\'alogos; Areces es
experto en m\'etodos de inferencia; y Estrella es experta en t\'ecnicas
de evaluci\'on de sistemas de procesamiento natural.
Los tres han desarrollado extensa investigaci\'on previa en sus respectivas
\'areas
de experiencia.

\item \textbf{Aspectos Pr\'acticos:}
Una vez se haya avanzado con las tareas de investigaci\'on te\'orica,
estas dar\'an origen al dise\~no de t\'ecnicas y algoritmos que se
implementar\'an en el framework de GIVE. Para ello se utilizar\'a la
metodolog\'ia
usual en ingenier\'ia del software (desarrollo por
m\'odulos con interfaces claras y bien documentadas,
que garanticen alta cohesi\'on y bajo acoplamiento).
Al finalizar la implementaci\'on
se espera tambi\'en tener un documento al estilo de manual de
desarrollador, en donde queden expl\'icitas las principales interfaces
de los m\'odulos desarrollados y la forma de interacci\'on con la
herramienta.


\item \textbf{An\'alisis Emp\'irico:}
Una vez
implementados los algoritmos, se proceder\'a a realizar tests
completos de unidad y de integraci\'on.

\fixme{mas sobre evaluacion y testing aca.}
\end{myitemize}
