\MySubSection{5. Objetivos Espec\'ificos}

Como discutimos en la secci\'on anterior, el objetivo
del proyecto es investigar temas en el \'area de pragm\'atica,
inferencia, y evaluaci\'on en un sistema de di\'alogo hombre-computadora con soporte para interacci\'on en tres dimensiones (3D) que generare autom\'aticamente instrucciones en lenguaje natural para guiar  al usuario durante la ejecuci\'on de una tarea determinada.

En el proyecto nos focalizaremos en el  m\'odulo encargado de la generaci\'on de  instrucciones e interpretaci\'on de las respuestas del usuario, el cual deber\'a ser capaz de una interacci\'on natural y debe poder adaptarse a posibles errores de interpretaci\'on o ejecuci\'on de las instrucciones, dando instrucciones correctivas en consecuencia. Adem\'as, el usuario se encontrar\'a en un universo en el que podr\'a interactuar con objetos y explorar el ambiente virtual provisto. As\'i, el usuario intentar\'a seguir las instrucciones provistas por el sistema,
realizando acciones f\'isicas en un mundo 3D.

En una primera etapa, el modelo a utilizar permitir\'a el flujo de informaci\'on ling\"u\'istica de manera unidireccional, es decir,
desde el sistema hacia el usuario, por lo que
el usuario no podr\'a pedir ningu\'n tipo de ayuda.  La restricci\'on a un
modelo unidireccional tiene como objetivo
simplificar la representaci\'on y manejo del contexto de la interacci\'on.

En etapas m\'as avanzadas del proyecto, el modelo unidireccional ser\'a extendido para
permitir intercambio ling\"u\'istico bidireccional: por ejemplo, el usuario podr\'a
pedir clarificaciones o redefinir el objetivo a alcanzar, en ambos casos usando lenguaje natural.

La generaci\'on de instrucciones para un dominio espec\'ifico requiere la adquisici\'on del conocimiento relevante y la compilaci\'on del mismo debe incluir cierta informaci\'on  ling\"u\'istica que permita su posterior procesamiento: en los aspectos morfol\'ogico y sint\'actico se requiere una gr\'amatica de lenguaje seleccionado (Espa\~nol en este caso), en el aspecto sem\'antico se necesita un repositorio de informaci\'on l\'exica organizada en forma ontol\'ogica y en lo pragm\'atico es necesaria una descripci\'on formal de las
acciones posibles en un universo dado (incluyendo precondiciones, efectos e informaci\'on contextual del estado de cada interacci\'on).

Esta informaci\'on relacionada al conocimiento de un dominio es utilizada por distintos componentes del sistema de di\'alogo que
gestionar\'an la interacci\'on con el usuario. En un sistema t\'ipico, esto sucede como se detalla a continuaci\'on: en la direcci\'on sistema $\rightarrow$ usuario, el componente encargado de \textit{planificar} genera una secuencia de acciones relevantes al momento de la interacci\'on, luego el componente encargado de  \textit{administrar el conocimiento} actualiza la informaci\'on contextual y,  finalmente, el componente encargado de
\textit{generar instrucciones}  transmite las mismas como expresiones
del lenguaje natural.  En la direcci\'on opuesta, es decir usuario $\rightarrow$ sistema,  el componente denominado \textit{discretizador}
transforma el flujo continuo de informaci\'on derivada del comportamiento del usuario, en acciones relevantes al universo en cuesti\'on, mientras que
los componentes de planeamiento y administraci\'on del conocimiento se ocupan
de mantener el contexto actualizado y de detectar posibles errores.

\MySubSubSection{Aspectos Pr\'acticos}

El dise\~no de una sistema como el descripto anteriormente presenta
desaf\'ios tanto te\'oricos como pr\'acticos.  En lo te\'orico, la necesidad de
 guiar la interacci\'on con el usuario (qu\'e decir,
cu\'ando y c\'omo, en un contexto dado) motiva la creaci\'on, adaptaci\'on o mejora de heur\'isticas pertinentes y, por otro lado,  la necesidad de describir la situaci\'on actual as\'i como tambi\'en  de indicar c\'omo alterarla para alcanzar un objetivo pre-establecido, incentivan el estudio de diversos m\'etodos
para la inferencia eficiente y precisa.
La calidad de cada una de estas capacidades afecta la percepci\'on que
el usuario tiene del sistema, y por lo tanto, es imperativo aplicar m\'etricas pertinentes que permitan evaluar cada uno de estos aspectos del sistema.  La complejidad del problema te\'orico se
refleja, en lo pr\'actico, en el desarrollo de un sistema de m\'ultiples componentes: un
generador de lenguaje natural, un componente de planeamiento, un entorno 3D, etc.

Dise\~nar e implementar todos estos componentes requerir\'ia un esfuerzo prohibitivo, por lo que proponemos utilizar recursos disponibles de forma libre y gratuita. En particular, se propone el uso de la plataforma desarrollada com parte de la competencia \textit{Generating Instructions in Virtual Environments} (GIVE\footnote{GIVE se encuentra disponible libremente en  \fnturl{http://www.give-challenge.org}, y posee el respaldo de los grupos de interes SIGSEM, SIGDIAL, y SIGGEN de ACL, Association for Computational Linguistics.}) que
provee un prototipo b\'asico de este tipo de sistemas. En este contexto ``competencia" (del Ingl\'es ``evaluation campaign" o ``evaluation challenge") se refiere a la organizaci\'on de un evento en el que participan distintos grupos de investigaci\'on, realizando todos el mismo ejercicio con sus sistemas propios y comparando al final los resultados obtenidos por cada grupo. Este tipo de eventos es muy popular en distinas \'areas del PLN (TA, reconocimiento del habla, recuperaci\'on de informac\'ion, entre otras) y benefician ampliamente cada \'area generando y compartiendo con la comunidad de investigadores nuevas tecnolog\'ias.

El objetivo principal de GIVE es actuar como medio de evaluaci\'on de sistemas de generaci\'on
de lenguaje natural, donde \'estos se eval\'uan de forma m\'as precisa y homog\'enea dado que todos los sistemas participantes utilizan los mismos recursos provistos por los organizadores. Adem\'as, desde un punto de vista general, la evaluaci\'on
tambi\'en tiene en cuenta el problema b\'asico de la interacci\'on en
entornos virtuales.  En el escenario propuesto por GIVE, el usuario
humano lleva acabo una tarea de `b\'usqueda del tesoro' en un entorno
virtual 3D y el trabajo del sistema participante es proveer, en tiempo
real, instrucciones en lenguaje natural que ayuden al usuario
a encontrar el tesoro escondido.

Por su dise\~no, el prototipo GIVE es una herramienta muy propicia para investigar distintos aspectos de la interacci\'on situada en un entorno virtual:
el problema de realizaci\'on sint\'actica y morfo\'ogica de las
instrucciones (en Ingl\'es \emph{surface realization}),
la representaci\'on del conocimiento existente en el contexto de
interacci\'on (contexto del discurso, ontolog\'ias, acciones posibles),
el tipo de inferencia requerido por las tareas involucradas
(planeamiento, construcci\'on y chequeo de modelos),
las reglas pragm\'aticas de interacci\'on requeridas por la tarea
(administraci\'on de la carga cognitiva, actualizaci\'on y uso de la informaci\'on contextual), entre otras. Adicionalmente, GIVE provee herramientas para registrar todos los detalles de la
interacci\'on, permitiendo de esta forma obtener f\'acilmente un corpus de interacci\'on en entornos virtuales anotado autom\'aticamente. \fixme{agregar la importancia de tales corpora}



\MySubSubSection{Tareas a Realizar}

Al comienzo del proyecto recopilaremos
un corpus de interacci\'on humano-humano (uni y bidireccional).  El
corpus contendr\'a la interacci\'on lig\"u\'istica alineada con
las acciones realizadas en el mundo virtual.  Este corpus
ser\'a utilizado para guiar el dise\~no de las pol\'iticas
de interacci\'on a implementar.

\fixme{cosas comentadas en el source aca abajo que pueden usarse}
% XXX
%
% Si se conceptualiza
% la estructura de la tarea como un \'arbol en el cual
% la ra\'iz
%
% (e.g., s\'olo pr\'oxima
% instrucci\'on a realizar, objetivo general a alcanzar m\'as
% proxima instrucci\'on, etc.).
%
%
%
% La teor\'ia
% de relevancia ~\citep{} distingue t
%
% (premisas implicadas, conclusiones implicadas
% y explicaturas)
%
%
% Dicho sistema de predicci\'on
% requerir\'a el uso de distintas tareas de inferencia.
% El contenido de reparaciones coherentes
% dentro de una interacci\'on tiene restricciones
% pragm\'aticas. Podemos clasificar las reparaciones
% coherentes en premisas implicadas, conclusiones implicadas,
% y explicaturas. Para predecir premisas implicadas se requiere
% planning con informaci\'on incompleta y planes alternativos,
% para predecir las
% conclusiones implicadas se requieren acciones posibles en
% un estado dado,
%
%
% Mapping entre tareas pragm\'aticas/cognitivas y servicios
% de inferencia / representaci\'on de conocimientos del sistema.



\emph{Generaci\'on de expresiones referenciales}. Un problema importante en
s\'intesis de lenguaje natural es el de la generaci\'on de expresiones
referenciales. Esto quiere decir, producir una frase nominal que describa
un\'ivocamente a un objeto (e.g. ``el jarr\'on que est\'a sobre la mesa'').
Estas expresiones, para ser aceptables, deben ser similares a las que podr\'ia
producir una persona en condiciones normales (no ser\'ia aceptable, en lugar del
ejemplo anterior, ``el jarr\'on que no est\'a arriba de la silla ni arriba del
sof\'a ni abajo de la mesa'').  En~\cite{AKS08} se propone utilizar la
minimizaci\'on simb\'olica del modelo que representa
el estado del mundo para as\'i obtener f\'ormulas l\'ogicas que describan
un\'ivocamente a cada objeto. En particular se observa que con un
lenguaje l\'ogico sin negaci\'on, no se pueden obtener resultados poco
naturales como el anteriormente expuesto. Esta \'area de aplicaci\'on se
investigar\'a en conjunto con el grupo TALARIS (INRIA,
Nancy, Francia).



\MySubSubSection{Aplicaciones}

El resultado del proyecto ser\'a un sistema capaz de dar instrucci\'ones
en lenguaje natural que deben ser llevadas a cabo por el usuario en un
entorno virtual 3D.  La tecnolog\'ia y los avances te\'oricos del proyecto
pueden utilizarse en distintas aplicaciones (en comercio electr\'onico,
soporte t\'ecnico, control de dispositivos por voz, etc.).  Durante el
\'ultimo a\~no del proyecto investigaremos su uso para la ense\~nanza a
distancia, adaptando el sistema para que funcione como tutor de lenguas
extranjeras.

Dada la arquitectura del sistema, es posible cambiar el lenguaje de
interacci\'on con el sistema (input y output), introduciendo
una gram\'atica y dem\'as recursos sint\'acticos (e.g., informaci\'on
morfol\'ogica) para el lenguaje correcto.  Es decir dados los recursos
sint\'acticos adecuados para, por ejemplo, el ingl\'es que cubran las
estructuras y el vocabulario usados en el sistema, se obtiene un sistema
de di\'alogo que interprete y/o produzca instrucciones en ingl\'es.

Un sistema unidireccional que genere instrucciones en ingl\'es puede
usarse para testear la comprensi\'on del usuario.  La correcta
interpretaci\'on de las instrucciones se puede evaluar a partir de la
correcta ejecuci\'on de las instrucciones dadas.  El sistema
bidireccional permitir\'a al usuario pedir aclaraciones sobre la
\'ultima instrucci\'on (en su lenguaje natal, en caso de no haber comprendido
la instrucci\'on, o en ingl\'es, si desea practicar su
capacidad de expresarse en el idioma extranjero).  El usuario tambi\'en
podr\'a redefinir el objetivo a alcanzar durante la interacci\'on, y
de esta forma seleccionar el vocabulario y el tipo de estructuras que desea
practicar.

Para estimar la eficacia de estos sistemas de tutoring se har\'a una evaluaci\'on comparativa (i.e., con otros
sistemas de tutoring existentes) o una evaluaci\'on orientada al usuario al ser utilizado, por ejemplo, por estudiantes de la universidad donde se desarrolla el proyecto.


