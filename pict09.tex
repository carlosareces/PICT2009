\documentclass[11pt]{article}

\usepackage{times,url,latexsym,amssymb,tabularx,a4wide,color}
\usepackage{colortbl}
\usepackage[draft,silent]{fixme}
%\usepackage[spanish]{babel}
%\usepackage[latin1]{inputenc}
%\usepackage[T1]{fontenc}

\usepackage{natbib}
%% Define a new 'leo' style for the package that will use a smaller font.
\makeatletter
\def\url@leostyle{%
  \@ifundefined{selectfont}{\def\UrlFont{\sf}}{\def\UrlFont{\small\sffamily}}}
\makeatother
%% Now actually use the newly defined style.
\urlstyle{leo}


\def\subsumptionchecker{\begin{picture}(100,100)
\thinlines    \multiput(5,100)(5,0){19}{\line(0,-1){100}}
              \multiput(0,5)(0,5){19}{\line(1,0){100}}
\thicklines   \put(0,0){\framebox(100,100){}}
              \put(0,100){\line(1,-1){100}}
              \put(0,0){\line(1,1){100}}
\end{picture}}


\newcounter{tbsnr}
\newenvironment{tbs}
{\addtocounter{tbsnr}{1}\par\bigskip \noindent\fbox{\thetbsnr}
\hspace*{\fill}\begin{minipage}{10cm}\tt \footnotesize}
%\typein{OPPASSEN GEBLAZEN}}
{\end{minipage}\hspace*{\fill}\bigskip}
\newcommand{\tb}[1]{\begin{tbs}{#1}\end{tbs}}

\newcommand{\todo}{\marginpar[$\bullet{\bullet}\bullet$]{$\bullet{\bullet}\bullet$}}

\makeatletter

\renewcommand{\refname}{\normalfont\large\sffamily\textbf{9. Literature}}

%\addtolength{\topmargin}{-1cm}
% \oddsidemargin 0.1 in
% \evensidemargin 0.15 in
% \marginparwidth 1 in
% \oddsidemargin 0.125 in
% \evensidemargin 0.125 in
% \marginparwidth 0.75 in
% \textwidth 6.125 in
%
%\addtolength{\textheight}{1.5cm}
% \addtolength{\textheight}{2cm}
% \addtolength{\voffset}{-1cm}
% % \addtolength{\textwidth}{2cm}
% \addtolength{\textwidth}{.4cm}
% % \addtolength{\hoffset}{-1cm}
% \addtolength{\hoffset}{-.4cm}

\makeatother

\pagestyle{plain}

\newcommand{\MySubSection}[1]{\vspace*{-.1\baselineskip}%
\subsection*{\sffamily\textbf
#1}\vspace*{-.2\baselineskip}}

\newcommand{\MySubSubSection}[1]{\vspace*{-.1\baselineskip}%
\subsubsection*{\sffamily\textbf
#1}\vspace*{-.2\baselineskip}}

\newcommand{\MyParagraph}[1]{\vspace*{-.35\baselineskip}\paragraph*{{\sffamily\textbf
#1}}}
\newcommand{\MySubParagraph}[1]{\vspace*{-.35\baselineskip}\subparagraph*{{\sffamily\textit{#1}}}}

\newenvironment{mylist}{%
  \renewcommand{\labelitemi}{\mbox{\tiny
$\blacksquare$}}%
  \renewcommand{\labelitemii}{$\bullet$}%
  \addtolength{\topsep}{-3\parskip}%
  \begin{itemize}\setlength{\itemsep}{-1.8pt}}%
  {\end{itemize}}

\newenvironment{myitemize}{%
  \renewcommand{\labelitemi}{\mbox{\tiny
$\blacksquare$}}%
  \renewcommand{\labelitemii}{$\bullet$}%
  \addtolength{\topsep}{-3\parskip}%
  \begin{itemize}\setlength{\itemsep}{-1.8pt}}%
  {\end{itemize}}

\newcommand{\nextchunk}{\vspace*{.65\baselineskip}\noindent}

\begin{document}
%\bibliographystyle{alpha}
\thispagestyle{plain}

\section*{\sffamily\textbf{Convocatioria PICT-2009}}
\mbox{}

\vspace*{-.5\baselineskip}
\MySubSection{1a. T\'itulo del Proyecto:
{\rm \emph{\Large Sistemas de Di\'alogo para Entornos Virtuales.}}}

\MySubSection{1b. Grupo Responsable}

\hspace*{.5cm} Dr.\ Carlos Areces (Investigador Responsable)

Dra.\ Paula Estrella (Investigador Integrante)

Lic.\ Luciana Benotti (Investigador Integrante)\footnote{La Lic.\ Benotti
obtendr\'a su t\'itulo de Dra.\ en Ciencias de la Computacion en Enero/2010.}

\MySubSection{1c. Grupo Colaborador}

\hspace*{.5cm}
Dra.\ Laura Alonso Alemani

Dr.\ Gabriel Infante L\'opez

Lic.\ Franco Luque

\MySubSection{2.  Palabras Clave}

Sistemas de Di\'alogo -- Representaci\'on del Conocimiento -- Planning
-- Evaluaci\'on


\MySubSection{3. Resumen}
En este proyecto se investigar\'an temas fundamentales respecto de la
interaccion hombre-maquina en entornos virtuales. Identificamos tres
\'areas principales en las podemos clasificar los resultados esperados
del proyecto: a) pragm\'atica de la interacci\'on, b) representaci\'on
de la informaci\'on e inferencia; y c) evaluaci\'on de sistemas
de di\'alogo.

En particular, tomaremos como primer modelo un sistema de di\'alogo con
soporte de interacci\'on 3D, en el que el sistema genera autom\'aticamente
instrucci\'ones en lenguaje natural para cumplir una tarea determinada, que
el usuario humano debe seguir.   El sistema debe ser capaz de interactuar naturalmente
con el usuario y debe poder adaptarse a posibles errores de interpretaci\'on o
ejecuci\'on del usuario.  En este primer modelo el flujo de informaci\'on
linguistica es unidireccional, desde el sistema hacia el usuario.
En las \'ultimas etapas del proyecto, este modelo ser\'a extendido para
permitir intercambio ling\"u\'istico bidireccional: por ejemplo, el usuario podr\'a
pedir clarificaciones o redefinir el objetivo a alcanzar usando lenguaje natural.

El dise\~no de una arquitectura como la que acabamos de describir presenta
desaf\'ios te\'oricos y pr\'acticos.  En lo te\'orico, necesita
heur\'isticas que gu\'ien la interacci\'on con el usuario (i.e., que decir,
cuando, y como, teniendo en cuenta el contexto actual) y debe contar con m\'etodos
de inferencia que le permitan no s\'olo describir la situaci\'on actual, sino
tambi\'en indicar como cambiarla para alcanzar un objetivo.
La calidad de cada una de estas capacidades afectar\'a la percepci\'on que
el usuario tendr\'a del sistema, y por lo tanto se necesitan m\'etricas que
permitan evaluar su desempe\~no.  La complejidad del problema te\'orico se
refleja, en lo pr\'actico, en un sistema de m\'ultiples componentes: un
generador de lenguaje natural, un sistema de planeamiento, un entorno 3D, etc.
Dise\~nar e implementar todos estos componentes requerir\'ia un esfuerzo prohibitivo.
En este proyecto utilizaremos la plataforma GIVE (\url{http://www.give-challenge.org}) que
provee un prototipo basico para este tipo de sistemas.

Sistemas como el descripto tienen aplicaciones en diferentes \'ambitos:
asistentes de comercio electr\'onico, sistemas de di\'alogo embebidos,
ense\~nanza a distancia, etc.  Una vez obtenido
un prototipo, planeamos investigar el uso del sistema como
tutores virtuales para el aprendizaje de idiomas.


\MySubSection{4. Objetivos Generales}

El objetivo te\'orico principal del proyecto es el estudio de
tres temas fundamentales que posibilitan el desarrollo
de sistemas de di\'alogo en entornos virtuales:
\begin{myitemize}
  \item \emph{Interacci\'on:} La interacci\'on humano-humano esta gobernada
  por numerosas reglas pragm\'aticas que definen nociones b\'asicas como
  las obligaciones conversacionales de los participantes (e.g., quien puede hablar en cada momento), o las implicaciones de un enunciado (e.g., si se
  hizo una pregunta se espera una respuesta).
  Estas reglas aplican tambi\'en a la interacci\'on humano-computadora en
  entornos virtuales, cuando el objetivo del sistema de di\'alogo es
  simular, en la medida de lo posible, el comportamiento humano.

  Modelar formalmente estas reglas pragm\'aticas, para un sistema de di\'alogo
  dado, es uno de los desaf\'ios que condicionan la efectividad del sistema.

  \item \emph{Inferencia:} En t\'erminos generales, podemos entender como
  inferencia toda operaci\'on que transforme informaci\'on impl\'icita en
  expl\'icita.  Bajo esta amplia definici\'on, es posible considerar como
  inferencia tanto inferencia cl\'asica (e.g., consequencia l\'ogica) como
  tareas habituales en inteligencia artificial (e.g., planning), o operaciones estad\'isitcas (e.g., obtener la media de un conjunto de datos).  Un sistema de di\'alogo realiza continuamente operaciones de inferencia, por un lado,
  para interpretar la informaci\'on recibida e incorporarla a su repositorio de informacion; y por otro, para decidir qu\'e parte de la informaci\'on disponible transmitir.

  El problema mismo de decidir qu\'e tarea de inferencia utilizar en una determinada situaci\'on es complejo, y las tareas de inferencia en s\'i son
  computacionalmente costosas.  El desaf\'io en este caso es encontrar el compromiso adecuado entre representaci\'on de la informaci\'on y m\'etodo
  de inferencia a utilizar.

  \item \emph{Evaluaci\'on:} Cuando podemos decir que una interacci\'on dada
  es correcta. Cuando podemos decir que es mejor que otra?  Que p\'arametros podemos utilizar para medir el buen desempe\~no de un sistema de di\'alogo?  Estas son preguntas abiertas que actualmente el \'area de evaluaci\'on sistemas de di\'alogo intenta contestar. \fixme{Paula, algo m\'as ac\'a?}

\end{myitemize}

En el proyecto dise\~naremos un sistema de di\'alogo hombre-computadora con
soporte de interacci\'on 3D.  El sistema generar\'a autom\'aticamente
instrucciones en lenguaje natural para ayudar al usuario a cumplir una tarea determinada.

En nuestro proyecto nos concentraremos en el estudio del m\'odulo generador de instrucciones que deber\'a ser capaz de interactuar naturalmente
con el usuario y adaptarse a posibles errores de interpretaci\'on o
ejecuci\'on de las instrucciones.  En una primera etapa, trabajaremos
con un modelo donde el flujo de informaci\'on ling\"u\'istica ser\'a unidireccional:
desde el sistema hacia el usuario.  El usuario se encontrar\'a en un entorno 3D
en el que puede interactuar con objetos y explorar un mundo virtual. El sistema
dar\'a instrucciones al usuario generadas autom\'aticamente en t\'erminos de
una gram\'atica de lenguaje natural, describiendo la tarea que el usuario
debe llevar a cabo.  El usuario XXX

La generaci'on de instrucciones requiere de informacion linguistica a
distintos niveles tanto a nivel morfologico y sintactico, como a nivel
semantico y pragmatico.



En las \'ultimas etapas del proyecto, este modelo ser\'a extendido para
permitir intercambio ling\"u\'istico bidireccional: por ejemplo, el usuario podr\'a
pedir clarificaciones o redefinir el objetivo a alcanzar usando lenguaje natural.

El dise\~no de una arquitectura como la que acabamos de describir presenta
desaf\'ios te\'oricos y pr\'acticos.  En lo te\'orico, necesita
heur\'isticas que gu\'ien la interacci\'on con el usuario (i.e., que decir,
cuando, y como, teniendo en cuenta el contexto actual) y debe contar con m\'etodos
de inferencia que le permitan no s\'olo describir la situaci\'on actual, sino
tambi\'en indicar como cambiarla para alcanzar un objetivo.
La calidad de cada una de estas capacidades afectar\'a la percepci\'on que
el usuario tendr\'a del sistema, y por lo tanto se necesitan m\'etricas que
permitan evaluar su desempe\~no.  La complejidad del problema te\'orico se
refleja, en lo pr\'actico, en un sistema de m\'ultiples componentes: un
generador de lenguaje natural, un sistema de planeamiento, un entorno 3D, etc.

Dise\~nar e implementar todos estos componentes requerir\'ia un esfuerzo prohibitivo.
En este proyecto utilizaremos la plataforma GIVE (\url{http://www.give-challenge.org}) que
provee un prototipo basico para este tipo de sistemas.
%Concretamente, exploraremos estos tres temas en el contexto del
%Challenge on Generating Instructions in Virtual Environments
%(GIVE, \url{http://www.give-challenge.org}).
El objetivo puntual
de GIVE es actuar como medio de evaluacion de sistemas de generaci\'on
de lenguaje natural. Aunque desde un punto de vista m\'as general, la evaluaci\'on
tambi\'en tiene en cuenta el problema b\'asico de la interaccion en
entornos virtuales.  En el escenario provisto por GIVE, el usuario
humano lleva acabo una tarea de `b\'usqueda del tesoro' en un entorno
virtual 3D.  El trabajo del sistema de di\'alogo es proveer, en tiempo
real, instrucciones en lenguaje natural que ayuden al usuario
a llevar a cabo la tarea.

\fixme{GIVE tiene el apoyo de SIGSEM SIGDIAL SIGGEN grupos de interes de
ACL, la Asociacion de Ling\"u\'istica Computacional.}

Por su dise\~no, GIVE provee un framework interesante para investigar
distintos aspectos de la interacci\'on situada en un entorno virtual:
el problema de realizaci\'on sint\'actica y morfo\'ogica de las
instrucciones (\emph{surface realization}),
la representaci\'on del conocimiento existente en el contexto de
interacci\'on (contexto del discurso, ontolog\'ias, acciones posibles),
el tipo de inferencia requerido por las tareas involucradas
(planning, model checking, construcci\'on de modelos),
las reglas pragm\'aticas de interacci\'on requeridas por la tarea
(administraci\'on de la carga cognitiva, actualizaci\'on y uso de la informaci\'on contextual), etc.

Finalmente, la calidad de la interacci\'on con un
sistema de di\'alogo dado puede medirse en esta tarea en t\'erminos
quantitativos (e.g., tiempo m\'aximo necesitado para completar la tarea)
y qualitativos (e.g., relevancia de la instrucci\'on dada en cada momento).
GIVE provee la infraestructura para loguear todos los detalles de la
interacci\'on y permite, de esta forma, obtener facilmente un corpus
de interacci\'on en entornos virtuales anotado autom\'aticamente con
datos medibles.

\MySubSection{5. Composici\'on del Grupo de Investigaci\'on y Areas de Especializaci\'on}

\begin{center}\small
    \begin{minipage}{\linewidth}
        \begin{center}
        \renewcommand{\thefootnote}{\thempfootnote}
        \begin{tabular}{|l|l|}
        \hline\hline
        Nombre &  Especializacion \\
        \hline
        Dr.\ Carlos Areces &
          L\'ogica Computacional\\
        & Representaci\'on de Conocimiento\\
        & Complejidad Algor\'itmica
      \\ \hline
        Dra.\ Paula Estrella &
          Lingu\'istica Computacional \\
        & Evaluacion de Sistemas de PLN
      \\ \hline
        Lic. Luciana Benotti\footnote{L. Benotti completar\'a sus estudios doctorales
          en la Universit\'e Henri Poincare en Enero del 2010.} &
          Pragm\'atica \\
        & Sistemas de Di\'alogo\\
        & L\'ogica Computacional\\
        \hline\hline
        \end{tabular}
        \end{center}
    \end{minipage}
\end{center}


\MyParagraph{Colaboraci\'on Internacional:}
Los investigadores del proyecto colaboran actualmente
con grupos de investigaci\'on internacionales relevantes para
el los objetivos a obtener.  Areces y Benotti fueron miembros
del grupo TALARIS \url{http://talaris.loria.fr}.

Dr.\ Patrick Blackburn \url{http://www.loria.fr/~blackbur} (INRIA Nancy Grand Est, France) has
extensive knowledge on computational and algorithmic aspects of
so-called hybrid languages~\citep{arec:hybr99,blac:hybr98c}.

GLyC \url{http://www.glyc.dc.uba.ar/} XXX

Agust\'in Gravano \url{http://www.glyc.dc.uba.ar/agustin/} y
Jose Casta\~no \url{XXX} XXX

Alexandre Koller \url{http://www.coli.uni-saarland.de/~koller/} XXX

Ron Petrick \url{http://homepages.inf.ed.ac.uk/rpetrick/} y
PKS \url{http://homepages.inf.ed.ac.uk/rpetrick/research/pks/} XXX

Paul Piwek \url{http://mcs.open.ac.uk/pp2464/} XXX

NLG Group at the Open University \url{http://mcs.open.ac.uk/nlg/}

\fixme{Paula: Agregar los grupos/gente con los que te interesa trabajar a vos.}

\MyParagraph{Publicaciones m\'as relevantes del grupo investigador:}
Luciana
\citep{benotti09c}
\citep{benotti09b}

Carlos

Referring Expressions
\citep{AKS08}
\citep{AF08}

Logic
\citep{ABM01}
\citep{arec:hybr05b}

%\citep{AFFM08}
Inference
\citep{AG06}
\citep{ANR01}

\fixme{Paula: Elegir las publicaciones que te parezcan m\'as importantes y escribir dos lineas de motivaci\'on.}
Paula
\citep{estr:femt09}
\citep{estr:expe05}
\citep{estr:impr08}
\citep{estr:newm07}
\citep{estr:howm07}
\citep{pope:estr07}
\citep{pope:mode06}
\citep{estr:find05}

% % % % % % % % % % % % % % % % % % % % % % % % % % % % % % % % % % % % % %
\MySubSection{6. Descripci\'on Detallada del Plan de Investigaci\'on Propuesto}

\vspace*{.2cm}%

\MySubSubSection{6.1. Definici\'on del Problema.}

well-known systems include FaCT~\citep{horr:fact99},
DLP~\citep{pate:dlps98} and RACE~\citep{haar:race99}.


\MySubSubSection{6.2. Resultados Espec\'ificos Esperados.}
Los resultados esperados pueden clasificarse usando los
tres temas principales que articulan el proyecto.


\MySubParagraph{Interacci\'on.}

\emph{Generaci\'on de expresiones referenciales}. Un problema importante en
s\'intesis de lenguaje natural es el de la generaci\'on de expresiones referenciales. Esto quiere decir, producir una frase nominal que describa un\'ivocamente a un objeto (e.g. ``el jarr\'on que est\'a sobre la mesa''). Estas expresiones, para ser aceptables, deben ser similares a las que podr\'ia producir una persona en condiciones normales (no ser\'ia aceptable, en lugar del ejemplo anterior, ``el jarr\'on que no est\'a arriba de la silla ni arriba del sof\'a ni abajo de la mesa'').  En~\cite{AKS08} se propone utilizar la minimizaci\'on simb\'olica del modelo que representa
el estado del mundo para as\'i obtener f\'ormulas l\'ogicas que describan
un\'ivocamente a cada objeto. En particular se observa que con un
lenguaje l\'ogico sin negaci\'on, no se pueden obtener resultados poco
naturales como el anteriormente expuesto. Esta \'area de aplicaci\'on se
investigar\'a en conjunto con el grupo TALARIS~\cite{TALARIS} (INRIA,
Nancy, Francia).

\MySubParagraph{Inferencia.}
\citep{arec:logi00}.


\MySubParagraph{Evaluaci\'on.}


\MySubSubSection{6.3. M\'etodo Scient\'ifico a Utilizar.}
\fixme{Mencionar Corpus}

Describimos a continuaci\'on el m\'etodo cient\'ifico a utilizar
respecto a los aspectos te\'oricos, pr\'acticos y emp\'iricos
que conciernen al proyecto.

\MySubParagraph{Aspectos Te\'oricos.}
La metodolog\'a de trabajo a utilizar para investigar los aspectos
te\'oricos del proyecto es el estandard en investigaci\'on de base.
Comenzaremos con el estudio de la bibliograf\'ia existente, para luego
familiarizarnos con las t\'ecnicas de demostraci\'on espec\'ificas de esta
\'area y poder con ellas demostrar los resultados deseados. El grupo
responsable esta formado por expertos en las tres \'areas principales
que el proyecto planea investigar: Benotti es experta en el \'area de
pragm\'atica de la interacci\'on y sistemas de di\'alogos (e.g., \citep{benotti09c,benotti09b}; Areces es
experto en m\'etodos de inferencia (e.g., \citep{}); y Estrella es experta en t\'ecnicas
de evaluci\'on de sistemas de procesamiento natural (e.g., \citep{}).  Los tres han desarrollado extensa investigaci\'on previa en sus respectivas \'areas
de experiencia.

\MySubParagraph{Aspectos Pr\'acticos.}
Una vez se haya avanzado con las tareas de investigaci\'on te\'orica,
estas dar\'an origen al dise\~no de t\'ecnicas y algoritmos que se
implementar\'an en el framework de GIVE. Para ello se utilizar\'a la metodolog\'ia
usual en ingenier\'ia del software (desarrollo por
m\'odulos con interfaces claras y bien documentadas,
que garanticen alta cohesi\'on y bajo acoplamiento).
Al finalizar la implementaci\'on
se espera tambi\'en tener un documento al estilo de manual de
desarrollador, en donde queden expl\'icitas las principales interfaces
de los m\'odulos desarrollados y la forma de interacci\'on con la
herramienta.


\MySubParagraph{An\'alisis Em\'irico.}
Una vez
implementados los algoritmos, se proceder\'a a realizar tests
completos de unidad y de integraci\'on.

MAS SOBRE TESTING AQUI.

\MySubSubSection{6.4. Cronograma de trabajo}

El cronograma de trabajo se estructura de la siguiente manera:

XXX

\MyParagraph{A\~no 1.} Durante los meses 1--12


\MyParagraph{A\~no 2.} Los meses 13--24

\MyParagraph{A\~no 3.} El \'ultimo a\~no



A continuaci\'on se presentan las tareas a desarrollar en los tres a\~nos de trabajo (organizadas trimestralmente):

{\footnotesize
\begin{center}
\begin{tabular}{|p{7cm}||p{2mm}|p{2mm}|p{2mm}|p{2mm}||p{2mm}|p{2mm}|p{2mm}|p{2mm}||p{2mm}|p{2mm}|p{2mm}|p{2mm}||}
\hline
 \rowcolor[rgb]{0.8,0.8,0.8}\hspace{3.5cm}Tarea & 1 & 2 & 3 & 4 & 1 & 2 & 3 & 4 & 1 & 2 & 3 & 4\\
\hline 1. Relevamiento bibliogr\'afico
& $\times$ & $\times$ &&&&&&&&&&\\
\hline 2. Estudio del material bibliogr\'afico
& $\times$ & $\times$ & $\times$ &  &&&&&&&&\\
\hline 3. Estudio de la plataforma GIVE
& & $\times$ & &&&&&&&&&\\
\hline 7. Implementaci\'on de algoritmos
& & & $\times$ & $\times$&&&$\times$&$\times$&&&$\times$&$\times$\\
\hline 8. Testing
&&&&&&&&&&&&\\
\hline 9. Documentaci\'on
&&&&&&&&&&&&\\
\hline 10. Elaborac.\ y presentaci\'on de resultados te\'oricos
&&&&$\times$&$\times$&&&$\times$&$\times$&&&$\times$\\
\hline 11. Elaborac.\ y presentaci\'on de resultados aplicados
&&&&$\times$&$\times$&&&$\times$&$\times$&&&$\times$\\\hline
\end{tabular}\end{center}
}

Se comienza con el relevamiento bibliogr\'afico (1) y estudio del
material (2). A continuaci\'on se desarrollan las investigaciones
te\'oricas en c\'alculo de complejidades y desarrollo de algoritmos para
extensiones de la l\'ogica modal b\'asica, en particular la l\'ogica
h\'ibrida (3 y 4) y para l\'ogicas modales sub-booleanas (5 y 6). Hacia los finales del primer a\~no (y una vez obtenidos avances te\'oricos sustanciales) se empieza paralelamente con la implementaci\'on de los algoritmos (7). Una vez que esta tarea termina, se procede con la etapa del testing (8). Se dedicar\'an tres meses a la documentaci\'on del software (9). El final del proyecto est\'a reservado para la elaboraci\'on y presentaci\'on final de los resultados te\'oricos (10) y aplicados (11).

\MySubSection{7. Trabajo Previo y Trabajos Relacionados.}



\MySubSection{8. Impacto Scient\'ifico.}

\MySubSection{9. Impacto Socio-Econ\'omico.}

FALTA INTRO

El tema es de relevancia en el panorama Argentino actual por, al menos, dos razones de peso. Por un lado, la educaci\'on a distancia es una herramienta
que contribuye a superar el problema de la centralizaci\'on de recursos
educativos en el pa\'is. Por otro lado, el proyecto
integra y desarrolla diferentes aspectos clave del \'area de ling\"u\'istica computacional (sintaxis, sem\'antica, pragm\'atica, representaci\'on,
inferencia, evaluaci\'on). El \'area de ling\"u\'istica computacional y su
aplicaci\'on al tratamiento autom\'atico del lenguaje natural han tenido un
gran desarrollo internacional en los \'ultimos a\~nos, con aplicaciones como los sistemas de b\'usqueda en la web, los sistemas de traducci\'on y resumen autom\'aticos, las interfaces de voz, etc. Sin embargo, el \'area es casi inexistente actualmente en Argentina.

Para desarrollar sistemas de ense\~nanza a distancia es esencial modelar el avance del aprendizaje del usuario. Esto requiere un sistema capaz de ser consciente de la evoluci\'on del usuario, y que tenga en cuenta sus logros y sus problemas. Este tipo de interacci\'on entre usuario y sistema puede modelarse como un dialogo, cuyo contexto registra el conocimiento adquirido (del usuario sobre el material del curso, y del sistema sobre el usuario).

La gesti\'on de este tipo de dialogo es particularmente interesante, ya que el sistema debe ser capaz de interpretar los requerimientos de, y generar respuestas adecuadas para, usuarios no expertos cuyo conocimiento evoluciona durante la interacci\'on. Adem\'as, el sistema debe ser capaz de representar apropiadamente tanto la informaci\'on concerniente al material del curso, como la informaci\'on concerniente a la evoluci\'on del usuario. Por ejemplo, el sistema debe ser capaz de diagnosticar que parte del material del curso debe ser revisada a partir de las respuestas err\'oneas del usuario.  Por \'ultimo, el sistema debe poder evaluar la interacci\'on con el usuario, para poder decidir
que objetivos del aprendizaje fueron alcanzados.



\MySubSection{10. Insersi\'on en el Programa de Investigaci\'on.}

El Procesamiento de Lenguaje Natural (PLN), y en particular el
desarrollo de Sistemas de Di\'alogo, es un \'area en crecimiento
en los pa\'ises desarrollados, por su gran aplicabilidad en el
actual contexto de la Sociedad de la Informaci\'on. Efectivamente,
dado el gran aumento de la informaci\'on textual en formato
digital, el procesamiento automatizado de los textos se ha
convertido en una capacidad estrat�gica para empresas,
instituciones y comunidades ling\"u\'isticas. Por esta raz\'on
el \'area de PLN se encuentra en crecimiento en las principales
universidades y empresas tecnol\'ogicas del mundo, con un
presupuesto que crece a\~no a a\~no. Sin embargo, en la Argentina
este \'area se encuentra muy poco desarrollada. Esto se puede
atribuir a diferentes factores:
\begin{myitemize}
\item la relativa juventud del \'area de PLN, lo que implica una relativa escasez de profesionales bien formados en todo el mundo,
\item el desarrollo insuficiente de todo el \'area de Ciencias de la Computaci\'on, por razones hist\'oricas y demanda de la industria,
\item el poco desarrollo del \'area de Inteligencia Artificial y
Ling\"u\'istica Formal en la Argentina, tambi�n por razones hist\'oricas
y acad�micas,
\item la escasa interacci\'on entre los pocos investigadores en PLN que se encuentran en la regi\'on.
\end{myitemize}

Parece claro que el PLN es un \'area de investigaci\'on estrat�gica
para la Argentina, en la que se puede alcanzar la excelencia acad�mica
e industrial a nivel internacional. Creemos que hay que apoyar el
desarrollo de este \'area favoreciendo los siguientes aspectos:
\begin{myitemize}
\item formaci\'on de recursos humanos, con una fuerte componente de
formaci\'on en el centros de excelencia extranjeros, ya sea a trav�s
de programas de doctorado o de cursos dictados por profesionales de
reconocido prestigio en la Argentina,
\item incorporaci\'on de recursos humanos formados, para contribuir
al aumento y diversificaci\'on de la masa cr\'itica en el \'area,
\item mejora de la interacci\'on entre los diversos grupos o investigadores aislados en PLN, a
trav�s de la organizaci\'on de workshops, cursos de profesores visitantes, co-tutor\'ias,
programas de especializaci\'on coordinados, etc.
\end{myitemize}

En FaMAF existe un grupo de PLN desde 2005 (\url{http://www.cs.famaf.unc.edu.ar/~pln}). Este
grupo est\'a desarrollando una importante labor de formaci\'on de recursos humanos, con el dictado de cursos de grado y de postgrado en la FaMAF y otras Universidades del pa\'is.
Tambi�n trabaja en el desarrollo de diversos proyectos de investigaci\'on y en la integraci\'on
con otros grupos de la regi\'on, tanto en Argentina como en Chile, Brasil y Uruguay. Este proyecto de investigaci\'on se integra al programa del grupo PLN de Procesamiento de Lenguaje Natural del
FaMAF.


\MySubSection{11. Diseminaci\'on.}

Durante el desarrollo de todo el proyecto, se organizar\'an seminarios
entre los integrantes del grupo y potenciales estudiantes de la carrera
interesados en temas afines.  La combinaci\'on de temas te\'oricos y
pr\'acticos se presta naturalmente a la definici\'on de proyectos de
grado que exploren subproblemas de los objetivos generales del proyecto.

Adem\'as, los resultados t\'eoricos y pr\'acticos ser\'an presentados en
congresos locales e internacionales del \'area.  Versiones finales, m\'as
completas, ser\'an publicadas en revistas cient\'ificas especializadas.


% % % % % % % % % % % % % % % % % % % % % % % % % % % % % % % % % % % % % %
% \MySubSection{8a.  Material Requerido} For day-to-day
% work standard machine equipment will be used.  Distributed
% computational experiments will be carried out on computing facilities
% available at SARA. Implementations will mostly be done in Standard ML
% of New Jersey and Allegro Common Lisp (for which licenses are
% available), while  dedicated Perl scripts will be used for
% data manipulation and preprocessing.


% % % % % % % % % % % % % % % % % % % % % % % % % % % % % % % % % % % % % %
%% header automatically generated from bbl file

\citep{byron09}
\citep{blaylock05}
\citep{lochbaum98}
\citep{litman90}
\citep{purver06}
\citep{fikes72}
\citep{gerevini05}
\citep{kautz99}
\citep{hoffmann01}
\citep{hsu06}
\citep{allwood95}
\citep{skantze07}
\citep{stoia07}
\citep{gabsdil03}
\citep{stoia08}
\citep{rieser05}
\citep{schegloff87b}
\citep{Grice75}
\citep{rodriguez04}

\citep{clark96}
\citep{ginzburg09}
\citep{levinson87}

\citep{BBW06}
\citep{KR99}
\citep{KR97}
\citep{PT87}
\citep{S08}
\citep{GLYC}
\citep{LSV}
\citep{TALARIS}
\citep{intohylo}

\citep{citeulike:386317}

\paragraph{BLA BLA}

La Universidad tiene un importante rol que desempe�ar en este nuevo espacio digital, multicultural y pluriling��stico llamado "Cibercultura", promoviendo una verdadera interacci�n entre:

   	El gobierno   	Las empresas   Las Universidades y Centros de Investigaci�n y Desarrollo
De esta manera se contribuye a la formaci�n de capital intelectual capaz de comprender los procesos de la nueva sociedad, desde un enfoque genuinamente interdisciplinario.
Las nuevas tecnolog�as ofrecen al presente y futuro de la Argentina, un sinf�n de posibilidades, por el hecho de atravesar todos los sectores, pol�ticos, econ�micos y sociales, siendo su "desarrollo estrat�gico" un pilar fundamental para crecimiento equitativo y sustentable del pa�s.

\url{http://www.uai.edu.ar/ciiti/2009/bsas/congreso.html}

\fixme{Cambiar el titulo de la bibliografia}
\bibliographystyle{plainnat}
\bibliography{pict09}

% % % % % % % % % % % % % % % % % % % % % % % % % % % % % % % % % % % % % %
% \newpage
% \MySubSection{10.  Requested Budget} We request funding for 3 years,
% for a full-time postdoc and a part-time scientific programmer.
%
% \addtocounter{footnote}{-1}
% \begin{center}\small
%     \begin{minipage}{\linewidth}
%     \renewcommand{\thefootnote}{\thempfootnote}
%     \begin{tabularx}{\linewidth}{@{}lXX@{= kf }r@{}}
%     \multicolumn{4}{@{}l}{Post-doc(s)} \\
%     \hline
%     a) & aanstelling (kf 100 pj, max 3 jaar) & 1 fte $\times$
%       3 $\times$ kf 100 & 300.00\\
%     b) & persoonsgebonden reisbudget & aantal fte $\times$ kf 7.35 &
%     7.35\\
%     c) & \multicolumn{2}{l}{additioneel reisbudget} & \\
% %     \footnote{YYY}} & \\
%     d) & projectgebonden appratuur/software \\
%     \cline{4-4}
%     \multicolumn{2}{@{}l}{Subtotaal Postdocs} & & 307.35
%     \\
%     \\
%     \multicolumn{4}{@{}l}{Overig (technisch personeel/programmeurs)} \\
%     \hline
%     &programmeur & 0.5 fte $\times$ 3 $\times$ kf 90 &  135.00
%     \\
%     \\
%     \multicolumn{4}{@{}l}{Investeringen: tussen kf 100 en kf 250} \\
%     \hline
%      & & & 0.00
%     \\
%     \\
%     \multicolumn{4}{@{}l}{Investeringen: tussen kf 250 en Mf 2} \\
%     \hline
%     a)& Totale investeringen & & 0.00 \\
%     b)& Eigen bijdrage & & 0.00 \\
%     c)& Gevraagde subsidie & & 0.00
%     \\
%     \\
%     \multicolumn{4}{@{}l}{\textbf{TOTAAL AANVRAAG}}\\
%     \multicolumn{2}{@{}l}{Totale kosten OiO's + Postdoc(s) + Overig} &
%     & 442.35\\
%     \multicolumn{2}{@{}l}{Totaal gevraagde subsidie voor investeringen} &
%     & 0.00
%     \end{tabularx}
%     \end{minipage}
% \end{center}



\end{document}
