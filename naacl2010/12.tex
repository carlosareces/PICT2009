
The PLN\footnote{\url{http://www.cs.famaf.unc.edu.ar/~pln}} research group, 
in which the describe scientific project will be carried out, was funded in 2005.
Te group is developing an important role in human resource training,
delivering courses to undergraduate and postgraduate studies at the FaMAF and
other universities. It also works in the development of various research
projects and integration with other groups in the region, both in Argentina and
Chile, Brazil and Uruguay. 

To begin with, some members of the group specialize in computational
logic, particularly in the theoretical and applied study of languages for
knowledge representation (e.g., modal, hybrid and description logics). They
have also developed  automated theorem provers for these
languages\footnote{\url{http://www.glyc.dc.uba.ar/intohylo/}}. In relation with 
the study of knowledge representation, they have also investigated and developed
algorithms for generating referring expressions~\cite{AKS08}.

The second line of research of the PLN group that is relevant for this project
is context-based evaluation. In particular, we have proposed a evaluation model
for machine translation (MT) systems which relates the context of use of the
system to potentially important quality
characteristics~\cite{estr:impr08,estr:femt09}. This model is general enough to
be applied to the other systems that produce natural language such as the system
proposed in this paper. 
% Given our background with MT systems we have experience in evaluating and
% comparing natural language output produced in different languages (Spanish and
% English in particular), which will be relevant for the development of the
% language tutor described in Section~\ref{applications}. 
Finally, we have
experience developing and evaluating   
 multimodal corpora as is the corpora described in Section~\ref{description}~\cite{multieval}.

The third line of research that is relevant for this project is pragmatics. In
this area we have implemented a conversational agent which is able to infer and
negotiate conversational implicatures using inference tasks such as
classical planning and planning under incomplete information~\cite{benotti09b}.
We have also investigated the inference of conversational implicatures triggered
by comparative utterances~\cite{benotti09a}. Recently we have done corpus-based
work, which shows what kinds of implicatures are inferred and negotiated by
human dialogue participants during a task situated in a 3D
virtual environment~\cite{benotti09c}. 

There are other lines of research in the PLN group that are not directly related
to this project at this stage but might become more relevant in the
future: grammar induction, text mining, statistical syntactic analysis
and ontology population from raw text. 





