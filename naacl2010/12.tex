%\MySubSection{12. Composici\'on del Grupo de Investigaci\'on y Areas de
% Especializaci\'on}

El Dr.\ Carlos Areces se especializa en l\'ogica
computacional,
en particular en el estudio te\'orico y aplicado de lenguajes para la
representaci\'on del conocimiento (e.g., l\'ogicas modales, h\'ibridas y
para la descripci\'on).  Las publicaciones
m\'as importantes sobre aspectos te\'oricos en estos lenguajes
son~\cite{ABM01,arec:hybr05b}.  Ha desarrollado tambi\'en demostradores
autom\'aticos para estos lenguajes~\cite{ANR01,arec:hylo02a,AG06,Hoffmann2007}
accessibles en \url{http://www.glyc.dc.uba.ar/intohylo/}.
Ha trabajado tambi\'en en el dise\~no de algoritmos para la generaci\'on de
expresiones referenciales~\cite{AKS08}, y en el estudio de lenguajes l\'ogicos
adecuados para la representaci\'on sem\'antica~\cite{AF08}.

La Dra.\ Paula Estrella ha realizado su tesis doctoral en la 
evaluaci\'on contextual de sistemas de TA~\cite{estr:impr08,estr:femt09},
proponiendo 
un modelo generalizable y aplicable a otras \'areas del PLN, como se muestra en 
\cite{Miller2008}. Tambi\'en ha trabajado en el desarrollo y mejora de sistemas
de TA 
estad\'istica desde y hacia varios idiomas, por ejemplo el
Espa\~nol~\cite{estr:expe05} 
y entre los idiomas Franc\'es, Ingl\'es, Japon\'es y Arabe
\cite{rayner-EtAl:2009:GEAF}. 
Adem\'as particip\'o en proyectos que involucran la creaci\'on y evaluaci\'on de
corpora 
correspondientes a anotaciones multimodales~\cite{pope:estr07} y trabaj\'o en un
prototipo 
para la evaluaci\'on autom\'atica de anotaciones de di\'alogos multimodales
\cite{multieval}, 
investigando tambi\'en la aplicaci\'on de m\'etricas provenientes de otras
\'areas; el 
prototipo est\'a disponible p\'ublicamente en
\url{http://www.issco.unige.ch:8080/multieval/}.


La Lic.\ Luciana Benotti ha trabajado en el desarrollo de un sistema de
di\'alogo completo situado en una aventura de
texto~\cite{benotti09b}. La implementaci\'on abarca desde tareas de an\'alisis
sint\'actico hasta tareas pragm\'aticas. Su investigaci\'on se ha enfocado en
tareas pragmaticas que requieren de tareas de inferencia practicas. En
particular ha utilizado la tarea de inferencia de planning clasico y de
planning con informacion incompleta. Luego extendi\'o el sistema de di\'alogo
desarrollado por el grupo de David Traum~\cite{TraumSigdialBook08} para
analizar pragmaticamente oraciones comparativas~\cite{benotti09a}. Finalmente,
trabaj\'o en el an\'alisis emp\'irico de corpora de di\'alogo humano-humano
utilizando
diferentes t\'ecnicas de planning para precedir sub-dialogos de
reparaci\'on~\cite{benotti09c}.








