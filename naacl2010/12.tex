
% En la FaMAF existe un grupo de PLN desde 2005
% (\url{http://www.cs.famaf.unc.edu.ar/~pln}). Este
% grupo est\'a desarrollando una importante labor de formaci\'on de recursos
% humanos, con el dictado 
% de cursos de grado y de postgrado en la FaMAF y en otras universidades del
% pa\'is.
% Tambi\'en trabaja en el desarrollo de diversos proyectos de investigaci\'on y
% en la integraci\'on
% con otros grupos de la regi\'on, tanto en Argentina como en Chile, Brasil y
% Uruguay. Este proyecto 
% de investigaci\'on se integra al programa del grupo de procesamiento de
% lenguaje % natural de la
% FaMAF.

The PLN~\footnote{\url{http://www.cs.famaf.unc.edu.ar/~pln}} research group, 
in which the describe scientific project will be carried out, was funded in 2005.
Te group is developing an important role in human resource training,
delivering courses to undergraduate and postgraduate studies at the FaMAF and
other universities. It also works in the development of various research
projects and integration with other groups in the region, both in Argentina and
Chile, Brazil and Uruguay. 


% El Dr.\ Carlos Areces se especializa en l\'ogica
% computacional,
% en particular en el estudio te\'orico y aplicado de lenguajes para la
% representaci\'on del conocimiento (e.g., l\'ogicas modales, h\'ibridas y
% para la descripci\'on).  Las publicaciones
% m\'as importantes sobre aspectos te\'oricos en estos lenguajes
% son~\cite{ABM01,arec:hybr05b}.  Ha desarrollado tambi\'en demostradores
% autom\'aticos para estos lenguajes~\cite{ANR01,arec:hylo02a,AG06,Hoffmann2007}
% accessibles en \url{http://www.glyc.dc.uba.ar/intohylo/}.
% Ha trabajado tambi\'en en el dise\~no de algoritmos para la generaci\'on de
% expresiones referenciales~\cite{AKS08}, y en el estudio de lenguajes l\'ogicos
% adecuados para la representaci\'on sem\'antica~\cite{AF08}.

There are three main lines of research in our team that are relevant for this
project. 

To begin with, some members of the group specialize in computational
logic, particularly in the theoretical and applied study of languages for
knowledge representation (e.g., modal, hybrid and description logics). They
have also developed  automated theorem provers for these
languages\footnote{\url{http://www.glyc.dc.uba.ar/intohylo/}}. In relation with 
the study of knowledge representation, they have also investigated and developed
algorithms for generating referring expressions~\cite{AKS08}.

% La Dra.\ Paula Estrella ha realizado su tesis doctoral en la 
% evaluaci\'on contextual de sistemas de TA~\cite{estr:impr08,estr:femt09},
% proponiendo 
% un modelo generalizable y aplicable a otras \'areas del PLN, como se muestra
% en 
% \cite{Miller2008}. Tambi\'en ha trabajado en el desarrollo y mejora de
% sistemas de TA 
% estad\'istica desde y hacia varios idiomas, por ejemplo el
% Espa\~nol~\cite{estr:expe05} 
% y entre los idiomas Franc\'es, Ingl\'es, Japon\'es y Arabe
% \cite{rayner-EtAl:2009:GEAF}. 
% Adem\'as particip\'o en proyectos que involucran la creaci\'on y evaluaci\'on
% de % corpora 
% correspondientes a anotaciones multimodales~\cite{pope:estr07} y trabaj\'o en
% un % prototipo 
% para la evaluaci\'on autom\'atica de anotaciones de di\'alogos multimodales
% \cite{multieval}, 
% investigando tambi\'en la aplicaci\'on de m\'etricas provenientes de otras
% \'areas; el 
% prototipo est\'a disponible p\'ublicamente en
% \url{http://www.issco.unige.ch:8080/multieval/}.

Paula Estrella has done his PhD at the contextual assessment of MT
systems~\cite{estr:impr08,estr:femt09}, suggesting a model generalizable and
applicable to other areas of NLP, as shown in~\cite{Miller2008}. She has also
worked on the development and improvement of
statistical MT system and from various languages, such as
Spanish~\cite{estr:expe05} and between the languages English, French, Japanese
and Arabic~\cite{rayner-EtAl:2009:GEAF}. Also participated in projects involving
the development and evaluation
corpora annotation for multimodal~\cite{pope:estr07} and worked on
a prototype for automatic evaluation of multimodal dialogues
annotations~\cite{multieval} also investigating the application of metrics from
other areas.\footnote{\url{http://www.issco.unige.ch:8080/multieval}}

% La Lic.\ Luciana Benotti ha trabajado en el desarrollo de un sistema de
% di\'alogo completo situado en una aventura de
% texto~\cite{benotti09b}. La implementaci\'on abarca desde tareas de an\'alisis
% sint\'actico hasta tareas pragm\'aticas. Su investigaci\'on se ha enfocado en
% tareas pragmaticas que requieren de tareas de inferencia practicas. En
% particular ha utilizado la tarea de inferencia de planning clasico y de
% planning con informacion incompleta. Luego extendi\'o el sistema de di\'alogo
% desarrollado por el grupo de David Traum~\cite{TraumSigdialBook08} para
% analizar pragmaticamente oraciones comparativas~\cite{benotti09a}. Finalmente,
% trabaj\'o en el an\'alisis emp\'irico de corpora de di\'alogo humano-humano
% utilizando
% diferentes t\'ecnicas de planning para precedir sub-dialogos de
% reparaci\'on~\cite{benotti09c}.

Luciana Benotti has investigated the addition of pragmatic abilities, in
particular the inference and negotiation of conversational implicatures, into
dialogue systems. For this purpose, she has implemented a conversational agent
(using a generic parser, realizer and knowledge base manager) which  is situated
in a text-adventure game, and she has extended it using inference tasks such as
classical planning and planning under incomplete information~\cite{benotti09b}.
She has also investigated the inference of conversational implicatures triggered
by comparative utterances~\cite{benotti09a}. She has recent
corpus-based work, which shows what kinds of implicatures are inferred and
negotiated by human dialogue participants during a task~\cite{benotti09c}. 





