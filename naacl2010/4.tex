This project aims to achieve a balance between a system which is at the 
same time sufficiently generic to be applicable in different areas, and 
specific enough in order to benefit from the efficient use of existing 
techniques for knowledge management, planning and natural language processing.

The resulting system could be used in different applications. It could be used, 
for example, in electronic commerce providing 
assistance on the web for buying a product; in technical support helping a
novice user to fix a problem; or as a virtual tutor for learning foreign 
languages. 

Implementing a dialogue system is a multidisciplinary effort. In this project
we will focus our research efforts in the following areas:

\paragraph{Pragmatics} is an interdisciplinary field which
integrates insights from linguistics (e.g., 
conversational implicatures~\cite{grice75}),
sociology (e.g., conversational analysis~\cite{schegloff87b}) and
philosophy (e.g., theory of speech acts~\cite{austin62}). It aims to explore how
the context (in which a conversation is situated) contributes to the meaning (of
everything that is said during that conversation). The meaning conveyed during
a conversation depends not only on linguistic information (entities in focus,
grammatical and morphological rules, etc.) but also on extralinguistic
information (physical situation of conversation, previous
experiences of speakers, etc.). As a result, the same sentence may mean
different things in different contexts. The area of pragmatics studies the
process by which a sentence is disambiguated using its context. In pragmatics,
there is a distinction between sentence (grammatical form that
the linguistic act takes) and utterance (sentence plus its context). The ability
to understand a sentence using its context, i.e. the ability to understand an
utterance, is called pragmatic competence. Explaining the pragmatic competence
involves explaining how a person makes inferences about a sentence and its
context to properly interpret the meaning that the speaker intends to convey.

A dialogue system needs to have pragmatic capabilities in order to interact in a
natural way with its users. In particular, a system must define what kind of
contextual information should be represented; and what inference tasks on a
sentence and context are necessary in order to interpret an utterance. Properly 
identifying this issues will have a crucial impact on the performance of a system
like the one we propose to develop. In such a system it is indispensable that
sentences makes explicit the right amount of information: too much information
will delay and bore the user, but if the information is not enough the user 
will not know how to perform the task and make mistakes.

One of the major contributions of the project in this area will be a virtual 
laboratory for pragmatic theories that consist of a controlled environment 
for studying the
interaction set in a world where physical actions and language intermingle.

The prototype will allow unidirectional investigate the impact of different
policies to give instructions (post or pre-order on the tree structure of the task)
on the successful completion of the task. Such studies have been performed
previously (e.g.,~\cite{foster-etal-ijcai2009}) but assuming a predetermined
task. Since our prototype allows the specification of the virtual world,
possible actions and the goal to achieve, we can determine when the impact
resulting from the use of a particular policy is dependent on the task.

The prototype will allow us to investigate the two-way phenomenon repair the
short and long term, designing and evaluating a prediction system repairs
contextualized. These repairs are usually caused by conversational implicatures.
Modeling these implicatures in a generic dialogue system is difficult. However,
since the present prototype provides a situated interaction and restricted to
the virtual world will be possible to test the relationship between
implicatures, the type of repairs that give rise and inference tasks necessary
to predict them. 

\paragraph{Inference} can be understood as any operation that
transforms implicit information in explicit information. This definition is
general enough to cover tasks ranging from logical inference (i.e., deduction in
a formal language) to inference tasks common in AI (e.g., planning and
non-monotonic inference), as well as statistical operations (e.g. obtaining
estimators on a data set). A dialogue system has to
continually perform inference operations. On the one hand, inference is needed
to interpret the information received from the user and incorporate it to 
the data repository of the system. On the other hand, it is also needed in order 
to decide how much of the
available information should be conveyed when addressing the user.

The very problem of deciding what kind of logical representation and what type
of inference to use in a given situation is complex (propositional logic vs.
first-order logic, validity vs. model checking, logical inference vs.
statistical inference). Independently of which type of inference it is used, they are usually computationally expensive. The challenge here is to find the appropriate
balance between the expressivity of the representation formalism and the
cost of the required inference methods.

The main contribution of the project in the area of logic and inference is in
the design, development and planning study of algorithms. A typical planning
system takes three inputs - an initial state, a specification of possible
actions and an expected objective - and returns a sequence of actions (a plan)
that when applied sequentially to the initial state, ends in a state that
satisfies the goal order. Different methods to obtain a scheme have been studied
(forward chaining, backward chaining, coding in terms of propositional
satisfiable, etc..) And are currently deployed systems that can solve this task
efficiently.

However, most of these systems assume conditions that simplify the problem
(deterministic atomic time, complete, absence of a background theory, etc.). And
return a single plan. In the course of the project will investigate algorithms
that eliminate some of the simplifications (in particular, investigate the case
of planning with incomplete information and theory based on a background) and
also provide extended services planning (return on alternative plans, plans
minimum conditional plans, incomplete plans, possible actions in a given state,
etc.).

\paragraph{Evaluation:} The evaluation of natural language generation systems
is one of the most difficult tasks in the area of NLP. A given concept can be 
expressed in many different ways, all of them correct. 
Hence, it is not possible to determining the quality of a
generated sentence simply by, for example, comparing 
the result with a gold standard. The problem of absence of gold standards is shared
with another area of the NLP, namely Machine Translation (MT), for which
various evaluation methodologies, both direct and indirect, have been proposed. 
Direct methods applies a metric to the text
generated by the system, while indirect methods evaluates the performance of the
system through the use of the generated text to perform some task. However, 
none of these methods is a standard and accepted methodology which has
been generally proven to be effective.

Since what is being evaluated in this project is a system that interacts via the
generation of natural language instructions, we can determine its performance
through quantitative metrics (such as average task completion time), 
qualitative metrics (e.g., general user satisfaction) and metrics based on the
context (how well the system addressed the user needs in particular situations).
Given our previous experience, it is particularly
interesting for us to study the portability of evaluation
techniques from the domain of MT
and multimodal human-computer interaction to the evaluation of the system
proposed in this project.

In this area it is expected that one of the main contributions is the
integration of assessment techniques from different areas on a methodology for
evaluating dialog systems for virtual environments so as to estimate its
usability and effectiveness. This methodology could be used both to determine
whether a system is suitable for a task type and user, and to compare the
performance of different systems of the same type.

Another contribution is the study and application of appraisal standards
developed software systems, creating a standardized quality model and proposing
a set of appropriate metrics to assess each of the aspects of the model. This
work will also include a thorough study of several metrics in late in the model
include some tips on quality metrics chosen, this study is considered a
meta-evaluation of the proposed model.

Finally, the annotated corpus of human-human interaction, over the corpus of
human-machine interaction collected during the project be made public. Such
corpora will serve, for example, to design more general platform for evaluating
dialog systems, which go beyond the aspects evaluated by currently existing
platforms as GIVE.

\paragraph{Impact in the Argentinean Landscape:} The topics investigated in the framework of this project are of relevance in the
current Argentine landscape for at least two reasons.

On the one hand, the project integrates and develops various key aspects of the
area of computational linguistics (syntax, semantics, pragmatics,
representation, inference, evaluation). The area of computational linguistics
and its application to automatic processing of natural language have had a major
international development in recent years, systems with applications like Web
search, translation systems and abstract machines, voice interfaces, etc. .
However, the area is almost nonexistent today in Argentina. This project will be
among the contributions that aim to reverse this situation. Moreover, the
ultimate goal of this project is to investigate the use of the platform
developed in the area of distance education (specifically, as a platform for
language learning).

Clearly, distance education is a valuable resource to overcome the problem of
centralization of educational resources in the country. However, developing
appropriate tools in this specific area is difficult.

To develop distance learning systems is essential to model the user's learning
progress. This requires a system capable of being aware of the evolution of the
user, and taking into account their achievements and their problems. This type
of interaction between user and system can be modeled as a dialogue, which
records the acquired knowledge context (the user about the course material, and
system to user). Managing this kind of dialogue is particularly interesting,
since the system must be able to interpret requirements, and generate
appropriate responses, non-experts whose knowledge evolves during the
interaction. Moreover, the system must be able to properly represent both the
information concerning the course material, such as information concerning the
evolution of the user. For example, the system must be able to diagnose what
part of the course material should be reviewed from the wrong answers of the
user. Finally, the system must be able to evaluate the user interaction in order
to decide which learning objectives were achieved. The theoretical and practical
results obtained during the project directly contribute to solving these
problems.




