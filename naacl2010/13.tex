The members of the PLN in general and the authors of this paper in particular
have several collaborations with national international research groups in CL
and related fields that are relevant for this project. 
 
We have collaborated and continue to do so with the \textit{Multilingual
Information
Processing Department at the University of Geneva} (ISSAC/TIM) and of the
\emph{Database management
and meeting
analysis} group of the Swiss National Center of Competence in Research  
\emph{Interactive Multimodal Information Management}. The joint research
project includes the evaluation of NLP systems and the
development of multilingual and multimodal HLT systems. Furthermore we
collaborate with some members of the \textit{Pervasive Artificial
Intelligence} group of the University of Fribourg. 

We currently collaborate with
the TALARIS\footnote{\url{http://talaris.loria.fr/}}
group of the \emph{Laboratoire Lorrain de Recherche en Informatique et ses
Applications (LORIA)}. The main research topic at TALARIS is computational
linguistics
with strong emphasis on semantics and inference. In the framework of this
collaboration we are participating
(at the moment of writing this paper) in the 2010 edition of the GIVE
Challenge. In the process of designing the systems that will participate in
the challenge we jointly investigated the use of different referring strategies
in situated instruction
giving~\cite{amoia10}. 

We have intensively collaborated with the GLyC group (\textit{Grupo de L\'ogica,
Lenguaje y Computabilidad})\footnote{\url{http://www.glyc.dc.uba.ar/}} on
knowledge representation and inference topics~\cite{AG06,AFFM08}. GLyC is part
of the Computer Science Department of the Universidad de Buenos Aires. 

We have also collaborated with the Virtual Humans group of the Institute for
Creative Technologies\footnote{\url{http://ict.usc.edu/projects/virtual_humans}}
from the University of Southern California. In particular we computationally
modeled the inference of conversational implicatures triggered by comparative
utterances~\cite{benotti09a}. The Institute for Creative Technologies offers
Internship programs every year that we plan to use in order to strengthen our
collaboration by sending PhD and Masters exchange students from the University
of C\'ordoba.

One of the authors has collaborated (organizing seminars in CL) with the
research group in Artificial Intelligence from the Universidad Nacional del
Comahue.\footnote{\url{http://www.uncoma.edu.ar/}} This is a University with
low resources. We plan to continue collaborating with them through student and
professors exchange programs financed by the Argentinean government. 

We are organizing the first school in Computational Linguistics in
Argentina, ELiC\footnote{\url{http://www.glyc.dc.uba.ar/elic2010/}} which will
take place in July 2010 in the Universidad de Buenos Aires. The main goals of
ELiC are the following. (a) Introduce the field of CL to graduate students in
Argentina. (b) Start a yearly event in CL in Argentina that will change
location every year inside the country or in other Latin-American countries.
(c) Organize a one-day workshop as a panel for researchers in the area
immediately after or before the school (starting from ELiC 2011). ELiC 2010 is
offering 10 student travel grants (for students not living in Buenos Aires)
due to the generous support of the North American Chapter of the Association for
Computational Linguistics (NAACL). ELiC 2010 will be co-located with the Escuela
de Ciencias Inform\'aticas
(ECI)\footnote{\url{http://www.dc.uba.ar/events/eci/2009/eci2009}}. ECI has a
long standing reputation as a high-quality winter school in Computer Science in
Argentina which is organized every year. ELiC 2010 will take advantage of the
organizational infrastructure of ECI, such as advertising, registration,
preparation of reading material, and classrooms; moreover, ECI will also offer
fee waivers to selected ELiC students.

The PLN group has being part of an international project of collaboration on
ontology population from raw text which might be relevant when trying to extend
our GIVE ontologies to new domains. The project was called 
MICROBIO\footnote{\url{http://www.microbioamsud.net/}}, was granted by
Stic-Amsud and finished last year. This project included universities and
laboratories from France, Brazil, Uruguay and Chile. 

Finally, the PLN group is planning to organize a workshop on Computational
Linguistics collocated with the Ibero-American Conference on Artificial
Intelligence (IBERAMIA
2010).\footnote{\url{http://cs.uns.edu.ar/iberamia2010/}} 

\fixme{add a closing paragraph mentioning PICT funding}