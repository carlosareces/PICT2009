This section briefly describes the interaction of the authors with other
international research groups. 

Dr. Areces was member of the TALARIS group, part of the \emph{Laboratoire
Lorrain de Recherche en Informatique et ses Applications}. The main research
topic at TALARIS is computational linguistics with strong emphasis on semantics
and inference and he is also collaborating with researchers and institutions
specialized in these areas, in particular with \textit{Patrick Blackburn}
(researcher in modal logics and computational linguistic applications), the GLyC
group (\textit{Grupo de L\'ogica, Lenguaje y Computabilidad}, part of the
Computer Science Department of the University of Buenos Aires) and \textit{Ron
Petrick} (developer of PKS, one of the few planning systems able to construct
conditional plans in the presence of incomplete knowledge). 

\fixme{Eliminar nombres y extender enfatizando la factibilidad de las
colaboraciones y linkeando con los temas del proyecto}

Dr.\ Estrella is a former member of the \textit{Multilingual Information
Processing Department at the University of Geneva} (ISSAC/TIM) and of the
\emph{Database management
and meeting
analysis} group of the Swiss National Center of Competence in Research                                                                                                                                                                                                                                                                                                                                                                                                                                                                                                                                                                                                                                                                                                                                            
\emph{Interactive Multimodal Information Management}. She continues
collaborating with TIM/ISSAC (focusing on the evaluation of NLP systems, the
development of multilingual HLT systems and multimodality) and is currently
collaborating with some members of the \textit{Pervasive Artificial
Intelligence} group of the University of Fribourg. 

We currently collaborate with the TALARIS\footnote{\url{talaris.loria.fr/}}
group of the \emph{Laboratoire Lorrain de Recherche en Informatique et ses
Applications}. In the framework of this collaboration we are participating
(at the moment of writing this paper) in the 2010 edition of the GIVE
Challenge. In the process of desigining the systems that will participate in
the challenge we jointly investigated and conducted several evaluation studies
on the use of different referring strategies in situated instruction
giving~\cite{amoia10}. 

\fixme{explain work with traum}
We have also collaborated with the Virtual Humans group of the Institute for
Creative Technologies\footnote{\url{http://ict.usc.edu/projects/virtual_humans}}
from the University of Southern California~\cite{benotti09a}. The Institute for
Creative Technologies offers Internship programs every year that we plan to use
in order to strenghten our collaboration by sending PhD and Masters exchange
students from the University of C\'ordoba.

% Dr. Benotti has recently obtained her PhD at the TALARIS group and is
% currently
% collaborating with groups working on NL generation and dialogue systems, such
% as
% \textit{Alexander Koller} (organizer of the GIVE challenge and coordinator for
% the implementation of the GIVE platform), \textit{David Traum} (researcher on
% dialogue systems and head of the Institute for Creative Technologies),
% \textit{Claire Gardent} (vice-director of TALARIS, working on NLG, lexical
% acquisition and grammar development) and \textit{Agust\'in Gravano}
% (researcher
% on prosodic variation in spoken dialogue).
