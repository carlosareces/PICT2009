
% En este proyecto se propone implementar un sistema de di\'alogo que genere
% autom\'aticamente
% instrucciones en lenguaje natural para ayudar a un usuario a cumplir una tarea
% determinada en un entorno virtual 
% en tres dimensiones. Para este fin, se propone investigar temas fundamentales
% sobre la
% in\-teracci\'on hombre-m\'aquina 
% en entornos virtuales. Como resultado de este proyecto, se espera la
% implementaci\'on, evaluaci\'on y aplicaci\'on de un sistema prototipo. Las
% tres
% \'areas principales en las que se pueden clasificar los resultados del
% proyecto
% son: (1) pragm\'atica de la interacci\'on, (2) representaci\'on
% de la informaci\'on e inferencia, (3) evaluaci\'on de sistemas
% de di\'alogo. Una vez obtenido
% un prototipo, se planea su aplicaci\'on a la tarea espec\'ifica del 
% aprendizaje de idiomas, utilizando el sistema como un ``profesor de idiomas"
% virtual.

The goal of this project is to implement a dialogue system which automatically
generates natural language instructions, in order to help a user to fulfill a
particular task in a 3D virtual environment. To this aim, we investigate
fundamental issues about human-computer interaction in virtual environments. We
can classify the expected results of the project in three areas: (1) pragmatics
of interaction, (2) information representation and inference, (3) evaluation of
dialogue systems. Once the working prototype is finished we will addapt it to
the specific task of language learning, using the system as a virtual language
teacher. 

% En una primera etapa investigaremos un modelo de interacci\'on 
% ling\"u\'istica unidireccional
% (i.e., el flujo de informaci\'on ling\"u\'istica ser\'a desde el sistema hacia
% el usuario).
% En etapas subsiguientes, el modelo ser\'a extendido para
% permitir intercambio ling\"u\'istico bidireccional, por ejemplo, para que el
% usuario pida clarificaciones o ayuda al sistema.

In a first stage we will investigate a model of unidirectional linguistic
interaction (i.e., linguistic information flows only from the system to the
user). In subsequent stages, the model will be extended to allow bidirectional
language exchange, for example, the user may ask for clarification or help to
the system.

% Dise\~nar la arquitectura de un sistema de di\'alogo presenta
% desaf\'ios tanto te\'oricos como pr\'acticos. En lo te\'orico, se necesitan
% heur\'isticas que gobiernen la interacci\'on en cuanto a qu\'e decir,
% cu\'ando, y c\'omo (teniendo en cuenta el contexto actual). Adem\'as, se debe
% contar con m\'etodos
% de inferencia que permitan al sistema adaptarse en funci\'on de la situaci\'on
% actual, para alcanzar un objetivo predefinido.
% La complejidad del problema te\'orico se
% refleja, en lo pr\'actico, en un sistema de m\'ultiples componentes: un
% generador de lenguaje natural, un sistema de planning, un entorno de 
% interacci\'on en tres dimensiones, etc.
% Dise\~nar e implementar todos estos componentes requerir\'ia un esfuerzo
% prohibitivo. En este proyecto adaptaremos herramientas ya
% implementadas y disponibles libremente para el prototipado de sistemas de este
% tipo, como la plataforma \emph{Generating Instructions in Virtual
% Environments} (GIVE).

Designing the architecture of a dialogue system presents both theoretical and
practical challenges. On the theoretical side, heuristics are needed to govern
the interaction in terms of what to say, when, and how (given the current
context). In addition, there must be inference methods that allow the system to
adapt to the current situation, to reach a predefined goal. The complexity of
the theoretical issues is reflected, in practice, in a system of
multiple components: a natural language generator, a planner,
a 3D interactive environment, to mention a few. Designing
and implementing all these components require a prohibitive effort. This project
will adapt tools already implemented and freely available for prototyping such
systems, such as the platform \emph{Generating Instructions in Virtual
Environments (GIVE)}\cite{byron09}. 

% La calidad de cada una de las capacidades del sistema afecta la
% percepci\'on que el usuario tiene de \'este. Por lo tanto, es imperativo
% evaluar cada una de ellas.
% Se planean adaptar y aplicar distintas t\'ecnicas y m\'etricas de evaluaci\'on
% del \'area de ingenier\'ia de software y de diversas \'areas del procesamiento
% de lenguaje natural (por ejemplo, la traducci\'on autom\'atica).

The quality of each of the capabilities of the system affects the perception
that users have of it. Therefore, it is imperative to evaluate each of such
capabilities. We plan to adapt and apply different evaluation techniques and
metrics from the area of software engineering and various areas of
natural language processing (e.g. machine translation).
