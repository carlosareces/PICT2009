
The goal of this project is to implement a dialogue system which automatically
generates instructions in order to help a user to fulfill a
given task in a 3D virtual environment. In this context, we will investigate
fundamental issues about human-computer interaction in virtual environments. We
can classify the expected results of the project in three areas: pragmatics
of interaction;  information representation and inference; and evaluation of
dialogue systems. Once a working prototype is finished, we will adapt it to
the specific task of language learning using the system as a virtual language
teacher. 

Initially, we will investigate a model of unidirectional linguistic
interaction (i.e., linguistic information flows only from the system to the
user). In subsequent stages, the model will be extended to allow bidirectional
language exchange. For example, the user may query the system for clarifications or help.

The quality of each of the capabilities of the system affects the perception
users will have of it. It is imperative to perform extensive evaluation. 
We plan to adapt and apply different evaluation techniques and
metrics from the area of software engineering and various areas of
natural language processing (e.g. machine translation) to assess the proper 
performance of the system.

The architecture of the envisioned dialogue system presents both theoretical and
practical challenges. On the theoretical side, heuristics are needed to govern
the interaction in terms of what to say, when, and how (given the current
context). In addition, the system should implement inference methods to 
adapt to the current situation and reach a predefined goal. The complexity of
the theoretical issues is reflected, in practice, in a system of
multiple components: a natural language generator, a planner,
a 3D interactive environment, to mention a few. Designing
and implementing all these components from scratch would 
require a prohibitive effort. Instead we will adapt tools already implemented 
and freely available for prototyping this kind of 
systems, such as the platform \emph{GIVE, Generating Instructions in Virtual
Environments}~\cite{byron09}. 


