%
% File naaclhlt2010.tex
%
% Contact: nasmith@cs.cmu.edu

\documentclass[11pt,letterpaper]{article}
\usepackage{naaclhlt2010}
\usepackage{times}
\usepackage{latexsym}
\usepackage[draft]{fixme}
\usepackage{url}
\setlength\titlebox{6.5cm}    % Expanding the titlebox

%% Define a new 'leo' style for the package that will use a smaller font.
\makeatletter
\def\url@leostyle{%
  \@ifundefined{selectfont}{\def\UrlFont{\sf}}{\def\UrlFont{\scriptsize\ttfamily}}}
\makeatother
%% Now actually use the newly defined style.
\urlstyle{leo}


\title{Dialogue Systems for Virtual Environments}

\author{Luciana Benotti, Paula Estrella, Carlos Areces \\
 Grupo de Procesamiento de Lenguaje Natural (PLN) \\
Secci\'on de Ciencias de la Computaci\'on \\
Facultad de Matem\'atica, Astronom\'ia y F\'isica \\
Universidad Nacional de C\'ordoba, Argentina \\
  {\tt surname@famaf.unc.edu.ar}
%   \And
%   TALARIS Team \\
%   INRIA Nancy Grand Est, France \\
%   {\tt areces@loria.fr}
}

\date{}

\begin{document}
\maketitle
\begin{abstract}
In this project we present an on-going research project carried out in the Universidad Nacional de C\'ordoba in Argentina. This project investigates theoretical and practical research questions related to the development of a dialogue system situated in a virtual environment. We describe, in general, the PLN research group in which this project is being developed and, in particular, we spell out the areas of expertise of the authors. Moreover, we discuss past, current and future collaborations of the research group which are relevant for this project.
\end{abstract}

\section{Introduction}\label{intro}

% En este proyecto se propone implementar un sistema de di\'alogo que genere
% autom\'aticamente
% instrucciones en lenguaje natural para ayudar a un usuario a cumplir una tarea
% determinada en un entorno virtual 
% en tres dimensiones. Para este fin, se propone investigar temas fundamentales
% sobre la
% in\-teracci\'on hombre-m\'aquina 
% en entornos virtuales. Como resultado de este proyecto, se espera la
% implementaci\'on, evaluaci\'on y aplicaci\'on de un sistema prototipo. Las
% tres
% \'areas principales en las que se pueden clasificar los resultados del
% proyecto
% son: (1) pragm\'atica de la interacci\'on, (2) representaci\'on
% de la informaci\'on e inferencia, (3) evaluaci\'on de sistemas
% de di\'alogo. Una vez obtenido
% un prototipo, se planea su aplicaci\'on a la tarea espec\'ifica del 
% aprendizaje de idiomas, utilizando el sistema como un ``profesor de idiomas"
% virtual.

This project aims to implement a dialogue system to automatically generate
natural language instructions to help a user to fulfill a particular task in a
3D virtual environment. For this purpose, we investigate fundamental issues
about human-computer interaction in virtual environments. As a result of this
project a prototype system will be implemented and evaluated. We can
classify the results of the project in three areas: (1) pragmatics of
interaction, (2) information representation and inference, (3) evaluation of
dialogue systems. Once the working prototype is finished it will be applied to
the specific task of language learning, using the system as a virtual language
teacher. 

% En una primera etapa investigaremos un modelo de interacci\'on 
% ling\"u\'istica unidireccional
% (i.e., el flujo de informaci\'on ling\"u\'istica ser\'a desde el sistema hacia
% el usuario).
% En etapas subsiguientes, el modelo ser\'a extendido para
% permitir intercambio ling\"u\'istico bidireccional, por ejemplo, para que el
% usuario pida clarificaciones o ayuda al sistema.

In a first stage we will investigate a model of unidirectional linguistic
interaction (i.e., linguistic information flows only from the system to the
user). In subsequent stages, the model will be extended to allow bidirectional
language exchange, for example, the user may ask for clarification or help to
the system.

% Dise\~nar la arquitectura de un sistema de di\'alogo presenta
% desaf\'ios tanto te\'oricos como pr\'acticos. En lo te\'orico, se necesitan
% heur\'isticas que gobiernen la interacci\'on en cuanto a qu\'e decir,
% cu\'ando, y c\'omo (teniendo en cuenta el contexto actual). Adem\'as, se debe
% contar con m\'etodos
% de inferencia que permitan al sistema adaptarse en funci\'on de la situaci\'on
% actual, para alcanzar un objetivo predefinido.
% La complejidad del problema te\'orico se
% refleja, en lo pr\'actico, en un sistema de m\'ultiples componentes: un
% generador de lenguaje natural, un sistema de planning, un entorno de 
% interacci\'on en tres dimensiones, etc.
% Dise\~nar e implementar todos estos componentes requerir\'ia un esfuerzo
% prohibitivo. En este proyecto adaptaremos herramientas ya
% implementadas y disponibles libremente para el prototipado de sistemas de este
% tipo, como la plataforma \emph{Generating Instructions in Virtual
% Environments} (GIVE).

Designing the architecture of a dialogue system presents both theoretical and
practical challenges. On the theoretical side, heuristics are needed to govern
the interaction in terms of what to say, when, and how (given the current
context). In addition, there must be inference methods that allow the system to
adapt to the current situation, to reach a predefined goal. The complexity of
the theoretical issues is reflected, in practice, in a system of
multiple components: a natural language generator, a planner,
a 3D interactive environment, to mention a few. Designing
and implementing all these components require a prohibitive effort. This project
will adapt tools already implemented and freely available for prototyping such
systems, such as the platform \emph{Generating Instructions in Virtual
Environments (GIVE)}\cite{byron09}. 

% La calidad de cada una de las capacidades del sistema afecta la
% percepci\'on que el usuario tiene de \'este. Por lo tanto, es imperativo
% evaluar cada una de ellas.
% Se planean adaptar y aplicar distintas t\'ecnicas y m\'etricas de evaluaci\'on
% del \'area de ingenier\'ia de software y de diversas \'areas del procesamiento
% de lenguaje natural (por ejemplo, la traducci\'on autom\'atica).

The quality of each of the capabilities of the system affects the perception
that users have of it. Therefore, it is imperative to evaluate each of such
capabilities. We plan to adapt and apply different evaluation techniques and
metrics from the area of software engineering and various areas of
natural language processing (e.g. machine translation).


\section{Description of the Project}\label{description}
As we mentioned above, we will use the GIVE platform as the basic architecture 
for our dialogue system.  In the scenario proposed by GIVE, a
human user carries out a ``treasure hunt'' in a 3D virtual environment
and the task of the generation system is to provide real-time, natural
language instructions that help the user find the hidden treasure.

For the correct definition of the interaction policies of our prototype we need
a corpus that provides examples of typical interactions in the domain. GIVE
provides tools for collecting such corpus in the form of a Wizard of Oz
platform that record all details of the interaction, thus
allowing to easily obtain a corpus of interaction in virtual environments
annotated automatically.

From the collected corpus we will begin the design, implementation and testing
of our dialogue system.  The main components that we will have to design and 
implement can be organized using the traditional four tasks that a dialogue 
system should address: (1) content planning, (2)
generation of referring expressions, (3) management of the interaction context, and
(4) interpretation of user responses. 

\emph{(1) Content Planning:} Given the envisioned setup we described in Section~\ref{intro},
the first task of the system is to obtain a plan to reach the desired goal, from the current state.
The plan will contain physical actions to be performed in the virtual environment. The second
step is to decide how to transmit this sequence of actions to the user. E.g, to decide
how many actions to communicate per instruction, and how to aggregate them
coherently. The result of the action aggregation process can be represented as a
tree describing the task structure at different levels of abstraction. The third
and final step is to decide how to navigate the tree of actions to verbalize the
instructions (for example, post or preorder as
explored in~\cite{foster-etal-ijcai2009}). We will investigate
different aggregation policies (e.g., aggregating actions that
manipulate similar objects) and innovative ways in which to navigate the task tree
(e.g., moving to a lower level of abstraction in case of misunderstandings).
Plan computation can be solved using classical planners~\cite{nau04}.
However, while there are planners that work well when optimized for certain
applications, none provides services such as the generation of alternative
plans, or the generation of incomplete plans in case of the absence of plan.
One of the goals of the project is to design and implement these extensions to 
classical planning algorithms. We will also study the theoretical behavior (e.g., complexity) of
these new algorithms. 

\emph{(2) Generation of Referring Expressions:} Once content planning is
complete, the next step is to generation adequate referring expressions. 
This task involves producing a phrase that describes a referable entity so that the user can
identify it (e.g., ``the vase on the table''). To be
acceptable, these expressions should be ``natural:'' they should be at the same time
sufficiently but not overly constrained, and they should not impose on the user a heavier 
cognitive load than necessary. For example, producing the expression 
``the vase that is not above the chair or sofa or under the
table'' would probably not be acceptable. Areces et al.~\shortcite{AKS08} propose to
use symbolic minimization of the model that represents the state of the world, in
order to obtain logical formulas that describe each object uniquely. In our
project we will implement this method and evaluate it within the dialogue system.

\emph{(3) Management of the Interaction Context:} To manage the use of
the interaction context we will use existing knowledge maintenance systems such as
RACER\footnote{\url{http://www.racer-systems.com}} or Pellet\footnote{\url{http://clarkparsia.com/pellet}}, which support inference tasks such as
definition, maintenance and querying of ontologies. These systems have been used
as inference engines in numerous applications in
the area and, in particular, in dialogue systems for text adventures~\cite{benotti09b}. Once we
have studied the behavior of these inference engines on the task, we will
analyze its limitations and investigate the required extensions.

\emph{(4) Interpretation of User Responses:} The interpretation of user
responses in the unidirectional system is relatively simple: it amounts to
discretizing the continuous flow of user behavior in the 3D world into actions
meaningful for the domain task. In a first
stage, we will use the discretizer provided by GIVE. After evaluating it we can
determine whether or not this module meets the requirements of
our task and what are its limitations. In the bidirectional system, however,
the interpretation of user responses is the task that will require more
attention.
To start with, the bidirectional system should be expanded with capabilities
for  processing statements coming from the user (namely, parsing, semantic
construction, resolution of references, etc.). We will study, in particular, two
types of user contributions: requests for clarification of the instruction
given (what we call `short-term repairs'), and for redefinition of goals (what we
call `long-term repairs'). We will implement short-term repairs
using the approach described in~\cite{purver06}. For long-term repairs we will use the
guidelines of~\cite{blaylock05a}. 

\subsection{Evaluation}
To determine the quality of the obtained prototypes we propose to create a
quality model following the ISO/IEC
9126 and 14528 standards for the evaluation of software
products~\cite{ISO9126-1,ISO14598-1}. These standards were successfully applied
to
the Machine Translation domain, resulting in the
\emph{FEMTI\footnote{\url{http://www.issco.unige.ch/femti/}}, Framework for the
Evaluation of
Machine Translation}~\cite{Est2005}. FEMTI
guides evaluators towards creating parameterized evaluation
plans that include various aspects of the to-be-evaluated system and offer a
relevant set of metrics. The identification of relevant metrics can be performed
using various methods, e.g., based on previous
experience~\cite{paradise06,Litman2002}, conducting
surveys or requirement specifications~\cite{Lecoeuche98}, or
collecting such data through Wizard of Oz
experiments~\cite{Dahlback93}.
After developing a quality model, several methodologies to assess
various aspects of the system can be applied: automatic metrics,
subjective metrics or metrics based on the task (to
evaluate both the contribution of each component and the quality of the whole
system). 

The GIVE platform is used every year as a unified framework for evaluating
generation systems. Systems have to generate natural language
instructions and be able to participate in a real-time interaction situated in a
3D environment. The GIVE Challenge is one of the shared tasks endorsed by
ACL's special interests groups in generation, dialogue and semantics. We plan
to participate in the challenge, which will serve as an additional
source of information about aspects of the system that need
improvement.
Once the prototype is evaluated and improved using the results
of this evaluation, we will investigate its use as a virtual language tutor as
described in the next section.

\subsection{An Application: A Virtual Tutor}\label{applications}

The project outcome will be a system capable of giving natural language
instructions situated in a virtual 3D environment. The technology and
theoretical advances of the project could be used in various applications, but
one of the most interesting characteristics we plan to investigate is that, 
a priori, by just changing the linguistic resources, the language of interaction
with the system (input and output) can be changed as desired. After obtaining a
first prototype of an instruction giving dialogue system, 
 we will investigate its use for distance learning,
adapting the system to operate as a foreign language tutor~\cite{Wik09}.

A one-way system that generates instructions in English can be used to test the
user understanding of a foreign language. The correct interpretation of the
instructions can be evaluated from the proper execution of the instructions. The
two-way system will allow the user to formulate clarifications (either in their
native language or in the foreign language). The user may also redefine the
objective to be achieved during the interaction, and thus select the type of
vocabulary he wants to practice.

Virtual worlds (like Second Life) are being rapidly incorporated into
education, both initial and superior~\cite{Doswell05,molk:lear09}. The use of a
virtual tutor has certain advantages over a human tutor.
Engwall~\shortcite{engwall1020} mentioned the following. (1) Amount of
practice: the chance to practice the new language is essential for learning, and
a virtual tutor provides opportunities only limited by the
technological resources. (2) Prestige: a student
may feel embarrassed about making mistakes with a human tutor, and this
might limit his willingness to speak in the foreign language. (3) Augmented
Reality: a virtual
tutor can provide additional material (e.g., examples in context, explanatory
images, etc.) with less effort than a human tutor.


Such a virtual tutor can be used in distance learning. To develop distance learning systems, it is essential to model the user's learning
progress. This requires a system aware of the evolution of the
user, and that takes into account their achievements and their problems. This type
of interaction between user and system can be modeled as a dialogue, which
records the acquired knowledge context (of the user about the course material, and
of the system about to user). The system must be able to interpret requirements, and generate
appropriate responses, for non-experts uses whose knowledge evolves during the
interaction. Moreover, the system must be able to properly represent both the
information concerning the course material, and information about the
evolution of the user. For example, the system must be able to diagnose what
part of the course material should be reviewed from the wrong answers of the
user. Finally, the system must be able to evaluate the user interaction in order
to decide which learning objectives have been achieved. The theoretical and practical
results of the project contribute to solving these difficult
problems.










\section{Impact of the Project}\label{impact}
\MySubSection{4. Objetivos Generales}

El objetivo \'ultimo de este proyecto es dise\~nar un sistema de di\'alogo hombre-computadora con soporte para interacci\'on en tres dimensiones (3D) que generare autom\'aticamente instrucciones en lenguaje natural para guiar  al usuario durante la ejecuci\'on de una tarea determinada. Esta propuesta apunta
a obtener un balance entre un sistema de uso general, aplicable en
diferentes \'ambitos, y un sitema lo suficientemente espec\'ifico
como para permitir el uso efectivo de las t\'ecnicas actuales de
administraci\'on del conocimiento, planneamiento y procesamiento de lenguaje natural (PLN).

El sistema resultante podr\'ia utilizarse en distintas situaciones, a modo de ejemplo:
en comercio electr\'onico  al usarse como avatar de asistencia en la web, en soporte t\'ecnico al proveer ayuda al usuario no experto, en
control de dispositivos por voz al usarse en sistemas de di\'alogo embebidos,
en la ense\~nanza a distancia al constituir tutores de lenguas extranjeras.  Una vez obtenido
un prototipo, se planea aplicarlo  como
tutor virtual para el aprendizaje de idiomas; para estimar la eficacia del prototipo  en esta tarea, se podr\'ia realizar una evaluaci\'on comparativa (compar\'andolo con otros sistemas existentes) o una evaluaci\'on orientada al usuario al ser utilizado, por ejemplo, estudiantes de FaMAF.


A fin de obtener un sistema como el descripto anteriormente, este proyecto propone estudiar las tres \'areas fundamentales detalladas a continuaci\'on, las cuales posibilitan el desarrollo de este tipo de sistemas:
\begin{myitemize}
  \item \emph{Interacci\'on:} La interacci\'on humano-humano est\'a gobernada
  por numerosas reglas pragm\'aticas que definen nociones b\'asicas como
  las obligaciones conversacionales de los participantes (qui\'en puede hablar en cada momento), o las implicaciones de un enunciado (por ejemplo, si se
  hizo una pregunta se espera una respuesta).
  Estas reglas se aplican tambi\'en a la interacci\'on humano-computadora en
  entornos virtuales cuando el objetivo del sistema de di\'alogo es
  simular, en la medida de lo posible, el comportamiento humano.

  Modelar formalmente estas reglas pragm\'aticas para un sistema de di\'alogo es uno de los desaf\'ios que condicionan la efectividad del mismo.

  \item \emph{Inferencia:} En t\'erminos generales, podemos entender como
  inferencia toda operaci\'on que transforme informaci\'on \textit{impl\'icita} en
  \textit{expl\'icita}.  Bajo esta amplia definici\'on, es posible considerar como
  inferencia tanto la cl\'asica (por caso
  tareas habituales en inteligencia artificial como la planificaci\'on) u operaciones estad\'isticas (por ejemplo para obtener estimadores sobre un conjunto de datos).  Un sistema de di\'alogo realiza continuamente operaciones de inferencia, por un lado   para interpretar la informaci\'on recibida e incorporarla a su repositorio de datos, y por otro, para decidir qu\'e parte de la informaci\'on disponible transmitir.

  El problema mismo de decidir qu\'e tipo de representaci\'on l\'ogica y que tipo de inferencia utilizar en una determinada situaci\'on (e.g., l\'ogica proposicional vs.\ l\'ogica de primer orden, validez vs.\ chequeo de modelos) es complejo, y las tareas de inferencia en s\'i son computacionalmente costosas.  El desaf\'io en este caso es encontrar el compromiso adecuado entre representaci\'on de la informaci\'on y el m\'etodo de inferencia a utilizar.

\item \emph{Evaluaci\'on:} La evaluaci\'on de sistemas de generaci\'on de lenguaje natural es una de las m\'as dif\'iciles dentro del \'area del Procesamiento de Lenguaje Natural (PLN) dado que una idea puede expresarse de muchas formas, todas ellas correctas, y, en general, determinar la calidad de tales frases no puede hacerse de manera simple y directa, por ejemplo comparando el resultado del sistema con un patr\'on (en Ingl\'es ``gold stadard"). El problema de la falta de patrones es compartido con otra \'area del PLN, la Traducci\'on Autom\'atica (TA), en la cual se han propuesto diversas metodolog\'ias para la evaluaci\'on de sistemas de forma directa (es decir, aplicando alguna m\'etrica al texto generado por un sistema) o indirecta (o sea, evaluando la perfomance del sistema a trav\'es de la utilizaci\'on del texto generado para realizaci\'on de alguna tarea). Sin embargo, en ninguna de estas \'areas existe una metodolog\'ia aceptada como est\'andar y demostrada eficaz de manera general.
En esta propuesta, dado que el objeto a evaluar es un sistema que interact\'ua via la generaci\'on de instrucciones en lenguaje natural, es necesario determinar su performance por medio de evaluaciones cuantitativas (como el tiempo de finalizaci\'on de la tarea), cualitativas (por ejemplo, la calidad de las interacciones) y basadas en el contexto (evaluaciones orientadas al usuario y sus necesidades en una situaci\'on particular). Dada la experiencia de los participantes es de especial inter\'es estudiar la  portabilidad y aplicaci\'on de t\'ecnicas de evaluaci\'on del dominio de la TA y la interacci\'on multimodal humano-computadora al sistema (en su totalidad y por componentes) planteado en este proyecto.
\end{myitemize}

El presente proyecto se focalizar\'a en el  m\'odulo del sistema encargado de generar las instrucciones, el cual deber\'a ser capaz de interactuar naturalmente con el usuario y adaptarse a posibles errores de interpretaci\'on o ejecuci\'on de las instrucciones, dando instrucciones correctivas en consecuencia. Adem\'as, el usuario se encontrar\'a en un universo en el que podr\'a interactuar con objetos y explorar el ambiente virtual provisto. As\'i, el usuario intentar\'a seguir las instrucciones provistas por el sistema,
realizando acciones f\'isicas en un mundo 3D.


En una primera etapa, el modelo a utilizar permitir\'a el flujo de informaci\'on ling\"u\'istica de manera unidireccional, es decir,
desde el sistema hacia el usuario, por lo que
el usuario no podr\'a pedir ningu\'n tipo de ayuda.  La restricci\'on a un
modelo unidireccional tiene como objetivo
simplificar la representaci\'on y manejo del contexto de la interacci\'on.

En etapas m\'as avanzadas del proyecto, el modelo unidireccional ser\'a extendido para
permitir intercambio ling\"u\'istico bidireccional: por ejemplo, el usuario podr\'a
pedir clarificaciones o redefinir el objetivo a alcanzar, en ambos casos usando lenguaje natural.

La generaci\'on de instrucciones para un dominio espec\'ifico requiere la adquisici\'on del conocimiento relevante y la compilaci\'on del mismo debe incluir cierta informaci\'on  ling\"u\'istica que permita su posterior procesamiento: en los aspectos morfol\'ogico y sint\'actico se requiere una gr\'amatica de lenguaje seleccionado (Espa\~nol en este caso), en el aspecto sem\'antico se necesita un repositorio de informaci\'on l\'exica organizada en forma ontol\'ogica y en lo pragm\'atico es necesaria una descripci\'on formal de las
acciones posibles en un universo dado (incluyendo precondiciones, efectos e informaci\'on contextual del estado de cada interacci\'on).

Esta informaci\'on relacionada al conocimiento de un dominio es utilizada por distintos componentes del sistema de di\'alogo que
gestionar\'an la interacci\'on con el usuario. En un sistema t\'ipico, esto sucede como se detalla a continuaci\'on: en la direcci\'on sistema $\rightarrow$ usuario, el componente encargado de \textit{planificar} genera una secuencia de acciones relevantes al momento de la interacci\'on, luego el componente encargado de  \textit{administrar el conocimiento} actualiza la informaci\'on contextual y,  finalmente, el componente encargado de
\textit{generar instrucciones}  transmite las mismas como expresiones
del lenguaje natural.  En la direcci\'on opuesta, es decir usuario $\rightarrow$ sistema,  el componente denominado \textit{discretizador}
transforma el flujo continuo de informaci\'on derivada del comportamiento del usuario, en acciones relevantes al universo en cuesti\'on, mientras que
los componentes de planeamiento y administraci\'on del conocimiento se ocupan
de mantener el contexto actualizado y de detectar posibles errores.


El dise\~no de una tal arquitectura presenta
desaf\'ios tanto te\'oricos como pr\'acticos.  En lo te\'orico, la necesidad de
 guiar la interacci\'on con el usuario (qu\'e decir,
cu\'ando y c\'omo, en un contexto dado) motiva la creaci\'on, adaptaci\'on o mejora de heur\'isticas pertinentes y, por otro lado,  la necesidad de describir la situaci\'on actual as\'i como tambi\'en  de indicar c\'omo alterarla para alcanzar un objetivo pre-establecido, incentivan el estudio de diversos m\'etodos
para la inferencia eficiente y precisa.
La calidad de cada una de estas capacidades afecta la percepci\'on que
el usuario tiene del sistema, y por lo tanto, es imperativo aplicar m\'etricas pertinentes que permitan evaluar cada uno de estos aspectos del sistema.  La complejidad del problema te\'orico se
refleja, en lo pr\'actico, en el desarrollo de un sistema de m\'ultiples componentes: un
generador de lenguaje natural, un componente de planeamiento, un entorno 3D, etc.

Dise\~nar e implementar todos estos componentes requerir\'ia un esfuerzo prohibitivo, por lo que proponemos utilizar recursos disponibles de forma libre y gratuita. En particular, se propone el uso de la plataforma desarrollada com parte de la competencia \textit{Generating Instructions in Virtual Environments} (GIVE\footnote{Disponible libremente en  \fnturl{http://www.give-challenge.org}}) que
provee un prototipo b\'asico de este tipo de sistemas. En este contexto ``competencia" (del Ingl\'es ``evaluation campaign" o ``evaluation challenge") se refiere a la organizaci\'on de un evento en el que participan distintos grupos de investigaci\'on, realizando todos el mismo ejercicio con sus sistemas propios y comparando al final los resultados obtenidos por cada grupo. Este tipo de eventos es muy popular en distinas \'areas del PLN (TA, reconocimiento del habla, recuperaci\'on de informac\'on, entre otras) y benefician ampliamente cada \'area generando y compartiendo con la comunidad de investigadores nuevas tecnolog\'ias.

El objetivo principal de GIVE es actuar como medio de evaluaci\'on de sistemas de generaci\'on
de lenguaje natural, donde \'estos se eval\'uan de forma m\'as precisa y homog\'enea dado que todos los sistemas participantes utilizan los mismos recursos provistos por los organizadores. Adem\'as, desde un punto de vista general, la evaluaci\'on
tambi\'en tiene en cuenta el problema b\'asico de la interacci\'on en
entornos virtuales.  En el escenario propuesto por GIVE, el usuario
humano lleva acabo una tarea de `b\'usqueda del tesoro' en un entorno
virtual 3D y el trabajo del sistema participante es proveer, en tiempo
real, instrucciones en lenguaje natural que ayuden al usuario
a encontrar el tesoro escondido.

\fixme{GIVE tiene el apoyo de SIGSEM SIGDIAL SIGGEN grupos de interes de
ACL, la Asociacion de Ling\"u\'istica Computacional.}

Por su dise\~no, el prototipo GIVE es una herramienta muy propicia para investigar distintos aspectos de la interacci\'on situada en un entorno virtual:
el problema de realizaci\'on sint\'actica y morfo\'ogica de las
instrucciones (en Ingl\'es \emph{surface realization}),
la representaci\'on del conocimiento existente en el contexto de
interacci\'on (contexto del discurso, ontolog\'ias, acciones posibles),
el tipo de inferencia requerido por las tareas involucradas
(planeamiento, construcci\'on y chequeo de modelos),
las reglas pragm\'aticas de interacci\'on requeridas por la tarea
(administraci\'on de la carga cognitiva, actualizaci\'on y uso de la informaci\'on contextual), entre otras. Adicionalmente, GIVE provee herramientas para registrar todos los detalles de la
interacci\'on, permitiendo de esta forma obtener f\'acilmente un corpus de interacci\'on en entornos virtuales anotado autom\'aticamente. \fixme{agregar la importancia de tales corpora}


%Finalmente, la calidad de la interacci\'on con un
%sistema de di\'alogo dado puede medirse en esta tarea en t\'erminos
%cuantitativos (por ejemplo el tiempo m\'aximo para completar la tarea)
%y cualitativos (como la relevancia de una instrucci\'on dada).

\fixme{Terminar con un comentario sobre tutoring!!!.}



\section{Introducing the Research Group}\label{group}
%\MySubSection{12. Composici\'on del Grupo de Investigaci\'on y Areas de
% Especializaci\'on}

% El Dr.\ Carlos Areces se especializa en l\'ogica
% computacional,
% en particular en el estudio te\'orico y aplicado de lenguajes para la
% representaci\'on del conocimiento (e.g., l\'ogicas modales, h\'ibridas y
% para la descripci\'on).  Las publicaciones
% m\'as importantes sobre aspectos te\'oricos en estos lenguajes
% son~\cite{ABM01,arec:hybr05b}.  Ha desarrollado tambi\'en demostradores
% autom\'aticos para estos lenguajes~\cite{ANR01,arec:hylo02a,AG06,Hoffmann2007}
% accessibles en \url{http://www.glyc.dc.uba.ar/intohylo/}.
% Ha trabajado tambi\'en en el dise\~no de algoritmos para la generaci\'on de
% expresiones referenciales~\cite{AKS08}, y en el estudio de lenguajes l\'ogicos
% adecuados para la representaci\'on sem\'antica~\cite{AF08}.

Carlos Areces specializes in computational logic, particularly in
theoretical and applied study of languages for knowledge representation (eg,
modal logics, hybrid and for the description). The most important publications
on theoretical aspects in these languages are~\cite{ABM01,arec:hybr05b}.
Demonstrators\footnote{Accessibles in
\url{http://www.glyc.dc.uba.ar/intohylo/}} also developed machines for these
languages~\cite{ANR01,arec:hylo02a,AG06,Hoffmann2007}. He also worked
on the design of algorithms for generating referring expressions ~\cite{AKS08}
and in the study of logical languages suitable for the semantic
representation~\cite{AF08}.

% La Dra.\ Paula Estrella ha realizado su tesis doctoral en la 
% evaluaci\'on contextual de sistemas de TA~\cite{estr:impr08,estr:femt09},
% proponiendo 
% un modelo generalizable y aplicable a otras \'areas del PLN, como se muestra
% en 
% \cite{Miller2008}. Tambi\'en ha trabajado en el desarrollo y mejora de
% sistemas de TA 
% estad\'istica desde y hacia varios idiomas, por ejemplo el
% Espa\~nol~\cite{estr:expe05} 
% y entre los idiomas Franc\'es, Ingl\'es, Japon\'es y Arabe
% \cite{rayner-EtAl:2009:GEAF}. 
% Adem\'as particip\'o en proyectos que involucran la creaci\'on y evaluaci\'on
% de % corpora 
% correspondientes a anotaciones multimodales~\cite{pope:estr07} y trabaj\'o en
% un % prototipo 
% para la evaluaci\'on autom\'atica de anotaciones de di\'alogos multimodales
% \cite{multieval}, 
% investigando tambi\'en la aplicaci\'on de m\'etricas provenientes de otras
% \'areas; el 
% prototipo est\'a disponible p\'ublicamente en
% \url{http://www.issco.unige.ch:8080/multieval/}.

Paula Estrella has done his PhD at the contextual assessment of MT
systems~\cite{estr:impr08,estr:femt09}, suggesting a model generalizable and
applicable to other areas of PLN, as shown in~\cite{Miller2008}. She has also
worked on the development and improvement of
statistical MT system and from various languages, such as
Spanish~\cite{estr:expe05} and between the languages English, French, Japanese
and Arabic~\cite{rayner-EtAl:2009:GEAF}. Also participated in projects involving
the development and evaluation
corpora annotation for multimodal~\cite{pope:estr07} and worked on
a prototype for automatic evaluation of multimodal dialogues
annotations~\cite{multieval} also investigating the application of metrics from
other areas.\footnote{The prototype is publicly available
\url{http://www.issco.unige.ch:8080/multieval/}.}

% La Lic.\ Luciana Benotti ha trabajado en el desarrollo de un sistema de
% di\'alogo completo situado en una aventura de
% texto~\cite{benotti09b}. La implementaci\'on abarca desde tareas de an\'alisis
% sint\'actico hasta tareas pragm\'aticas. Su investigaci\'on se ha enfocado en
% tareas pragmaticas que requieren de tareas de inferencia practicas. En
% particular ha utilizado la tarea de inferencia de planning clasico y de
% planning con informacion incompleta. Luego extendi\'o el sistema de di\'alogo
% desarrollado por el grupo de David Traum~\cite{TraumSigdialBook08} para
% analizar pragmaticamente oraciones comparativas~\cite{benotti09a}. Finalmente,
% trabaj\'o en el an\'alisis emp\'irico de corpora de di\'alogo humano-humano
% utilizando
% diferentes t\'ecnicas de planning para precedir sub-dialogos de
% reparaci\'on~\cite{benotti09c}.

Luciana Benotti has investigated the addition of pragmatic abilities, in
particular the inference and negotiation of conversational implicatures, into
dialogue systems. For this purpose, she has implemented a conversational agent
(using a generic parser, realizer and knowledge base manager) which  is situated
in a text-adventure game, and she has extended it using inference tasks such as
classical planning and planning under incomplete information~\cite{benotti09b}.
She has also investigated the inference of conversational implicatures triggered
by comparative utterances~\cite{benotti09a}. She has recent
corpus-based work, which shows what kinds of implicatures are inferred and
negotiated by human dialogue participants during a task~\cite{benotti09c}. 








\section{Ongoing and Future Collaborations}\label{collaboration}
The members of the PLN in general and the authors of this paper in particular
have several collaborations with national international research groups in CL
and related fields that are relevant for this project. 
 
We have collaborated and continue to do so with the \textit{Multilingual
Information
Processing Department at the University of Geneva} (ISSAC/TIM) and of the
\emph{Database management
and meeting
analysis} group of the Swiss National Center of Competence in Research  
\emph{Interactive Multimodal Information Management}. The joint research
project includes the evaluation of NLP systems and the
development of multilingual and multimodal HLT systems. Furthermore we
collaborate with some members of the \textit{Pervasive Artificial
Intelligence} group of the University of Fribourg. 

We currently collaborate with
the TALARIS\footnote{\url{http://talaris.loria.fr/}}
group of the \emph{Laboratoire Lorrain de Recherche en Informatique et ses
Applications (LORIA)}. The main research topic at TALARIS is computational
linguistics
with strong emphasis on semantics and inference. In the framework of this
collaboration we are participating
(at the moment of writing this paper) in the 2010 edition of the GIVE
Challenge. In the process of designing the systems that will participate in
the challenge we jointly investigated the use of different referring strategies
in situated instruction
giving~\cite{amoia10}. 

We have intensively collaborated with the GLyC group (\textit{Grupo de L\'ogica,
Lenguaje y Computabilidad})\footnote{\url{http://www.glyc.dc.uba.ar/}} on
knowledge representation and inference topics~\cite{AG06,AFFM08}. GLyC is part
of the Computer Science Department of the Universidad de Buenos Aires. 

We have also collaborated with the Virtual Humans group of the Institute for
Creative Technologies\footnote{\url{http://ict.usc.edu/projects/virtual_humans}}
from the University of Southern California. In particular we computationally
modeled the inference of conversational implicatures triggered by comparative
utterances~\cite{benotti09a}. The Institute for Creative Technologies offers
Internship programs every year that we plan to use in order to strengthen our
collaboration by sending PhD and Masters exchange students from the University
of C\'ordoba.

One of the authors has collaborated (organizing seminars in CL) with the
research group in Artificial Intelligence from the Universidad Nacional del
Comahue.\footnote{\url{http://www.uncoma.edu.ar/}} This is a University with
low resources. We plan to continue collaborating with them through student and
professors exchange programs financed by the Argentinean government. 

We are organizing the first school in Computational Linguistics in
Argentina, ELiC\footnote{\url{http://www.glyc.dc.uba.ar/elic2010/}} which will
take place in July 2010 in the Universidad de Buenos Aires. The main goals of
ELiC are the following. (a) Introduce the field of CL to graduate students in
Argentina. (b) Start a yearly event in CL in Argentina that will change
location every year inside the country or in other Latin-American countries.
(c) Organize a one-day workshop as a panel for researchers in the area
immediately after or before the school (starting from ELiC 2011). ELiC 2010 is
offering 10 student travel grants (for students not living in Buenos Aires)
due to the generous support of the North American Chapter of the Association for
Computational Linguistics (NAACL). ELiC 2010 will be co-located with the Escuela
de Ciencias Inform\'aticas
(ECI)\footnote{\url{http://www.dc.uba.ar/events/eci/2009/eci2009}}. ECI has a
long standing reputation as a high-quality winter school in Computer Science in
Argentina which is organized every year. ELiC 2010 will take advantage of the
organizational infrastructure of ECI, such as advertising, registration,
preparation of reading material, and classrooms; moreover, ECI will also offer
fee waivers to selected ELiC students.

The PLN group has being part of an international project of collaboration on
ontology population from raw text which might be relevant when trying to extend
our GIVE ontologies to new domains. The project was called 
MICROBIO\footnote{\url{http://www.microbioamsud.net/}}, was granted by
Stic-Amsud and finished last year. This project included universities and
laboratories from France, Brazil, Uruguay and Chile. 

Finally, the PLN group is planning to organize a workshop on Computational
Linguistics collocated with the Ibero-American Conference on Artificial
Intelligence (IBERAMIA
2010).\footnote{\url{http://cs.uns.edu.ar/iberamia2010/}} 

\fixme{add a closing paragraph mentioning PICT funding}

\bibliographystyle{naaclhlt2010}
\bibliography{pict09}


\end{document}
