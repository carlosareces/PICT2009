%\MySubSection{11. Insersi\'on en el Programa de Investigaci\'on}

% El procesamiento de lenguaje natural, y en particular el
% desarrollo de sistemas de di\'alogo, es un \'area en crecimiento
% en los pa\'ises desarrollados, por su gran aplicabilidad en el
% actual contexto de la sociedad de la informaci\'on. Efectivamente,
% dado el gran aumento de la informaci\'on textual en formato
% digital, el procesamiento automatizado de los textos se ha
% convertido en una capacidad estrat\'egica para empresas,
% instituciones y la comunidad en general. Por esta raz\'on
% el \'area de PLN se encuentra en crecimiento en las principales
% universidades y empresas tecnol\'ogicas del mundo, con un
% presupuesto que crece a\~no a a\~no. Sin embargo, 
% este \'area se encuentra muy poco desarrollada en la Argentina. Esto se puede
% atribuir a diferentes factores:

% \begin{itemize}
% \item la relativa juventud del \'area de PLN, lo que implica una relativa
% escasez de profesionales bien formados en todo el mundo,
% \item el desarrollo insuficiente de todo el \'area de investigaci\'on en
% Ciencias de la Computaci\'on, por razones hist\'oricas y demanda de la
% industria,
% \item el poco desarrollo del \'area de Inteligencia Artificial y
% Ling\"u\'istica Formal en la Argentina, tambi\'en por razones hist\'oricas
% y acad\'emicas,
% \item la escasa interacci\'on entre los pocos investigadores en PLN que se
% encuentran en la regi\'on.
% \end{itemize}

Natural language processing, and in particular the development of dialogue
systems is a growth area in developed countries, for their great applicability
in the current context of information society. Indeed, given the large increase
of textual information in digital form, the automated processing of texts has
become a strategic capability for companies, institutions and the wider
community. For this reason the area is growing PLN at major universities and
technology companies in the world with a budget that grows every year. However,
this area is very underdeveloped in Argentina. This can be attributed to several
factors. (a) The relative youth of the area of PLN, which implies a relative
dearth of trained professionals throughout the world. (b) The underdevelopment
of the whole area of research in computer science, for historical reasons and
industry demand. (c) The undeveloped area of Artificial Intelligence and
Formal Linguistics in Argentina, also for historical and academic. (d) Poor
interaction between the few researchers in NLP that are in the region. 

% Parece claro que el PLN es un \'area de investigaci\'on estrat\'egica
% para la Argentina, en la que se puede alcanzar la excelencia acad\'emica
% e industrial a nivel internacional. Creemos que hay que apoyar el
% desarrollo de este \'area favoreciendo los siguientes aspectos:
% 
% \begin{itemize}
% \item formaci\'on de recursos humanos a trav\'es
% de programas de doctorado y de cursos dictados en la Argentina por
% profesionales de
% reconocido prestigio internacional,
% 
% \item incorporaci\'on de recursos humanos formados, para contribuir
% al aumento y diversificaci\'on de la masa cr\'itica en el \'area,
% 
% \item mejora de la interacci\'on entre los diversos grupos o investigadores
% aislados en PLN, a
% trav\'es de la organizaci\'on de workshops, cursos de profesores visitantes,
% co-tutor\'ias,
% programas de especializaci\'on coordinados, etc.
% \end{itemize}

It seems clear that PLN is a strategic research area for Argentina, which can
achieve academic excellence and industry worldwide. We believe in supporting the
development of this area by promoting the following. (a) Training of human
resources through doctoral programs and courses taught in Argentina by
internationally renowned professionals. (b) Incorporation of trained human
resources to contribute to the growth and diversification of the critical mass
in the area. (c) Improving the interaction between various groups or
individual researchers in NLP, through the organization of workshops, courses,
visiting professors, co-tutoring, coordinated specialty programs, etc.

% En la FaMAF existe un grupo de PLN desde 2005
% (\url{http://www.cs.famaf.unc.edu.ar/~pln}). Este
% grupo est\'a desarrollando una importante labor de formaci\'on de recursos
% humanos, con el dictado 
% de cursos de grado y de postgrado en la FaMAF y en otras universidades del
% pa\'is.
% Tambi\'en trabaja en el desarrollo de diversos proyectos de investigaci\'on y
% en la integraci\'on
% con otros grupos de la regi\'on, tanto en Argentina como en Chile, Brasil y
% Uruguay. Este proyecto 
% de investigaci\'on se integra al programa del grupo de procesamiento de
% lenguaje % natural de la
% FaMAF.


In FaMAF there is an NLP group since 2005 (\url{http://www.cs.famaf.unc.edu.ar/
~ pln}). This group is developing an important role in human resource training,
delivering courses to undergraduate and postgraduate studies at the FaMAF and
other universities. It also works in the development of various research
projects and integration with other groups in the region, both in Argentina and
Chile, Brazil and Uruguay. This research program is integrated into the group of
natural language processing of FaMAF. 


