\MySubSection{3. Resumen}
En este proyecto se investigar\'an temas fundamentales respecto de la
interaccion hombre-maquina en entornos virtuales. Identificamos tres
\'areas principales en las podemos clasificar los resultados esperados
del proyecto: a) pragm\'atica de la interacci\'on, b) representaci\'on
de la informaci\'on e inferencia; y c) evaluaci\'on de sistemas
de di\'alogo.

En particular, tomaremos como primer modelo un sistema de di\'alogo con
soporte de interacci\'on 3D, en el que el sistema genera autom\'aticamente
instrucci\'ones en lenguaje natural para cumplir una tarea determinada, que
el usuario humano debe seguir.   El sistema debe ser capaz de interactuar naturalmente
con el usuario y debe poder adaptarse a posibles errores de interpretaci\'on o
ejecuci\'on del usuario.  En este primer modelo el flujo de informaci\'on
linguistica es unidireccional, desde el sistema hacia el usuario.
En las \'ultimas etapas del proyecto, este modelo ser\'a extendido para
permitir intercambio ling\"u\'istico bidireccional: por ejemplo, el usuario podr\'a
pedir clarificaciones o redefinir el objetivo a alcanzar usando lenguaje natural.

El dise\~no de una arquitectura como la que acabamos de describir presenta
desaf\'ios te\'oricos y pr\'acticos.  En lo te\'orico, necesita
heur\'isticas que gu\'ien la interacci\'on con el usuario (i.e., que decir,
cuando, y como, teniendo en cuenta el contexto actual) y debe contar con m\'etodos
de inferencia que le permitan no s\'olo describir la situaci\'on actual, sino
tambi\'en indicar como cambiarla para alcanzar un objetivo.
La calidad de cada una de estas capacidades afectar\'a la percepci\'on que
el usuario tendr\'a del sistema, y por lo tanto se necesitan m\'etricas que
permitan evaluar su desempe\~no.  La complejidad del problema te\'orico se
refleja, en lo pr\'actico, en un sistema de m\'ultiples componentes: un
generador de lenguaje natural, un sistema de planeamiento, un entorno 3D, etc.
Dise\~nar e implementar todos estos componentes requerir\'ia un esfuerzo prohibitivo.
En este proyecto utilizaremos la plataforma GIVE (\url{http://www.give-challenge.org}) que
provee un prototipo basico para este tipo de sistemas.

Sistemas como el descripto tienen aplicaciones en diferentes \'ambitos:
asistentes de comercio electr\'onico, sistemas de di\'alogo embebidos,
ense\~nanza a distancia, etc.  Una vez obtenido
un prototipo, planeamos investigar el uso del sistema como
tutores virtuales para el aprendizaje de idiomas.
