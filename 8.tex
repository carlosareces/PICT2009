\MySubSection{8. Cronograma de trabajo}

El cronograma de trabajo se estructura de la siguiente manera:

\paragraph{A\~no 1.} El objetivo de los primeros meses es obtener un
prototimo del sistema unidireccional.  Comenzaremos con un relevamiento
de la bibliograf\'ia.  Dado que el grupo de investigaci\'on est\'a
constitu\'io de expertos en las distintas \'areas pertinentes al
proyecto, nos concentraremos durante los primeros meses en obtener un
entendimiento com\'un de los distintos aspectos del problema, donde
cada experto contribuir\'a bibliograf\'ia adecuada de su \'area para los
dem\'as miembros del grupo.  A continuaci\'on comenzaremos a estudiar
la plataforma GIVE, y a adaptar sus componentes para nuestro objetivo
espec\'ifico.  Como parte del proyecto, planeamos participar del GIVE
challenge, lo que nos impondr\'a deadlines concretos para obtener un
primer prototipo.  La siguiente tarea es la definici\'on  y el dise\~no,
por un lado, de los par\'ametros de interacci\'on que queremos incorporar
en el sistema, y por otro, los esquemas de representaci\'on de la
informaci\'on que el sistema necesita, y las tareas de inferencia a
utilizar.

\fixme{Agregar algo concreto aca sobre `parametros de interacci\'on'}

Una vez que obtengamos una version del sistema con su
correspondiente testeo y documentaci\'on, podemos comenzar con
su evaluaci\'on.  Debe notarse de todas formas que es preciso tener en cuenta
el tipo de evaluaci\'on a realizar (que por lo tanto debe estar definido
adecuadamente) durante el periodo de desarrollo, para asegurar que el sistema
es capaz de prover la informaci\'on necesaria requerida durante la evaluaci\'on
(e.g., logueo de eventos, etc.).

Al final del a\~no comenzar\'a la preparaci\'on de articulos y reportes para
la presentaci\'on de los resultados obtenidos hasta el momento.

\paragraph{A\~no 2.} El objetivo principal del segundo a\~no del proyecto es,
por un lado, extender y completar el prototipo de sistema con informaci\'on
unidireccional en base a la evaluaci\'on realizada al fin del a\~no anterior,
y utilizando el feedback obtenido de la
participaci\'on en el GIVE challenge.  Por otro lado, se comenzar\'a a agregar
capacidades de interacci\'on ling\"u\'istica bidireccional.  Un sistema
bidireccional es mucho m\'as complejo que un sistema unidireccional.  Por
empezar, deberemos integrar al sistema un m\'odulo de interpretaci\'on de
lenguaje natural (e.g., parser, construcci\'on sem\'antica, etc.).  Nos
concentraremos en dos tipos espec\'ificos de instrucciones que el usuario
podr\'a dar al sistema: a) pedidos de aclaraci\'on de la \'ultima
instrucci\'on dada, y b) redefinici\'on del goal de la interacci\'on.  Estos
dos tipos de instrucci\'on son representativos de `reparaciones' a corto y
largo plazo.  Cada uno de ellos requerir\'a la redefinici\'on adecuada de
las reglas pragm\'aticas que gobiernan la interacci\'on y de servicios
de infer\'encia.  Una vez m\'as, a medida que
estas nuevas capacidades sean agregadas al sistema, deberemos llevar a
cabo testeo y documentaci\'on, para finalmente dar paso a la evaluaci\'on.



\paragraph{A\~no 3.} El objetivo del \'ultimo a\~no es transformar el prototipo
de sistema de di\'alogo bidireccional obtenido durante el a\~no anterior, en un
tutor virtual para el aprendizaje de idiomas.  Adem\'as del beneficio claro de
obtener un sistema para una aplicaci\'on concreta que pueda ser utilizado
m\'as alla del \'ambito acad\'emico, XXX \fixme{terminar.}

\medskip

\noindent
A continuaci\'on se sumarizan las tareas a desarrollar en los tres a\~nos de
trabajo (organizadas trimestralmente):

{\footnotesize
\begin{center}
\begin{tabular}{|p{7cm}||p{2mm}|p{2mm}|p{2mm}|p{2mm}||p{2mm}|p{2mm}|p{2mm}|p{2mm
}||p{2mm}|p{2mm}|p{2mm}|p{2mm}||}
\hline
 \rowcolor[rgb]{0.8,0.8,0.8}\hspace{3.5cm}Tarea & 1 & 2 & 3 & 4 & 1 & 2 & 3 & 4
& 1 & 2 & 3 & 4\\
\hline 1. Relevamiento bibliogr\'afico & $\times$ & $\times$ &&&&&&&&&&\\
\hline 2. Estudio del material bibliogr\'afico & $\times$ & $\times$ & $\times$ &  &&&&&&&&\\
\hline 3. Estudio de la plataforma GIVE & & $\times$ &$\times$&&&&&&&&&\\
\hline 4. Recopilaci\'on corpus de interacci\'on & & & $\times$ &$\times$&&&&&&&&\\
\hline 5. Dise\~no, impl., testing alg.\ interacc. unidireccional & & & $\times$ & $\times$&$\times$&$\times$&&&&&&\\
\hline 6. Dise\~no, impl., testing alg.\ interacc. bidireccional & & &  & &&$\times$&$\times$&$\times$&$\times$&&&\\
\hline 7. Dise\~no e impl., testing alg.\ inferencia & & & $\times$ & $\times$&&&$\times$&$\times$&&&&\\
%\hline 7. Testing &&&&$\times$&&&&$\times$&&&&$\times$\\
\hline 8. Desarrollo de tutor virtual &&&&&&&&&$\times$&$\times$&$\times$&\\
\hline 9. Evaluaci\'on &&&&$\times$&$\times$&&&$\times$&$\times$&&$\times$&$\times$\\
\hline 10. Documentaci\'on &&&&&&&&&&&$\times$&$\times$\\
\hline 11. Diseminaci\'on de resultados &&&&$\times$&$\times$&&&$\times$&$\times$&&&$\times$\\\hline
%\hline 12. Elaborac.\ y presentaci\'on de resultados aplicados &&&&$\times$&$\times$&$\times$&&$\times$&$\times$&$\times$&&$\times$\\\hline
\end{tabular}\end{center}
}

%Se comienza con el relevamiento bibliogr\'afico (1) y estudio del
%material (2). A continuaci\'on se desarrollan las investigaciones
%te\'oricas en c\'alculo de complejidades y desarrollo de algoritmos para
% extensiones de la l\'ogica modal b\'asica, en particular la l\'ogica
% h\'ibrida (3 y 4) y para l\'ogicas modales sub-booleanas (5 y 6). Hacia los
% finales del primer a\~no (y una vez obtenidos avances te\'oricos sustanciales)
% se empieza paralelamente con la implementaci\'on de los algoritmos (7). Una vez
% que esta tarea termina, se procede con la etapa del testing (8). Se dedicar\'an
% tres meses a la documentaci\'on del software (9). El final del proyecto est\'a
% reservado para la elaboraci\'on y presentaci\'on final de los resultados
% te\'oricos (10) y aplicados (11).

