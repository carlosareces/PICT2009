\MySubSubSection{6. Metodolog\'ia de trabajo}

Describimos a continuaci\'on el m\'etodo cient\'ifico a utilizar
respecto a los aspectos te\'oricos, de desarrollo y de evaluaci\'on
que conciernen al proyecto.

\begin{myitemize}
\item \textbf{Aspectos Te\'oricos.}
La metodolog\'ia de trabajo a utilizar para investigar los aspectos
te\'oricos del proyecto es la est\'andar en investigaci\'on de base.
Comenzaremos con el estudio de la bibliograf\'ia existente, para luego
familiarizarnos con las t\'ecnicas de demostraci\'on espec\'ificas del
\'area y poder con ellas demostrar los resultados deseados.

El grupo
responsable est\'a formado por expertos en las tres \'areas principales
que el proyecto planea investigar: Areces es
experto en m\'etodos de inferencia y lenguajes para la representaci\'on del
conocimiento; Benotti es experta en el \'area de
pragm\'atica de la interacci\'on y sistemas de di\'alogos; y Estrella es experta en t\'ecnicas
de evaluaci\'on de sistemas de procesamiento de lenguaje natural.
Los tres han desarrollado extensa investigaci\'on previa en sus respectivas
\'areas
de experiencia.

\item \textbf{Desarrollo:}
Una vez se haya avanzado con las tareas de investigaci\'on te\'orica,
\'estas dar\'an origen al dise\~no de t\'ecnicas y algoritmos que se
implementar\'an en la plataforma de GIVE. Para ello se utilizar\'a la
metodolog\'ia
usual en ingenier\'ia del software (desarrollo por
m\'odulos con interfaces claras y bien documentadas,
que garanticen alta cohesi\'on y bajo acoplamiento).  El grupo
responsable ha participado extensamente en tareas de desarrollo de
software, tanto en el \'area de sistemas de di\'alogo, como en el
de sistemas de inferencia.

Al finalizar la implementaci\'on
se espera tambi\'en tener un documento al estilo de manual de
desarrollador, en donde queden expl\'icitas las principales interfaces
de los m\'odulos desarrollados y la forma de interacci\'on con la
herramienta.


\item \textbf{Evaluaci\'on:}
Una vez implementados los algoritmos, se proceder\'a a realizar tests
completos de unidad y de integraci\'on.  Verificar la correctitud de los
algoritmos implementados respecto de una especificaci\'on adecuada es una
tarea que requiere tiempo y esfuerzo, pero de clara soluci\'on.  
Por otro lado, evaluar la calidad de las pol\'iticas de interacci\'on es, como
discutimos en las secciones anteriores, mucho m\'as dif\'icil.  Intentar
definir metodolog\'ias para una adecuada evaluaci\'on de sistemas de
di\'alogo es una de las contribuciones principales de este proyecto.

Un punto que merece mencionarse separadamente es la recopilaci\'on
de corpus de interacci\'on.  La recopilaci\'on de este corpus es
importante para permitir la correcta definici\'on de pol\'iticas de
interacci\'on en este proyecto y esperamos que, una vez hecho p\'ublico, 
ser\'a de ayuda tambi\'en para la comunidad cient\'ifica en general. La
plataforma GIVE provee herramientas que simplifican esta tarea.  GIVE
provee interfaces que permiten recopilar interaciones hombre-m\'aquina u
hombre-hombre, alineando el material ling\"u\'istico con las acciones
realizadas en el mundo virtual.  Adem\'as, al proveer
accesso mediante internet, los usuarios que proveer\'an los datos pueden
estar distribu\'idos geogr\'aficamente.
\end{myitemize}
