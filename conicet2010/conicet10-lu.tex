\documentclass[11pt]{article}

\usepackage{url,latexsym,amssymb,color}
\usepackage{a4wide}
\usepackage{xcolor}
\usepackage{natbib}
\usepackage{xspace}
\usepackage[silent,draft]{fixme}
\usepackage{ifpdf}
\usepackage{colortbl}

\newcommand{\tup}[1]{\langle #1 \rangle}
\newcommand{\glyc}{\textsf{GLyC}\xspace}
\newcommand{\talaris}{\textsf{TALARIS}\xspace}
\newcommand{\pln}{\textsf{PLN}\xspace}

\newcommand{\hylores}{\textsf{HyLoRes}\xspace}
\newcommand{\htab}{\textsf{HTab}\xspace}

%\addtolength{\topmargin}{-.6cm}
%\addtolength{\textheight}{1.2cm}

\renewcommand{\refname}{\normalfont\large\textbf{Referencias}}

\ifpdf % are we compiling via pdflatex?
  \usepackage[backref=page, pdftex]{hyperref}
  \pdfadjustspacing=1                %%% force LaTeX-like character spacing
\else
  % NO!, it must be latex + dvips + ps2pdf
  \usepackage{ps2pdf,backref=page,pdfcreator  = {LaTeX with hyperref package},
pdfproducer = {dvips + ps2pdf}]{hyperref}
\fi

\hypersetup{
    colorlinks=true,   % color instead of boxes
    citecolor=blue,    % cite links in blue
    linkcolor=gray,    % other internal links in gray
%    linktocpage,         % in TOC, LOF and LOT, the link is on the page number
}


\renewcommand*{\backref}[1]{
  % default interface
  % #1: backref list
  %
  % We want to use the alternative interface,
  % therefore the definition is empty here.
}
\renewcommand*{\backreflastsep}{, }
\renewcommand*{\backreftwosep}{, }

\renewcommand*{\backrefalt}[4]{%
  % alternative interface
  % #1: number of distinct back references
  % #2: backref list with distinct entries
  % #3: number of back references including duplicates
  % #4: backref list including duplicates
  [citado en p\'ag.: #2]
}

\title{Plan de Trabajo}
\author{Luciana Benotti}
%\date{}

\begin{document}

\maketitle

\section*{T\'itulo: \emph{Pragm\'atica de Sistemas de Di\'alogo Situados}} 

\section{Objetivos} \label{objetivos}

La \emph{pragm\'atica} es un \'area de
estudio interdisplinaria en la que convergen las \'areas de ling\"u\'istica,
filosof\'ia
del lenguaje y socio-ling\"u\'istica. El \'area de pragm\'atica
estudia c\'omo el contexto influye en la interpretaci\'on del significado del
lenguaje natural. Un \emph{sistema de di\'alogo situado} es un sistema que emula
capacidades conversacionales humanas y que se encuentra
embebido en un entorno f\'isico que puede ser el mundo real (como es el caso,
por ejemplo, para los robots con capacidades conversacionales) o un mundo
virtual. Tal sistema no s\'olo debe ser capaz de comunicarse mediante
lenguaje natural con un interlocutor humano sino que tambi\'en debe ser
conciente
del estado de su entorno y de los cambios que ocurren en el mismo. 

El objetivo pr\'actico principal de mi plan de trabajo para los pr\'oximos tres
a\~nos es dise\~nar e implementar un sistema de di\'alogo que ayude a un usuario
a alcanzar una meta dada en un entorno virtual en tres dimensiones (3D). 
El objetivo pr\'actico da marco al tema te\'orico principal que investigar\'e:
el dise\~no de algoritmos pragm\'aticamente motivados que utilicen el
contexto conversacional para generar e interpretar oraciones en lenguaje
natural. Este objetivo te\'orico incluye el estudio y uso de t\'ecnicas de
representaci\'on del conocimiento e inferencia necesarias para consultar y
actualizar el contexto conversacional. 

En una primera etapa implementar\'e un prototipo con
capacidades de generaci\'on de lenguaje natural, es decir
ser\'a capaz de verbalizar instrucciones contextualizadas pero s\'olo podr\'a
interpretar contribuciones f\'isicas (manipulaci\'on del entorno 3D a trav\'es
del mouse y el teclado) y no ling\"u\'isticas de parte del usuario. En una
segunda etapa el prototipo ser\'a extendido para incluir no
s\'olo generaci\'on sino tambi\'en interpretaci\'on de languaje natural, es
decir el sistema ser\'a capaz de
comprender contribuciones ling\"u\'isticas generadas por el usuario. Por
ejemplo, el usuario podr\'a pedir aclaraciones al sistema o negociar la meta
de la interacci\'on. Finalmente el sistema de di\'alogo
implementado durante este proyecto ser\'a adaptado como herramienta para la
tarea de Ense\~nanza de Lenguas Asistida por Computador o CALL.\footnote{CALL es
acr\'onimo de \emph{Computer-Assisted Language
Learning}.}  

La interpretaci\'on y generaci\'on de dominio abierto de lenguaje natural
hablado es un desaf\'io del \'area de ling\"u\'istica computacional que est\'a
muy lejos de ser resuelto. Mis objetivos en esta \'area estar\'an
restringidos de tres formas motivadas por mi inter\'es en el estudio de
aspectos pragm\'aticos del lenguaje (en contraste con aspectos fonol\'ogicos,
l\'exicos, sint\'acticos, etc.). Para empezar, el sistema de di\'alogo no
ser\'a basado en habla sino en texto y aceptar\'a contribuciones
ling\"u\'isticas en un formato al estilo de la mensajer\'ia instant\'anea.
En segundo lugar, las contribuciones entendidas por el sistema estar\'an
restringidas al dominio del entorno virtual, por ejemplo, el usuario s\'olo
podr\'a referirse a objetos f\'isicos presentes en el entorno virtual; 
simplificando radicalmente el l\'exico y las ontolog\'ias requeridas por el
sistema. Tercero y \'ultimo, el sistema restringir\'a las
construcciones sint\'acticas y sem\'anticas entendidas por el sistema a
fragmentos del lenguaje natural como se explica en la
Secci\'on~\ref{antecedentes}.

%  (adaptando t\'ecnicas simb\'olicas como
% \emph{Conceptual Authoring}~\citep{hallett07} o estad\'isticas como
% \emph{Maximum Entropy}~\citep{sagae09}).

\section{Antecedentes y actividades planeadas} \label{antecedentes}

El \'area de pragm\'atica de lenguaje natural es un \'area
interdisciplinaria a
la que contribuyen teor\'ias ling\"u\'isticas (e.g., implicaturas
conversacionales~\citep{grice75}), sociol\'ogicas (e.g., an\'alisis
conversa\-cio\-nal~\citep{schegloff87b}) y filos\'oficas (e.g., teor\'ia de los
actos de habla~\citep{austin62}). Su objetivo es estudiar c\'omo el contexto en
el que una conversaci\'on esta situada contribuye al significado de cada cosa
que se diga durante esa conversaci\'on. La transmisi\'on de significado depende
no s\'olo de la informaci\'on ling\"u\'istica (entidades en foco, reglas
gramaticales y morfol\'ogicas, etc.), sino tambien extraling\"u\'istica
(entorno f\'isico donde la comunicaci\'on est\'a situada, experiencias
previas de los hablantes, objetivo de la conversaci\'on, etc.). Por lo tanto,
una misma oraci\'on puede significar cosas diferentes en distintos contextos; el
\'area de pragm\'atica estudia el proceso por el cual una oraci\'on es
desambig\"uada usando su contexto. En pragm\'atica se distingue entre oraci\'on
(forma gramatical que toma el acto ling\"u\'istico) y enunciado (oraci\'on m\'as
su contexto). La habilidad de entender una oraci\'on usando su contexto, es
decir, la habilidad de entender un enunciado, se conoce como competencia
pragm\'atica.
Estudiar las habilidades pragm\'aticas de una persona implica estudiar c\'omo
la persona hace
inferencias sobre una oraci\'on y su contexto para interpretar adecuadamente el
enunciado que el emisor intenta transmitir. 

Para que un sistema de di\'alogo interact\'ue de una forma natural con sus
usuarios, debe demostrar habilidades pragm\'aticas. En dicho sistema es
indispensable que las oraciones hagan expl\'icita la cantidad de informaci\'on
justa para que el enunciado sea interpretado de la manera deseada: si la
informaci\'on es demasiada, se retrasar\'a y aburrir\'a al usuario, si la
informaci\'on es muy
poca, el usuario no sabr\'a c\'omo llevar a cabo la tarea y cometer\'a errores. 
Por lo tanto, dicho sistema
debe definir qu\'e tipo de informacion contextual se debe representar y
qu\'e tareas de inferencia sobre la oraci\'on y el contexto son necesarias para
interpretar un enunciado. 
% Hacer estas dos tareas de
% forma correcta tendr\'a un
% impacto crucial sobre el desempe\~no de un sistema como el que proponemos
% desarrollar. 

Podemos entender como \emph{inferencia} toda operaci\'on que transforme
informaci\'on
impl\'icita en expl\'icita.  Esta definici\'on es lo
suficientemente general como para cubrir tareas que van desde la inferencia
l\'ogica (e.g., tareas de inferencia en
\emph{Description Logics}~\citep{DBLP:conf/dlog/2003handbook}), hasta tareas de
inferencia habituales en inteligencia artificial (e.g.,
\emph{Planning}~\citep{nau04}), y modelos de inferencia estad\'isticos (e.g.,
\emph{Maximum Entropy Models}~\citep{berger96}). Un sistema de di\'alogo realiza
continuamente operaciones de inferencia, tanto para interpretar como para
generar lenguaje natural. 

Se necesita inferencia para interpretar la informaci\'on recibida e incorporarla
al repositorio de datos. Como mencionamos en la Secci\'on~\ref{objetivos}, el
sistema restringir\'a las
construcciones sint\'acticas y sem\'anticas entendidas por el sistema a
fragmentos del lenguaje natural. Con este objetivo adaptar\'e t\'ecnicas
simb\'olicas como
\emph{Conceptual Authoring} sobre \emph{Description
Logics}~\citep{hallett07,piwek07} en una primera etapa y luego experimentar\'e
con
\emph{Maximum Entropy Models} siguiendo trabajo
reciente en el area de interpretaci\'on~\citep{sagae09}. Para la
inferencia de las implicaturas conversacionales necesarias para integrar el
significado literal en el contexto conversacional continuar\'e investigando el
uso de t\'ecnicas de planning no
cl\'asicas~\citep{benotti08, benotti09a, benotti10}. En trabajo previo he
expermientado con el uso de \emph{parsers} y constructores sem\'anticos
simb\'olicos
para esta tarea, los cuales son extremadamente costosos de
configurar para alcanzar una cobertura confiable~\citep{benotti09b,benotti10}. 

Inferencia es necesaria para decidir qu\'e parte de la
informaci\'on
disponible se debe transmitir y para modelar expectativas de c\'omo el usuario
reaccionar\'a a la informaci\'on transmitida. Una de las tareas de inferencia
principales en la selecci\'on del contenido a transmitir que es particularmente
importante en conversaciones situadas es la generaci\'on de expresiones
referenciales. Para esta tarea hemos investigado y evaluado el uso de algoritmos
de inferencia ad-hoc~\citep{amoia10}. En este proyecto planeo investigar el uso
de algoritmos de inferencia y lenguajes con propiedades y complejidad bien
estudiadas como el de~\cite{AKS08}, e investigar la extension de dicho
algoritmo a processos referenciales multi-sentencia (actualmente el algoritmo
s\'olo soporta generaci\'on de espresiones referenciales
uni-sentencia)~\citep{areces10,benotti10b}. Para modelar las expectativas de
c\'omo el usuario reaccionar\'a (ya sea con actos f\'isicos o actos del habla) a
la informaci\'on transmitida planeo continuar investigando el uso y las
limitaciones de \emph{planning} cl\'asico~\citep{benotti09c}. 

El sistema propuesto en este plan de trabajo es un sistema complejo, de
m\'ultiples componentes: un generador de lenguaje natural, un sistema de
\emph{planning}, un entorno 3D interactivo, entre otros. Dise\~nar e implementar
todos estos componentes desde cero requerir\'ia un esfuerzo prohibitivo. Uno de
los objetivos secundarios de este proyecto es utilizar herramientas de uso
general, como sistemas de \emph{planning}, como por ejemplo \emph{FF
planner}~\citep{hoffmann01}, y entornos virtuales para interacci\'on 3D como
\emph{GIVE\footnote{\url{http://www.give-challenge.org}}, Generating
Instructions in Virtual Environments}~\citep{byron09}. El uso de estos sistemas
(sobre los que ya tengo experiencia) permitir\'a un prototipado r\'apido de los
componentes que no constituyen el objetivo de investigaci\'on de este proyecto. 


El objetivo principal de GIVE no es proveer un entorno de interacci\'on 3D
sino actuar como medio de evaluaci\'on de sistemas de generaci\'on de lenguaje
natural situados en un entorno 3D. En el escenario propuesto por GIVE, el
usuario
humano lleva acabo una tarea de ``b\'usqueda del tesoro" en un entorno
virtual 3D y el trabajo del sistema participante es proveer, en tiempo
real, instrucciones en lenguaje natural que ayuden al usuario
a encontrar el tesoro escondido.  Por lo tanto, GIVE no s\'olo eval\'ua
la generaci\'on de lenguaje natural sino tambi\'en la habilidad del
sistema de participar en una interacci\'on situada en un entorno 3D.
Cada a\~no, GIVE funciona como un
\emph{challenge} en el cual sistemas de generaci\'on situados participan con el
objetivo de ser evaluados con respecto a distintas m\'etricas objetivas
(e.g., porcentaje de participantes que completaron la tarea exitosamente) y
subjetivas (e.g., deseo del usuario de volver a usar el sistema). Como parte de
este proyecto planeo participar del \emph{GIVE Challenge} (ver
Secci\'on~\ref{cronograma}). 


\section{Cronograma de actividades} \label{cronograma}



El cronograma de trabajo se estructura de la siguiente manera:

\paragraph{A\~no 1.} El objetivo de los primeros meses es obtener un
prototipo del sistema unidireccional.  Comenzaremos con un relevamiento
de la bibliograf\'ia.  Dado que el grupo de investigaci\'on en el que me
insertar\'e (ver Secci\'on~\ref{recursos-locales}) est\'a
constitu\'ido de expertos en distintas \'areas pertinentes al
proyecto, me concentrar\'e en obtener un
entendimiento com\'un de los distintos aspectos del problema, donde
cada experto contribuir\'a bibliograf\'ia adecuada de su \'area.  A
continuaci\'on comenzar\'e a estudiar
la plataforma GIVE, y a dise\~nar el algoritmo de interacci\'on
unidireccional.  Para esto ser\'a necesario definir y dise\~nar los
par\'ametros de interacci\'on a incorporar
en el sistema, los esquemas de representaci\'on de la
informaci\'on que el sistema necesita, y de las tareas de inferencia a
utilizar. El dise\~no de estos de este algoritmo estar\'a informado por el
corpus GIVE de interacciones humano-humano sobre el dominio de
GIVE~\citep{gargett10}.
Una vez que obtenga una version del sistema con su
correspondiente testeo y documentaci\'on, podemos comenzar con
su evaluaci\'on.  El tipo de evaluaci\'on a realizar debe ser definido
adecuadamente durante el periodo de desarrollo, para asegurar que el sistema
es capaz de prover la informaci\'on necesaria requerida durante la evaluaci\'on
(e.g., logueo de eventos, etc.). Como parte de la evaluaci\'on planeo participar
de la competencia GIVE, lo que impondr\'a deadlines concretos para obtener
un primer prototipo. 

Al final del a\~no comenzar\'a la preparaci\'on de art\'iculos y reportes para
la presentaci\'on de los resultados obtenidos hasta el momento.

\paragraph{A\~no 2.} El objetivo principal del segundo a\~no del proyecto es,
por un lado, extender y completar el prototipo de sistema con informaci\'on
unidireccional en base a la evaluaci\'on realizada al fin del a\~no anterior,
y utilizando el feedback obtenido de la
participaci\'on en el GIVE challenge.  Por otro lado, se comenzar\'a a agregar
capacidades de interacci\'on ling\"u\'istica bidireccional.
Un sistema bidireccional es mucho m\'as complejo que un sistema unidireccional. 
Nuestros objetivos en esta \'area estar\'an
restringidos de tres formas motivadas por nuestro foco en el estudio de
aspectos pragm\'aticos del lenguaje (en contraste con aspectos fonol\'ogicos,
l\'exicos, sint\'acticos, etc.). Para empezar, nuestro sistema de di\'alogo no
ser\'a basado en habla sino en texto y aceptar\'a contribuciones
ling\"u\'isticas en un formato al estilo de la mensajer\'ia instant\'anea.
En segundo lugar, las contribuciones entendidas por el sistema estar\'an
restringidas al dominio del entorno virtual, por ejemplo, el usuario s\'olo
podr\'a referirse a objetos f\'isicos presentes en el entorno virtual; 
simplificando radicalmente el l\'exico y las ontolog\'ias requeridas por el
sistema. Tercero y \'ultimo, el sistema restringir\'a las
construcciones sint\'acticas y sem\'anticas entendidas por el sistema a
fragmentos del lenguaje natural como se explica en la
Secci\'on~\ref{antecedentes}. Crucialmente, antes de comenzar con esta etapa de
dise\~no del algoritmo bidireccional recolectaremos un corpus bidireccional
usando la plataforma de \emph{Wizard of Oz} de GIVE~\citep{gargett10}. 

% Nos concentraremos en dos tipos espec\'ificos de instrucciones que el usuario
% podr\'a dar al sistema: (1) pedidos de aclaraci\'on de la \'ultima
% instrucci\'on dada, y (2) redefinici\'on del objetivo de la interacci\'on. 
% Estos
% dos tipos de instrucci\'on son representativos de `reparaciones' a corto y
% largo plazo.  Cada uno de ellos requerir\'a la redefinici\'on adecuada de
% las reglas pragm\'aticas que gobiernan la interacci\'on y de servicios
% de inferencia.  
Una vez m\'as, a medida que
estas nuevas capacidades sean agregadas al sistema, deberemos llevar a
cabo testeo y documentaci\'on, para finalmente dar paso a la evaluaci\'on.



\paragraph{A\~no 3.} El objetivo del \'ultimo a\~no es transformar el prototipo
de sistema de di\'alogo bidireccional obtenido durante el a\~no anterior, en un
tutor virtual para el aprendizaje de idiomas.  Dado que
\'este es un objetivo que incluye nuevos desaf\'ios (e.g., consideraciones
pedag\'ogicas en la presentaci\'on de la interacci\'on) comenzaremos otra
vez con un nuevo relevamiento bibliogr\'afico y consultas con expertos en
el \'area.  Adaptar los sistemas uni y bi direccionales para que permitan
interacci\'on tanto en castellano como en ingl\'es es una tarea de poca
complejidad (dada la simplicidad del dominio).  El mayor desaf\'io de esta etapa
es el dise\~no de
actividades y extensiones del sominio dentro de las capacidades del prototipo
que brinden al
usuario oportunidades para practicar el nuevo idioma.

Al final del a\~no comenzar\'a la preparaci\'on de art\'iculos y reportes para
la presentaci\'on de los resultados obtenidos durante el proyecto adem\'as de
la especificaci\'on de trabajo futuro que el proyecto motive.

\medskip

\noindent
A continuaci\'on se resumen las tareas a desarrollar en los tres a\~nos de
trabajo (organizadas trimestralmente):

{\footnotesize
\begin{center}
\begin{tabular}{|p{7cm}||p{2mm}|p{2mm}|p{2mm}|p{2mm}||p{2mm}|p{2mm}|p{2mm}|p{2mm
}||p{2mm}|p{2mm}|p{2mm}|p{2mm}||}
\hline
 \rowcolor[rgb]{0.8,0.8,0.8}\hspace{3.5cm}Tarea & 1 & 2 & 3 & 4 & 1 & 2 & 3 & 4
& 1 & 2 & 3 & 4\\
\hline 1. Relevamiento bibliogr\'afico & $\times$ & $\times$
&&&&&&&$\times$&&&\\
\hline 2. Estudio del material bibliogr\'afico & $\times$ & $\times$ & $\times$
&  &&&&&&&&\\
\hline 3. Estudio de la plataforma GIVE & & $\times$ &$\times$&&&&&&&&&\\
\hline 4. Dise\~no, impl., test. alg.\ interacc. unidir. & & & $\times$
& $\times$&$\times$&$\times$&&&&&&\\
\hline 5. Recopilaci\'on corpus de interacci\'on & & & &&$\times$
&$\times$&&&&&&\\
\hline 6. Dise\~no, impl., testing alg.\ interacc. bidir. & & &  &
&&$\times$&$\times$&$\times$&$\times$&&&\\
% \hline 7. Dise\~no e impl., test. alg.\ inferencia & & & $\times$ &
% $\times$&&&$\times$&$\times$&&&&\\
\hline 7. Testing &&&&$\times$&&&&$\times$&&&&$\times$\\
\hline 8. Desarrollo de tutor virtual &&&&&&&&&$\times$&$\times$&$\times$&\\
\hline 9. Evaluaci\'on
&&&&$\times$&$\times$&&&$\times$&$\times$&&$\times$&$\times$\\
\hline 10. Documentaci\'on &&&&&&&&&&&$\times$&$\times$\\
\hline 11. Diseminaci\'on de resultados
&&&&$\times$&$\times$&&&$\times$&$\times$&&&$\times$\\
\hline 12. Elaborac.\ y presentaci\'on de resultados 
&&&&$\times$&$\times$&$\times$&&$\times$&$\times$&$\times$&&$\times$\\\hline
\end{tabular}\end{center}
}

\section{Factibilidad y Disponibilidad de Recursos}

La gran mayor parte de los recursos necesarios para llevar a cabo el presente
proyecto son recursos humanos y no recursos materiales. Por lo tanto la
colaboraci\'on con expertos en las diferentes \'areas relevantes a este plan
interdisciplinario es crucial asi como la formaci\'on de recursos humanos y
promoci\'on del \'area de Ling\"u\'istica Computacional en Argentina, un \'area
que actualmente esta extremadamente subsdesarrollada en nuestro pa\'is y que
tiene es necasario desarrollar (como argumento en Secci\'on~\ref{economico}). 

Los recursos materiales necesarios para llevar a cabo este proyecto son
computadoras (un servidor y unas pocas computadoras personales) y
bibliograf\'ia que se comprar\'an son los recursos del
proyecto~\citep{areces10b}. Dicho proyecto cuenta con aproximadamente 15.000
d\'olares estadounidenses para este fin. 

El resto de los recursos del proyecto~\citep{areces10b} (aproximadamente 100.000
d\'olares estadounidenses) ser\'an usados para financiar distintos tipos de
recursos humanos necesarios para este proyecto (incluyendo horas de
programadores, vi\'aticos para asistir a conferencias y para visitar
colaboradores en otros pa\'ises, costos de los participantes en experimentos,
etc.). Dichos recursos ser\'an compartidos en partes iguales con los otros dos
investigadores responsables del proyecto. 

El recurso m\'as importante para llevar a cabo este proyecto son las \'areas de
especialidad de los distintos colaboradores con los que interactuar\'e durante
el mismo. La factibilidad de este
proyecto est\'a basada en tres fundamentos: mi inter\'es y experiencia en el
\'area (ver Secci\'on~\ref{antecedentes}), el grupo en el que trabajar\'e en la
Universidad Nacional de C\'ordoba y las colaboraciones con otros grupos de
investigaci\'on (vers subsecciones siguientes). 


\subsection{Recursos humanos locales} \label{recursos-locales}

El grupo de investigaci\'on
PLN\footnote{\url{http://www.cs.famaf.unc.edu.ar/~pln}} (es decir,
Procesamiento de Lenguaje Natural), en el cual este plan de trabajo ser\'a
llevado a cabo, fue fundado en el a\~no 2005. Desde entonces el grupo est\'a
desarrollando un importante trabajo de formaci\'on de recursos humanos, dando
cursos de grado y de post-grado a estudiantes de Licenciatura y del Doctorado de
Ciencias de la Computaci\'on de la Universidad Nacional de C\'ordoba. Adem\'as,
el grupo PLN trabaja en el desarrollo de varios proyectos de investigaci\'on e
integraci\'on con otros grupos de investigaci\'on (ver
Secci\'on~\ref{colaboraciones}). 

Este plan de trabajo se beneficiar\'a principalmente de dos \'areas de
experiencia de los miembros del grupo PLN: (1) Representaci\'on de conocimiento
e inferencia y (2) evaluaci\'on de sistemas que procesan lenguaje natural. 

Con respecto a la representaci\'on de conocimiento e inferencia la experiencia
del grupo es particularmente en el estudio te\'orico y aplicado de lenguajes
(con publicaciones como~\citep{arec:logi00,arec:reso01,arec:reso10} entre otras)
para la representaci\'on del conocimiento (e.g. \emph{Description Logics}). Asi
como la implementaci\'on de motores de inferencia sobre estos lenguajes
libremente disponibles\footnote{\url{http://www.glyc.dc.uba.ar/intohylo/}}.
Ademas, han investigado el uso de t\'ecnicas de inferencia no cl\'asicas sobre
estos lenguajes para la tarea de generaci\'on de expresiones
referenciales~\citep{AKS08}.

La segunda l\'inea de investigaci\'on del grupo de PLN que es directamente
relevante para este plan de trabajo es evaluaci\'on de sistemas de lenguaje
natural dependientes del contexto. El grupo de PLN tiene experiencia en el
desarrollo de modelos de evaluaci\'on de sistemas de traducci\'on
autom\'atica (con publicaciones como~\citep{estrella08,estrella10} entre
otras). Recientement, t\'ecnicas de evaluaci\'on de traducci\'on autom\'atica
han sido aplicadas con \'exito a la evaluaci\'on de sistemas de generaci\'on
contextualizada de lenguaje natural como el investigado en este
plan~\citep{reiter09}. Por lo tanto, la experiencia del grupo PLN en esta
\'area ser\'a \'util para la evaluaci\'on de nuestro sistema. 

Otras l\'ineas de investigaci\'on del grupo PLN que no estan directamente
relacionadas con los objetivos esperables dentro de los tres a\~nos de
investigaci\'on descriptos en este plan pero que pueden ser relevantes en un
futuro incluyen: inducci\'on de gram\'aticas, poblaci\'on autom\'atica de
ontolog\'ias desde texto y an\'alisis sint\'actico estad\'istico de
mon\'ologos.  
          


\subsection{Colaboraciones} \label{colaboraciones}

Los miembros del grupo PLN y en particular, yo misma, tenemos varias
colaboraciones con grupos de investigaci\'on nacionales e internacionales en
ling\"u\'istica computacional y campos relacionados que son directamente
relevantes para este plan. 

Los miembros del grupo PLN han colaborado intensamente con el grupo
TALARIS\footnote{\url{http://talaris.loria.fr}}
del \emph{Laboratoire Lorrain de Recherche en Informatique et ses
Applications (LORIA)}, del que fuimos miembros. El tema de investigaci\'on
principal de TALARIS es ling\"u\'istica computacional con un fuerte \'enfasis
en sem\'antica e inferencia. En el marco de esta colaboraci\'on estamos
participando conjuntamente en la edici\'on 2010 de la competencia GIVE. Durante
el proceso de dise\~no y desarrollo nuestros dos sistemas que est\'an
participando en la competencia investigamos conjuntamente el uso de diferentes
estrategias de generaci\'on de referencias situadas~\citep{amoia10}. 

Adem\'as, he colaborado del grupo de investgaci\'on \emph{Virtual Humans} del
\emph{Institute for Creative
Technologies (ICT)}\footnote{\url{http://ict.usc.edu/projects/virtual_humans}}
de la \emph{University of Southern California} (como investigadora invitada
durante tres meses). En particular hemos trabajado en la modelizaci\'on
computacional de implicaturas asociadas a sentencias comparativas usando
t\'ecnicas de planning no cl\'asicas. Todos los a\~nos ICT ofrece programas de
intercambio que usar\'e para mandar estudiantes y fortalecer nuestra
colaboraci\'on. 


% All these collaborations are directly related to the main theme of the
% project 
% described in this article.  The PLN group has also research collaborations
% with 
% other international research teams in the framework of other scientific
% programs. 
% For example, the PLN group has being part of a recently finished international
% project 
% MICROBIO\footnote{\url{http://www.microbioamsud.net}} on ontology population
% from raw text.
% The project was funded by the
% Stic-Amsud\footnote{\url{http://www.sticamsud.org}} program, 
% a scientific-technological cooperation program integrated by France,
% Argentine,
% Brazil, 
% Chile, Paraguay, Peru and Uruguay. The expertise obtained during this project
% might
% be useful in the future when trying to extend our GIVE ontologies to new
% domains. 

A nivel nacional PLN colabora con
GLyC\footnote{\url{http://www.glyc.dc.uba.ar}}, \emph{Grupo de L\'ogica,
Lenguaje y Computabilidad} en el \'area de representaci\'on del conocimiento e
inferencia. GLyC es parte del Departamento de Ciencias de la Computaci\'on en
la Universidad de Buenos Aires. Durante los a\~nos 2008 y 2009 colaboramos bajo
el marco de un proyecto bilateral financiado por MinCyT-INRIA para investigar
t\'ecnicas de gesti\'on del conocimiento en sistemas de
di\'alogo~\cite{benotti10b}. Actualmente, estamos organizando conjuntamente
ELiC\footnote{\url{http://www.glyc.dc.uba.ar/elic2010}}, 
la \emph{Primera Escuela en Ling\"u\'istica Computacional} en Argentina, que
ocurrir\'a en Julio en la Universidad de Buenos Aires.  ELiC 2010 ser\'a
co-organizada junto con 
ECI\footnote{\url{http://www.dc.uba.ar/events/eci/2009/eci2009}},
Escuela de Ciencias Inform\'aticas que se realiza anualmente. 
El objetivo de ELiC es crear un espacio para introducir el
campo de ling\"u\'istica computacional a estudiantes en Argentina. Gracias al
soporte de la \emph{North American Chapter of the Association for
Computational Linguistics (NAACL)}, y de las Universidades de C\'ordoba y de
Buenos Aires, ELiC ofrece becas de viaje para estudiantes del interior del
pa\'is.

Finalmente, junto a la Universidad Nacional de San Luis estamos proponiendo un
workshop en Ling\"u\'istica computacional como evento sat\'elite de IBERAMIA
2010\footnote{\url{http://cs.uns.edu.ar/iberamia2010}}, 
la Conferencia Ibero-Americana en Inteligencia Artificial, que ser\'a
organizada por primera vez en Argentina en la Universidad del Sur
en la ciudad de Bah\'ia Blanca.            


\section{Impacto Socio-Econ\'omico} \label{economico}

Los temas a investigar en el marco de este proyecto son de relevancia en el
panorama Argentino actual por, al menos, dos razones de peso.

Por un lado, el proyecto
integra y desarrolla diferentes aspectos clave del \'area de ling\"u\'istica
computacional (sintaxis, sem\'antica, pragm\'atica, representaci\'on,
inferencia, evaluaci\'on). El \'area de ling\"u\'istica computacional y su
aplicaci\'on al tratamiento autom\'atico del lenguaje natural han tenido un
gran desarrollo internacional en los \'ultimos a\~nos, con aplicaciones como los
sistemas de b\'usqueda en la web, los sistemas de traducci\'on y resumen
autom\'aticos, las interfaces de voz, etc. Sin embargo, el \'area es casi
inexistente actualmente en Argentina. Este proyecto se contar\'a entre 
las contribuciones que apuntan a revertir esta situaci\'on.
Por otro lado, el objetivo \'ultimo de este proyecto es investigar el uso de la
plataforma
desarrollada en el \'area de educaci\'on a distancia (concretamente, como
plataforma de aprendizaje de idiomas).

Claramente, la educaci\'on a distancia es un recurso valioso para superar el 
problema de la centralizaci\'on de recursos
educativos en el pa\'is. Sin embargo, desarrollar herramientas adecuadas en
este \'area espec\'ifica es dif\'icil.

Para desarrollar sistemas de ense\~nanza a distancia es esencial modelar el
avance del aprendizaje del usuario. Esto requiere un sistema capaz de ser
consciente de la evoluci\'on del usuario, y que tenga en cuenta sus logros y sus
problemas. Este tipo de interacci\'on entre usuario y sistema puede modelarse
como un di\'alogo, cuyo contexto registra el conocimiento adquirido (del usuario
sobre el material del curso, y del sistema sobre el usuario). La gesti\'on de
este tipo de di\'alogo es particularmente interesante, ya que el sistema debe
ser
capaz de interpretar los requerimientos de, y generar respuestas adecuadas para,
usuarios no expertos cuyo conocimiento evoluciona durante la interacci\'on.
Adem\'as, el sistema debe ser capaz de representar apropiadamente tanto la
informaci\'on concerniente al material del curso, como la informaci\'on
concerniente a la evoluci\'on del usuario. Por ejemplo, el sistema debe ser
capaz de diagnosticar qu\'e parte del material del curso debe ser revisada a
partir de las respuestas err\'oneas del usuario.  Por \'ultimo, el sistema debe
poder evaluar la interacci\'on con el usuario, para poder decidir
que objetivos del aprendizaje fueron alcanzados.
Los resultados te\'oricos y pr\'acticos obtenidos durante el proyecto,
contribuyen directamente a la soluci\'on de estos problemas.

% \bibliographystyle{named}
% \bibliography{pict09}

%\begin{thebibliography}{} \renewcommand{\itemsep}{1mm}
  
\begin{thebibliography}{} \renewcommand{\itemsep}{1mm}

\bibitem[\protect\citeauthoryear{Amoia \bgroup \em et al.\egroup
  }{Submitted}]{amoia10}
M.~Amoia, A.~Denis, L.~Benotti, and C.~Gardent.
\newblock Evaluating referring strategies in situated instruction giving.
\newblock {\em Topics in Cognitive Science}, Submitted.

\bibitem[\protect\citeauthoryear{Areces and Figueira}{2010}]{areces10}
C.~Areces and S.~Figueira.
\newblock Characterization, complexity and optimized algorithms.
\newblock Technical Report PICT-2010-0688, Proyectos de Investigación
  Cient\'ifica y Tecnol\'ogica (PICT), Agencia Nacional de Promoci\'on
  cient\'ifica y Tecnol\'ogica, Argentina, 2010.
\newblock Submitido a la convocatoria PICT Bicentenario.

\bibitem[\protect\citeauthoryear{Areces and Gor{\'\i}n}{2010}]{arec:reso10}
C.~Areces and D.~Gor{\'\i}n.
\newblock Resolution with order and selection for hybrid logics.
\newblock {\em Accepted for publication in the Journal of Automated Reasoning},
  2010.

\bibitem[\protect\citeauthoryear{Areces \bgroup \em et al.\egroup
  }{2001}]{arec:reso01}
C.~Areces, H.~de~Nivelle, and M.~de~Rijke.
\newblock Resolution in modal, description and hybrid logic.
\newblock {\em Journal of Logic and Computation}, 11(5):717--736, 2001.

\bibitem[\protect\citeauthoryear{Areces \bgroup \em et al.\egroup
  }{2008}]{AKS08}
C.~Areces, A.~Koller, and K.~Striegnitz.
\newblock Referring expressions as formulas of description logic.
\newblock In {\em Proc.\ of INLG-08}, 2008.

\bibitem[\protect\citeauthoryear{Areces \bgroup \em et al.\egroup
  }{2010}]{areces10b}
C.~Areces, L.~Benotti, and P.~Estrella.
\newblock Sistemas de di\'alogo para entornos virtuales.
\newblock Technical Report PICT-2010-0306, Agencia Nacional de Promoci\'on
  cient\'ifica y Tecnol\'ogica, Argentina, 2010.
\newblock Submitido a la convocatoria PICT asociada al Proyecto de
  Investigación y Desarrollo para la Radicaci\'on de Investigadores (PIDRI).

\bibitem[\protect\citeauthoryear{Areces}{2000}]{arec:logi00}
C.~Areces.
\newblock {\em Logic Engineering: The Case of Description and Hybrid Logics}.
\newblock PhD thesis, ILLC, University of Amsterdam, 2000.

\bibitem[\protect\citeauthoryear{Austin}{1962}]{austin62}
J.~Austin.
\newblock {\em How to do Things with Words}.
\newblock Oxford University Press, 1962.

\bibitem[\protect\citeauthoryear{Baader \bgroup \em et al.\egroup
  }{2003}]{DBLP:conf/dlog/2003handbook}
F.~Baader, D.~Calvanese, D.~McGuinness, D.~Nardi, and P.~Patel-Schneider,
  editors.
\newblock {\em The Description Logic Handbook: Theory, Implementation, and
  Applications}. Cambridge University Press, 2003.

\bibitem[\protect\citeauthoryear{Benotti and Traum}{2009}]{benotti09a}
L.~Benotti and D.~Traum.
\newblock A computational account of comparative implicatures for a spoken
  dialogue agent.
\newblock In {\em Proc.\ of IWCS-8}, 2009.

\bibitem[\protect\citeauthoryear{Benotti \bgroup \em et al.\egroup
  }{2010}]{benotti10b}
L.~Benotti, P.~Estrella, and C.~Areces.
\newblock Dialogue systems for virtual environments.
\newblock In {\em Proc.\ of the NAACL-HLT Young Investigators in the Americas
  Workshop}. Association of Computational Linguistics, 2010.

\bibitem[\protect\citeauthoryear{Benotti}{2008}]{benotti08}
L.~Benotti.
\newblock Accommodation through tacit sensing.
\newblock In {\em Proc.\ of the {SEMDIAL} 2008 Workshop on the Semantics and
  Pragmatics of Dialogue ({LONDIAL} 2008)}, 2008.

\bibitem[\protect\citeauthoryear{Benotti}{2009a}]{benotti09c}
L.~Benotti.
\newblock Clarification potential of instructions.
\newblock In {\em Proceedings of the SIGDIAL 2009 Conference}, pages 196--205.
  ACL, 2009.

\bibitem[\protect\citeauthoryear{Benotti}{2009b}]{benotti09b}
L.~Benotti.
\newblock Frolog: An accommodating text-adventure game.
\newblock In {\em Proceedings of the Demonstrations Session at the 12th
  Conference of the European Chapter of the ACL (EACL 2009)}, pages 1--4. ACL,
  2009.

\bibitem[\protect\citeauthoryear{Benotti}{2010}]{benotti10}
L.~Benotti.
\newblock {\em Implicature as an Interactive Process}.
\newblock PhD thesis, Universit\'e Henri Poincar\'e, INRIA Nancy Grand Est,
  Francia, 2010.

\bibitem[\protect\citeauthoryear{Berger \bgroup \em et al.\egroup
  }{1996}]{berger96}
A.~Berger, V.~Pietra, and S.~Pietra.
\newblock A maximum entropy approach to natural language processing.
\newblock {\em Comput. Linguist.}, 22(1):39--71, 1996.

\bibitem[\protect\citeauthoryear{Byron \bgroup \em et al.\egroup
  }{2009}]{byron09}
D.~Byron, A.~Koller, K.~Striegnitz, J.~Cassell, R.~Dale, J.~Moore, and
  J.~Oberlander.
\newblock Report on the {1}st {GIVE} challenge.
\newblock In {\em Proc.\ of ENLG}, pages 165--173, 2009.

\bibitem[\protect\citeauthoryear{Estrella \bgroup \em et al.\egroup
  }{2010}]{estrella10}
P.~Estrella, A.~Popescu-Belis, and M.~King.
\newblock The femti guidelines for contextual mt evaluation: principles and
  tools.
\newblock {\em Evaluation of Translation Technology, Linguistica Antverpiensia,
  New Series}, 2010.
\newblock Submitted.

\bibitem[\protect\citeauthoryear{Estrella}{2008}]{estrella08}
P.~Estrella.
\newblock {\em Evaluation of Machine Translation: towards a user-customised
  quality model}.
\newblock PhD thesis, Escuela de Traducción e Interpretación, Universidad de
  Ginebra, Suiza, 2008.
\newblock Menci\'on Tr\`es honorable avec f\'elicitation du jury.

\bibitem[\protect\citeauthoryear{Gargett \bgroup \em et al.\egroup
  }{2010}]{gargett10}
A.~Gargett, K.~Garoufi, A.~Koller, and K.~Striegnitz.
\newblock The give-2 corpus of giving instructions in virtual environments.
\newblock In {\em Proceedings of the 7th International Language Resources and
  Evaluation Conference (LREC)}, 2010.

\bibitem[\protect\citeauthoryear{Grice}{1975}]{grice75}
P.~Grice.
\newblock Logic and conversation.
\newblock In P.~Cole and J.~Morgan, editors, {\em Syntax and Semantics: Vol. 3:
  Speech Acts}, pages 41--58. Academic Press, 1975.

\bibitem[\protect\citeauthoryear{Hallett \bgroup \em et al.\egroup
  }{2007}]{hallett07}
C.~Hallett, D.~Scott, and R.~Power.
\newblock Composing questions through conceptual authoring.
\newblock {\em Comput. Linguist.}, 33(1):105--133, 2007.

\bibitem[\protect\citeauthoryear{Hoffmann and Nebel}{2001}]{hoffmann01}
J.~Hoffmann and B.~Nebel.
\newblock The {FF} planning system: Fast plan generation through heuristic
  search.
\newblock {\em JAIR}, 14:253--302, 2001.

\bibitem[\protect\citeauthoryear{Nau \bgroup \em et al.\egroup }{2004}]{nau04}
D.~Nau, M.~Ghallab, and P.~Traverso.
\newblock {\em Automated Planning: Theory \& Practice}.
\newblock Morgan Kaufmann Publishers Inc., 2004.

\bibitem[\protect\citeauthoryear{Piwek and Power}{2007}]{piwek07}
D.~Hardcastle Piwek, P. and R.~Power.
\newblock Dialogue games for crosslingual communication.
\newblock In {\em Proc.\ of the {SEMDIAL} Workshop on the Semantics and
  Pragmatics of Dialogue ({DECALOG} 2007)}, pages 117--124, 2007.

\bibitem[\protect\citeauthoryear{Reiter and Belz}{2009}]{reiter09}
E.~Reiter and A.~Belz.
\newblock An investigation into the validity of some metrics for automatically
  evaluating natural language generation systems.
\newblock {\em Comput. Linguist.}, 35(4):529--558, 2009.

\bibitem[\protect\citeauthoryear{Sagae \bgroup \em et al.\egroup
  }{2009}]{sagae09}
K.~Sagae, G.~Christian, D.~DeVault, and D.~Traum.
\newblock Towards natural language understanding of partial speech recognition
  results in dialogue systems.
\newblock In {\em Proceedings of HLT-NAACL 2009}, pages 53--56. Association for
  Computational Linguistics, 2009.

\bibitem[\protect\citeauthoryear{Schegloff}{1987}]{schegloff87b}
E.~Schegloff.
\newblock Some sources of misunderstanding in talk-in-interaction.
\newblock {\em Linguistics}, 8:201--218, 1987.

\end{thebibliography}




\end{document}
