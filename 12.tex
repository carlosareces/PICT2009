\MySubSection{12. Composici\'on del Grupo de Investigaci\'on y Areas de Especializaci\'on}

\begin{center}\small
    \begin{minipage}{\linewidth}
        \begin{center}
        \begin{tabular}{|l|l|}
        \hline\hline
        Nombre &  Especializaci\'on \\
        \hline
        Dr.\ Carlos Areces &
          L\'ogica Computacional\\
        & Representaci\'on de Conocimiento\\
        & Complejidad Algor\'itmica
      \\ \hline
        Dra.\ Paula Estrella &
          Ling\"u\'istica Computacional \\
        & Evaluaci\'on de Sistemas de PLN
      \\
		& Traducci\'on Autom\'atica \\ \hline
        Lic.\ Luciana Benotti &
          Pragm\'atica \\
        & Sistemas de Di\'alogo\\
        & L\'ogica Computacional\\
        \hline\hline
        \end{tabular}
        \end{center}
    \end{minipage}
\end{center}

\MySubSection{12.1. Publicaciones m\'as relevantes del grupo investigador}
\begin{itemize}

\item[$\triangleright$] El Dr.\ Carlos Areces se especializa en l\'ogica computacional,
en particular en el estudio te\'orico y aplicado de lenguajes para
representaci\'on del conocimiento (e.g., l\'ogicas modales, h\'ibridas y
para la descripci\'on).  Las publicaciones
m\'as importantes sobre aspectos te\'oricos en estos lenguajes
son~\citep{ABM01,arec:hybr05b}.  Ha desarrollado tambi\'en demostradores
autom\'aticos para estos lenguajes~\citep{ANR01,arec:hylo02a,AG06,Hoffmann2007}
accessibles en \url{http://www.glyc.dc.uba.ar/intohylo/}.
Ha trabajado tambi\'en en el dise\~no de algoritmos para la generaci\'on de
expresiones referenciales~\citep{AKS08}, y en el estudio de lenguajes l\'ogicos
adecuados para la representaci\'on sem\'antica~\citep{AF08}.

\item[$\triangleright$] La Dra.\ Paula Estrella ha realizado su tesis doctoral en la evaluaci\'on contextual de sistemas de TA~\citep{estr:impr08, estr:femt09}, proponiendo un modelo generalizable y aplicable a otras otras \'areas del PLN, como se muestra en \citep{Miller2008}. Tambi\'en ha trabajado en el desarrollo y mejora de sistemas de TA estad\'istica desde y hacia varios idiomas, por ejemplo el Espa\~nol~\citep{estr:expe05} y entre los idiomas \{Franc\'es, Ingl\'es, Japon\'es, Arabe\} \citep{rayner-EtAl:2009:GEAF}. Adem\'as ha participado en proyectos que involucran la creaci\'on y evaluaci\'on de corpora correspondientes a anotaciones multimodales~\citep{pope:estr07} y trabaj\'o en un prototipo para la evaluaci\'on autom\'atica de anotaciones de di\'alogos multimodales \citep{multieval}, investigando tambi\'en la aplicaci\'on de m\'etricas provenientes de otras \'areas; el prototipo est\'a disponible p\'ublicamente en \url{http://www.issco.unige.ch:8080/multieval/}.


\item[$\triangleright$] La Lic.\ Luciana Benotti ha trabajado en el desarrollo de un sistema de
di\'alogo completo, llamado Frolog, situado en una aventura de
texto~\citep{benotti09b} (la implementacion abarca tareas de analisis
sint\'actico hasta tareas pragmaticas). Tambi\'en ha extendido sistemas de
dialogo ya existentes con habilidades pragmaticas. Primero extendi\'o un
sistema desarrollado por Alexander Koller~\citep{koller04} usando un sistema de
planning clasico~\citep{benotti07} y un sistema de planning con informaci\'on
incompleta~\citep{benotti08}. Y luego extendi\'o el sistema de di\'alogo
desarrollado por el grupo de David Traum usando la arquitectura cognitiva
SOAR~\citep{benotti09a} (para el caso de oraciones comparativas). Finalmente,
trabaj\'o en el analisis empirico de corpora de dialogo humano-humano utilizando
diferentes tecnicas de planning para precedir los sub-dialogos de reparacion que
ocurren en los corpora~\citep{benotti09c}.

\end{itemize}

\MySubSection{12.2. Colaboraci\'on Internacional}
A continuaci\'on se detallan grupos de investigaci\'on relevantes con los que los  participantes del presente proyecto colaboran.


\begin{itemize}
    \item[$\triangleright$] El Dr. Carlos Areces fue miembro
del grupo TALARIS \url{http://talaris.loria.fr}, parte del
\textit{Laboratoire Lorrain de Recherche en Informatique et ses Applications} (LORIA - \url{http://www.loria.fr}). El principal tema de investigaci\'on de TALARIS es la ling\"u\'istica computacional con \'enfasis en sem\'antica e inferencia. Actualmente, colabora con los siguientes investigadores e instituciones especialistas en el \'area de inferencia y l\'ogica computacional:
\begin{itemize}
    \item[-]  Patrick Blackburn \url{http://www.loria.fr/~blackbur} (director de TALARIS) es especialista en el \'area de l\'ogicas modales y aplicaciones
en ling\"u\'istica computacional.
\item[-] Grupo de L\'ogica, Lenguaje y Computabilidad (GLyC) \url{http://www.glyc.dc.uba.ar/} del Departamento de Computaci\'on de la Universidad
de Buenos Aires.
\item[-] Ron Petrick \url{http://homepages.inf.ed.ac.uk/rpetrick/} es el
desarrollador de PKS, uno de los pocos sistemas de planning con informaci\'on
incompleta funcionales hoy en d\'ia.
\end{itemize}

\item[$\triangleright$] La Dra. Paula Estrella fue miembro del grupo\textit{Traitement Informatique Multilingue} (ISSCO/TIM) de la Facultad de Traducci\'on e Interpretaci\'on de la Universidad de Ginebra y del grupo \textit{Database management and meeting analysis} (IM2.DMA) dentro del Polo de Investigaci\'on \textit{Interactive Multimodal Information Management} (IM2 - \url{http://www.im2.ch/}). Acutalmente colabora con las siguientes instituciones especialistas en las \'areas de Interacci\'on Hombre-M\'aquina y  Ling\"u\'istica Computacional:
\begin{itemize}
    \item[-]  Instituto de Investigaci\'on Idiap, Martigny, Suiza -  \url{http://www.idiap.ch/}
\item[-] Grupo de Tratamiento Inform\'atico Multiling\"ue, Universidad de Ginebra, Suiza - \\ \url{http://www.issco.unige.ch/}
\item[-]  Grupo de Inteligencia Artificial y Pervasiva, Universidad de Friburgo, Suiza - \\ \url{http://diuf.unifr.ch/pai/wiki/doku.php}
\end{itemize}

\item[$\triangleright$] La Lic. Luciana Benotti realiza su doctorado en el
grupo TALARIS y colabora actualmente con los siguientes investigadores e instituciones
especialistas en generaci\'on de lenguaje natural y sistemas de di\'alogo:
\begin{itemize}
    \item[-]  Alexander Koller \url{http://www.coli.uni-saarland.de/~koller/}.
Alexander Koller es uno de los organizadores del GIVE Challenge y el
coordinador de la implementaci\'on de la plataforma GIVE.
    \item[-] David Traum \url{http://people.ict.usc.edu/~traum/} es, hoy en d\'ia, uno
de los investigadores mas renombrados del \'area de sistemas de di\'alogo.
Adem\'as es el director del grupo de sistemas de di\'alogo en el ``Institute
for Creative Techologies'' \url{http://ict.usc.edu/}.
\item[-] Claire Gardent \url{http://www.loria.fr/~gardent/} (vice-directora de
TALARIS) es especialista en generaci\'on de lenguaje natural, adquisici\'on
lexica y desarrollo de gram\'aticas.
\item[-] Agust\'in Gravano \url{http://www.glyc.dc.uba.ar/agustin/} es especialista
en el estudio de variaci\'ones pros\'odicas en sisteams de di\'alogo.

\end{itemize}
\end{itemize}



