\MySubSection{11. Impacto Socio-Econ\'omico}

Los temas a investigar en el marco de este proyecto son de relevancia en el panorama Argentino actual por, al menos, dos razones de peso.

Por un lado, el proyecto
integra y desarrolla diferentes aspectos clave del \'area de ling\"u\'istica computacional (sintaxis, sem\'antica, pragm\'atica, representaci\'on,
inferencia, evaluaci\'on). El \'area de ling\"u\'istica computacional y su
aplicaci\'on al tratamiento autom\'atico del lenguaje natural han tenido un
gran desarrollo internacional en los \'ultimos a\~nos, con aplicaciones como los sistemas de b\'usqueda en la web, los sistemas de traducci\'on y resumen autom\'aticos, las interfaces de voz, etc. Sin embargo, el \'area es casi inexistente actualmente en Argentina.


Por otro lado, durante las \'ultimas etapas
del proyecto se propone investigar el uso de la plataforma
desarrollada en el \'area de educaci\'on a distancia (concretamente, como
plataforma de aprendizaje de idiomas).
Claramente, la educaci\'on a distancia es una herramienta
que contribuye a superar el problema de la centralizaci\'on de recursos
educativos en el pa\'is, pero desarrollar herramientas adecuadas en
este \'area espec\'ifica es dif\'icil.

Para desarrollar sistemas de ense\~nanza a distancia es esencial modelar el avance del aprendizaje del usuario. Esto requiere un sistema capaz de ser consciente de la evoluci\'on del usuario, y que tenga en cuenta sus logros y sus problemas. Este tipo de interacci\'on entre usuario y sistema puede modelarse como un dialogo, cuyo contexto registra el conocimiento adquirido (del usuario sobre el material del curso, y del sistema sobre el usuario). La gesti\'on de este tipo de dialogo es particularmente interesante, ya que el sistema debe ser capaz de interpretar los requerimientos de, y generar respuestas adecuadas para, usuarios no expertos cuyo conocimiento evoluciona durante la interacci\'on. Adem\'as, el sistema debe ser capaz de representar apropiadamente tanto la informaci\'on concerniente al material del curso, como la informaci\'on concerniente a la evoluci\'on del usuario. Por ejemplo, el sistema debe ser capaz de diagnosticar que parte del material del curso debe ser revisada a partir de las respuestas err\'oneas del usuario.  Por \'ultimo, el sistema debe poder evaluar la interacci\'on con el usuario, para poder decidir
que objetivos del aprendizaje fueron alcanzados.

Los resultados te\'oricos y pr\'acticos obtenidos durante el proyecto,
contribuyen directamente a la soluci\'on de estos problemas.

